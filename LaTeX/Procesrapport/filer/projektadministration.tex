\chapter{Projektadministration}

\section{Samarbejdsaftale}
I projektets spæde opstart, var det første som blev udformet og klarlagt, en samarbejdsaftale, fungerende som forventningsafstemning i gruppen. Denne aftale blev anvendt som et værktøj til at få diskuteret vigtige parametre for samarbejdet, herunder mødetider, arbejdsform, målsætning, konflikthåndtering mm. \citep{RefWorks:12}. Samarbejdsaftalen fremgår af bilag €€. 
Det konkluderes, at samarbejdsaftalen har fungeret efter hensigten og, at emnerne har været klarlagte, idet det ikke har været nødvendigt, at referere til aftalen under projektforløbet.   
	
\section{Samarbejdspartnere}

	Projektoplægget er udarbejdet af Pavia Lumholt (PL), speciallæge i plastikkirurgi på OPA Privathospital Aarhus, i samarbejde med Samuel Alberg Thrysøe (SAT). PL har ageret som kunde i projektet, og har henvendt sig med en idé, som han ønsker at få medicinsk godkendt til klinisk anvendelse. I projektets opstart blev der afholdt samarbejdspartnermøde, hvor der blev idé- og erfaringsudvekslet viden. Inden mødet sørgede projektgruppen for at fremsende en mødeindkaldelse samt at klarlægge roller som hhv. ordstyrer og referent. Der blev lagt stor vægt på at fremstå professionelle idet gruppen repræsenterer uddannelsesinstitutionen.        
	
\section{Projektplanlægning}
	Dette afsnit beskrives de anvendte planlægningsmetoder, som er anvendt i projektet. Hensigten er at belyse, hvordan projektgruppen har anvendt metoderne samt hvilke resultater der var forventet og erfaringerne heraf.  

\subsection{Kommunikation}

	\subsubsection{Mail}
	Med ønsket om fremstå strukturerede og organiserede, oprettede projektgruppen en fælles mail, tilknyttet projektet. Her foregik al korrespondance med samarbejdspartner, vejleder samt implicerede fagfolk. På denne måde kunne mailkorrespondancer holdes adskilt fra private anliggender samt logges et samlet sted. 
	\subsubsection{Ekstern fildeling}
	For at gøre det hurtigt og simpelt at dele viden og udveksle filer, blev der oprettet en fælles fildelingstjeneste på Google Drev, som kunne tilgås af PL samt projektgruppen.  
	Projektgruppen har gjort PL bekendt med, at der forefindes risici ved at benytte en online tjeneste som Google Drev. PL er indforstået med dette og har accepteret brugen. 
	\subsubsection{Mødeindkaldelser og aktionsreferater}
	Som tidligere beskrevet, har det været vigtigt for projektgruppen at fremstå professionelle i det udadvendte arbejde. Således er der opbygget og oprettet en skabelon for mødeindkaldelser, som struktureret belyser informationer vedr. mødet. Her beskrives emne, formål samt hvad mødets resultat skal anvendes til. Yderemere beskrives mødedetaljer som tidspunkt og sted, mødedeltagere, samt hvad der skal forberedes inden mødet, og hvad der evt. skal medbringes. Derudover stilles dagordenen, og en ansvarlig sættes for hvert punkt. Til sidst estimeres mødets varighed. Hensigten med at udsende disse informationer inden mødet, er at der foretages en forventningsafstemning inden mødet og deltagere ved, hvad der skal være forberedt og medbringes. Mødeindkaldelsesskabelonen fremgår af bilag €€ 
	
	Efter et møde, udsendte projektgruppen et aktionsreferat fra det pågældende møde. Også her blev der udarbejdet en struktureret skabelon, som beskrev emne samt formålet med mødet, mødeleder, referent og tidspunkt samt varighed. Ud fra dagsordenen blev der skrevet et resume til hvert punkt, og endvidere blev beslutninger og aktioner sat op, hvor en ansvarlig blev tilknyttet samt en deadline. På denne måde blev det overskueliggjort, hvem der havde hvilke ansvar inden et givent tidspunkt. Dette lettede samarbejdet med implicerede mødedeltagere. Aktionsreferatskabelonen fremgår af bilag €€.     
	
\subsection{Den statiske tidsplan}

	I projektets indledende faser, hvor der blev arbejdet med konceptudvikling, udkast til kravspecifikation samt accepttest, viste Stage Gate-modellen at være en hensigtsmæssig tidsplansmodel. Fordele ved at anvende Stage Gate modellen er opdeling, specificering og eksekvering af de foreliggende opgaver, og giver derfor mulighed for at danne et helhedsbillede af projektets tidsmæssige ramme. Det er siden erfaret at projektets udviklingsfase (herunder design, implementering samt integrationstest) ikke følger en lineær udvikling, og disse faser ikke eksekveres som Stage Gate-modellen foreskriver. Der blev foretaget refleksioner over hvorvidt Stage Gate-modellen blev anvendt forkert eller om projektet havde udviklet sig i en retning, hvor modellen ikke længere være hensigtsmæssig at benytte. Konklusionen er, at Stage Gate-modellen afspejler vandfaldsmodellen, hvilket er uhensigtsmæssigt i projektets udviklingsfase. Det er efter sparring med SAT valgt at gå videre med ASE-modellen, som afspejler en iterativ udviklingsproces. Der blev efterfølgende reflekteret og overvejet over, hvorledes ASE-modellen kunne bruges til at understøtte projektets tidsplan, og det blev konkluderet, at modellen ikke alene kunne understøtte projektets behov for tidsplan. Der blev udført brainstorming på tavlen, hvor ASE-modellen blev tilpasset projektets behov, og modellens iterative proces blev udvidet, så denne omfavnede projektets specifikation af accepttest og integrationstest. Denne brainstorming vises i figur \ref{fig:ASEbrain}.  
	
	\begin{figure}[htb]
		\centering
		\includegraphics[width=4in]{ASEbrain.jpg}
		\caption{Brainstorming på tavlen, hvor ASE-modellen blev tilpasset.}
		\label{fig:ASEbrain}	
	\end{figure}
	
	 Dog afspejler ASE-modellen et projektflow og  giver ikke et tidsmæssigt overblik over projektets faser. Det vigtige overblik er højt prioriteret, og det blev besluttet, \textit{ikke} at tilpasse sig en model, men at modellen måtte tilpasses projektet. Ud fra det daværende kendskab fandtes der ikke en tidplansmodel, som opfyldte de væsentligste behov, og det blev dermed konkluderet, at der måtte udvikles en brugbar model, som tog udgangspunkt i en overskuelig tidsplan og den iterative og agile tilgang. Der blev foretaget en illustrativ inspirationssøgning på hjemmesiden www.google.com, hvor der under “Billeder” blev søgt på stregen \texttt{“scrum+agile+stage+gate”}. Søgning resulterede i et inspirerende diagram af en projektstyringsmetode, som vises i figur \ref{fig:inspidia}.  
	
	\begin{figure}[htb]
		\centering
		\includegraphics[width=5in]{inspirerendediagram}
		\caption{€€€Figurtekst}
		\label{fig:inspidia}	
	\end{figure}

	Diagrammet afspejler The Agile-Stage-Gate model, som er en integration af agile udviklingsmetoder og professor Robert G. Coopers traditionelle Stage-Gate model. Den Agile Stage-Gate model er målrettet til produktion af nye fysiske produkter. Den Agile Stage-Gate model er under udvikling i et samarbejde mellem Cooper og Dansk Industri (DI), Danmarks Tekniske Universitet (DTU) og GEMBA Innovation. I denne udviklingsproces sidder et ekspertpanel bestående af virksomhederne LEGO, Coloplast, Grundfos, Danfoss og IT-virksomheden ForNAV. Evidensen på denne nye udviklingsmetode er begrænset og består hovedsageligt af tidligere evidens, hvor der er eksperimenteret med Stage-Gate og Scrum inden for softwareudvikling samt nyere empirisk evidens fra udviklingsprocesser i førende produktionsvirksomheder. Ved at anvende den Agile Stage-Gate model, opnås et stort potentiale for at sikre en struktureret udviklingsproces, reducering af udviklingstiden samt give et større overblik og en bedre kvalitet. Disse punkter er yderst fordelagtige i udviklingen af et nyt produkt, og det er derfor besluttet at udarbejde en tilpasset Agile Stage-Gate model i dette projektforløb. Den Agile Stage-Gate model dækker både mikro- og makroplanlægning, og det forventes derfor, at modellen vil opfylde behovet for klare milepæle og faste beslutningspunkter samt hastighed og flexibilitet. 
	   
	€€€HUSK BILLEDER AF AGILE-STAGE-GATE 
	
\subsection{Den dynamiske tidsplan} 
I projektets begyndelse anvendte projektgruppen et online projektplanlægningsværktøj, Teamweek, som fungerede som gruppens dynamiske tidsplan og kalender. Denne blev tilpasset, og større opgaver fra Stage-Gate modellen v.0.1 blev lagt ind, og tidsplanen virkede derved som en let udgave af et Gantt-diagram, som gav overblik over tidsmæssige overlap mellem udviklingsfaser. I løbet af udviklings- og testprocessen, hvor behovet for agilitet og dag-til-dag planlægning voksede, blev Stage-Gate modellen videreudviklet til den Agile Stage-Gate model, og behovet for Teamweek forsvandt. Den Agile Stage-Gate gav det overordnede overblik, og projektgruppen fandt det ikke længere nødvendigt med en dynamisk tidsplan. Det blev derfor besluttet at fravælge dette projektplanlægningsværktøj.   

\section{Projektstyring}
	€€PivotalTracker, Planning poker, logbog OG agilt: opslagstavle, tavler, analoge oversigter

\section{Udviklingsværktøjer}
	€€LaTeX+ RefWorks, LabVIEW, Visio, Creately, 

\section{Versionsstyring}
 	€€Dropbox og GitHub

\section{Arbejdsfordeling}

\section{Opnåede erfaringer}