\chapter{Projektadministration}

\section{Samarbejdsaftale}

	\subsection{Konflikthåndtering}

\section{Planlægning}
\section{Møder}
\section{Projektstyring}

	\subsection{Erfaringer vedr. Stage Gate}

	I projektets indledende faser, hvor der blev arbejdet med konceptudvikling, udkast til kravspecifikation samt accepttest, viste Stage Gate-modellen at være en hensigtsmæssig tidsplansmodel. Fordele ved at anvende Stage Gate modellen er opdeling, specificering og eksekvering de foreliggende opgaver, og giver derfor mulighed for at danne et helhedsbillede af projektets tidsmæssige ramme. Det er siden erfaret at projektets udviklingsfase (herunder design, implementering samt integrationstest) ikke følger en lineær udvikling, og disse faser ikke eksekveres som Stage Gate-modellen foreskriver. Det blev foretaget refleksioner over hvorvidt Stage Gate-modellen blev anvendt forkert eller om projektet havde udviklet sig i en retning, hvor modellen ikke længere være hensigtsmæssig at benytte. Konklusionen er, at Stage Gate-modellen afspejler vandfaldsmodellen, hvilket er uhensigtsmæssigt i projektets udviklingsfase. Det er efter sparring med SAT valgt at gå videre med ASE-modellen, som afspejler en iterative udviklings- og testproces i projektet. Der blev efterfølgende reflekteret og overvejet over, hvorledes ASE-modellen kunne bruges til at understøtte projektets tidsplan, og det blev konkluderet, at modellen ikke alene kunne understøtte projektets behov for tidsplan. Der blev udført brainstorming på tavlen hvor ASE-modellen blev tilpasset projektets behov, og modellens iterative proces blev udvidet, så denne omfavnede projektets specifikation af accepttest og integrationstest. ASE-modellen afspejler processen i projektet og giver ikke et tidsmæssigt overblik over projektets faser. Det vigtige overblik var højt prioriteret, og det blev besluttet, ikke at tilpasse sig en model, men at modellen måtte tilpasses projektet. Ud fra det nuværende kendskab fandtes der ikke en tidplansmodel, som opfyldte de væsentligste behov, og det blev dermed konkluderet, at der måtte udvikles en brugbar model, som tog udgangspunkt i en overskuelig tidsplan og den iterative og agile tilgang. Der blev foretaget en illustrativ inspirationssøgning på hjemmesiden www.google.com, hvor der under “Billeder” blev søgt på stregen “scrum+agile+stage+gate”. Søgning resulterede i et inspirerende diagram, som vises i figur €€. Diagrammet afspejler en tidsplansproces med gates sammenkoblet med en agil tilgang, og stammer fra en artikel, som beskriver   Diagrammet illustrerer refleksionerne om en tidsplan med gates, kendt fra Stage-Gate-modellen, sammenkoblet med en iterativ tilgang. Billedet hører til en artikel, som beskriver denne tilgang, og ud fra referencerne i artiklen var det muligt at finde uddybende litteratur dertil. Litteraturen beskriver en Agile-Stage-Gate hybrid model, som integrerer agile udviklingsmetoder med den traditionelle Stage-Gate model. Professor Robert G. Cooper, som er ophavsmanden af Stage-Gate modellen, står bag den fundne evidens.  
 har indgået et samarbejde med DI,DTU og GEMBA Innovation for at videreudvikle denne model. Hvor Lego,Coloplast,Grundfos, Danfoss og it-virksomheden For NAV. 




  foregangsmand for denne nye tilgang til   produktudvikling.Tidsplansmodellen er videreudvi en nyere metode, og det er derfor begrænset med videnskabelig evidens


Klare milepæle og faste beslutningspunkter
hastighed og flexibilitet. 

	
	
\section{Arbejdsfordeling}

\section{Opnåede erfaringer}