\chapter{Projektadministration}

\section{Samarbejdsaftale}

	\subsection{Konflikthåndtering}

\section{Planlægning}
\section{Møder}
\section{Projektstyring}

	\subsection{Erfaringer vedr. Stage Gate}

	Det er i procesforløbet erfaret, at Stage Gate-modellen ikke er et optimal  projektstyringsværktøj for projektforløbets samtlige faser.
	I projektets indledende faser, hvor der blev arbejdet med konceptudvikling, kravspecifikation samt accepttest, viste Stage Gate-modellen at være en hensigtsmæssig tidsplanmodel. Fordele ved at anvende Stage Gate modellen er opdeling, specificering og eksekvering de foreliggende opgaver, og giver derfor mulighed for at danne et helhedsbillede af projektets tidsmæssige ramme. Det er siden erfaret at projektets design- og implementeringsfase ikke følger en lineær udvikling, og disse faser ikke eksekveres som Stage Gate-modellen foreskriver. Det blev foretaget refleksioner over hvorvidt Stage Gate-modellen blev anvendt forkert eller om projektet havde udviklet sig i en retning, hvor modellen ikke længere være hensigtsmæssig at benytte. Konklusionen er, at Stage Gate-modellen afspejler vandfaldsmodel, hvilket er uhensigtsmæssigt i projektets udviklingsfase. Det er valgt at gå videre med ASE-modellen, som afspejler den iterative udviklings- og testproces i projektet. 
	
\section{Arbejdsfordeling}

\section{Opnåede erfaringer}