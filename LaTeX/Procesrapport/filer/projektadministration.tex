\chapter{Projektadministration}

\section{Projektorganisering}
\subsection{Samarbejdsaftale}
I projektets spæde opstart, var det første som blev udformet og klarlagt, en samarbejdsaftale, fungerende som forventningsafstemning i gruppen. Denne aftale blev anvendt som et værktøj til at få diskuteret vigtige parametre for samarbejdet, herunder mødetider, arbejdsform, målsætning, konflikthåndtering mm. \citep{RefWorks:12}. Samarbejdsaftalen fremgår af bilag A. 
Det konkluderes, at samarbejdsaftalen har fungeret efter hensigten og, at emnerne har været klarlagte, idet det ikke har været nødvendigt, at referere til aftalen under projektforløbet.   
	
\subsection{Samarbejdspartnere}

	Projektoplægget er udarbejdet af Pavia Lumholt (PL), speciallæge i plastikkirurgi på OPA Privathospital Aarhus, i samarbejde med Samuel Alberg Thrysøe (SAT). PL har ageret som kunde i projektet, og har henvendt sig med en idé, som han ønsker at få medicinsk godkendt til klinisk anvendelse. I projektets opstart blev der afholdt samarbejdspartnermøde, hvor der blev idé- og erfaringsudvekslet viden. Inden mødet sørgede projektgruppen for at fremsende en mødeindkaldelse samt at klarlægge roller som hhv. ordstyrer og referent. Der blev lagt stor vægt på at fremstå professionelle idet gruppen repræsenterer uddannelsesinstitutionen.        
	
\subsection{Kommunikation}

	\subsubsection{Mail}
	Med ønsket om fremstå strukturerede og organiserede, oprettede projektgruppen en fælles mail, tilknyttet projektet. Her foregik al korrespondance med samarbejdspartner, vejleder samt implicerede fagfolk. På denne måde kunne mailkorrespondancer holdes adskilt fra private anliggender samt logges ét samlet sted. 
	
	\subsubsection{Ekstern fildeling}
	For at gøre det lettilgængeligt at dele viden og udveksle filer, blev der anvendt en fælles fildelingstjeneste på Google Drev, som kunne tilgås af PL samt projektgruppen.  
	Projektgruppen har gjort PL bekendt med, at der forefindes risici ved at benytte en online tjeneste som Google Drev. PL er indforstået med dette, og har accepteret brugen. 
	
	\subsubsection{Mødeindkaldelser og aktionsreferater}
	Som tidligere beskrevet, har det været vigtigt for projektgruppen at fremstå professionelle i projektarbejdet. Således er der opbygget og oprettet en skabelon for mødeindkaldelser, som struktureret belyser informationer vedr. mødet. Her beskrives emne, formål samt hvad mødets resultat skal anvendes til. Yderemere beskrives mødedetaljer som tidspunkt, sted, mødedeltagere samt hvad der skal forberedes inden mødet, og hvad der evt. skal medbringes. Derudover stilles dagordenen, og en ansvarlig sættes for hvert punkt. Til sidst estimeres mødets varighed. Hensigten med at udsende disse informationer inden mødet, er at der foretages en forventningsafstemning inden mødet, og deltagere ved, hvad der skal være forberedt og medbringes. Mødeindkaldelsesskabelonen fremgår af bilag B.  
	
	Efter et endt møde, udsendte projektgruppen et aktionsreferat fra det pågældende møde. Også her blev der udarbejdet en struktureret skabelon, som beskrev emne samt formålet med mødet, mødeleder, referent og tidspunkt samt varighed. Ud fra dagsordenen blev der skrevet et resume til hvert punkt, og endvidere blev beslutninger og aktioner sat op, hvor en ansvarlig samt en deadline blev tilknyttet. På denne måde blev det overskueliggjort, hvem der havde hvilke ansvar inden et givent tidspunkt. Dette lettede samarbejdet med implicerede mødedeltagere. Aktionsreferatskabelonen fremgår af bilag C.     

\section{Projektplanlægning}
	I dette afsnit beskrives de anvendte planlægningsmetoder, som er benyttet i projektet. Hensigten er at belyse, hvordan projektgruppen har anvendt metoderne samt hvilke resultater der var forventet og erfaringerne heraf.
	
\subsection{Den statiske tidsplan}

	I projektets indledende faser, hvor der blev arbejdet med konceptudvikling, udkast til kravspecifikation samt accepttest, viste Stage Gate-modellen at være en hensigtsmæssig tidsplansmodel. Fordele ved at anvende Stage Gate modellen er opdeling, specificering og eksekvering af de foreliggende opgaver, og giver derfor mulighed for at danne et helhedsbillede af projektets tidsmæssige ramme. Det er siden erfaret at projektets udviklingsfase (herunder design, implementering samt integrationstest) ikke følger en lineær udvikling, og disse faser ikke eksekveres som Stage Gate-modellen foreskriver. Der blev foretaget refleksioner over hvorvidt Stage Gate-modellen blev anvendt forkert eller om projektet havde udviklet sig i en retning, hvor modellen ikke længere være hensigtsmæssig at benytte. Konklusionen er, at Stage Gate-modellen afspejler vandfaldsmodellen, hvilket er uhensigtsmæssigt i projektets udviklingsfase. Det er efter sparring med SAT valgt at gå videre med ASE-modellen, som afspejler en iterativ udviklingsproces. Der blev efterfølgende reflekteret og overvejet over, hvorledes ASE-modellen kunne bruges til at understøtte projektets tidsplan, og det blev konkluderet, at modellen ikke alene kunne understøtte projektets behov for tidsplan. Der blev udført brainstorming på tavlen, hvor ASE-modellen blev tilpasset projektets behov, og modellens iterative proces blev udvidet, så denne omfavnede projektets specifikation af accepttest og integrationstest. Denne brainstorming vises i figur \ref{fig:ASEbrain}.  

\newpage
\begin{landscape}
\begin{figure}[htb]
\centering	
\includegraphics[width=9.5in]{stagegate.png}
\caption{Den anvendte Stage-Gate model}
\label{fig:stagegate}
\end{figure}
\end{landscape}

	\begin{figure}[htb]
		\centering
		\includegraphics[width=4in]{ASEbrain.jpg}
		\caption{Brainstorming på tavlen, hvor ASE-modellen blev tilpasset.}
		\label{fig:ASEbrain}	
	\end{figure}
	
	 Dog afspejler ASE-modellen et projektflow og giver ikke et tidsmæssigt overblik over projektets faser. Det vigtige overblik er højt prioriteret, og det blev besluttet, \textit{ikke} at tilpasse sig en model, men at modellen måtte tilpasses projektet. Ud fra det daværende kendskab fandtes der ikke en tidplansmodel, som opfyldte de væsentligste behov, og det blev dermed konkluderet, at der måtte udvikles en brugbar model, som tog udgangspunkt i en overskuelig tidsplan og den iterative og agile arbejdstilgang. Der blev foretaget en illustrativ inspirationssøgning på hjemmesiden \texttt{www.google.com}, hvor der under \textit{Billeder} blev søgt på stregen \texttt{“scrum+agile+stage+gate”}. Søgningen resulterede i et inspirerende diagram af en projektstyringsmetode, som vises i figur \ref{fig:inspidia}.  
	
	\begin{figure}[htb]
		\centering
		\includegraphics[width=5in]{inspirerendediagram}
		\caption{Inspirerende billede fundet under en inspirationssøgning på \texttt{www.google.com} under \textit{Billeder}}
		\label{fig:inspidia}	
	\end{figure}

	Diagrammet afspejler The Agile-Stage-Gate model, som er en integration af agile udviklingsmetoder og professor Robert G. Coopers traditionelle Stage-Gate model. Der er efterfølgende søgt litteratur omkring den Agile Stage-Gate model, hvor der er fundet, at den Agile Stage-Gate model er målrettet produktion af nye fysiske produkter. Den Agile Stage-Gate model er under udvikling i et samarbejde mellem Cooper og Dansk Industri (DI), Danmarks Tekniske Universitet (DTU) og GEMBA Innovation. I denne udviklingsproces sidder et ekspertpanel bestående af virksomhederne LEGO, Coloplast, Grundfos, Danfoss og IT-virksomheden ForNAV. Evidensen på denne nye udviklingsmetode er begrænset og består hovedsageligt af tidligere evidens, hvor der er eksperimenteret med Stage-Gate og Scrum inden for softwareudvikling samt nyere empirisk evidens fra udviklingsprocesser i førende produktionsvirksomheder. Ved at anvende den Agile Stage-Gate model, opnås et stort potentiale for at sikre en struktureret udviklingsproces, reducering af udviklingstiden samt at give et større overblik og en bedre kvalitet \citep{{RefWorks:8},{RefWorks:9}}. Disse punkter er yderst fordelagtige i udviklingen af et nyt produkt, og det er derfor besluttet at udarbejde en tilpasset Agile Stage-Gate model i dette projektforløb. Den Agile Stage-Gate model dækker både mikro- og makroplanlægning, og det forventes derfor, at modellen vil opfylde behovet for klare milepæle og faste beslutningspunkter samt hastighed og flexibilitet. 
	   
\newpage
\begin{landscape}
\begin{figure}[ht]
\centering
\includegraphics[width=10in]{agilstagegate}
\caption{Den anvendte Agile Stage-Gate model}
\label{fig:agilstagegate}
\end{figure}
\end{landscape}
	
\subsection{Den dynamiske tidsplan} 
I projektets begyndelse anvendte projektgruppen et online projektplanlægningsværktøj, Teamweek, som fungerede som gruppens dynamiske tidsplan og interne kalender. Teamweek blev tilpasset, og større opgaver fra Stage-Gate modellen v.0.1 blev lagt ind. Den dynamiske tidsplanen virkede derved som en let udgave af et Gantt-diagram, som gav overblik over tidsmæssige overlap mellem udviklingsfaser. Figur \ref{fig:teamweek} viser et billede af den dynamiske tidsplans opbygning. I løbet af udviklings- og testprocessen, hvor behovet for agilitet og dag-til-dag planlægning voksede, blev Stage-Gate modellen videreudviklet til den Agile Stage-Gate model, og behovet for Teamweek forsvandt. Den Agile Stage-Gate gav det overordnede overblik, og projektgruppen fandt det ikke længere nødvendigt med en dynamisk tidsplan. Det blev derfor besluttet at fravælge dette projektplanlægningsværktøj.  

\begin{figure}[htb]
\centering
\includegraphics[width=5in]{teamweek.png}	
\caption{Den dynamiske tidsplan med overlap, øjebliksbillede fra d. 29.09.16,}
\label{fig:teamweek}
\end{figure}

\section{Projektstyring}
	€€PivotalTracker, Planning poker, logbog OG agilt: opslagstavle, tavler, analoge oversigter
	
	Dette projekt omhandler, hvorledes man med en systematisk og struktureret tilgang, kan udvikle og teste sig frem mod en ny løsning til en given problemstilling. Brystvolumenmåleren er en ny løsning inden for dets anvendelsesområde, og det har derfor været yderst fordelagtigt at opbygge udviklings- samt testprocessen med en agil arbejdstilgang. Den agile tilgang har gjort det muligt at søge viden, opstille en testhypotese og derefter hurtigt at afprøve den(€€ref til vores diagram?). Dette afsnit beskriver anvendelsen af projektstyringsværktøjer i arbejdsprocessen. 
	

	\subsection{Scrum}
	\label{subsec:scrum}
	Der er i projektet anvendt elementer fra Scrum. Hver morgen afholdes \textit{Daily Scrum Meetings}, således gruppemedlemmer er opdateret på, hvad der er lavet siden sidst, hvad planen er for den pågældende dag samt eventuelle hindringer. Med henblik på at strukturere og overskueliggøre den dynamiske arbejdsproces, beskrevet i afsnit \ref{subsec:statisk}, er der i projektet anvendt den kendte iterative arbejdsmetode fra Scrum, hvor der løbende bliver prioriteret mellem opgaver. Herefter revuderes og planlægges delopgaver, og disse styres ud fra 7-dages-sprints. Dette gør, at produktet og resultater evalueres og testes løbende. I det efterfølgende afsnit, afsnit \ref{subsec:pivotal}, uddybes det, hvorledes denne styringsproces er anvendt.  
	
	\subsection{Pivotal Tracker}
	\label{subsec:pivotal}
	Pivotal Tracker er et webbaseret projektstyringsværktøj, som muliggør denne agile arbejdstilgang. I Pivotal Trackers icebox, er samtlige arbejdsopgaver defineret. Dette giver et overblik over foreliggende opgaver, og giver samtidig en ro over, at intet forglemmes. Arbejdsopgaverne defineres med en kort beskrivelse og tildeles points. Pointtildelingen sker ved brug af \textit{Planning poker}, som fremgår i figur \ref{fig:planningpoker}, hvorved der opnås enighed om opgavens arbejdsbyrde samt omfang. Denne arbejdsmetode skaber stor gennemsigtighed i arbejdsprocessen, og samtidig et fælles overblik over indholdet i opgaverne. 
	
	\begin{figure}[htb]
	\centering
	\includegraphics[width=2in]{Planningpoker}
	\caption{Anvendelse af Planning poker ved tildeling af points til arbejdsopgaver}
	\label{fig:planningpoker}	
	\end{figure}
	 
	Definererede arbejdsopgaverne ligger herefter med en kort beskrivelse samt pointestimat for omfanget i projektets icebox, klar til at blive flyttet over i backloggen. Backologgen indeholde de opgaver, som prioriteres, og Pivotal Tracker tilføjer automatisk opgaver til det igangværende sprint indtil \textit{Velocity}-grænsen opnås. Velocity er gennemsnittet af points, som gennemføres i løbet af et sprint. En opgaves status defineres ud fra en række forskellige states, herunder \textit{unstarted}, \textit{started}, \textit{finished}, \textit{delivered}, \textit{rejected} og \textit{accepted}. Denne arbejdsproces gør det dermed muligt, at en færdiggjort opgave kan afleveres til review hos det andet gruppemedlem, som derefter afviser eller godkender opgaven. Samtidig medvirker denne arbejdsproces til, at projektmedlemmer er inde over alt indhold gennem projektprocessen.     
	
	Ved brug af \textit{Burnup chart'et} i Pivotal Tracker, kan der dannes et overblik over projektets fremgang, hvor der stræbes efter en lineær fremgang, således man undgår en tung arbejdsbyrde mod projektets slutning. Processen sammenholdes med tidsplanen, og ved en eksponentiel fremgang i Burnup chart'et, må en revidering af tidsplanen overvejes, for at opnå en realistisk arbejdsbyrde mod projektet udgang.  
	
	Pivotal Tracker har også den fordel, at den indeholder en komplet historik over de afsluttede sprints med dertilhørende opgaver. I denne log fremgår det, hvilke opgaver, der er udført i hvilken uge, og på den vis kan loggen benyttes som en opgave-logbog. Dog er der i projektet prioriteret at anvende en traditionel logbog, da overvejelser, refleksioner og erfaringer vægtes meget højt i arbejdsprocessen.      		
	
	\subsection{Logbog}
	Logbogen er anvendt som et højt prioriteret værktøj i arbejdsprocessen, da projektets store omdrejningspunkt er udviklings- samt testproces. Logbogen er benyttet til at dokumentere refleksioner, overvejelser og beslutninger, som er gjort under projektarbejdet. Hver morgen er startet med, at logbogen er blevet åbnet, og i forlængelse af Daily Scrum meeting, er dagordenen blevet  fastlagt. Logbogens opbygning, som fremgår af bilag D, lægger op til en reflekterende og evaluerende granskning af procesforløbet. Således er procesforløbet løbende blevet evalueret og revideret i forhold til passende arbejdsmetoder. Projektgruppen har fundet denne arbejdsmetode tung, men yderst fordelagtig, da ofte vigtige refleksioner og overvejelser hurtigt kan blive forglemt. 

\section{Udviklingsværktøjer}
	€€LaTeX+ RefWorks, LabVIEW, Visio, Creately, 

	\subsection{\LaTeX}
Det blev i projektets indledende uger, prioriteret at bruge tid på at lære at anvende tekstformateringsprogrammet \LaTeX. Fordelene ved at anvende LaTeX, er at der kan fokuseres på at skabe det tekstuelle indhold, da der under skrivningen kun angives strukturelle og logiske kommandoer, som LaTeX derved bruger til at lave indholdfortegnelse, afsnitsinddeling, krydsreferencer, bibliografi mm. Den stilmæssige udformning af layoutet defineres i en særskilt fil, og på denne måde opnås en ensartet typografisk kvalitet, som er klar til udprintning.   
	\subsection{RefWorks}
Det online referenceværktøj RefWorks, er benyttet til at holde styr på kilder fra anvendt litteratur. Projektgruppen har oprettet en fælles account til RefWorks, så alle referencer er samlet i én online database, og på denne måde kan tilgås fra enhver computer. Referencerne i RefWorks-databasen eksporteres til bibliografien i LaTeX, som danner en litteraturliste. På denne måde har det i rapportskrivningen været problemfrit at referere til anvendt litteratur.  
	
	



\section{Versionsstyring}
 	€€Dropbox og GitHub

\section{Arbejdsfordeling}

\section{Opnåede erfaringer}