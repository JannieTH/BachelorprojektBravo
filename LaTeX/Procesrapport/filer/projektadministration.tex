\chapter{Projektadministration}

\section{Samarbejdsaftale}

	\subsection{Konflikthåndtering}

\section{Planlægning}
\section{Møder}
\section{Projektstyring}

	\subsection{Erfaringer vedr. Stage Gate}

	I projektets indledende faser, hvor der blev arbejdet med konceptudvikling, udkast til kravspecifikation samt accepttest, viste Stage Gate-modellen at være en hensigtsmæssig tidsplansmodel. Fordele ved at anvende Stage Gate modellen er opdeling, specificering og eksekvering af de foreliggende opgaver, og giver derfor mulighed for at danne et helhedsbillede af projektets tidsmæssige ramme. Det er siden erfaret at projektets udviklingsfase (herunder design, implementering samt integrationstest) ikke følger en lineær udvikling, og disse faser ikke eksekveres som Stage Gate-modellen foreskriver. Der blev foretaget refleksioner over hvorvidt Stage Gate-modellen blev anvendt forkert eller om projektet havde udviklet sig i en retning, hvor modellen ikke længere være hensigtsmæssig at benytte. Konklusionen er, at Stage Gate-modellen afspejler vandfaldsmodellen, hvilket er uhensigtsmæssigt i projektets udviklingsfase. Det er efter sparring med SAT valgt at gå videre med ASE-modellen, som afspejler en iterativ udviklingsproces. Der blev efterfølgende reflekteret og overvejet over, hvorledes ASE-modellen kunne bruges til at understøtte projektets tidsplan, og det blev konkluderet, at modellen ikke alene kunne understøtte projektets behov for tidsplan. Der blev udført brainstorming på tavlen, hvor ASE-modellen blev tilpasset projektets behov, og modellens iterative proces blev udvidet, så denne omfavnede projektets specifikation af accepttest og integrationstest. Denne brainstorming vises i figur \ref{fig:ASEbrain}.  
	
	\begin{figure}[htb]
		\centering
		\includegraphics[width=4in]{ASEbrain.jpg}
		\caption{Brainstorming på tavlen, hvor ASE-modellen blev tilpasset.}
		\label{fig:ASEbrain}	
	\end{figure}
	
	 Dog afspejler ASE-modellen et projektflow og  giver ikke et tidsmæssigt overblik over projektets faser. Det vigtige overblik er højt prioriteret, og det blev besluttet, \textit{ikke} at tilpasse sig en model, men at modellen måtte tilpasses projektet. Ud fra det daværende kendskab fandtes der ikke en tidplansmodel, som opfyldte de væsentligste behov, og det blev dermed konkluderet, at der måtte udvikles en brugbar model, som tog udgangspunkt i en overskuelig tidsplan og den iterative og agile tilgang. Der blev foretaget en illustrativ inspirationssøgning på hjemmesiden www.google.com, hvor der under “Billeder” blev søgt på stregen \texttt{“scrum+agile+stage+gate”}. Søgning resulterede i et inspirerende diagram af en projektstyringsmetode, som vises i figur \ref{fig:inspidia}.  
	
	\begin{figure}[htb]
		\centering
		\includegraphics[width=5in]{inspirerendediagram}
		\caption{€€€Figurtekst}
		\label{fig:inspidia}	
	\end{figure}

	Diagrammet afspejler The Agile-Stage-Gate model, som er en integration af agile udviklingsmetoder og professor Robert G. Coopers traditionelle Stage-Gate model. Den Agile Stage-Gate model er under udvikling i et samarbejde mellem Cooper og Dansk Industri (DI), Danmarks Tekniske Universitet (DTU) og GEMBA Innovation. I denne udviklingsproces sidder et ekspertpanel bestående af virksomhederne LEGO, Coloplast, Grundfos, Danfoss og IT-virksomheden ForNAV. Evidensen på denne nye udviklingsmetode er begrænset og består hovedsageligt af tidligere evidens, hvor der er eksperimenteret med Stage-Gate og Scrum inden for softwareudvikling samt nyere empirisk evidens fra udviklingsprocesser i førende produktionsvirksomheder.
	
	 "\textit{If the recent evidence can be trusted, this new approach promises to be the most significant change to our thinking about how new-product development should be done since the introduction of today's popular gating systems 30 years ago}" €€KILDE. 
	 En af fordelene ved Agil Stage-Gate modellen er, at man bliver bedre og langt hurtigere til at vise produkter til relevante brugere. Samtidig opnåes værdifuldt feedback, der kan indarbejdes i den videre udvikling. Endvidere forventes det, at den Agile Stage-Gate model vil kunne opnå over 20 pct. reducering af udviklingstiden samt give et større overblik og en bedre kvalitet. Disse punkter er yderst fordelagtige i udviklingen af et nyt produkt, og det er derfor besluttet at udarbejde en tilpasset Agile Stage-Gate model i dette projektforløb. Da den Agile Stage-Gate model dækker både mikro- og makroplanlægning, forventes det, at modellen vil opfylde behovet for klare milepæle og faste beslutningspunkter samt hastighed og flexibilitet. 
	 
	 
	
	
\section{Arbejdsfordeling}

\section{Opnåede erfaringer}