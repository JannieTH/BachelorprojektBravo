\chapter{Testprocessen}

\section{Testforløbets fem faser}

\section{Mikrofonproblemet}

Under udførslen af bordtest nr. 5, blev det observeret, at resultatet i VI'en \texttt{optagefrekvenssignal0.2.vi} blev opfanget af PC'ens indbyggede mikrofon og ikke Minijack PC Mikrofonen.
Der opstod en mistanke om problemet da resultaterne var ens uanset mikronfonens placering indeni samt uden for resonatoren. Der blev derefter testet ved at udtage Minijack PC Mikrofonen fra PC'en, hvorefter resultaterne stadig var ens. Dette medvirkede til en ny enhedstest, hvor mikrofonen blev placeret i et andet rum med en lukket branddør imellem. Da der ikke blev opfanget et signal i LabVIEW fra mikrofonen blev det konstateret, at mikrofonen ikke var aktiv. Årsagen til problemstillingen skyldes, at mikrofonen har et 3-pols stik, og mangler derfor en pol til lyd input. PC'en indlæser derfor mikrofonen som en højtaler, og forsøger dermed at udsende lyd gennem mikrofonen.
Løsningen på denne problemstilling er at anvende en mikrofon med 4-pols stik, en adaptaer eller en mikrofon med USB-stik.
Det blev forsøgt at optage lyd med et headset med indbygget mikrofon som havde et 4-pols jackstik. Headsettet blev indlæst på PC´en som et headset og derfor var det ikke muligt at vælge headsetmikrofonen som lydkilde under controlpanel → sound. Det var heller ikke muligt at få forbindelse til mikrofonen igennem LabView.  Konklusionen på denne problemstilling er at der må siddet et 3 pols jack hun stik i PC´en. Der er foretaget en internetsøgning på indholdet af stik i en Macbook Pro 2009 model for at understøttet denne konklusion. Det lykkedes ikke at finde specifikationer som klart udspecificerer hvilket hun jackstik som er indbygget i omhandlende PC. Det vælges at gå videre til test med webkameraet da dette kan være en hurtig løsning af problemstillingen. 
USB kameraet med indbygget mikrofon blev dernæst testet. USB kameraet blev koblet til computeren og under >>sound<< modulet i kontrolpanelet, var det nu muligt at vælge mikrofonen på webkameraet som lydkilde. De andre indbyggede lydkilder fravalgte(Måske et billede af kontrolpanlet) og det blev nu forsøgt at optage en lyd i labview. (enhedstest €€€). Den blev konkluderet at det er muligt at anvende en mikrofon med USB stik til indlæsning af lydsignaler i LabView. Med denne nye viden til rådighed blev det undersøgt om det muligt at
anvende en adaptor hvor vi kan tilslutte vores nuværende mikrofon(€€€) og få en USB-udgang. Det lykkedes at finde denne model(€€€) som har den ønskede funktion. 
Det blev overvejet grundigt om adaptoren skulle anvendes eller om det var tid til at få tilsluttet Sparkfun mikrofonen til arduino´en og få den op at køre med LabView. Der er leveringstid på adaptoren og det vurderes at det ikke vil være så besværligt at få gang i sparkfun mikrofonen. Det kan dog ende med at arduinoen slet ikke kan bruges hvis problemet med de harmoniske overtoner ikke bliver løst med anvendelse af resonatoren. Det vælges alligevel at gå i kast med sparkfun mikrofonen da den er indkøbt og arbejdet kan gå i gang med det samme. 
Grundet denne nye viden, udføres enhedstest samt samtlige bordtest igen således resultaterne anses for at være valide.
