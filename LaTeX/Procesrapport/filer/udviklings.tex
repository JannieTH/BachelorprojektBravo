\chapter{Udviklingsforløb}

\section{Litteratursøgning}

Der er praktiseret en omhyggelig, systematisk tilgang til litteratursøgningen, så resultatet af projektarbejdet bliver repræsentativt og uden bias. Søgeresultaterne danner grobund for den empiri, som projektarbejdet bygger videre på, og det er derfor vigtigt, at det er solidt nok til at bære analyser og konklusioner. Der er med andre ord, søgt, analyseret og vurderet ny viden som er relevant inden for projektets fagområde og rammer. For at overskueliggøre litteratursøgningen, er denne inddelt i en søgeproces bestående af fire faser, som vises i figur \ref{fig:littpro}, og som endvidere beskrives i nedenstående afsnit.  \\

	\begin{figure}[htb]
			\centering
				\includegraphics[width=5in]{Littproces}
				\caption{Litteratursøgningsprocessens fire faser}	
				\label{fig:littpro}
	\end{figure}
	
	\subsection{Søgestrategi}
	Søgestrategien, beskrevet i søgeprotokollen bilag €€, er udarbejdet med tanke på, at fremsøge det mest relevante information, ud fra gigantiske datamængder. Endvidere er søgestrategien udarbejdet med henblik på at gøre søgningen reproducerbar for at sikre troværdighed.  
	En søgestrategi der kombinerer ord i artiklernes titel og resumé med udvalgte emneord blev anvendt til at finde og screene artikler vedrørende den specifikke problemstilling. 
	
	 €€Her defineres forskningsspørgsmålet, søgeord og søgestrenge, og nævner de databaser man vil søge i. Der skal også være en beskrivelse af øvrige søgemetoder, man anvender, som fx håndsøgninger, citation tracking, konferencepapirer, Scholar Google, hjemmesider, personlige kontakter til eksperter, fagbøger osv. 
		Nævn også: booleske operatorer, synonymer 
	
	\subsection{Dataindsamling}
	I screeningen beskrevet ovenfor, blev artikler udvalgt, hvis ordene i artiklens titel og resumé matchede de udvalgte emneord. Derefter blev de udvalgte artikler selvstændigt bedømt af JH og JR, og artikeludvælgelsen blev foretaget sammenholdt med følgende inklusionskriterier:
	
		\begin{enumerate}
			\item Intervention
			\item Målgruppe
			\item Succeskriterier 
			Valgte emneord med dertilhørende problemstillinger synes besvaret
			\item €€Noget GRADE-agtigt 
			\item €€Noget med studie-typen
			\item Etc. 
		\end{enumerate} 
		
	Alle artikler, som opfyldte ovenstående inklusionskriterier gik videre til dataudvælgelse. 	
		
	\subsection{Dataudvælgelse}	
	Information blev ekstraheret selvstændigt af JH og JR ved at bruge de foruddefinerede kriterier.  
	
	€€€Eksempler på, hvad der er udvalgt (tabeller, citater, diagrammer etc.
	
	
	€€Beskriv: Hvad gjorde vi ved manglende fund af litt.? 
	
	\subsection{Dataanalyse} 
 

\section{Design}
\section{Testproces}	

			\begin{figure}[htb]
			\centering
				\includegraphics[width=4in]{AgileTesting}
				\caption{Diagrammet viser €€€}	
				\label{fig:agiletesting}
			\end{figure}

\subsection{Beskrivelse af testforløbet}


Projektet er et udviklingsprojekt, hvor der systematisk testes frem til en produktløsning. Projektet er derfor præget af et omfattende testforløb, hvor der med en systematisk tilgang er lagt vægt på reproducerbarhed samt sporbarhed. 


Det indledende testforløb blev udarbejdet med inspiration fra “Projekteringshåndbogen”, skrevet af Søren Lyngsø-Petersen, som beskriver test af produktionsudstyr til Health Care branchen. Lyngsø-Petersen beskriver, hvorledes et testforløb overordnet kan inddeles i fem faser;  

	\begin{enumerate}
		\item Indledende funktionstest
		\item Pre-FAT
		\item SAT
		\item Kvalificeringstest
	\end{enumerate}

€€€[Beskrivelse af hvad faserne er?]


Det blev overvejet at følge disse faser, da det var et ønske at arbejde med et testforløb fra Health Care industrien. Et muligt testforløb blev skitseret, [Billede af tavle] og efter en grundig gennemgang med mange spændende refleksioner over, hvorledes projektets testforløb kunne tilpasses det beskrevede testforløb fra “Projekteringshåndbogen”, blev det konkluderet, at bogens testforløb er for rettet mod produktionen af det medicinske udstyr i forhold til projektet, og derfor var uhensigtsmæssigt at følge. Dog blev det besluttet at anvende enkelte tilpassede metoder i den videre udarbejdelse af projektets testforløb. Således blev testforløb-modellen tilpasset projektets testforløb, som endte ud med at være et testforløb inddelt i følgende fem faser:    

	\begin{enumerate}
		\item Enhedstest
		\item Integrationstest
		\item Accepttest
		\item Lab PoC
		\item Kvalificeringstest 
	\end{enumerate}

I første fase, enhedstest, testes de indgående komponenter, for at sikre disses funktion. I anden fase, integrationstest, er hovedformålet at foretage verificeringer af de forskellige funktionaliteter og processer i elementer, som anvendes i det videre testforløb. I tredje fase, accepttest, eftervises alle specificerede krav fra Kravspecifikationen. I fjerde fase, Lab PoC, testes diverse forhold og hypoteser, og der udvikles mod en ny og bedre version af produktet. I femte fase, kvalificeringstest, foretages en systemvalidering til/med kunden. Alle fem faser er struktureret ud fra???


Testforløbet afspejler “God testpraksis”, som er et begreb anvendt inden for Health Care Industrien (KILDE). “God testpraksis” beskriver en testprocedure, hvor man udfører og dokumenterer sine tests på en måde, som gør dem valide, hvilket afspejles i den måde, hvorpå dokumentationen af de foretagede tests i projektet er opbygget. Projektets testudførelse består derfor af følgende tre forhold;

	\begin{enumerate}
	\item beskrivelse af, hvordan testen skal udføres (testprocedure) 
	\item Selve udførelsen af testen 
	\item Dokumentation af testresultatet
	\end{enumerate}

Testproceduren beskriver den praktiske udførelse af testen, således den er reproducerbar og alle/ fagfolk vil have mulighed for at udføre testen, uden at have nogen specifik baggrundsviden. Dokumentationen af hver enkelte test starter med en kort beskrivelse af, hvad formålet med er test er. Derefter specificeres det anvendte udstyr og komponenter, og testopstillingen samt -opsætningen beskrives meget udførligt for at sikre en korrekt udførelse af testen. Selve udførelsen af testen beskrivelses med en høj detaljegrad og kan muligvis forekomme nedladende, men da resultatet kan afhænge af, hvordan testen udføres er dette et nødvendigt forhold. Slutvis fremvises testresultaterne, og disse diskuteres efterfølgende for at sikre en refleksion over de opnåede resultater. Testen afrundes med en konklusion af resultatet i sammenhold med formålet, og der planlægges en aktion for næste skridt.   

\subsection{Beskrivelse af road blocks}

€€Her skal beskrives og dokumenteres for at vi ikke har siddet fast undervejs selvom vi er stødt på road blocks. 
Nævn, at vi har sat tidsbegrænsninger på os selv i forbindelse med LabVIEW, og vi har prioriteret arbejdet, vi har kørt med en plan B (DAQ), søgt hjælp ved fagfolk samt gået igang med andre opgaver. --> endda stoppet sprint, tidsplan mm. 

\begin{figure}[htb]
			\centering
				\includegraphics[width=4in]{flowdiagramelyd}
				\caption{Diagrammet viser €€€}	
				\label{fig:flowdiagramelyd}
			\end{figure}

\begin{figure}[htb]
			\centering
				\includegraphics[width=4in]{flowdiagrameop}
				\caption{Diagrammet viser €€€}	
				\label{fig:flowdiagrameop}
			\end{figure}

\subsubsection{Mikrofonproblemet}

Under udførslen af bordtest nr. 5, blev det observeret, at resultatet i VI'en \texttt{optagefrekvenssignal0.2.vi} blev opfanget af PC'ens indbyggede mikrofon og ikke Minijack PC Mikrofonen.
Der opstod en mistanke om problemet da resultaterne var ens uanset mikronfonens placering indeni samt uden for resonatoren. Der blev derefter testet ved at udtage Minijack PC Mikrofonen fra PC'en, hvorefter resultaterne stadig var ens. Dette medvirkede til en ny enhedstest, hvor mikrofonen blev placeret i et andet rum med en lukket branddør imellem. Da der ikke blev opfanget et signal i LabVIEW fra mikrofonen blev det konstateret, at mikrofonen ikke var aktiv. Årsagen til problemstillingen skyldes, at mikrofonen har et 3-pols stik, og mangler derfor en pol til lyd input. PC'en indlæser derfor mikrofonen som en højtaler, og forsøger dermed at udsende lyd gennem mikrofonen.
Løsningen på denne problemstilling er at anvende en mikrofon med 4-pols stik, en adaptaer eller en mikrofon med USB-stik.
Det blev forsøgt at optage lyd med et headset med indbygget mikrofon som havde et 4-pols jackstik. Headsettet blev indlæst på PC´en som et headset og derfor var det ikke muligt at vælge headsetmikrofonen som lydkilde under controlpanel → sound. Det var heller ikke muligt at få forbindelse til mikrofonen igennem LabView.  Konklusionen på denne problemstilling er at der må siddet et 3 pols jack hun stik i PC´en. Der er foretaget en internetsøgning på indholdet af stik i en Macbook Pro 2009 model for at understøttet denne konklusion. Det lykkedes ikke at finde specifikationer som klart udspecificerer hvilket hun jackstik som er indbygget i omhandlende PC. Det vælges at gå videre til test med webkameraet da dette kan være en hurtig løsning af problemstillingen. 
USB kameraet med indbygget mikrofon blev dernæst testet. USB kameraet blev koblet til computeren og under >>sound<< modulet i kontrolpanelet, var det nu muligt at vælge mikrofonen på webkameraet som lydkilde. De andre indbyggede lydkilder fravalgte(Måske et billede af kontrolpanlet) og det blev nu forsøgt at optage en lyd i labview. (enhedstest €€€). Den blev konkluderet at det er muligt at anvende en mikrofon med USB stik til indlæsning af lydsignaler i LabView. Med denne nye viden til rådighed blev det undersøgt om det muligt at
anvende en adaptor hvor vi kan tilslutte vores nuværende mikrofon(€€€) og få en USB-udgang. Det lykkedes at finde denne model(€€€) som har den ønskede funktion. 
Det blev overvejet grundigt om adaptoren skulle anvendes eller om det var tid til at få tilsluttet Sparkfun mikrofonen til arduino´en og få den op at køre med LabView. Der er leveringstid på adaptoren og det vurderes at det ikke vil være så besværligt at få gang i sparkfun mikrofonen. Det kan dog ende med at arduinoen slet ikke kan bruges hvis problemet med de harmoniske overtoner ikke bliver løst med anvendelse af resonatoren. Det vælges alligevel at gå i kast med sparkfun mikrofonen da den er indkøbt og arbejdet kan gå i gang med det samme. 
Grundet denne nye viden, udføres enhedstest samt samtlige bordtest igen således resultaterne anses for at være valide.

\subsubsection{Fra Arduino til DAQ}

I forbindelse med det indledende design af testopstilligen, blev Arduino Mega 2560 valgt frem for DAQ'en grundet økonomiske omstændigheder. Dog blev det hurtigt konstateret, at Arduinoen forårsagede uhensigtsmæssige udfordringer i forbindelse med frekvenssignalet, idet det kun er muligt at generere et firkantsignal igennem LINX MakerHub i LabVIEW. Det ønskede resultat var grundtonens frekvens, men i stedet blev firkantssignalets harmoniske overtoner opfanget som maksimum frekvens i  FFT'en. €€Ref til test€€ Efter samtale med lektor Tore Arne Skogberg, blev det påpeget, at det burde være muligt at måle brystvolumen med firkantsignaler. De harmoniske overtoner kunne muligvis dæmpes ved hjælp af resonatoren da denne har funktion som et lavpasfilter, og dette skulle hermed testes som en løsning til denne problemstilling. Således blev det besluttet at arbejde videre med Arduino'en da de økonomiske fordele stadig talte for. Der opstod endvidere en udfordring forårsaget af Arduino'en begrænsede ydeevne. Arduino´ens højeste Loop Rate ligger på omkring 122Hz. I og med at den valgte højtaler €€REF€€ er egnet til et frekvensområde fra 100 Hz op til 2 kHz vil der ikke kunne genereres et frekvenssignal, som er muligt at opfange. For at få en korrekt digital repræsentation af det analoge signal, der samples med det dobblete af indgangssignalet. Således vil det indsendte signal være begrænset til maksimalt 50 Hz, for at undgå aliasering. €€Nyquist frekvens€€. Til sidst blev det konkluderet, at udfordringerne ikke blev opvejet af de økonomiske fordele, og det blev derfor besluttet at udskifte Arduino'en til en DAQ da der kan samples med højere frekvens, og samtidig anvendes sinussignal for at undgå udfordringer med harmoniske overtoner.      

\subsubsection{DAQ}

Udviklingen af et VI til at generer en lyd igennem daq´en til højtaleren var mere udfordrende en først antaget. Det blev fundet flere guides på NI.com til at løse problemstillingen og koden dertil fandtes meget simpel. Disse guides blev brugt til at bygge \texttt{generefrekvenssignal0.4} VI´et men der opstod fejl i forbindelse med opsætning at DAQ Assistant-modulet. 

\begin{figure}[htb]
			\centering
				\includegraphics[width=4in]{FejlDAQAssistant}
				\caption{Diagrammet viser €€€}	
				\label{fig:FejlDAQAssistant}
			\end{figure}
			
			
Dette blev løst ved at vælge generation mode til 1 sample(on Demand) og signal output range min til 0. Løsningen blev fundet ved brug af trail and error metoden sammen med vejleder. Der opstod herefter en ny fejl med kørslen af VI´et. Under fejlsøgning af denne problemstilling blev det opdaget at daq´en har en maksimal samplingsrate på 150Hz. Fejlsøgning blev afbrudt da denne nye viden sat udviklingen af VI´et i et nyt perspektiv. Med brug af dag´en vil der kunne generes  et maksimalt signal på 75Hz, når Nyquist-teorien er taget i betragtning og det er ønsket at kunne genere et bredt frekvensbånd fra 100Hz op til 1000Hz. Højtalerens begrænsede frekvensspektre fra 100Hz til 2kHz  gør det ikke muligt at anvende daq´en sammen med højtaleren. 
På denne begrund blev det besluttet ikke at anvende daq´en og højtaleren i det videre projektforløb. 	
	
\section{Lovgivning}

Ønskes medicinsk udstyr markedsført på det europæsiske marked skal det CE-mærkes. Hertil stilles der omfattende krav til produkterne styret af \textit{the European Medical Devices Directive 93/42/EEC}{}(MDD 93/42/EEC). I udgangen af år 2016 eller begyndelsen af 2017 bliver MDD erstattet af \textit{the European Medical Device Regulation}{}(MDR), hvis regulativer skal være implementeret inden udgangen af 2019. 


Dette kapitel beskriver vejen til CE-mærket for brystvolumenmåleren. CE-mærkningen starter ved at definere produktet som medicinsk udstyr og derefter at klassificere produktet ud fra \textit{the European Commission's official guidance for Medical Devices - MEDDEV 2.4/1 Rev.9}. Ud fra klassifikationen tydeliggøres det i MEDDEV, hvilke bilag fra MDD, som skal opfyldes for at opnå CE-mærkning.
Udover klassificeringen, findes gennerelle krav, som ethvert medicinsk udstyr skal opfylde. Disse væsentlige krav indeholder bl.a. en risikoanalyse, som vil blive udarbejdet for brystvolumenmåleren. 
Når dokumentationen for overensstemmelse med gældende krav er udarbejdet, kan CE-mærkning opnåes.


\subsection{Definition af BVM som medicinsk udstyr}

BMV kategoriseres som et medicinsk udstyr ud fra definitionen af medicinsk udstyr i MDD 93/42/ EEC, artikel 1.2 (a) og B(e). BMV's anvendelsesformål er bestemmelse af volumen af et bryst med henblik på modificering af anatomien på en patient. Dette anvendelsesformål definerer derved brystvolumenmåleren som værende et medicinsk udstyr. 

\subsection{Klassificeringen af brystvolumenmåleren}

Ud fra MEDDEV guidelines klassificeres BMV'en som et klasse I produkt, ud fra regel 1.  
\textit{"Devices that either do not touch the patient or contact intact skin only."} 
Ydermere har produktet en målefunktion og derved skærpes kravene til CE-godkendelsen. Klassificeringen bliver derved en klasse Im.

\subsection{Vejen til CE-mærkning}

For klasse Im-udstyr, og dermed BVM, er der flere veje til at opnå CE-mærkning. I figur \ref{fig:Klas} fremgår det, hvilke bilag i MDD 93/42/EEC, som beskriver kravene til opfyldelse af overensstemmelseserklæring. Der gøres opmærksom på, at bilag VII \textit{skal} opfyldes sammen med \textit{enten} bilag II (pånær sektion 4), bilag IV, bilag V eller bilag VI. Bilag VII er en EF overensstemmelseserklæring, som blandt andet indeholder al den tekniske dokumentation. I bilag II, IV, V og VI stilles der krav til kvalitetssikringssystemer, hvor forskellen er omfanget af kravene til kvalitetssystemerne. Bilag II beskriver kravene til en fuld kvalitetssikring, hvor bilag IV, bilag V og bilag VI beskriver kvalitetssikringskrav til hhv. produktionverifikation, produktion og produkt, hvor kun de de metrologiske aspekter medtages. Producenten må derfor vurdere, hvordan graden af kvalitetssikring og økonomiske omkostninger skal afbalanceres.   

    Den harmoniserede standard DS/EN ISO 13485:2016 kan følges for at sikre overensstemmelse med kvalitetskravene i MDD. Følges hele standarden, er producenten fuldt ud i overensstemmelse med bilag II. 

\begin{figure}[htb]
\centering
\includegraphics[width=5in]{Klassificering}
\caption{€€REF€€Klassificeringen af det medicinske udstyr har indflydelse på hvilke bilag som skal opfyldes for at opnå CE-mærkningen.}
\label{fig:Klas}
\end{figure}

\subsection{CE-mærkningen}
   
Når bilag VII samt en af de ovenstående kvalitetssikringsbilag er opfyldt, gennemgår og vurderer et selvvalgt bemyndiget organ, om produktet lever op til kravene i MDD. Disse bemyndigede organer er private virksomheder, som er udvalgt af nationale sundhedsmyndigheder i EU. Når dokumentationen godkendes udstedes et certifikat, som giver producenten lov til at påsætte CE-mærket på sine produkter, og dermed markedsføre dem.    
 Producenten skal opbevare sin overensstemmelseserklæring i mindst fem år efter produktionen af sidste produkt. Producenten har samtidig det fulde ansvar for at holde sin dokumentation opdateret, således de opfyldte bilag til hver en tid kan godkendes af det bemyndigede organ. Det bemyndigede organ har ansvaret for løbende at vurdere dokumentationen.
Det gøres opmærksom på, at producenten er forpligtiget til at vedligeholde sit markedsovervågningssystem, som oprettes jvf. bilag VII. Efter markedsføringen skal producenten fortsat systematisk indsamle og vurdere erfaringer, som opnås ved brug af det medicinske udstyr på markedet. 

\subsection{Risikoanalyse}




