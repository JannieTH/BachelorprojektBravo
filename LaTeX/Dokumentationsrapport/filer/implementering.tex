\chapter{Implementering}

\section{Indledning}
	Dette kapitel indeholder enhedstest,integrationstest samt €€€€€€€€€€acceptest. Alle udførte tests er reproducerbare, hvilket afspejles i den høje detaljegrad som testene er beskrevet i. For at kunne reproducere beskrevne tests, forventes et kendskab til Labview, herunder LINX [LabVIEW MakerHub]. 
	
	    	  
		\subsection{Formål}
	
		\subsection{Læsevejledning}	
	
		
		\subsection{Versionshistorik}

 
\section{Enhedstest}
	Dette afsnit beskriver de indledende funktionstests, hvor hver enkelte selvstændige funktion i systemet afprøves. 
	
	\subsection{Højtaler ABS-224-RC}
	\label{etha}
		\subsubsection{Testformål}
		Det afprøves, om højtaleren kan generere en lyd.
		\subsubsection{Produktspecifikationer}
	
	
		\textit{Hardware:}\\
		\abshøj\\
		\højtalerkabel\\
		\pins\\
		\arduino\\
		\PC\\
		\usbkabel\\
	
		\textit{Software:}\\
		\labview\\
		\visa\\
		\vi\\
		\ardsw\
	
		\subsubsection{Opstilling og opsætning}
		Højtaleren er loddet til højtalerkablets ene ende, og pin headerne er loddet til kablets anden ende. 
		Pin headerne er isat Arduino'en i pin 46 (PL3(OC5A)), som et er digitalt PWM output, og til ground (GND). 
		Arduino'en er med et USB kabel koblet til PC'en. Testopstillingen er vist i figur \ref{fig:etha}.\\ 
	  
			\begin{figure}[htb]
			\centering
				\includegraphics[width=3in]{haArduino}
				\caption{Testopstilling for enhedstest af Højtaler ABS-224-RC.}	
				\label{fig:etha}
			\end{figure}
	
		På PC'en er VI'et \texttt{genererfrekvenssignal0.2.vi} åbnet i LabVIEW. Blokdiagrammet for VI'et er vist i figur \ref{fig:bdgenerer}.   \\   
	
			\begin{figure}[htb]
			\centering
				\includegraphics[width=4in]{genererfrekvenssignal02}
				\caption{I blokdiagrammet \texttt{genererfrekvenssignal0.2.vi} er der anvendt følgende LINX-VI'er; Initialize, Digital Write Square Wave samt Close.}	
				\label{fig:bdgenerer}
			\end{figure}	  
	
		\subsubsection{Udførsel}
			\begin{enumerate}
				\item I \textit{Enter Frequency} på frontpanelet i \texttt{genererfrekvenssignal0.2.vi}, indtastes 500. 
				\item Der trykkes på \textit{Run}. 
				\item Der lyttes efter lydsignal fra højtaleren.  
			\end{enumerate}
			
		\subsubsection{Resultater}
		Lydsignal blev generet og afspillet. 
		\subsubsection{Diskussion} 
		-
		\subsubsection{Konklusion}
		Højtaleren opfylder testen idet der generes en lyd. Samtidig konkluderes, at øvrigt anvendt HW og SW i denne test virker tilfredsstillende, og der vil derfor ikke blive lavet yderligere enhedstests på disse komponenter. 
	
		\subsubsection{Aktion}
		- 
	\subsection{Minijack PC Mikrofon}
		\subsubsection{Testformål}
		Det afprøves, om mikrofonen kan optage en tone.
		\subsubsection{Produktspecifikationer}
	
			\textit{Hardware:}\\
			\mikrofon\\
			\PC\\
	
			\textit{Software:}\\
			\labview\\
	
		\subsubsection{Opstilling og opsætning}
		Mikrofonen er sat i PC'ens minijack-stik. Testopstillingen er vist i figur \ref{fig:etmik}.\\ 
	  
			\begin{figure}[htb]
			\centering
				\includegraphics[width=3in]{mikopstilling}
				\caption{Testopstilling for enhedstest af Minijack PC Mikrofon.}	
				\label{fig:etmik}
			\end{figure}
	
			På PC'en er VI'et \texttt{optagefrekvenssignal0.1.vi} åbnet i LabVIEW. Blokdiagrammet for VI'et er vist i figur \ref{fig:bdoptage}.   \\   
	
			\begin{figure}[htb]
			\centering
				\includegraphics[width=4in]{optagefrekvenssignal01}
				\caption{I blokdiagrammet \texttt{optagefrekvenssignal0.1.vi} opsamles lydsignalet, som vises i en graf på frontpanelet.}	
				\label{fig:bdoptage}
			\end{figure}	  
	
		\subsubsection{Udførsel}
			\begin{enumerate}
				\item I VI'et \texttt{optagefrekvenssignal0.1.vi}, trykkes på \textit{Run}.  
				\item Der indtales en tone i mikrofonen. 
				\item På frontpanelet i VI'et ses frekvensudsving på en graf.  
			\end{enumerate}
		
		\subsubsection{Resultater}
		Mikronen har opfanget et frekvenssignal. 
		\subsubsection{Diskussion} 
		-
		\subsubsection{Konklusion}
		Mikrofonen opfylder testen idet der optages en tone. 
		\subsubsection{Aktion}
		- 
	
	\subsection{Tores højtaler}
		\subsubsection{Testformål}
		Det afprøves, om højtaleren kan generere en lyd.
		\subsubsection{Produktspecifikationer}
	
	
		\textit{Hardware:}\\
		\tores\\
		\højtalerkabel\\
		\kabelsko\\
		\pins\\
		\arduino\\
		\PC\\
		\usbkabel\\
	
		\textit{Software:}\\
		\labview\\
		\visa\\
		\vi\\
		\ardsw\
	
		\subsubsection{Opstilling og opsætning}
		Højtalerkablets ene ende er påsat kabelsko, som er påsat højtaleren. Til kablets anden ende er pin headere loddet fast. Pin headerne er isat Arduino'en i pin 46 (PL3(OC5A)), som et er digitalt PWM output, og til ground (GND). 
		Arduino'en er med et USB kabel koblet til PC'en. Testopstillingen er vist i figur \ref{fig:ethb}.\\ 
	  
			\begin{figure}[htb]
			\centering
				\includegraphics[width=3in]{hbArduino}
				\caption{Testopstilling for enhedstest af Tores højtaler.}	
				\label{fig:ethb}
			\end{figure}
	
		På PC'en er VI'et \texttt{genererfrekvenssignal0.2.vi} åbnet i LabVIEW. Blokdiagrammet for VI'et er vist i figur \ref{fig:bdgenerer}.   \\   
	
		
	
		\subsubsection{Udførsel}
			Udføres på samme vis, som ved enhedstest af \ref{etha} Højtaler ABS-224-RC. 
			
		\subsubsection{Resultater}
		Lydsignal blev generet og afspillet. 
		\subsubsection{Diskussion} 
		-
		\subsubsection{Konklusion}
		Højtaleren opfylder testen idet der generes en lyd. 
		\subsubsection{Aktion}
		- 
		
		\subsection{Ny enhedstest af Minijack PC Mikrofon d. 25.10.16}
		\subsubsection{Testformål}
		Det afprøves, om mikrofonen overhovedet opfanger et signal.
		\subsubsection{Produktspecifikationer}
	
			\textit{Hardware:}\\
			\mikrofon\\
			\PC\\
	
			\textit{Software:}\\
			\labview\\
	
		\subsubsection{Opstilling og opsætning}
		Mikrofonen er med en ledning tilsluttet PC'ens minijack-stik. Ledningens længde er godt en meter €€€SKAL MÅLES, og muliggør dermed, at PC'en er placeret i et rum, og mikrofonen med ledningen gennem den lukkede dør, er placeret i et andet rum. 
	
			På PC'en er VI'et \texttt{optagefrekvenssignal0.1.vi} åbnet i LabVIEW. Blokdiagrammet for VI'et er vist i figur \ref{fig:bdoptage}.   \\   
			
			Testopstillingen er vist i figur €€€€.\\ 
	
		\subsubsection{Udførsel}
			\begin{enumerate}
				\item I VI'et \texttt{optagefrekvenssignal0.1.vi}, trykkes på \textit{Run}.  
				\item Der indtales en tone i mikrofonen. 
				\item På frontpanelet i VI'et afventes frekvensudsving på en graf.  
			\end{enumerate}
		
		\subsubsection{Resultater}
		Mikronen har ikke opfanget et frekvenssignal. 
		\subsubsection{Diskussion} 
		-
		\subsubsection{Konklusion}
		Mikrofonen opfanger intet frekvensudsving og det konkluderes, at den virker til at være deaktiv eller ude af funktion.
		\subsubsection{Aktion}
		Denne problemstilling må undersøges yderligere med henblik på at få en fungerende, aktiv mikrofon, som opfanger et frekvenssignal. 
		
\subsection{Elektret Mikrofon}
		\subsubsection{Testformål}
		Det afprøves, om mikrofonen opfanger et signal.
		
		\subsubsection{Produktspecifikationer}
	
			\textit{Hardware:}\\
			\elektret\\
			\pins\\
			\mikrofonkabel\\
			\krympeflex\\
			\daq\\
			\daqusb\\			
			\PC\\
	
			\textit{Software:}\\
			\labview\\
	
		\subsubsection{Opstilling og opsætning}
		På mikrofonkablet er pins fastloddet, som igen er loddet fast til mikrofonprintet. Den røde ledning forbinder mikrofonprintets "AUD" til "+AI0"-indgangen på DAQ'en. Den sorte ledning forbinder mikrofonprintet til "GND". Den blå ledning forbinder "VCC" til  IC power-supply  
		Rød ledning --> "AUD" --> "+AI0" 
		Sort ledning --> "GND" --> "GND"
		Blå ledning --> VCC --> "+5V"
		
		 mi ene ende er påsat kabelsko, som er påsat højtaleren. Til kablets anden ende er pin headere loddet fast. Pin headerne er isat Arduino'en i pin 46 (PL3(OC5A)), som et er digitalt PWM output, og til ground (GND). 
		Arduino'en er med et USB kabel koblet til PC'en. Testopstillingen er vist i figur \ref{fig:ethb}.\\
		
		
		er med en ledning tilsluttet PC'ens minijack-stik. Ledningens længde er godt en meter €€€SKAL MÅLES, og muliggør dermed, at PC'en er placeret i et rum, og mikrofonen med ledningen gennem den lukkede dør, er placeret i et andet rum. 
	
			På PC'en er VI'et \texttt{optagefrekvenssignal0.1.vi} åbnet i LabVIEW. Blokdiagrammet for VI'et er vist i figur \ref{fig:bdoptage}.   \\   
			
			Testopstillingen er vist i figur €€€€.\\ 
	
		\subsubsection{Udførsel}
			\begin{enumerate}
				\item I VI'et \texttt{optagefrekvenssignal0.1.vi}, trykkes på \textit{Run}.  
				\item Der indtales en tone i mikrofonen. 
				\item På frontpanelet i VI'et afventes frekvensudsving på en graf.  
			\end{enumerate}
		
		\subsubsection{Resultater}
		Mikronen har ikke opfanget et frekvenssignal. 
		\subsubsection{Diskussion} 
		-
		\subsubsection{Konklusion}
		Mikrofonen opfanger intet frekvensudsving og det konkluderes, at den virker til at være deaktiv eller ude af funktion.
		\subsubsection{Aktion}
		Denne problemstilling må undersøges yderligere med henblik på at få en fungerende, aktiv mikrofon, som opfanger et frekvenssignal. 
		


\section{Integrationstest}

	\subsection{Bordtest nr. 1} %% Skal måske omdøbes?!
	\label{bordtest1}
		\subsubsection{Testformål}
		Det afprøves, at generere et frekvenssignal som udsendes gennem højtaleren, og derefter opfanges af mikrofonen, hvor den højst målte frekvens til sidst angives.  
		\subsubsection{Produktspecifikationer}
		
		\textit{Hardware:}\\
		\abshøj\\
		\højtalerkabel\\
		\pins\\
		\arduino\\
		\PC\\
		\usbkabel\\
	
		\textit{Software:}\\
		\labview\\
		\visa\\
		\vi\\
		\ardsw\
		
		\subsubsection{Opstilling og opsætning}
		\textit{1. delopstilling}:\\
		Højtaleren er loddet til højtalerkablets ene ende, og pin headerne er loddet til kablets anden ende. 
		Pin headerne er isat Arduino'en i pin 46 (PL3(OC5A)), som et er digitalt PWM output, og til ground (GND). 
		Arduino'en er med et USB kabel koblet til PC'en. 		
		På PC'en er VI'et \texttt{genererfrekvenssignal0.2.vi} åbnet i LabVIEW. Testopstillingen for denne del er vist i figur \ref{fig:etha}, og blokdiagrammet er vist i figur \ref{fig:bdgenerer}.\\ 
 
		\textit{2. delopstilling}:\\
		Mikrofonen er sat i PC'ens minijack-stik. På PC'en er VI'et \texttt{optagefrekvenssignal0.1.vi} åbnet i LabVIEW. Testopstillingen for denne del er vist i figur \ref{fig:etmik}, og blokdiagrammet er vist i figur \ref{fig:bdoptage}.\\ 
		
		\subsubsection{Udførsel}
			\begin{enumerate}
				\item Højtaleren holdes manuelt således membranen står i lodret position. 
				\item Mikrofonen holdes manuelt, vendt mod højtaleren, i en afstand på 5 cm. 
				\item I VI'et \texttt{genererfrekvenssignal0.2.vi} indtastes den ønskede frekvens i den numeriske kontrol \textit{Enter Frequency}. 
					\begin{enumerate}
						\item Koden eksekveres ved at trykke på \textit{Run}. 
					\end{enumerate} 
				\item I VI'et \texttt{optagefrekvenssignal0.1.vi} trykkes på \textit{Run}. 
					\begin{enumerate}
						\item Den maksimale optagede frekvens aflæses i \textit{Max Frequency}. 
					\end{enumerate}	  
			\end{enumerate}
			
			Punkt 1-4 gentages med frekvenser på: 100 Hz, 150 Hz, 200 Hz, 400 Hz, 500 Hz, 600 Hz og 700 Hz. 
			
			\subsubsection{Resultater}
			Den maksimale optagede frekvens var ikke tilnærmelsesvis frekvensen på den udsendte tone.
			\subsubsection{Diskussion}
			Det ønskede resultat er frekvensen på den udsendte tone, hvilket ikke var tilfældet i denne test. Hvor er fejlen opstået; er der fejl i LabVIEW-kode eller hardware? 
			\subsubsection{Konklusion}
			Det er nødvendigt at undersøge om fejlen opstår i vores hardware eller software. 
			\subsubsection{Aktion}
			Det skal med en online tonegenerator undersøges, hvor fejlen er opstået. 

	\subsection{Bordtest nr. 2} %% Skal måske omdøbes?!
		\subsubsection{Testformål}
		Det afprøves, at generere et frekvenssignal fra en online tonegenerator, som udsendes gennem PC'ens højtaler, og derefter opfanges af mikrofonen, hvor den højst målte frekvens til sidst angives.  
		
		\subsubsection{Produktspecifikationer}
		
		\textit{Hardware:}\\
		\mikrofon\\
		\PC\\
	
		\textit{Software:}\\
		\labview\\
		\visa\\
		\vi\\
		\ardsw\\
		\texttt{onlinetonegenerator.com}
		
		\subsubsection{Opstilling og opsætning}
		Mikrofonen er sat i PC'ens minijack-stik. På PC'en er VI'et \texttt{optagefrekvenssignal0.1.vi} åbnet i LabVIEW. Testopstillingen for denne del er vist i figur \ref{fig:etmik}, og blokdiagrammet er vist i figur \ref{fig:bdoptage}.\\ 
		I en internetbrowser er hjemmesiden \texttt{www.onlinetonegenerator.com} åbnet, og PC'ens højtalere er slået til. 
		
		\subsubsection{Udførsel}
			\begin{enumerate}
				\item Mikrofonen holdes manuelt, vendt mod PC'ens højtaler, i en afstand på 5 cm. 
				\item I \texttt{onlinetonegenerator.com} genereres et signal med den ønskede frekvens. 
				\item I VI'et \texttt{optagefrekvenssignal0.1.vi} trykkes på \textit{Run}. 
					\begin{enumerate}
						\item Den maksimale optagede frekvens aflæses i \textit{Max Frequency}. 
					\end{enumerate}	  
			\end{enumerate}
			
			Punkt 1-4 gentages med frekvenser på: 100 Hz, 150 Hz, 200 Hz, 400 Hz, 500 Hz, 600 Hz og 700 Hz. 
			
			\subsubsection{Resultater}
			Den optagede frekvens var den generede udsendte frekvens (+/- 0.5 Hz). 
			\subsubsection{Diskussion}
			Der opnås nu pæne resultater, og der reflekteres over om resultaterne i \ref{bordtest1} skyldes fejl i højtaler i fejl i VI'et \texttt{genererfrekvenssignal0.2.vi}. 
			\subsubsection{Konklusion}
			Det konkluderes, at der ikke er fejl i VI'et \texttt{optagefrekvenssignal0.1.vi}. 
			\subsubsection{Aktion}
			Det skal undersøges, hvilken forskel der er på frekvenssignalet fra onlinetonegenerator.com og det generede frekvenssignal udsendt fra højtaleren ABS-224-RC.  
	
		\subsection{Bordtest nr. 3} %% Skal måske omdøbes?!
		\subsubsection{Testformål}
		Det undersøges, hvilken forskel der er på frekvenssignalet fra onlinegenerator.com og det genererede frekvenssignal udsendt fra højtaleren ABS-224-RC.  
		
		\subsubsection{Produktspecifikationer}
		
		\textit{Hardware:}\\
		\abshøj\\
		\højtalerkabel\\
		\pins\\
		\krympeflex
		\arduino\\
		\usbkabel\\
		\PC\\
		\mikrofon\\
	
		\textit{Software:}\\
		\labview\\
		\visa\\
		\vi\\
		\ardsw\\
		\texttt{onlinetonegenerator.com}
		
		\subsubsection{Opstilling og opsætning}
		\textit{1. delopstilling}:\\
		Højtaleren er loddet til højtalerkablets ene ende, og pin headerne er loddet til kablets anden ende. 
		Pin headerne er isat Arduino'en i pin 46 (PL3(OC5A)), som et er digitalt PWM output, og til ground (GND). 
		Arduino'en er med et USB kabel koblet til PC'en. 		
		På PC'en er VI'et \texttt{genererfrekvenssignal0.2.vi} åbnet i LabVIEW. Testopstillingen for denne del er vist i figur \ref{fig:etha}, og blokdiagrammet er vist i figur \ref{fig:bdgenerer}.\\ 
 
		\textit{2. delopstilling}:\\
		Mikrofonen er sat i PC'ens minijack-stik. På PC'en er VI'et \texttt{optagefrekvenssignal0.1.vi} åbnet i LabVIEW. Testopstillingen for denne del er vist i figur \ref{fig:etmik}, og blokdiagrammet er vist i figur \ref{fig:bdoptage}.\\  
		
		\textit{3. delopstilling}:\\
		I en internetbrowser er hjemmesiden \texttt{www.onlinetonegenerator.com} åbnet, og PC'ens højtalere er slået til. 
		
		\subsubsection{Udførsel}
			
			\textit{1. deltest}
			\begin{enumerate}
				\item Højtaleren holdes manuelt således membranen står i lodret position. 
				\item Mikrofonen holdes manuelt, vendt mod højtaleren, i en afstand på 5 cm. 
				\item I VI'et \texttt{genererfrekvenssignal0.2.vi} indtastes den ønskede frekvens i den numeriske kontrol \textit{Enter Frequency}. 
					\begin{enumerate}
						\item Koden eksekveres ved at trykke på \textit{Run}. 
					\end{enumerate} 
				\item I VI'et \texttt{optagefrekvenssignal0.1.vi} trykkes på \textit{Run}. 
					\begin{enumerate}
						\item Den maksimale optagede frekvens aflæses i \textit{Max Frequency}. 
					\end{enumerate}	 	
			\end{enumerate}
			
			
			\textit{2. deltest}			
			\begin{enumerate}
				\item Mikrofonen holdes manuelt, vendt mod PC'ens højtaler, i en afstand på 5 cm. 
				\item I \texttt{onlinetonegenerator.com} genereres et signal med den ønskede frekvens. 
				\item I VI'et \texttt{optagefrekvenssignal0.1.vi} trykkes på \textit{Run}. 
					\begin{enumerate}
						\item Den maksimale optagede frekvens aflæses i \textit{Max Frequency}. 
					\end{enumerate}	  
			\end{enumerate}
		
			Deltestene gentages med frekvenser på: 100 Hz, 150 Hz, 200 Hz, 400 Hz, 500 Hz, 600 Hz, 700, 1000 0g 1200 Hz, og resultaterne sammenholdes. 
			
			\subsubsection{Resultater}
			 Det blev observeret, at resultaterne fra det generede frekvenssignal i VI'et \texttt{genererfrekvenssignal0.2.vi} var grundtonens harmoniske overtoner, idet frekvensen udsendes som et firkantsignal. Kun ved højfrekvente signaler (<1 kHz), blev grundtonen opfanget. 
			 Ved at benytte \texttt{onlinetonegenerator.com}, kunne der udsendes et sinussignal med en given frekvens, som blev korrekt opfanget (+/- 0.5 Hz).     
			\subsubsection{Diskussion}
			Der opnås pæne resultater ved at bruge et sinussignal, men der er desværre meget kompliceret at generere sinussignaler til en Arduino.   
			\subsubsection{Konklusion}
			Det konkluderes, at der ikke er fejl i software og hardware, og de unøjagtige resultater skyldes firkantsignalets harmoniske overtoner.  
			\subsubsection{Aktion}
			Det skal undersøges, om det er muligt at filtrere firkantssignalets harmoniske overtoner fra, således firkantssignalet kan benyttes. 
		
		\subsection{Bordtest nr. 4} %% Skal måske omdøbes?!
		\subsubsection{Testformål}
		Det undersøges, om en resonatorlignende beholder kan dæmpe de harmoniske overtoner fra firkantsignalet. 
		
		\subsubsection{Produktspecifikationer}
		
		\textit{Hardware:}\\
		\tores\\
		\højtalerkabel\\
		\kabelsko\\
		\pins\\
		\krympeflex\\
		\arduino\\
		\mikrofon\\
		\PC\\
		\usbkabel\\
		\sprøjtebeholder (resonator)\\
		Lineal\\
	
		\textit{Software:}\\
		\labview\\
		\visa\\
		\vi\\
		\ardsw\\
		
		
		\subsubsection{Opstilling og opsætning}
		
		Højtalerkablets ene ende er påsat kabelsko, som er påsat højtaleren. Til kablets anden ende er pin headere loddet fast og forsejlet med krympeflex. Pin headerne er isat Arduino'en i pin 46 (PL3(OC5A)), som et er digitalt PWM output, og til ground (GND). 
		Arduino'en er med et USB kabel koblet til PC'en.	
		Mikrofonen er sat i PC'ens minijack-stik og er ført ned i resonatoren hvor den ligger i bunden. Højtaleren placeres i en afstand på en halv diameter af højtalerens membran, ovenfor resonatorens hals. 
		
		På PC'en er VI'erne \texttt{genererfrekvenssignal0.2.vi} og \texttt{optagefrekvenssignal0.2.vi} åbnet i LabVIEW og blokdiagrammerne er vist i figur \ref{fig:bdgenerer} og \ref{fig:bdoptage} \\ Testopstillingen kan ses på figur \ref{fig:bt4}.  
		
		\begin{figure}[htb]
			\centering
				\includegraphics[width=3in]{bordtest4}
				\caption{Testopstilling for bordtest 4}	
				\label{fig:bt4}
			\end{figure}
		

		
		\subsubsection{Udførsel}
			
			\begin{enumerate}
				\item Tryksprøjtedelen afmonteres af beholderen og fungere nu som resonator. Resonatoren stilles på et bord med halsen pegende opad. 
				\item Mikrofonen føres ned i resonatoren og ligger i resonatorens bund. 
				\item Linealen påsættes resonatorens hals så den fungere som afstandsmåler fra halsåbningen.
				\item Højtaleren holdes manuelt over resonatorhalsen i en afstand på en halv diameter af højtalermembranen. Ved anvendelse ef den specificerede højtaler er afstanden to centimeter. 
				\item I VI'et \texttt{genererfrekvenssignal0.2.vi} indtastes den ønskede frekvens i den numeriske kontrol \textit{Enter Frequency}. 
					\begin{enumerate}
						\item Koden eksekveres ved at trykke på \textit{Run}. 
					\end{enumerate} 
				\item I VI'et \texttt{optagefrekvenssignal0.2.vi} trykkes på \textit{Run}. 
					\begin{enumerate}
						\item Den maksimale optagede frekvens aflæses i \textit{Max Frequency}. 
					\end{enumerate}	 	
			\end{enumerate}
			
			
			Testen udføres med en frekvens på 200 Hz, 500 Hz, 1000 Hz og 1200 Hz. Der afprøves to gange med hvert frekvens  
			
			\subsubsection{Resultater}
			Det blev observeret ved begge forsøg, at resultatet fra det generede frekvenssignal i VI'et \texttt{genererfrekvenssignal0.2.vi} på 500 Hz var  en af grundtonens harmoniske overtoner på 1500 Hz. Dette ses i figur \ref{fig:bt4500}. 

			 Det blev observeret i første forsøg, at resultatet fra det generede frekvenssignal i VI'et \texttt{genererfrekvenssignal0.2.vi} på 200 Hz var en af grundtonens harmoniske overtone på 1803 Hz og i andet forsøg observeres en harmonisk overtone 1402,33 Hz. Resultatet fra første forsøg vises i figur \ref{fig:bt4200}.
			 
			 Det blev observeret i første forsøg, at resultatet fra det generede frekvenssignal i VI'et \texttt{genererfrekvenssignal0.2.vi} på 1000 Hz var en af grundtonens harmoniske overtoner på 3000,33 Hz i. Dette ses i figur \ref{fig:bt41000}. I det andet forsøg observeres det at resultatet stemmer overens med den afspillede grundtone. Dette ses på figur \ref{fig:bt41000b}.  
			 
			 Det blev observeret, at resultatet fra den generede frekvenssignal i VI'et \texttt{genererfrekvenssignal0.2.vi} på 1200 Hz stemmer overens med den afspillede grundtone. Dette ses på figur \ref{fig:bt41200}.  
			 
			\begin{figure}[htb]
			\centering
				\includegraphics[width=4in]{Bordtest4500Hz}
				\caption{Resultat for for bordtest 4 ved anvendelse af 500 Hz}	
				\label{fig:bt4500}
			\end{figure} 
			
			\begin{figure}[htb]
			\centering
				\includegraphics[width=4in]{Bordtest4200Hz}
				\caption{Resultat for for bordtest 4 ved anvendelse af 200 Hz}	
				\label{fig:bt4200}
			\end{figure} 
			
			\begin{figure}[htb]
			\centering
				\includegraphics[width=4in]{Bordtest41000Hz}
				\caption{Resultat for for bordtest 4 ved anvendelse af 1000 Hz}	
				\label{fig:bt41000}
			\end{figure} 
			
			\begin{figure}[htb]
			\centering
				\includegraphics[width=4in]{Bordtest41000Hzb}
				\caption{Resultat for for bordtest 4 ved anvendelse af 1000 Hz}	
				\label{fig:bt41000b}
			\end{figure} 
			
			\begin{figure}[htb]
			\centering
				\includegraphics[width=4in]{Bordtest41200Hz}
				\caption{Resultat for for bordtest 4 ved anvendelse af 1200 Hz}	
				\label{fig:bt41200}
			\end{figure} 
	
			  
			\subsubsection{Diskussion}
			Ved forsøget med 1000 Hz observeres, at de to resultater ikke er tilnærmelsesvis ens. I første forsøg blev en harmonisk overtone opfanget, som den maksimale frekvens, hvor der i andet forsøg blev opfanget grundtonen på den genererede frekvens. Disse to resultater skal være ens og derfor er de ikke tilfredsstillende.
			
			\subsubsection{Konklusion}
			Ved generering af frekvenser lavere end 1000 Hz opfanges harmoniske overtoner, i stedet for grundtonen, som er det ønskede resultat. 
			Ved generering af frekvenser lig 1000 Hz opfanges ustabile resultater. 
			Ved generering af frekvenser højere end 1000 Hz, opnås pæne resultater, hvor den genererede frekvens er lig den opfangede frekvens.  
			Dermed konkluderes, at resonatoren i dette tilfælde ikke virker dæmpende på harmoniske overtoner på frekvenser lavere end 1000 Hz. 
			  
			\subsubsection{Aktion}
			Det skal undersøges, om målinger genereret med en frekvens lig eller højere end 1000 Hz er stabile. 
			
			\subsection{Bordtest nr. 5} %% Skal måske omdøbes?!
		\subsubsection{Testformål}
		Det undersøges, om firkantsignaler genereret med frekvenser lig eller højere end 1000 Hz er stabile. 
		
		\subsubsection{Produktspecifikationer}
		
		\textit{Hardware:}\\
		\tores\\
		\højtalerkabel\\
		\kabelsko\\
		\pins\\
		\krympeflex\\
		\arduino\\
		\mikrofon\\
		\PC\\
		\usbkabel\\
		\sprøjtebeholder (resonator)\\
		Lineal\\
	
		\textit{Software:}\\
		\labview\\
		\visa\\
		\vi\\
		\ardsw\\
		
		
		\subsubsection{Opstilling og opsætning}
		
		Højtalerkablets ene ende er påsat kabelsko, som er påsat højtaleren. Til kablets anden ende er pin headere loddet fast og forsejlet med krympeflex. Pin headerne er isat Arduino'en i pin 46 (PL3(OC5A)), som et er digitalt PWM output, og til ground (GND). 
		Arduino'en er med et USB kabel koblet til PC'en.	
		Mikrofonen er sat i PC'ens minijack-stik og er ført ned i resonatoren hvor den ligger i bunden. Højtaleren placeres i en afstand på en halv diameter af højtalerens membran, ovenfor resonatorens hals. 
		
		På PC'en er VI'erne \texttt{genererfrekvenssignal0.2.vi} og \texttt{optagefrekvenssignal0.2.vi} åbnet i LabVIEW og blokdiagrammerne er vist i figur \ref{fig:bdgenerer} og \ref{fig:bdoptage} \\ Testopstillingen er den samme som i bordtest 4, og vises i figur \ref{fig:bt4}.  
		

		\subsubsection{Udførsel}
			
			\begin{enumerate}
				\item Tryksprøjtedelen afmonteres af beholderen og fungere nu som resonator. Resonatoren stilles på et bord med halsen pegende opad. 
				\item Mikrofonen føres ned i resonatoren og ligger i resonatorens bund. 
				\item Linealen påsættes resonatorens hals så den fungere som afstandsmåler fra halsåbningen.
				\item Højtaleren holdes manuelt over resonatorhalsen i en afstand på en halv diameter af højtalermembranen. Ved anvendelse ef den specificerede højtaler er afstanden to centimeter. 
				\item I VI'et \texttt{genererfrekvenssignal0.2.vi} indtastes den ønskede frekvens i den numeriske kontrol \textit{Enter Frequency}. 
					\begin{enumerate}
						\item Koden eksekveres ved at trykke på \textit{Run}. 
					\end{enumerate} 
				\item I VI'et \texttt{optagefrekvenssignal0.2.vi} trykkes på \textit{Run}. 
					\begin{enumerate}
						\item Den maksimale optagede frekvens aflæses i \textit{Max Frequency}. 
					\end{enumerate}	 	
			\end{enumerate}
			
			Testen udføres med en frekvens på 950 Hz, 1000 Hz, 1100 Hz, 1200 Hz, 1300 Hz. Testen fortages to gange med hver frekvens.  
			
			\subsubsection{Resultater}
			
			\begin{table}[]
				\centering
				\caption{Tabel over resultater}
				\label{bordtest5resultater}
				\begin{tabular}{lll}
					\multicolumn{1}{l|}{\textbf{Udsendt frekvens {[}Hz{]}}} & 	
					\multicolumn{1}{l|}{\textbf{1. resultat}} & \textbf{2. resultat} \\ \hline
					\multicolumn{1}{c|}{950}& 
					\multicolumn{1}{c|}{954.33}&954.33\\
					\multicolumn{1}{c|}{1000}& 
					\multicolumn{1}{c|}{1000.00}&1000.00\\
					\multicolumn{1}{c|}{1100}& 
					\multicolumn{1}{c|}{1101.33}&1101.33\\
					\multicolumn{1}{c|}{1200}& 
					\multicolumn{1}{c|}{1202.00}&1202.00\\
					\multicolumn{1}{c|}{1300}& 
					\multicolumn{1}{c|}{1302.00}&1302.00\\
                   
				\end{tabular}
			\end{table}

			Det blev observeret ved begge forsøg, at resultatet fra det generede frekvenssignal i VI'et \texttt{genererfrekvenssignal0.2.vi} på 950 Hz var  en af grundtonens harmoniske overtoner på 1500 Hz. Dette ses i figur \ref{fig:bt4500}. 

			 Det blev observeret i første forsøg, at resultatet fra det generede frekvenssignal i VI'et \texttt{genererfrekvenssignal0.2.vi} på 200 Hz var en af grundtonens harmoniske overtone på 1803 Hz og i andet forsøg observeres en harmonisk overtone 1402,33 Hz. Resultatet fra første forsøg vises i figur \ref{fig:bt4200}.
			 
			 Det blev observeret i første forsøg, at resultatet fra det generede frekvenssignal i VI'et \texttt{genererfrekvenssignal0.2.vi} på 1000 Hz var en af grundtonens harmoniske overtoner på 3000,33 Hz i. Dette ses i figur \ref{fig:bt41000}. I det andet forsøg observeres det at resultatet stemmer overens med den afspillede grundtone. Dette ses på figur \ref{fig:bt41000b}.  
			 
			 Det blev observeret, at resultatet fra den generede frekvenssignal i VI'et \texttt{genererfrekvenssignal0.2.vi} på 1200 Hz stemmer overens med den afspillede grundtone. Dette ses på figur \ref{fig:bt41200}.  
			 
			\begin{figure}[htb]
			\centering
				\includegraphics[width=4in]{Bordtest4500Hz}
				\caption{Resultat for for bordtest 4 ved anvendelse af 500 Hz}	
				\label{fig:bt4500}
			\end{figure} 
			
				
			  
			\subsubsection{Diskussion}
			Ved forsøget med 1000 Hz observeres, at de to resultater ikke er tilnærmelsesvis ens. I første forsøg blev en harmonisk overtone opfanget, som den maksimale frekvens, hvor der i andet forsøg blev opfanget grundtonen på den genererede frekvens. Disse to resultater skal være ens og derfor er de ikke tilfredsstillende.
			
			\subsubsection{Konklusion}
			Ved generering af frekvenser lavere end 1000 Hz opfanges harmoniske overtoner, i stedet for grundtonen, som er det ønskede resultat. 
			Ved generering af frekvenser lig 1000 Hz opfanges ustabile resultater. 
			Ved generering af frekvenser højere end 1000 Hz, opnås pæne resultater, hvor den genererede frekvens er lig den opfangede frekvens.  
			  
			\subsubsection{Aktion}
			Det skal undersøges, om målinger genereret med en frekvens lig eller højere end 1000 Hz er stabile. 


		
	
	
	

	