\chapter{Implementering}
\renewcommand{\arraystretch}{1.5}


\section{Indledning}
Implementeringsfasen forløber sig efter en iterativ proces, hvor de enkelte dele er udviklet og testet før de
sammensættes med andre delelementer. På denne måde er prototypen gradvist samlet og løbende testet i gennem forløbet.
4.1.1 Formål
Formålet med dette dokument er, at beskrive i detaljer hvordan de enkelte delelementer
fra designdokumentet konkret er implementeret.
	Dette kapitel indeholder udførte enhedstest samt integrationstest samt. Alle udførte tests er reproducerbare, hvilket afspejles i den høje detaljegrad som testene er beskrevet i. For at kunne reproducere beskrevne tests, forventes et kendskab til Labview, herunder LINX [LabVIEW MakerHub]. 
	
		\subsection{Formål}
		Formålet med dette dokument er at beskrive i detaljer, hvordan de enheds- samt intergrationstests er udført. 
		
		\subsection{Læsevejledning}	
	
	
	€ Her indsættes beskrivelse samt oversigt at navngivning til tests (i en smuk tabel)
	
	\begin{table}[htb]
				\centering
				\caption{Tabeloversigt over udførte enhedstests} 
				\label{enhedstests 2}
				\begin{tabular}{lll}
					\multicolumn{1}{l|}{\textbf{Test-ID}} & 	
					\multicolumn{1}{l|}{\textbf{Emne}} & \textbf{ID: navn} \\ \hline
					\multicolumn{1}{l|}{E01-H1}& 
					\multicolumn{1}{l|}{Højtaler}&H1: ABS-224-RC\\
					\multicolumn{1}{l|}{E02-M1}& 
					\multicolumn{1}{l|}{Mikrofon}&M1: Minijack PC Mikrofon\\
					\multicolumn{1}{l|}{E03-H2}& 
					\multicolumn{1}{l|}{Højtaler}&H2: Wide Band 2,5'' SB65WBAC25-4\\
					\multicolumn{1}{l|}{E04-M1}& 
					\multicolumn{1}{l|}{Mikrofon}&M1: (udvidet test)\\
					\multicolumn{1}{l|}{E05-M2}& 
					\multicolumn{1}{l|}{Mikrofon}&M2: Logitech HD WEBCAM C270\\	
					\multicolumn{1}{l|}{E06-M3}& 
					\multicolumn{1}{l|}{Mikrofon}&M3: Electret Microphone BOB12758\\
						\multicolumn{1}{l|}{E07-M4}& 
					\multicolumn{1}{l|}{Mikrofon}&M4: Electret Microphone Amplifier MAX4466 (2.5 V strømforsyning)\\
					\multicolumn{1}{l|}{E08-M4}& 
					\multicolumn{1}{l|}{Mikrofon}&M4: (5 V strømforsyning)\\
					\multicolumn{1}{l|}{E09-M4}& 
					\multicolumn{1}{l|}{Mikrofon}&M4: (frekvensbånd)\\
					\multicolumn{1}{l|}{E10-H3}& 
					\multicolumn{1}{l|}{Højtaler}&H3: Multimedia USB Speaker HP-1800\\
						
					\multicolumn{1}{l|}{E11-VI02G}& 
					\multicolumn{1}{l|}{LabVIEW VI}&VI02G: LabVIEW VI: genererfrekvenssignal02\\
					\multicolumn{1}{l|}{E12-VI01\underline{O}}& 
					\multicolumn{1}{l|}{LabVIEW VI}&VI01\underline{O}: optagefrekvenssignal01.vi\\
					\multicolumn{1}{l|}{E13-VI02\underline{O}}& 
					\multicolumn{1}{l|}{LabVIEW VI}&VI02\underline{O}: optagefrekvenssignal02.vi\\
					\multicolumn{1}{l|}{E14-VI05\underline{O}}& 
					\multicolumn{1}{l|}{LabVIEW VI}&VI05\underline{O}: optagefrekvenssignal05.vi\\
					\multicolumn{1}{l|}{E15-VI06\underline{O}}& 
					\multicolumn{1}{l|}{LabVIEW VI}&VI06\underline{O}: optagefrekvenssignal06.vi\\
					\multicolumn{1}{l|}{E16-VI08\underline{O}}& 
					\multicolumn{1}{l|}{LabVIEW VI}&VI08\underline{O}: optagefrekvenssignal08.vi\\
								
				\end{tabular}
			\end{table}
	
	
	
 
\section{Enhedstest}
	Dette afsnit beskriver de indledende funktionstests, hvor hver enkelte selvstændige funktion i systemet afprøves. 
	
	Ved samtlige test med lydgenererende funktion er der anvendt maksimalt lydtryk på den lydgivende enhed.
	
	\subsection{E01-H1}
	\label{subsec:E01}
		\subsubsection{Testhypotese}
		Der kan genereres en lyd gennem højtaleren ABS-224-RC ved brug af \arduino{} og \labview.		
		\subsubsection{Produktspecifikationer}
	
	
		\textit{Hardware:}\\
		\abshøj\\
		\hojtalerkabel\\
		\pins\\
		\arduino\\
		\PC\\
		\usbkabel
	
		\textit{Software:}\\
		\labview\\
		\visa\\
		\vi\\
		\ardsw\
	
		\subsubsection{Opstilling og opsætning}
		Højtaleren er loddet til højtalerkablets ene ende, og pin headerne er loddet til kablets anden ende. 
		Pin headerne er isat Arduino'en i pin 46 (PL3(OC5A)), som et er digitalt PWM output, og til ground (GND). 
		Arduino'en er med et USB kabel koblet til PC'en. Delopstillingen er vist i figur \ref{fig:E01}.\\ 
	  
			\begin{figure}[htb]
			\centering
				\includegraphics[width=3in]{haArduino}
				\caption{Delopstilling af \ref{subsec:E01} E01-H1, hvor højtaler er tilkoblet Arduino.}	
				\label{fig:E01}
			\end{figure}
	
		På PC'en er \texttt{genererfrekvenssignal02.vi} åbnet i LabVIEW. \textit{Serial port}, \textit{Channel} og \textit{Duration} er sat op, som givet i blokdiagrammet, vist i figur \ref{fig:bdgenerer}.   \\   
	
			\begin{figure}[htb]
			\centering
				\includegraphics[width=4in]{genererfrekvenssignal02}
				\caption{I blokdiagrammet \texttt{genererfrekvenssignal02.vi} er der anvendt følgende LINX-VI'er; Initialize, Digital Write Square Wave samt Close.}	
				\label{fig:bdgenerer}
			\end{figure}	  
	
		\subsubsection{Udførsel}
			\begin{enumerate} 
				\item I \textit{Enter Frequency} på frontpanelet i \texttt{genererfrekvenssignal02.vi}, indtastes 500. 
				\item Der trykkes på \textit{Run}. 
				\item Der lyttes efter lydsignal fra højtaleren.  
			\end{enumerate}
			
		\subsubsection{Resultater}
		Der høres en lyd fra højtaleren. 
		\subsubsection{Diskussion} 
		-
		\subsubsection{Konklusion}
		Hypotesen er accepteret idet der kan genereres en lyd gennem højtaleren ved brug af \arduino{} og \labview. 
		
		\subsubsection{Aktion}
		- 
		
	\subsection{E02-M1}
		\subsubsection{Testhypotese}
		Der kan gennem \mikrofon opfanges en lyd. 
		
		\subsubsection{Produktspecifikationer}
	
			\textit{Hardware:}\\
			\mikrofon\\
			\PC
	
			\textit{Software:}\\
			\labview\\
			\onlineg\\
	
		\subsubsection{Opstilling og opsætning}
		Mikrofonen er sat i PC'ens minijack-stik. 
	  
			På PC'en er \texttt{optagefrekvenssignal01.vi} åbnet i LabVIEW. VI'et er opbygget af \textit{Graphics and sound} komponenter. Komponenterne, forbindelserne og de angivede værdier er vist i blokdiagrammet, i figur \ref{fig:bdoptage}.   \\   
	
			\begin{figure}[htb]
			\centering
				\includegraphics[width=5in]{of01.PNG}
				\caption{I blokdiagrammet \texttt{optagefrekvenssignal01.vi} opsamles lydsignalet, som vises i en graf på frontpanelet.}	
				\label{fig:bdoptage}
			\end{figure}	  
			
			\texttt{www.onlinetonegenerator.com} er åbnet i en browser på PC´en. 
	
		\subsubsection{Udførsel}
			\begin{enumerate}
				\item I VI'et \texttt{optagefrekvenssignal01.vi}, trykkes på \textit{Run}.  
				\begin{enumerate}
				\item Der genereres en hørbar lyd til mikrofonen fra onlinetonegenerator.com. 
				\item På frontpanelet i VI'et observeres om mikrofonen opfanger en lyd, ved at grafen viser amplitudeudsving.
				\end{enumerate}
				\item I VI'et \texttt{optagefrekvenssignal01.vi}, trykkes på \textit{Run}. 
				\begin{enumerate}
				\item Mikronfonen dækkes nu med en hånd mhp. at mikrofonen ikke opfanger en lyd. 
				\item Der generes en hørbar lyd til mikrofonen fra onlinetonegenerator.com. 
				\item På frontpanelet i VI'et observeres om mikrofonen opfanger en lyd, ved at grafen viser amplitudeudsving.
				\end{enumerate}
			\end{enumerate}
			
		\subsubsection{Resultater}
		Ad1. Der er observeret et amplitudeudsving på grafen. Mikronen har opfanget en lyd. \\
		Ad2. Der er ikke observeret et amplitudeudsving på grafen. Mikrofonen har ikke opfanget en lyd. 
		
		\subsubsection{Diskussion} 
		- 
		
		\subsubsection{Konklusion}
		Idet der ikke opfanges en hørbar lyd i Ad2., konkluderes det, at den hørbare lyd, opfanges gennem Ad1. 
		Hypotesen accepteres idet mikrofonen opfanger en lyd. 
		\subsubsection{Aktion}
		- 
	
	\subsection{E03-H2}
	\label{subsec:E03}
		\subsubsection{Testhypotese}
		Der kan genereres en lyd gennem højtaleren ved brug af \arduino{} og \labview.
		\subsubsection{Produktspecifikationer}
	
	
		\textit{Hardware:}\\
		\tores\\
		\hojtalerkabel\\
		\kabelsko\\
		\pins\\
		\arduino\\
		\PC\\
		\usbkabel
	
		\textit{Software:}\\
		\labview\\
		\visa\\
		\vi\\
		\ardsw\
	
		\subsubsection{Opstilling og opsætning}
		Højtalerkablets ene ende er påsat kabelsko, som er påsat højtaleren. Til kablets anden ende er pin headere loddet fast. Pin headerne er isat Arduino'en i pin 46 (PL3(OC5A)), som et er digitalt PWM output, og til ground (GND). 
		Arduino'en er med et USB kabel koblet til PC'en. Delopstillingen er vist i figur \ref{fig:tores}.\\ 
	  
			\begin{figure}[htb]
			\centering
				\includegraphics[width=3in]{hbArduino}
				\caption{Delopstilling af \ref{subsec:E03} E03-H2, hvor højtaler er tilkoblet Arduino.}	
				\label{fig:tores}
			\end{figure}
	  
		På PC'en er \texttt{genererfrekvenssignal02.vi} åbnet i LabVIEW. \textit{Serial port}, \textit{Channel} og \textit{Duration} er sat op, som givet i blokdiagrammet, vist i figur \ref{fig:bdgenerer}.   \\   
	
	\subsubsection{Udførsel}
			\begin{enumerate} 
				\item I \textit{Enter Frequency} på frontpanelet i \texttt{genererfrekvenssignal02.vi}, indtastes 500. 
				\item Der trykkes på \textit{Run}. 
				\item Der lyttes efter lydsignal fra højtaleren.  
			\end{enumerate}
			
		\subsubsection{Resultater}
		Der høres en lyd fra højtaleren. 
		\subsubsection{Diskussion} 
		-
		\subsubsection{Konklusion}
		Hypotesen er accepteret idet der kan genereres en lyd gennem højtaleren ved brug af \arduino{} og \labview. 
		
		\subsubsection{Aktion}
		- 
		
		\subsection{E04-M1}
		\label{subsec:E04}
		\subsubsection{Testhypotese}
		Der kan gennem \mikrofon opfanges en lyd. 
		
		\subsubsection{Produktspecifikationer}
	
			\textit{Hardware:}\\
			\mikrofon{} (med 2 meter ledning)\\
			\PC\\
			Smartphone 
	
			\textit{Software:}\\
			\labview\\
			\onlineg\\
	
		\subsubsection{Opstilling og opsætning}
		Mikrofonen er med en ledning tilsluttet PC'ens minijack-stik. PC'en er placeret i et rum, og mikrofonen med ledningen gennem den lukkede dør, er placeret i et andet rum. I en browser på smartphonen åbnes onlinetonegenerator.com. Smartphonen placeres ca. 2 cm fra mikrofonen. 
	
			På PC'en er \texttt{optagefrekvenssignal01.vi} åbnet i LabVIEW. VI'et er opbygget af \textit{Graphics and sound} komponenter. Komponenterne, forbindelserne og de angivede værdier er vist i blokdiagrammet, i figur \ref{fig:bdoptage}.   \\    
	
		\subsubsection{Udførsel}
			\begin{enumerate}
				\item I VI'et \texttt{optagefrekvenssignal01.vi}, trykkes på \textit{Run}.  
				\item Der generes en lyd i onlinetonegenerator.com til mikrofonen. 
				\item På frontpanelet i VI'et observeres amplitudeudsving på grafen.  
			\end{enumerate}
		
		\subsubsection{Resultater}
		Der blev ikke observeret et amplitudeudsving på grafen. 
		\subsubsection{Diskussion} 
		-
		\subsubsection{Konklusion}
		Mikrofonen har ikke opfanget en lyd, og hypotesen forkastes. 
		\subsubsection{Aktion}
		Det undersøges, hvilke årsager der er skyld i, at mikrofonen ikke opfanger et frekvenssignal.  
		
	\subsection{E05-M2}
		\subsubsection{Testhypotese}
		Der kan gennem \webcammic{} opfanges en lyd.  
		
		\subsubsection{Produktspecifikationer}
	
			\textit{Hardware:}\\
			\webcammic{} (med 1,5 meter ledning)\\
			\PC\\
			Smartphone
	
			\textit{Software:}\\
			\labview\\
			\onlineg\\
	
		\subsubsection{Opstilling og opsætning}
		Webkameraet er tilsluttet PC'ens USB-port. PC'en er placeret i et rum, og webkameraet med ledningen gennem den lukkede dør, er placeret i et andet rum. I en browser på smartphonen åbnes onlinetonegenerator.com. Smartphonen placeres ca. 2 cm fra mikrofonen.  
			
			På PC'en er \texttt{optagefrekvenssignal01.vi} åbnet i LabVIEW. VI'et er opbygget af \textit{Graphics and sound} komponenter. Komponenterne, forbindelserne og de angivede værdier er vist i blokdiagrammet, i figur \ref{fig:bdoptage}.   \\ 
			 
		\subsubsection{Udførsel}
			\begin{enumerate}
				\item I VI'et \texttt{optagefrekvenssignal01.vi}, trykkes på \textit{Run}.  
				\item Der generes en lyd i onlinetonegenerator.com til mikrofonen. 
				\item På frontpanelet i VI'et observeres amplitudeudsving på grafen.  
			\end{enumerate}
		
		\subsubsection{Resultater}
		Der blev observeret et amplitudeudsving på grafen. 
		\subsubsection{Diskussion} 
		-
		\subsubsection{Konklusion}
		Mikrofonen har opfanget en lyd, og hypotesen accepteres. 
		
		\subsubsection{Aktion}
		Denne mikrofon er ikke hensigtsmæssig til brug i den videre test, da den er for stor til at kunne påsættes resonatoren. Næste skridt er at finde en mindre mikrofon med USB-stik. 
		
\subsection{E06-M3}
\label{subsec:E06}
		\subsubsection{Testhypotese}
		Der kan gennem \elektret{} opfanges lydsignaler i frekvensbåndet i intervallet 20 Hz - 10 kHz.
	
		
		\subsubsection{Produktspecifikationer}
	
			\textit{Hardware:}\\
			\elektret\\
			\pinstre\\
			\mikrofonkabel\\
			\krympeflex\\
			\daq\\
			\daqusb\\			
			\PC
	
			\textit{Software:}\\
			\labview\\
			\daqsoft\\
			\onlineg\\
	
		\subsubsection{Opstilling og opsætning}
		På kablerne er pins fastloddet, som igen er loddet fast til mikrofonprintet. Det røde kabel forbinder mikrofonprintets "AUD"{} til "+AI0"{}-indgangen på DAQ'en. Det sorte kabel forbinder mikrofonprintets "GND"{} til "GND"{}-indgangen på DAQ'en. Det blå kabel forbinder mikrofonprintets "VCC"{} til "+5V"{} på DAQ'en. DAQ'en er med USB-kablet tilkoblet PC'en, hvorpå \texttt{optagefrekvenssignal04.vi} er åbnet i LabVIEW. Blockdiagrammet for VI'et er vist i figur \ref{fig:op04}. I en browser på PC´en er onlinetonegenerator.com åbnet.    
		
		\begin{figure}[htb]
			\centering
			\includegraphics[width=3in]{optagefrekvenssignal04}
			\caption{Blokdiagram for VI'et \texttt{optagefrekvenssignal04.vi}.}
			\label{fig:op04}
		\end{figure}
		
		Testopstillingen er vist i figur \ref{fig:webop}.\\
		
		\begin{figure}[htb]
			\centering
			\includegraphics[width=3in]{webcamop.jpg}
			\caption{Testopstilling for \ref{subsec:E06} E06-M3}
			\label{fig:webop}
		\end{figure}
	
		\subsubsection{Udførsel}
			\begin{enumerate}
				\item I VI'et \texttt{optagefrekvenssignal04.vi}, trykkes på \textit{Run}.  
				\item Der generes et signal med følgende frekvens: 100 Hz, 500 Hz, 1 kHz , 5 kHz og 10 KHz fra onlinetonegenerator.com
				\item På frontpanelet i VI'et observeres amplitudeudsving på grafen.  
				\item Resultatet aflæses i \textit{Maks. frekvens} på VI'ets frontpanel.
			\end{enumerate}
			
			Testen fortages to gange for hver frekvens.  
		
		\subsubsection{Resultater}
		Resultaterne vises i tabel \ref{E06 resultater}
		\begin{table}[htb]
				\centering
				\caption{Tabel over resultater af \ref{subsec:E06} E06-M3.} 
				\label{E06 resultater}
				\begin{tabular}{ccc}
					\multicolumn{1}{c|}{\textbf{Udsendt frekvens {[}Hz{]}}} & 	
					\multicolumn{1}{c|}{\textbf{1. resultat}} & \textbf{2. resultat} \\ \hline
					\multicolumn{1}{c|}{100}& 
					\multicolumn{1}{c|}{20}&40\\
					\multicolumn{1}{c|}{500}& 
					\multicolumn{1}{c|}{500}&500\\
					\multicolumn{1}{c|}{1000}& 
					\multicolumn{1}{c|}{1000}&1000\\
					\multicolumn{1}{c|}{5000}& 
					\multicolumn{1}{c|}{3000}&3000\\
					\multicolumn{1}{c|}{10000}& 
					\multicolumn{1}{c|}{20}&50\\
                   
				\end{tabular}
			\end{table}
		
		\subsubsection{Diskussion} 
		Skyldes de unøjagtige resultater ved 100 Hz, 500 Hz og 10000 Hz, at PC højtalerens frekvensbånd ikke er bredt nok til at udsende disse signaler?
		 \subsubsection{Konklusion}
	Ved et signal med en frekvens på hhv. 100 Hz, 5 kHz og 10 kHz er resultaterne ikke tilnærmelsesvis de udsendte frekvenser. 
	Det kan ikke konkluderes, hvorvidt det skyldes mikrofonen eller PC'ens højtalere. Hypotesen må derfor forkastes.  
		\subsubsection{Aktion}
		Det undersøges, hvilke årsager der er skyld i, at der opnås unøjagtige resultater.  
		
		
\subsection{E07-M4}
\label{subsec:E07}
		\subsubsection{Testhypotese}
		Der kan gennem \elektretto{} opfanges lydsignaler i frekvensbåndet i intervallet 20 Hz - 10 kHz med 2.5 V strømforsyning.
		
		\subsubsection{Produktspecifikationer}
	
			\textit{Hardware:}\\
			\elektretto\\
			\pinstre\\
			\mikrofonkabel\\
			\krympeflex\\
			\daq\\
			\daqusb\\			
			\PC
	
			\textit{Software:}\\
			\labview\\
			\daqsoft\\
			\onlineg\\

	
		\subsubsection{Opstilling og opsætning}
		På kablerne er pins fastloddet, som igen er loddet fast til mikrofonprintet. Det røde kabel forbinder mikrofonprintets "AUD"{} til "+AI0"{}-indgangen på DAQ'en. Det sorte kabel forbinder mikrofonprintets "GND"{} til "GND"{}-indgangen på DAQ'en. Det blå kabel forbinder mikrofonprintets "VCC"{} til "+2.5V"{} på DAQ'en. DAQ'en er med USB-kablet tilkoblet PC'en, hvorpå \texttt{optagefrekvenssignal04.vi} er åbnet i LabVIEW. Blockdiagrammet for VI'et er vist i figur \ref{fig:op04}. I en browser på PC´en er onlinetonegenerator.com åbnet.      
	
		\subsubsection{Udførsel}
			\begin{enumerate}
				\item I VI'et \texttt{optagefrekvenssignal04.vi}, trykkes på \textit{Run}.  
				\item Der generes et signal med følgende frekvens: 20 Hz, 50 Hz, 100 Hz, 1 kHz og 10 KHz fra onlinetonegenrator.com.
				\item På frontpanelet i VI'et observeres efter amplitudeudsving på grafen.   
				\item Resultatet aflæses i \textit{Maks. frekvens} på VI´ets frontpanel.
			\end{enumerate}
			
			Testen fortages to gange for hver frekvens.  
		
		\subsubsection{Resultater}
		Resultaterne vises i tabel \ref{elektrettoresultater}. 
		\begin{table}[htb]
				\centering
				\caption{Tabel over resultater af \ref{subsec:E07} E07-M4.} 
				\label{elektrettoresultater}
				\begin{tabular}{ccc}
					\multicolumn{1}{c|}{\textbf{Udsendt frekvens {[}Hz{]}}} & 	
					\multicolumn{1}{c|}{\textbf{1. resultat}} & \textbf{2. resultat} \\ \hline
					\multicolumn{1}{c|}{20}& 
					\multicolumn{1}{c|}{96}&96\\
					\multicolumn{1}{c|}{50}& 
					\multicolumn{1}{c|}{22}&15\\
					\multicolumn{1}{c|}{100}& 
					\multicolumn{1}{c|}{95}&95\\
					\multicolumn{1}{c|}{1000}& 
					\multicolumn{1}{c|}{1000}&1000\\
					\multicolumn{1}{c|}{10000}& 
					\multicolumn{1}{c|}{4}&15\\
                   
				\end{tabular}
			\end{table}
		
		\subsubsection{Diskussion} 
		I beskrivelsen af mikrofonens specifikationer anbefales det at bruge 2.5 V, som er den mest støjsvage forsyning.€€€ https://www.adafruit.com/products/1063 Skyldes de afvigende resultater, at der kun bruges 2.5 V strømforsyning til at opsamle lyden?
		\subsubsection{Konklusion}
	Ved et signal med en frekvens på hhv. 20 Hz, 50 Hz, 100 Hz, og 10 kHz er resultaterne ikke tilnærmelsesvis de udsendte frekvenser. 
	Det kan ikke konkluderes, hvorvidt det skyldes mikrofonen, et begrænset frekvensbånd på PC'ens højtaler eller en for lav strømforsyning til mikrofonen. Hypotesen må derfor forkastes. 
		\subsubsection{Aktion}
		Der foretages en ny enhedstest af \elektretto{} med en strømforsyning på 5 V. 
		
\subsection{E08-M4}
\label{subsec:E08}
		\subsubsection{Testhypotese}
		Der kan gennem \elektretto{} opfanges lydsignaler i frekvensbåndet i intervallet 20 Hz - 10 kHz med 5 V strømforsyning.
		
		\subsubsection{Produktspecifikationer}
	
			\textit{Hardware:}\\
			\elektretto\\
			\pinstre\\
			\mikrofonkabel\\
			\krympeflex\\
			\daq\\
			\daqusb\\			
			\PC
	
			\textit{Software:}\\
			\labview\\
			\daqsoft\\
			\onlineg\\

	
		\subsubsection{Opstilling og opsætning}
		På kablerne er pins fastloddet, som igen er loddet fast til mikrofonprintet. Det røde kabel forbinder mikrofonprintets "AUD"{} til "+AI0"{}-indgangen på DAQ'en. Det sorte kabel forbinder mikrofonprintets "GND"{} til "GND"{}-indgangen på DAQ'en. Det blå kabel forbinder mikrofonprintets "VCC"{} til "5V"{} på DAQ'en. DAQ'en er med USB-kablet tilkoblet PC'en, hvorpå \texttt{optagefrekvenssignal04.vi} er åbnet i LabVIEW. Blockdiagrammet for VI'et er vist i figur \ref{fig:op04}. I en browser på PC´en er onlinetonegenerator.com åbnet.      
	
		\subsubsection{Udførsel}
			\begin{enumerate}
				\item I VI'et \texttt{optagefrekvenssignal04.vi}, trykkes på \textit{Run}.  
				\item Der generes et signal med følgende frekvens: 20 Hz, 50 Hz, 100 Hz, 1 kHz og 10 KHz fra onlinetonegenrator.com.
				\item På frontpanelet i VI'et observeres efter amplitudeudsving på grafen.   
				\item Resultatet aflæses i \textit{Maks. frekvens} på VI´ets frontpanel.
			\end{enumerate}
			
			Testen fortages to gange for hver frekvens.  
		
		\subsubsection{Resultater}
		Resultaterne vises i tabel \ref{E08resultater}. 
		\begin{table}[htb]
				\centering
				\caption{Tabel over resultater af \ref{subsec:E08} E08-M4.} 
				\label{E08resultater}
				\begin{tabular}{ccc}
					\multicolumn{1}{c|}{\textbf{Udsendt frekvens {[}Hz{]}}} & 	
					\multicolumn{1}{c|}{\textbf{1. resultat}} & \textbf{2. resultat} \\ \hline
					\multicolumn{1}{c|}{20}& 
					\multicolumn{1}{c|}{20}&40\\
					\multicolumn{1}{c|}{50}& 
					\multicolumn{1}{c|}{80}&100\\
					\multicolumn{1}{c|}{100}& 
					\multicolumn{1}{c|}{20}&700\\
					\multicolumn{1}{c|}{1000}& 
					\multicolumn{1}{c|}{1000}&1000\\
					\multicolumn{1}{c|}{10000}& 
					\multicolumn{1}{c|}{40}&20\\ 
				\end{tabular}
			\end{table}
		
		\subsubsection{Diskussion} 
		Strømforsyningen på 5 V gjorde ikke afvigelserne i resultaterne mindre. Kan det være frekvensbåndet i PC´ens højtalere som ikke er bredt nok til at afspille det ønskede frekvensbånd? 
		\subsubsection{Konklusion}
	Ved et signal med en frekvens på hhv. 20 Hz, 50 Hz, 100 Hz, og 10 kHz er resultaterne ikke tilnærmelsesvis de udsendte frekvenser. 
	Det kan ikke konkluderes, hvorvidt det skyldes mikrofonen eller et begrænset frekvensbånd på PC'ens højtaler. Hypotesen må defor forkastes. 
		\subsubsection{Aktion}
		Frekvensbåndet mellem 100 - 1000 Hz ønskes testet, da det muligvis kan være tilstrækkeligt til det videre testforløb.  
		
\subsection{E09-M4}
		\label{subsec:E09}
		\subsubsection{Testhypotese}
		Der kan gennem \elektretto{} opfanges lydsignaler i frekvensbåndet i intervallet 100 Hz - 1 kHz med 5 V strømforsyning.
		
		\subsubsection{Produktspecifikationer}
	
			\textit{Hardware:}\\
			\elektretto\\
			\pinstre\\
			\mikrofonkabel\\
			\krympeflex\\
			\daq\\
			\daqusb\\			
			\PC
	
			\textit{Software:}\\
			\labview\\
			\daqsoft\\
			\onlineg\\

	
		\subsubsection{Opstilling og opsætning}
		På kablerne er pins fastloddet, som igen er loddet fast til mikrofonprintet. Det røde kabel forbinder mikrofonprintets "AUD"{} til "+AI0"{}-indgangen på DAQ'en. Det sorte kabel forbinder mikrofonprintets "GND"{} til "GND"{}-indgangen på DAQ'en. Det blå kabel forbinder mikrofonprintets "VCC"{} til "5V"{} på DAQ'en. DAQ'en er med USB-kablet tilkoblet PC'en, hvorpå \texttt{optagefrekvenssignal04.vi} er åbnet i LabVIEW. Blockdiagrammet for VI'et er vist i figur \ref{fig:op04}. I en browser på PC´en er onlinetonegenerator.com åbnet.       
		
	
		\subsubsection{Udførsel}
			\begin{enumerate}
				\item I VI'et \texttt{optagefrekvenssignal04.vi}, trykkes på \textit{Run}.  
				\item Der generes et signal med følgende frekvens: 100 Hz, 150 Hz, 200 Hz, 300, 400 Hz, 500 Hz, 600 Hz, 700 Hz, 800 Hz, 900 Hz og 1 kHz fra onlinetonegenrator.com.
				\item På frontpanelet i VI'et observeres efter amplitudeudsving på grafen.   
				\item Resultatet aflæses i \textit{Maks. frekvens} på VI´ets frontpanel.
			\end{enumerate}
			
			Testen fortages to gange for hver frekvens.  
		
		\subsubsection{Resultater}
		Resultaterne vises i tabel \ref{E09resultater}. 
		\begin{table}[]
				\centering
				\caption{Tabel over resultater af \ref{subsec:E09} E09-M4.} 
				\label{E09resultater}
				\begin{tabular}{ccc}
					\multicolumn{1}{c|}{\textbf{Udsendt frekvens {[}Hz{]}}} & 	
					\multicolumn{1}{c|}{\textbf{1. resultat}} & \textbf{2. resultat} \\ \hline
					\multicolumn{1}{c|}{100}& 
					\multicolumn{1}{c|}{20}&80\\
					\multicolumn{1}{c|}{150}& 
					\multicolumn{1}{c|}{150}&150\\
					\multicolumn{1}{c|}{200}& 
					\multicolumn{1}{c|}{200}&200\\
					\multicolumn{1}{c|}{300}& 
					\multicolumn{1}{c|}{300}&300\\
					\multicolumn{1}{c|}{400}& 
					\multicolumn{1}{c|}{400}&400\\
					\multicolumn{1}{c|}{500}& 
					\multicolumn{1}{c|}{500}&500\\
					\multicolumn{1}{c|}{600}& 
					\multicolumn{1}{c|}{600}&600\\
					\multicolumn{1}{c|}{700}& 
					\multicolumn{1}{c|}{700}&700\\
					\multicolumn{1}{c|}{800}& 
					\multicolumn{1}{c|}{800}&800\\
					\multicolumn{1}{c|}{900}& 
					\multicolumn{1}{c|}{900}&900\\
					\multicolumn{1}{c|}{1000}& 
					\multicolumn{1}{c|}{1000}&1000\\
                   
				\end{tabular}
			\end{table}
		
		\subsubsection{Diskussion} 
		Resultaterne af det udsendte frekvenssignal på 100 Hz er afvigende. Det kan muligvis skyldes et begrænset frekvensbånd i PC-højtaleren. 
		\subsubsection{Konklusion}
	Mikrofonen kan ikke opfange et frekvenssignal på 100 Hz. Hypotesen må derfor forkastes.  
		\subsubsection{Aktion}
		Da mikrofonen kan opfange et frekvensbånd fra 150 Hz - 1 kHz med en nøjagtighed på 100 pct., besluttes det at anvende mikrofonen i det videre testforløb. 
		
\subsection{E10-H3}
\label{subsec:E10}
		\subsubsection{Testhypotese}
		Højtaleren \usbhøj kan generere signaler med frekvenser i intervallet 20 Hz - 1500 Hz.
		
		\subsubsection{Produktspecifikationer}
	
			\textit{Hardware:}\\
			\elektretto\\
			\pinstre\\
			\mikrofonkabel\\
			\krympeflex\\
			\daq\\
			\daqusb\\	
			\usbhøj\\		
			\PC
	
			\textit{Software:}\\
			\labview\\
			\daqsoft\\
			\onlineg\\
	
		\subsubsection{Opstilling og opsætning}
		På kablerne er pins fastloddet, som igen er loddet fast til mikrofonprintet. Det røde kabel forbinder mikrofonprintets "AUD"{} til "+AI0"{}-indgangen på DAQ'en. Det sorte kabel forbinder mikrofonprintets "GND"{} til "GND"{}-indgangen på DAQ'en. Det blå kabel forbinder mikrofonprintets "VCC"{} til "5V"{} på DAQ'en. DAQ'en er med USB-kablet tilkoblet PC'en, hvorpå \texttt{optagefrekvenssignal04.vi} er åbnet i LabVIEW. Blockdiagrammet for VI'et er vist i figur \ref{fig:op04}. Højtaleren er tilsluttet PC'ens USB-port og minijack-stik, og er placeret ca. 2 cm. fra mikrofonen. I en browser på PC´en er onlinetonegenerator.com åbnet.       
	
		\subsubsection{Udførsel}
			\begin{enumerate} 
				\item Der generes et signal med henholdvis følgende frekvens: 20 Hz, 75 Hz, 80 Hz, 90 Hz, 100 Hz, 150 Hz, 200 Hz, 300, 400 Hz, 500 Hz, 600 Hz, 700 Hz, 800 Hz, 900 Hz, 1000 Hz, 1100 Hz, 1200 Hz, 1300 Hz, 1400 Hz og 1500 Hz fra onlinetonegenrator.com.
				\item I VI'et \texttt{optagefrekvenssignal04.vi}, trykkes på \textit{Run} efter hvert genereret frekvens.  
				\item På frontpanelet i VI'et aflæses frekvensen i \textit{Maks. frekvens} for hvert genereret frekvens. 
			\end{enumerate}
			
			Testen fortages to gange for hver frekvens.  
		
		\subsubsection{Resultater}
		Resultaterne vises i tabel \ref{usbhojresul}. 
		\begin{table}[]
				\centering
				\caption{Tabel over resultater af \ref{subsec:E10} E10-H3.}
				\label{usbhojresul}
				\begin{tabular}{ccc}
					\multicolumn{1}{c|}{\textbf{Udsendt frekvens {[}Hz{]}}} & 	
					\multicolumn{1}{c|}{\textbf{1. resultat}} & \textbf{2. resultat} \\ \hline
					\multicolumn{1}{c|}{20}& 
					\multicolumn{1}{c|}{20}&50\\
					\multicolumn{1}{c|}{75}& 
					\multicolumn{1}{c|}{70}&80\\
					\multicolumn{1}{c|}{80}& 
					\multicolumn{1}{c|}{90}&80\\
					\multicolumn{1}{c|}{90}& 
					\multicolumn{1}{c|}{90}&90\\
					\multicolumn{1}{c|}{100}& 
					\multicolumn{1}{c|}{100}&100\\
					\multicolumn{1}{c|}{150}& 
					\multicolumn{1}{c|}{150}&150\\
					\multicolumn{1}{c|}{200}& 
					\multicolumn{1}{c|}{200}&200\\
					\multicolumn{1}{c|}{300}& 
					\multicolumn{1}{c|}{300}&300\\
					\multicolumn{1}{c|}{400}& 
					\multicolumn{1}{c|}{400}&400\\
					\multicolumn{1}{c|}{500}& 
					\multicolumn{1}{c|}{500}&500\\
					\multicolumn{1}{c|}{600}& 
					\multicolumn{1}{c|}{600}&600\\
					\multicolumn{1}{c|}{700}& 
					\multicolumn{1}{c|}{700}&700\\
					\multicolumn{1}{c|}{800}& 
					\multicolumn{1}{c|}{800}&800\\
					\multicolumn{1}{c|}{900}& 
					\multicolumn{1}{c|}{900}&900\\
					\multicolumn{1}{c|}{1000}& 
					\multicolumn{1}{c|}{1000}&1000\\
					\multicolumn{1}{c|}{1100}& 
					\multicolumn{1}{c|}{1100}&1100\\
					\multicolumn{1}{c|}{1200}& 
					\multicolumn{1}{c|}{1200}&1200\\
					\multicolumn{1}{c|}{1300}& 
					\multicolumn{1}{c|}{1200}&1200\\
					\multicolumn{1}{c|}{1400}& 
					\multicolumn{1}{c|}{1100}&1100\\
					\multicolumn{1}{c|}{1500}& 
					\multicolumn{1}{c|}{1000}&1000\\
                   \end{tabular}
			\end{table}
		
		\subsubsection{Diskussion} 
		Frekvenser op til 90 Hz og frekvenser fra 1300 Hz til 1500 Hz afviger fra de udsendte frekvenser. Kan det muligvis skyldes mikrofonen ikke kan optage disse signaler, eller skyldes det højtaleren ikke kan generere frekvenserne? 
		\subsubsection{Konklusion}
	Hypotesen forkastes da højtaleren \usbhøj ikke kan generere signaler med frekvenser i intervallet 20 Hz - 1500 Hz.
	    \subsubsection{Aktion}
	    	Da højtaleren kan levere frekvenser fra 90 - 1200 Hz, besluttes det at anvende denne højtaler til det videre testforløb, hvor resonansfrekvens i et tomt kammer måles. 
	    
\subsection{E11-VI02G}
\subsubsection{Testhypotese} 
Der kan igennem LINX LabVIEW Makerhub oprettes en forbindelse til Arduino Mega 2560 r3. 
\subsubsection{Produktspecifikationer}

\textit{Hardware:}\\
		\abshøj\\
		\hojtalerkabel\\
		\pins\\
		\arduino\\
		\PC\\
		\usbkabel\\
	
		\textit{Software:}\\
		\labview\\
		\visa\\
		\vi\\
		\ardsw\


\subsubsection{Opstilling og opsætning}

		Højtaleren er loddet til højtalerkablets ene ende, og pin headerne er loddet til kablets anden ende. 
		Pin headerne er isat Arduino'en i pin 46 (PL3(OC5A)), som et er digitalt PWM output, og til ground (GND). 
		Arduino'en er med et USB kabel koblet til PC'en. Testopstillingen er vist i figur \ref{fig:etha1}.\\ 
	  
			\begin{figure}[htb]
			\centering
				\includegraphics[width=3in]{haArduino}
				\caption{Arduino med tilkoblet højtaler}	
				\label{fig:etha1}
			\end{figure}

I LabVIEW er det opbygget et VI \textit{genererfrekvenssignal.VI} af LINX MakerHub komponenterne \textit{Initialize.VI}, \textit{Write square signal.VI} og \textit{Close.VI}. I blokdiagrammet angives \textit{serial port},\textit{Channel} og \textit{Duration}.  	Blokdiagrammet ses i figur \ref{fig:gf02} 

\begin{figure}[htb]
			\centering
				\includegraphics[width=3in]{gf02}
				\caption{Blokdiagram for \textit{genererfrekvnessignal02.VI}}	
				\label{fig:gf02}
			\end{figure}
				
\subsubsection{Udførsel}

\begin{enumerate}
				\item I \textit{Enter Frequency} på frontpanelet i \texttt{genererfrekvenssignal02.vi}, indtastes en selvvalgt frekvens. 
				\item Der trykkes på \textit{Run}. 
				\item Der lyttes efter lydsignal fra højtaleren.  
			\end{enumerate}

\subsubsection{Resultat}

Der observeres en lyd ud fra højtaleren

\subsubsection{Diskusion}
-
\subsubsection{Konklusion}

Det konkluderes at det har været muligt at skabe en forbindelse  
igennem LINX LabVIEW Makerhub til Arduino Mega 2560 r3.
\subsubsection{Aktion}
-



\subsection{E12-VI01\underline{O}}
\subsubsection{Testhypotese}
Det er muligt at opfange en lyd ved brug af LabVIEWs \textit{Graphics and sound} VI´er. 
\subsubsection{Produktspecifikationer}

		\textit{Hardware:}\\
		\PC\\
	
		\textit{Software:}\\
		\labview\\

\subsubsection{Opstilling og opsætning}

I LabVIEW er det opbygget et VI \textit{optagefrekvenssignal01.VI} af \textit{Graphics and sound} komponenter. Komponenterne, forbindelserne og de angivede værdier er vist i figur \ref{fig:of01}

\begin{figure}[htb]
			\centering
				\includegraphics[width=3in]{of01}
				\caption{Blokdiagram for \textit{optagefrekvenssignal01.VI}}	
				\label{fig:of01}
			\end{figure}

\subsubsection{Udførsel}
\begin{enumerate} 
				\item Der trykkes på \textit{Run}. 
				\item Det snakkes ind i PC mikrofonen.
				\item Der observeres efter udsving af amplituden i grafen på frontpanelet.  
			\end{enumerate}

\subsubsection{Resultat}
I grafen på frontpanelet ses der udsving på amplituden når der snakkes til PC mikrofonen. 
\subsubsection{Diskusion}
-
\subsubsection{Konklusion}
Hypotesen accepteres. Det har været muligt at opfange en lyd ved brug af LabVIEWs \textit{Graphics and sound} VI´er. 
\subsubsection{Aktion}
-

\subsection{E13-VI02\underline{O}}
\subsubsection{Testhypotese}
Det er muligt at lave en FFT på et lydsignal og detektere maks. frekvensen.  

\subsubsection{Produktspecifikationer}
\textit{Hardware:}\\
		\PC\\
	
\textit{Software:}\\
		\labview\\
		\onlineg\\

\subsubsection{Opstilling og opsætning}

I LabVIEW er det opbygget et VI \textit{optagefrekvenssignal02.VI} af \textit{Graphics and sound} og Textit{Waveforms Measurements}-komponenter. Komponenterne, forbindelserne og de angivede værdier er vist i figur \ref{fig:VIo2}

\begin{figure}[htb]
			\centering
				\includegraphics[width=3in]{VI02o}
				\caption{Blokdiagram for \textit{optagefrekvenssignal02.VI}}	
				\label{fig:VIo2}
			\end{figure}
			
I onlinetonegeneratoren åbnes DMTF generatoren som er vist i figur \ref{fig:DTMF}  
			
\subsubsection{Udførsel}

\begin{enumerate}
			\item En DTMF tone 1 vælges i tonegeneratoren. Den højste frekvens i DTMF tonen noteres.
			\item  Tonen afspilles og der trykkes på run i VI´et som nu optager tonen.
			\item Maksfrekvensen aflæses på frontpanlet i \textit{Maks frekvens} og sammenholdes med den tidligere noteret maksfrekvens. Disse skulle gerne stemme overens. 
\end{enumerate}

			Denne udførsel gentages med tone 5, 9 samt D fra DTMF generatoren.

\subsubsection{Resultat}

\begin{table}[h]
				\centering
				\caption{Tabel over resultater}
				\label{FFTresultater}
				\begin{tabular}{lll}
					\multicolumn{1}{l|}{\textbf{DTMF tone}} & 	
					\multicolumn{1}{l|}{\textbf{Højeste frekvens}} & \textbf{Resultat} \\ \hline
					\multicolumn{1}{c|}{1}& 
					\multicolumn{1}{c|}{1209}&1209\\
					\multicolumn{1}{c|}{5}& 
					\multicolumn{1}{c|}{1336}&1336\\
					\multicolumn{1}{c|}{9}& 
					\multicolumn{1}{c|}{1477}&1477\\
					\multicolumn{1}{c|}{D}& 
					\multicolumn{1}{c|}{1633}&1633\\
					
	\end{tabular}
			\end{table}
\subsubsection{Diskusion}

-

\subsubsection{Konklusion}
Resultaterne er lig med de afsendte frekvenser. Testhypotesen er godkendt. 
\subsubsection{Aktion}
-

\subsection{E14-VI04\underline{O}}	
\subsubsection{Testhypotese}
Det er muligt at lave en FFT på et lydsignal og detektere maks. frekvensen med lydsignal fra DAQ.

		\subsubsection{Produktspecifikationer}
	
		\textit{Hardware:}\\
		\PC\\
		\daq\\
	
		\textit{Software:}\\
		\labview\\
		\visa\\
		\vi\\
		\daqsoft\\
		
	
		\subsubsection{Opstilling og opsætning}
		\textit{optagefrekvenssignal0.4.vi} er åbnet og onlinetonegeneratoren er åbnet i browseren på PC´en. Blokdiagrammet er vist i figur \ref{fig:vio5}. I blokdiagrammet i \texttt{optagefrekvenssignal0.4.vi} åbnes DAQ assistant modulet og i \textit{Samples to read} indtastes 800 og i \textit{Rate(Hz)} indtastes 8000. Dette vises i figur \ref{fig:daq}.   
		 I onlinetonegeneratoren åbnes DMTF generatoren som er vist i figur \ref{fig:DTMF}  
		
\begin{figure}[htb]
			\centering
			\includegraphics[width=3in]{DAQassistant}
			\caption{DAQ assistant modulet med korrekt indtastet værdier}
			\label{fig:daq}
		\end{figure}		
		
		\begin{figure}[htb]

			\centering
			\includegraphics[width=3in]{optagefrekvenssignal04}
			\caption{Blokdiagram for VI'et \texttt{optagefrekvenssignal0.4.vi}.}
			\label{fig:vio4}
		\end{figure}
		
		\begin{figure}[htb]

			\centering
			\includegraphics[width=3in]{DTMF}
			\caption{DTMF generatoren}
			\label{fig:DTMF}
		\end{figure}
	
		\subsubsection{Udførsel}
			\begin{enumerate}
			\item En DTMF tone 1 vælges i tonegeneratoren. Den højste frekvens i DTMF tonen noteres.
			\item  Tonen afspilles og der trykkes på run i VI´et som nu optager tonen.
			\item Maksfrekvensen aflæses på frontpanlet i \textit{Maks frekvens} og sammenholdes med den tidligere noteret maksfrekvens. Disse skulle gerne stemme overens. 
			\end{enumerate}
			Denne udførsel gentages med tone 5, 9 samt D fra DTMF generatoren.
	
			
		\subsubsection{Resultater}
		
			\begin{table}[htb]
				\centering
				\caption{Tabel over resultater}
				\label{SWhelmholtzresultater}
				\begin{tabular}{lll}
					\multicolumn{1}{l|}{\textbf{DTMF tone}} & 	
					\multicolumn{1}{l|}{\textbf{Højeste frekvens}} & \textbf{Resultat} \\ \hline
					\multicolumn{1}{c|}{1}& 
					\multicolumn{1}{c|}{1209}&1209,6\\
					\multicolumn{1}{c|}{5}& 
					\multicolumn{1}{c|}{1336}&1332,16\\
					\multicolumn{1}{c|}{9}& 
					\multicolumn{1}{c|}{1477}&1475,84\\
					\multicolumn{1}{c|}{D}& 
					\multicolumn{1}{c|}{1633}&1633,4\\
                   
				\end{tabular}
			\end{table}
	
		\subsubsection{Diskussion} 
		
		-
	
		\subsubsection{Konklusion}
		 
Resultaterne af testen ligger acceptabelt tæt på de afsendte frekvenser. Hypotesen er godkendt.
	
	    \subsubsection{Aktion}
	   Det besluttes at koden bruges i det videre udviklingsforløb.

\subsection{E15-VI05\underline{O}}
\subsubsection{Testhypotese}
Det er muligt at foretager n antal målinger i samme runtime og gemme disse i en .CSV fil hvor signalet er opsamlet med DAQ´en. 
\subsubsection{Produktspecifikationer}
\textit{Hardware:}\\
		\PC\\
		\daq\\
	
		\textit{Software:}\\
		\labview\\
		\visa\\
		\vi\\
		\daqsoft\\

\subsubsection{Opstilling og opsætning}
		\textit{optagefrekvenssignal0.5.vi} er åbnet. Blokdiagrammet er vist i figur \ref{fig:vio5}. I blokdiagrammet i \texttt{optagefrekvenssignal0.5.vi} åbnes DAQ assistant modulet og i \textit{Samples to read} indtastes 800 og i \textit{Rate(Hz)} indtastes 8000. Dette vises i figur \ref{fig:daq}.   
		
		\begin{figure}[htb]

			\centering
			\includegraphics[width=3in]{VI05o}
			\caption{Blokdiagram for VI'et \texttt{optagefrekvenssignal0.5.vi}.}
			\label{fig:vio5}
		\end{figure}

\subsubsection{Udførsel}
	\begin{enumerate}
			\item I blokdiagrammet indtastes 5 i \textit{Loop Count} og den ønskede sti til den gemte .CSV fil angiver i \textit{file path} i \textit{Write To Spreadsheet File.VI}´et	
			\item På frontpanelt indtasten 4 sekunder i \textit{Tid} 		
			\item Der trykkes på run i VI´et som nu optager lyd igennnem DAQ´en.
			\item VI´et optager omkringværende lyd
			\item På frontpanelt aflæses om der ligger 5 målinger i array´et
			\item I den førangivet sti kontrolleres om der er dannet en .CSV fil med de 5 målinger 
			\end{enumerate}
			
\subsubsection{Resultat}

På frontpanlet i VI´et ser der 5 målinger og der er ligger i array´et. 
I den valgt sti ligger der ligeledes en .CSV fil med de 5 målinger.
\subsubsection{Diskusion}
-
\subsubsection{Konklusion}
Da resultatet er fuldt tilfredsstillende, godkendes testhypotesen. 
\subsubsection{Aktion}
- 

\subsection{E16-VIHRVF\underline{O}}
\subsubsection{Testhypotese}
		 Helmholtzresonansformel i LabVIEW udregner korrekt volumen ud fra kendte givet værdier. 
		\subsubsection{Produktspecifikationer}
	
		\textit{Hardware:}\\
		\PC\\
		Texas Instruments TI-89
	
		\textit{Software:}\\
		\labview\\
	
		\subsubsection{Opstilling og opsætning}
		\textit{HRVF.vi} er åbnet. Blokdiagrammet for VI´et ses i figur \ref{fig:HHRF} 
		
		\begin{figure}[htb]
			\centering
			\includegraphics[width=3in]{HelmholtzformelLabVIEW}
			\caption{Blokdiagram for VI'et \texttt{HRVF}.}
			\label{fig:HHRF}
		\end{figure}
		
		På TI-89´eren indtastes formlen for udregningen af volumen ud fra helmholtzresonansen. Formel defineres således 
		
		\begin{equation}
		W = V\left(1-\left(\frac{f_{0}}{f_{b}}\right)^2\right)
		\end{equation}
		
		Indtastningen op TI-89´eren ses på figur \ref{fig:TI89},hvor værdierne er angivet således at $v=1671$, $f_0=150$ og $f_b=250$. Resultatet af indtastning giver volumet $W=1069.44$
		
		\begin{figure}[htb]
			\centering
			\includegraphics[width=3in]{TI-89}
			\caption{Indtastning af formel på TI-89 }
			\label{fig:TI89}
		\end{figure}
		
	
	
		\subsubsection{Udførsel}
			 
         \begin{enumerate}
         \item De overnævnte værdier indtastes i blokdiagrammet på de korrekte pladser.
         \item Der trykkes på \textit{Run} 
         \item Resultat af udregningen aflæses på frontpanelet i feltet \textit{Resultat}. 
         \item Resultatet fra TI-89 og VI´et sammenlignes.  
         \end{enumerate}
         
        \subsubsection{Resultat}
        
        Volumet, regnet på TI-89´erne, angav at $W=1069.44$.
        Volumet, regnet i VI´et, angav at $W=1069.44$.

		\subsubsection{Diskussion} 
	-
	
	
		\subsubsection{Konklusion}
		 
Da resultatet af udregningen er ens for begge metoder, godkendes testenhypotesen.  
	
	   \subsubsection{Aktion}
Det besluttes at den kodede formel fra VI´et kan bruges i det videre udviklingsforløb i LabVIEW. 


\subsection{E17-VI07\underline{O}}
\subsubsection{Testhypotest}
Det er muligt at foretager n antal målinger i samme runtime og gemme disse i en .CSV fil hvor signalet er opsamlet med DAQ´en. 
\subsubsection{Produktspecifikationer}
\textit{Hardware:}\\
		\PC\\
		\daq\\
	
		\textit{Software:}\\
		\labview\\
		\visa\\
		\vi\\
		\daqsoft\\
\subsubsection{Opstilling og opsætning}
\textit{optagefrekvenssignal0.7.vi} er åbnet. Blokdiagrammet er vist i figur \ref{fig:vio5}. I blokdiagrammet i \texttt{optagefrekvenssignal0.7.vi} åbnes DAQ assistant modulet og i \textit{Samples to read} indtastes 800 og i \textit{Rate(Hz)} indtastes 8000. Dette vises i figur \ref{fig:vio5}.   
		
		\begin{figure}[htb]

			\centering
			\includegraphics[width=3in]{VI07o}
			\caption{Blokdiagram for VI'et \texttt{optagefrekvenssignal0.7.vi}.}
			\label{fig:vio5}
		\end{figure}

\subsubsection{Udførsel}
\begin{enumerate}
			\item I blokdiagrammet indtastes 5 i \textit{Loop Count} og den ønskede sti til den gemte .CSV fil angiver i \textit{file path} i \textit{Write To Spreadsheet File.VI}´et	
			\item På frontpanelt indtasten 4 sekunder i \textit{Tid} 		
			\item Der trykkes på run i VI´et som nu optager lyd igennnem DAQ´en.
			\item VI´et optager omkringværende lyd
			\item På frontpanelt aflæses om der ligger 5 målinger i array´et
			\item I den førangivet sti kontrolleres om der er dannet en .CSV fil med de 5 målinger 
			\end{enumerate}

\subsubsection{Resultat}
På frontpanlet i VI´et ser der 5 målinger og der er ligger i array´et. 
I den valgt sti ligger der ligeledes en .CSV fil med de 5 målinger.
\subsubsection{Diskusion}
-
\subsubsection{Konklusion}
Da resultatet er fuldt tilfredsstillende, godkendes testhypotesen. 
 \subsubsection{Aktion}
Det besluttes at dette \textit{optagefrekvenssignal0.7.vi} bruges i det videre udvilingsforløb

\section{Integrationstest}

Dette afsnit..
Ved samtlige test med lydgenererende funktion er der anvendt maksimalt lydtryk på den lydgivende enhed.\\
			
\begin{table}[htb]
				\centering
				\caption{Tabeloversigt over udførte integrationstests} 
				\label{integrationstest}
				\begin{tabular}{cll}
					\multicolumn{1}{c|}{\textbf{Test-ID}} &
					\multicolumn{1}{l|}{\textbf{Indgående enheder}} & 	  \textbf{Testformål} \\ \hline
					\multicolumn{1}{c|}{I01} &
					\multicolumn{1}{l|}{H1/M1/VI02G/VI01\underline{O}}&Generere og opfange lydsignal\\	
					
					\multicolumn{1}{c|}{I02} &
					\multicolumn{1}{l|}{M1/VI02G}&Optage lydsignal gennem M1\\	
					
					\multicolumn{1}{c|}{I03} &
					\multicolumn{1}{l|}{H1/M1/VI02G/VI02\underline{O}}&Sammenligning af frekvenssignaler\\
					
					\multicolumn{1}{c|}{I04} &
					\multicolumn{1}{l|}{H2/M1/VI02G/VI02\underline{O}}&Dæmpning af overtoner\\
					
					\multicolumn{1}{c|}{I05} &
					\multicolumn{1}{l|}{H2/M1/VI02G/VI02\underline{O}}&Test af resonansfrekvens ved højfrekvenser\\	
					
					\multicolumn{1}{c|}{I06} &
					\multicolumn{1}{l|}{H3/M4/VI05\underline{O}}&Resonansfrekvens i tom resonator\\
					
					\multicolumn{1}{c|}{I07} &
					\multicolumn{1}{l|}{H3/M4/VI05\underline{O}}&Resonansfrekvens i tom resonator med højtalerholder\\	
					
					\multicolumn{1}{c|}{I08} &
					\multicolumn{1}{l|}{H3/M4/VI06\underline{O}}&Volumenmåling af ballon\\
					
					\multicolumn{1}{c|}{I09} &
					\multicolumn{1}{l|}{H3/M4/VI08\underline{O}}&Volumenmåling af balloner med og uden højtalerholder\\	
						
				\end{tabular}
			\end{table}
	



	\subsection{I01-H1/M1/VI02G/VI01\underline{O}} 
	\label{bordtest1}
		\subsubsection{Testhypotese}
		Kan der genereres et kendt frekvenssignal som udsendes gennem højtaleren via arudino´en, og derefter opfanges af PC mikrofonen og hvor (maks) frekvensen angives i LabVIEW. 
		\subsubsection{Produktspecifikationer}
		
		\textit{Hardware:}\\
		\abshøj\\
		\hojtalerkabel\\
		\pins\\
		\arduino\\
		\PC\\
		\usbkabel
	
		\textit{Software:}\\
		\labview\\
		\visa\\
		\vi\\
		\ardsw\
		
		\subsubsection{Opstilling og opsætning}
		\textit{1. delopstilling}:\\
		Højtaleren er loddet til højtalerkablets ene ende, og pin headerne er loddet til kablets anden ende. 
		Pin headerne er isat Arduino'en i pin 46 (PL3(OC5A)), som et er digitalt PWM output, og til ground (GND). 
		Arduino'en er med et USB kabel koblet til PC'en. 		
		På PC'en er VI'et \texttt{genererfrekvenssignal02.vi} åbnet i LabVIEW.     		Blokdiagrammet er vist i figur \ref{fig:1gf02}.\\ 
 
\begin{figure}[htb]
			\centering
				\includegraphics[width=3in]{gf02}
				\caption{Blokdiagrammet for \textit{genererfrekvenssignal02.vi}}	
				\label{fig:1gf02}
			\end{figure} 
 
		\textit{2. delopstilling}:\\
		Mikrofonen er sat i PC'ens minijack-stik. På PC'en er VI'et \texttt{optagefrekvenssignal0.1.vi} åbnet i LabVIEW. Blokdiagrammet er vist i figur \ref{fig:1of01}.\\
		
		\begin{figure}[htb]
			\centering
				\includegraphics[width=3in]{of01}
				\caption{Blokdiagram for \textit{optagefrekvenssignal01.VI}}	
				\label{fig:1of01}
			\end{figure}
		
		\subsubsection{Udførsel}
			\begin{enumerate}
				\item Højtaleren holdes manuelt således membranen står i lodret position. 
				\item Mikrofonen holdes manuelt, vendt mod højtaleren, i en afstand på 5 cm. 
				\item I VI'et \texttt{genererfrekvenssignal02.vi} indtastes den ønskede frekvens i den numeriske kontrol \textit{Enter Frequency}. 
				
						\item Koden eksekveres ved at trykke på \textit{Run}. 
					
				\item I VI'et \texttt{optagefrekvenssignal0.1.vi} trykkes på \textit{Run}. 
					
						\item Den maksimale optagede frekvens aflæses i \textit{Max Frequency}.   
			\end{enumerate}
			
			Punkt 1-4 gentages med frekvenser på: 100 Hz, 150 Hz, 200 Hz, 400 Hz, 500 Hz, 600 Hz og 700 Hz. 
			
			\subsubsection{Resultater}
			Den maksimale optagede frekvens var ikke tilnærmelsesvis frekvensen på den udsendte tone ved alle afprøvede frekvenser. 
			\subsubsection{Diskussion}
			Det ønskede resultat er frekvensen på den udsendte tone, hvilket ikke var tilfældet i denne test. Hvor er fejlen opstået; er der fejl i LabVIEW-kode eller hardware? 
			\subsubsection{Konklusion}
			Da resultatet ikke er tilfredsstillende forkastes hypotesen. 
			\subsubsection{Aktion}
			Det er nødvendigt at undersøge om fejlen opstår i vores hardware eller software. Onlinetonegenerator anvendes til at undersøge dette. 
			 
	\subsection{I02-M1/VI01} 
	
		\subsubsection{Testhypotese}
		Kan der genereres et frekvenssignal fra en online tonegenerator, som udsendes gennem PC'ens højtaler, og derefter opfanges af mikrofonen, hvor den højst målte frekvens til sidst angives i LabVIEW.  
		
		\subsubsection{Produktspecifikationer}
		
		\textit{Hardware:}\\
		\mikrofon\\
		\PC
	
		\textit{Software:}\\
		\labview\\
		\texttt{onlinetonegenerator.com}
		
		\subsubsection{Opstilling og opsætning}
		Mikrofonen er sat i PC'ens minijack-stik. På PC'en er VI'et \texttt{optagefrekvenssignal0.1.vi} åbnet i LabVIEW. Blokdiagrammet er vist i figur \ref{fig:2of01}.\\ 
		
		
		\begin{figure}[htb]
			\centering
				\includegraphics[width=3in]{of01}
				\caption{Blokdiagram for \textit{optagefrekvenssignal01.VI}}	
				\label{fig:2of01}
			\end{figure}
		
		I en internetbrowser er hjemmesiden \texttt{www.onlinetonegenerator.com} åbnet, og PC'ens højtalere er slået til og volumen er sat til maksimum.
		
		\subsubsection{Udførsel}
			\begin{enumerate}
				\item Mikrofonen holdes manuelt, vendt mod PC'ens højtaler, i en afstand på 5 cm. 
				\item I \texttt{onlinetonegenerator.com} genereres et signal med den ønskede frekvens. 
				\item I VI'et \texttt{optagefrekvenssignal0.1.vi} trykkes på \textit{Run}. 
					
					\item Den maksimale optagede frekvens aflæses i \textit{Max Frequency}. 
					  
			\end{enumerate}
			
			Punkt 1-4 gentages med frekvenser på: 100 Hz, 150 Hz, 200 Hz, 400 Hz, 500 Hz, 600 Hz og 700 Hz. 
			
			\subsubsection{Resultater}
			Den optagede frekvens var den generede udsendte frekvens (+/- 0.5 Hz). 
			\subsubsection{Diskussion}
			Der opnås nu pæne resultater, og der reflekteres over om resultaterne i \ref{bordtest1} skyldes fejl i højtaler eller fejl i VI'et \texttt{genererfrekvenssignal02.vi}. 
			\subsubsection{Konklusion}
			Testresultaterne er tilfredsstillende og testhypotesen accepteres. Det konkluderes, at der ikke er fejl i VI'et \texttt{optagefrekvenssignal0.1.vi}. 
			\subsubsection{Aktion}
			Det skal undersøges, hvilken forskel der er på frekvenssignalet fra onlinetonegenerator.com og det generede frekvenssignal udsendt fra højtaleren ABS-224-RC.  
	
 		\subsection{I03-H1/M1/VI02G/VI02\underline{O}}
		
		\subsubsection{Testhypotese}
		 Frekvenssignalet fra onlinegenerator.com og det genererede frekvenssignal udsendt fra højtaleren ABS-224-RC er forskellige ved analyse i LabVIEW.
		
		\subsubsection{Produktspecifikationer}
		
		\textit{Hardware:}\\
		\abshøj\\
		\hojtalerkabel\\
		\pins\\
		\krympeflex
		\arduino\\
		\usbkabel\\
		\PC\\
		\mikrofon
	
		\textit{Software:}\\
		\labview\\
		\visa\\
		\vi\\
		\ardsw\\
		\onlineg\\
		
		\subsubsection{Opstilling og opsætning}
		\textit{1. delopstilling}:\\
		Højtaleren er loddet til højtalerkablets ene ende, og pin headerne er loddet til kablets anden ende. 
		Pin headerne er isat Arduino'en i pin 46 (PL3(OC5A)), som et er digitalt PWM output, og til ground (GND). 
		Arduino'en er med et USB kabel koblet til PC'en. 		
		På PC'en er VI'et \texttt{genererfrekvenssignal02.vi} åbnet i LabVIEW. Blokdiagrammet er vist i figur \ref{fig:2gf02}.\\ 
 
 \begin{figure}[htb]
			\centering
				\includegraphics[width=3in]{gf02}
				\caption{Blokdiagrammet for \textit{genererfrekvenssignal02.vi}}	
				\label{fig:2gf02}
			\end{figure} 
			
		\textit{2. delopstilling}:\\
		Mikrofonen er sat i PC'ens minijack-stik. På PC'en er VI'et \texttt{optagefrekvenssignal0.1.vi} åbnet i LabVIEW. Blokdiagrammet er vist i figur \ref{fig:3of01}.\\  

		\begin{figure}[htb]
			\centering
				\includegraphics[width=3in]{of01}
				\caption{Blokdiagram for \textit{optagefrekvenssignal01.VI}}	
				\label{fig:3of01}
			\end{figure}		
		
		\textit{3. delopstilling}:\\
		I en internetbrowser er hjemmesiden \texttt{www.onlinetonegenerator.com} åbnet, og PC'ens højtalere er slået til og volumen er sat til maksimum. 
		
		\subsubsection{Udførsel}
			
			\textit{1. deltest}
			\begin{enumerate}
				\item Højtaleren holdes manuelt således membranen står i lodret position. 
				\item Mikrofonen holdes manuelt, vendt mod højtaleren, i en afstand på 5 cm. 
				\item I VI'et \texttt{genererfrekvenssignal02.vi} indtastes den ønskede frekvens i den numeriske kontrol \textit{Enter Frequency}. 
					
					\item Koden eksekveres ved at trykke på \textit{Run}. 
				
				\item I VI'et \texttt{optagefrekvenssignal0.1.vi} trykkes på \textit{Run}. 
					
				\item Den maksimale optagede frekvens aflæses i \textit{Max Frequency}. 
				
			\end{enumerate}
			
			
			\textit{2. deltest}			
			\begin{enumerate}
				\item Mikrofonen holdes manuelt, vendt mod PC'ens højtaler, i en afstand på 5 cm. 
				\item I \texttt{onlinetonegenerator.com} genereres et signal med den ønskede frekvens. 
				\item I VI'et \texttt{optagefrekvenssignal0.1.vi} trykkes på \textit{Run}. 
					
						\item Den maksimale optagede frekvens aflæses i \textit{Max Frequency}. 
					  
			\end{enumerate}
		
			Deltestene gentages med frekvenser på: 100 Hz, 150 Hz, 200 Hz, 400 Hz, 500 Hz, 600 Hz, 700, 1000 0g 1200 Hz, og resultaterne sammenholdes. 
			
			\subsubsection{Resultater}
			 Det blev observeret, at de afsendte frekvenser fra det generede frekvenssignal i VI'et \textit{genererfrekvenssignal02.vi} ikke stemte overens med de aflæste i maks frekvens-feltet.  
			 Det erfaredes at det var grundtonens harmoniske overtoner, idet frekvensen udsendes som et firkantsignal. Kun ved højfrekvente signaler (<1 kHz), blev grundtonen opfanget. 
			 Ved at benytte \texttt{onlinetonegenerator.com}, kunne der udsendes et sinussignal med en given frekvens, som blev korrekt opfanget (+/- 0.5 Hz) i maks frekvens-feltet.     
			\subsubsection{Diskussion}
			Der opnås pæne resultater ved at bruge et sinussignal. Det overvejes om det er muligt at genere sinussignaler til en Arduino, men det viser sig, at være kompliceret at generere sinussignaler til en Arduino.   
			\subsubsection{Konklusion}
			Testhypotesen accepteres. Det konkluderes, at der ikke er fejl i software og hardware, og de unøjagtige resultater skyldes firkantsignalets harmoniske overtoner.  
			\subsubsection{Aktion}
			Det skal undersøges, om det er muligt at filtrere firkantssignalets harmoniske overtoner fra vha. en resonator, således firkantssignalet kan benyttes. 
		
		\subsection{I04-H2/M1/VIO2G/VIO2\underline{O}} 
		\subsubsection{Testhypotese}
		Kan en resonatorlignende beholder dæmpe de harmoniske overtoner fra firkantsignalet. 
		
		\subsubsection{Produktspecifikationer}
		
		\textit{Hardware:}\\
		\tores\\
		\hojtalerkabel\\
		\kabelsko\\
		\pins\\
		\krympeflex\\
		\arduino\\
		\mikrofon\\
		\PC\\
		\usbkabel\\
		\sprøjtebeholder (resonator)\\
		Lineal\\
	
		\textit{Software:}\\
		\labview\\
		\visa\\
		\vi\\
		\ardsw\\
		
		
		\subsubsection{Opstilling og opsætning}
		
		Højtalerkablets ene ende er påsat kabelsko, som er påsat højtaleren. Til kablets anden ende er pin headere loddet fast og forsejlet med krympeflex. Pin headerne er isat Arduino'en i pin 46 (PL3(OC5A)), som et er digitalt PWM output, og til ground (GND). 
		Arduino'en er med et USB kabel koblet til PC'en.	
		Mikrofonen er sat i PC'ens minijack-stik og er ført ned i resonatoren hvor den ligger i bunden. Højtaleren placeres i en afstand på en halv diameter af højtalerens membran, ovenfor resonatorens hals. 
		
		På PC'en er VI'erne \texttt{genererfrekvenssignal02.vi} og \texttt{optagefrekvenssignal02.vi} åbnet i LabVIEW og blokdiagrammerne er vist i figur \ref{fig:4gf02} og \ref{fig:4of01} \\ Testopstillingen kan ses på figur \ref{fig:bt4}.  
		
\vspace{5mm}
		
		\begin{figure}[htb]
			\centering
				\includegraphics[width=3in]{gf02}
				\caption{Blokdiagrammet for \textit{genererfrekvenssignal02.vi}}	
				\label{fig:4gf02}
			\end{figure} 
\vspace{5mm}
			\begin{figure}[htb]
			\centering
				\includegraphics[width=3in]{of01}
				\caption{Blokdiagram for \textit{optagefrekvenssignal01.VI}}	
				\label{fig:4of01}
			\end{figure}
\vspace{5mm}

		\begin{figure}[h]
			\centering
				\includegraphics[width=3in]{bordtest4}
				\caption{Testopstilling for test \textit{I04-H2/M1/VI02G/VI02\underline{O}}}	
				\label{fig:bt4}
			\end{figure}
		

		
		\subsubsection{Udførsel}
			
			\begin{enumerate}
				\item Tryksprøjtedelen afmonteres beholderen, og beholderen fungerer nu som resonator. Resonatoren stilles på et bord med halsen pegende opad. 
				\item Mikrofonen føres ned i resonatoren og ligger i resonatorens bund. 
				\item Linealen påsættes resonatorens hals så den fungerer som afstandsmåler fra halsåbningen.
				\item Højtaleren holdes manuelt over resonatorhalsen i en afstand på en halv diameter af højtalermembranen. Ved anvendelse ef den specificerede højtaler er afstanden to centimeter. 
				\item I VI'et \texttt{genererfrekvenssignal02.vi} indtastes den ønskede frekvens i den numeriske kontrol \textit{Enter Frequency}. 
					
						\item Koden eksekveres ved at trykke på \textit{Run}. 
					
				\item I VI'et \texttt{optagefrekvenssignal02.vi} trykkes på \textit{Run}. 
					
						\item Den maksimale optagede frekvens aflæses i \textit{Max Frequency}. 
					
			\end{enumerate}
			
			
			Testen udføres med en frekvens på 200 Hz, 500 Hz, 1000 Hz og 1200 Hz. Der afprøves to gange med hvert frekvens  
			
			\subsubsection{Resultater}
			
			 Det blev observeret i første forsøg, at resultatet fra det generede frekvenssignal i VI'et \texttt{genererfrekvenssignal02.vi} på 200 Hz var en af grundtonens harmoniske overtone på 1803 Hz og i andet forsøg observeres en harmonisk overtone 1402,33 Hz. Resultatet fra første forsøg vises i figur \ref{fig:bt4200}.
			 
			 Det blev observeret ved begge forsøg, at resultatet fra det generede frekvenssignal i VI'et \texttt{genererfrekvenssignal02.vi} på 500 Hz var en af grundtonens harmoniske overtoner på 1500 Hz. Dette ses i figur \ref{fig:bt4500}. 
			 
			 Det blev observeret i første forsøg, at resultatet fra det generede frekvenssignal i VI'et \texttt{genererfrekvenssignal02.vi} på 1000 Hz var en af grundtonens harmoniske overtoner på 3000,33 Hz i. Dette ses i figur \ref{fig:bt41000}. I det andet forsøg observeres det at resultatet stemmer overens med den afspillede grundtone. Dette ses på figur \ref{fig:bt41000b}.  
			 
			 Det blev observeret, at resultatet fra den generede frekvenssignal i VI'et \texttt{genererfrekvenssignal02.vi} på 1200 Hz stemmer overens med den afspillede grundtone. Dette ses på figur \ref{fig:bt41200}.  
			 
			
			
			\begin{figure}[htb]
			\centering
				\includegraphics[width=4in]{Bordtest4200Hz}
				\caption{Resultat for for bordtest 4 ved anvendelse af 200 Hz}	
				\label{fig:bt4200}
			\end{figure} 
			
\begin{figure}[htb]
			\centering
				\includegraphics[width=4in]{Bordtest4500Hz}
				\caption{Resultat for\textit{I04-H2/M1/VI02G/VI02\underline{O}} ved anvendelse af 500 Hz}	
				\label{fig:bt4500}
			\end{figure} 			
			
			\begin{figure}[htb]
			\centering
				\includegraphics[width=4in]{Bordtest41000Hz}
				\caption{Resultat for for \textit{I04-H2/M1/VI02G/VI02\underline{O}} ved anvendelse af 1000 Hz}	
				\label{fig:bt41000}
			\end{figure} 
			
			\begin{figure}[htb]
			\centering
				\includegraphics[width=4in]{Bordtest41000Hzb}
				\caption{Resultat for for \textit{I04-H2/M1/VI02G/VI02\underline{O}} ved anvendelse af 1000 Hz}	
				\label{fig:bt41000b}
			\end{figure} 
			
			\begin{figure}[htb]
			\centering
				\includegraphics[width=4in]{Bordtest41200Hz}
				\caption{Resultat for \textit{I04-H2/M1/VI02G/VI02\underline{O}} ved anvendelse af 1200 Hz}	
				\label{fig:bt41200}
			\end{figure} 
	
			  
			\subsubsection{Diskussion}
			Ved forsøget med 1000 Hz observeres, at de to resultater ikke er tilnærmelsesvis ens. I første forsøg blev en harmonisk overtone opfanget, som den maksimale frekvens, hvor der i andet forsøg blev opfanget grundtonen på den genererede frekvens. Det er et meget divergerene resultater men hvad kan årsagen være til at der måles så forskellige tal?
			
			\subsubsection{Konklusion}
			testhypotesen afkastes. 
			Ved generering af frekvenser lavere end 1000 Hz opfanges harmoniske overtoner, i stedet for grundtonen, som er det ønskede resultat. 
			Ved generering af frekvenser lig 1000 Hz opfanges ustabile resultater. 
			Ved generering af frekvenser højere end 1000 Hz, opnås pæne resultater, hvor den genererede frekvens er lig den opfangede frekvens.  
			Dermed konkluderes, at resonatoren i dette tilfælde ikke virker dæmpende på harmoniske overtoner på frekvenser lavere end 1000 Hz. 
			  
			\subsubsection{Aktion}
			Det skal undersøges, om målinger genereret med en frekvens $\geq$ 1000 Hz er stabile. 
			
			\subsection{I05-H2/M1/VI02G/VI02\underline{O}}
			
		\subsubsection{Testhypotese}
		Firkantsignaler genereret med frekvenser $\geq$ 1000 Hz er stabile, og Helmholtzresonatoren har på indvirkning på signalet. 
		
		\subsubsection{Produktspecifikationer}
		
		\textit{Hardware:}\\
		\tores\\
		\hojtalerkabel\\
		\kabelsko\\
		\pins\\
		\krympeflex\\
		\arduino\\
		\mikrofon\\
		\PC\\
		\usbkabel\\
		\sprøjtebeholder (resonator)\\
		Lineal\\
	
		\textit{Software:}\\
		\labview\\
		\visa\\
		\vi\\
		\ardsw\\
		
		
		\subsubsection{Opstilling og opsætning}
		
		Højtalerkablets ene ende er påsat kabelsko, som er påsat højtaleren. Til kablets anden ende er pin headere loddet fast og forsejlet med krympeflex. Pin headerne er isat Arduino'en i pin 46 (PL3(OC5A)), som et er digitalt PWM output, og til ground (GND). 
		Arduino'en er med et USB kabel koblet til PC'en.	
		Mikrofonen er sat i PC'ens minijack-stik og er ført ned i resonatoren hvor den ligger i bunden. Højtaleren placeres i en afstand på en halv diameter af højtalerens membran, ovenfor resonatorens hals. 
		
		På PC'en er VI'erne \texttt{genererfrekvenssignal02.vi} og \texttt{optagefrekvenssignal02.vi} åbnet i LabVIEW og blokdiagrammerne er vist i figur \ref{fig:5gf02} og \ref{fig:5of01} \\ Testopstillingen er den samme som i bordtest 4, og vises i figur \ref{fig:bt4}.  
		

\begin{figure}[htb]
			\centering
				\includegraphics[width=3in]{gf02}
				\caption{Blokdiagrammet for \textit{genererfrekvenssignal02.vi}}	
				\label{fig:5gf02}
			\end{figure} 
\vspace{5mm}
			\begin{figure}[htb]
			\centering
				\includegraphics[width=3in]{of01}
				\caption{Blokdiagram for \textit{optagefrekvenssignal01.VI}}	
				\label{fig:5of01}
			\end{figure}
\vspace{5mm}
		

		\subsubsection{Udførsel}
			
			\begin{enumerate}
				\item Tryksprøjtedelen afmonteres af beholderen og fungere nu som resonator. Resonatoren stilles på et bord med halsen pegende opad. 
				\item Mikrofonen føres ned i resonatoren og ligger i resonatorens bund. 
				\item Linealen påsættes resonatorens hals så den fungere som afstandsmåler fra halsåbningen.
				\item Højtaleren holdes manuelt over resonatorhalsen i en afstand på en halv diameter af højtalermembranen. Ved anvendelse ef den specificerede højtaler er afstanden to centimeter. 
				\item I VI'et \texttt{genererfrekvenssignal02.vi} indtastes den ønskede frekvens i den numeriske kontrol \textit{Enter Frequency}. 
					
						\item Koden eksekveres ved at trykke på \textit{Run}. 
				
				\item I VI'et \texttt{optagefrekvenssignal02.vi} trykkes på \textit{Run}. 
					
						\item Den maksimale optagede frekvens aflæses i \textit{Max Frequency}. 
					
			\end{enumerate}
			
			Testen udføres med en frekvens på 950 Hz, 1000 Hz, 1100 Hz, 1200 Hz og 1300 Hz. Testen fortages to gange med hver frekvens.  
			
			\subsubsection{Resultater}
			
			\begin{table}[]
				\centering
				\caption{Tabel over resultater}
				\label{bordtest5resultater}
				\begin{tabular}{lll}
					\multicolumn{1}{l|}{\textbf{Udsendt frekvens {[}Hz{]}}} & 	
					\multicolumn{1}{l|}{\textbf{1. resultat}} & \textbf{2. resultat} \\ \hline
					\multicolumn{1}{c|}{950}& 
					\multicolumn{1}{c|}{954.33}&954.33\\
					\multicolumn{1}{c|}{1000}& 
					\multicolumn{1}{c|}{1000.00}&1000.00\\
					\multicolumn{1}{c|}{1100}& 
					\multicolumn{1}{c|}{1101.33}&1101.33\\
					\multicolumn{1}{c|}{1200}& 
					\multicolumn{1}{c|}{1202.00}&1202.00\\
					\multicolumn{1}{c|}{1300}& 
					\multicolumn{1}{c|}{1302.00}&1302.00\\
                   
				\end{tabular}
			\end{table}
			
			Ud fra tabellen ses at de opfangede signaler er stabile ved alle udsendte frekvenser. Det ses ydermere at de afsendte signaler ikke ser ud til at være påvirket af Helmholtzresonatoren
				
			  
			\subsubsection{Diskussion}
			 Variationen på de opfangede frekenser er tilfredsstillende til brug i det videre udviklingsforløb. De optagede frekvenser afviger ikke fra de udsendte frekvenser selvom de udsendte frekvenser er opfanget i resonatoren. Det er Helmholtz resonansen som ønskes opfanget og den formodes at afvige fra den udsendte frekvens. Dette er dog ikke tilfældet i denne tests resultater. Der undres over de optagede frekvenser ikke er påvirkede af resonatoren og der sås nu tvivl minijack mikrofonen opfanger signalerne eller om det er PC mikrofonen som opfanger signalerne
			
			\subsubsection{Konklusion}
			Hypotesen afkastestes. Frekvenser over 950 Hz kan opfanges som grundtonen og det opfangede signal er ikke forstyrret at de harmoniske overtoner. Det ser ikke ud til at de opfangede frekvenser er påvirkede af resonatoren, hvilket er forventeligt. Det konkluderes, at det er nødvendigt at undersøge om minijack mikrofonen opfanger de afsendte signaler. 
			
			\subsubsection{Aktion}
			 Der udføres en ny enhedstest af minijack mikrofonen for at klarlægge om mikrofonen opfanger signalerne. 
			 
 	\subsection{I06-H3/M4/VI05\underline{O}}
 	\label{subsec:I06}
	
	\subsubsection{Testhypotese}
	Resonansfrekvensen $f_{0}$ i den tomme resonator, kan verificeres i forhold til resonatorens kendte volumen. 
		
	\subsubsection{Teoretisk baggrundsviden}	
Volumen af resonatoren skal kendes for at verificere den målte resonansfrekvens i det tomme kammer. 
Volumen af resonatoren $V$ er ekskl. port/ hals. Volumen er udregnet ved at hælde vand i resonatoren, og derefter måle vandets vægt. Ud fra vandets vægt kan volumen gives vha. følgende formel:  
\begin{equation}
  V=\frac{m}{\rho}
  \label{eq:vformel}
\end{equation}
$\Downarrow$
\begin{equation}
V_{resonator}=\frac{\SI{1,671}{\kilo\gram}}{\SI{1000}{\kilo \gram \per \meter^{3}}}={\SI{0,001671}{\meter^{3}}}
\end{equation}


Helmholtz resonansen i en tom resonator er givet ved følgende formel: \f

hvor 
\begin{description}[align=left, labelwidth=1in,labelindent=0.5cm]
\item $f_{0}$: resonansfrekvens i en tom resonator [$Hz$],\\
\item $c$: lydens hastighed i luft [$m/s$],\\
\item $S_{p}$: tværsnitsareal af port [$m^2$],\\
\item $V$: statisk volumen af resonator [$m^3$],\\
\item $l_{p}$: længde af port [$m$],\\
\item $\Delta l$: endekorrektion [$m$]\\
\end{description}
Lydens hastighed i luft $c$ er givet ved formlen: 
\c
hvor $T_{K}$ er givet ved \T. 
Derved kan lydens hastighed i luft ved en temperatur på 23\degree C bestemmes: 
\begin{equation}
		c = {\SI{331,5}{\meter / \second}} \cdot
		\sqrt{\frac{\SI{296,15}{\kelvin}}{\SI{273,15}{\kelvin}}} = {\SI{345,175}{\meter / \second}} 	\end{equation}
		
Tværsnitsarealet $S_{p}$ af porten bestemmes ved: 
\Sp 
hvor ${r}$ er radius er porten. 
Derved bliver $S_{p}$ ved en radius på ${\SI{1.75}{\centi \meter}}$:  
\begin{equation}
S_{p} = ({\SI{0,0175}{\meter}})^{2}\pi = {\SI{0,000962}{\meter^{2}}}	
\end{equation}  

Endekorrektionen $\Delta l$ er en værdi som tillægges, som korrektion for den medsvingende luft, der i resonatoren fungerer som masse. 

$\Delta l$ gives ved: \deltal
$\Downarrow$
\begin{equation}
		\Delta l = 0,6 \cdot {\SI{0,0175}{\meter}} + \frac{8}{3\pi} \cdot {\SI{0,0175}{\meter}} = {\SI{0,025354}{\meter}} 
\end{equation}

Resonansfrekvens $f_{0}$ i en tom resonator ved en lufttemperatur på 23\degree C er altså:

\f
$\Downarrow$   
\begin{equation}
		f_{0} = \frac{\SI{345,175}{\meter / \second}}{2\pi}\sqrt{\frac{\SI{0,000962}{\meter^{2}}}{{\SI{0,001671}{\meter^{3}}}({\SI{0,034}{\meter}}+{\SI{0,025354}{\meter}})}} 
	\end{equation}	
$\Downarrow$
\begin{equation}
	f_{0} = {\SI{171,094}{\second}^{-1}} \approxeq {\SI{171,094}{\hertz}} 	
	\end{equation}


	\subsubsection{Produktspecifikationer}
	\textit{Hardware:}\\
	\usbhøj\\
	\elektretto\\
	\daq\\
	\sprøjtebeholder\\	
	\snot\\	
	Lineal\\
	\modellervoks\\
	\plade\\
	\PC\\
	
	\textit{Software:}\\
	\labview\\
	\daqsoft\\
	\onlineg\\
	
	\subsubsection{Opstilling og opsætning}
	Flasken er skruet af tryksprøjten, og bunden er savet af i et lige snit. Denne flaske har funktionen som resonator i testen. Resonatoren vises i figur \ref{fig:resonator}. Resonatoren er placeret på pladen, og er tætnet med modellervoks i kanten mellem resonator og plade. Trykventilen på flasken er også tætnet med modellervoks. En lineal er påsat flaskehalsen, og benyttes til afstandsmåling. Mikrofonen er monteret på resonatorens inderside med klæbemasse. Mikrofonen placeres mindre end resonatorhalsens længde fra resonatorhalsens indre åbning. Mikrofonen er endvidere tilkoblet DAQ'en, som er tilsluttet PC'en. På PC'en er LabVIEW VI'et \texttt{optagefrekvenssignal05.vi} åbnet. Blokdiagrammet vises i figur \ref{fig:optagefrekvenssignal05}. Højtaleren er tilsluttet PC'en, og \onlineg er åbnet i en browser, hvor pink noise genereres. Højtaleren holdes i en afstand på en halv diameter af halsåbningen fra halsens indgangen.  
	
	\begin{figure}[htb]
	\centering
	\includegraphics[width=3in]{resonator.jpg}
	\caption{Resonator fremstillet af \sprøjtebeholder.}
	\label{fig:resonator}	
	\end{figure}

	\begin{figure}[htb]
	\centering
	\includegraphics[width=4in]{optagefrekvenssignal05.png}
	\caption{Blokdiagram for LabVIEW VI'et \texttt{optagefrekvenssignal05.vi}.}
	\label{fig:optagefrekvenssignal05}	
	\end{figure}
	
	
	
	\subsubsection{Udførsel}
			
			\begin{enumerate}
			\item Pink noise genereres fra \onlineg  ud gennem højtaleren. 
			\item Gentagelse af antal loops i blokdiagrammet sættes til 10, og der optages 4 sekunder af gangen. 
			\item Der trykkes \textit{Run} i VI'et. 
			\item Resonansfrekvens $f_{0}$ aflæses i \textit{Resonansfrekvens af kammer med objekt}.
			\end{enumerate}
			
	\subsubsection{Resultater}

			
		\begin{table}[htb]
\centering
\caption{Tabel over resultater}
\label{bordtest6resultater}
\begin{tabular}{c|l}
\textbf{\#}& \textbf{$f_{0}$ {[Hz]}} \\
\hline
1  &    179,394              \\
2  &    183,692             \\
3  &    204,948              \\
4  &    216,2              \\
5  &    218,8              \\
6  &    232,2              \\
7  &    240,6              \\
8  &    260,5              \\
9  &    255,9              \\
10 &    251,3             
\end{tabular}
\end{table}
			
	\subsubsection{Diskussion}
	Den teoretiske udregnede resonansfrevkens i den tomme resonator er 171 Hz. Den første måling har en relativ afvigelse på 5 \%, og de efterfølgende målinger ser ud til at stige for hver foretagede måling.  
	
	\subsubsection{Konklusion}
	Hypotesen afkastes. Der må være en fejl i softwaren siden målinger tilnærmelsesvis stiger for hver måling. 
	
	\subsubsection{Aktion}
	Det skal undersøges, om der er en fejl i softwaren.	
	
	\subsection{I07-H3/M4/VI05\underline{O}}		\subsubsection{Testhypotese}
	Resonansfrekvensen $f_{0}$ i den tomme resonator, kan verificeres i forhold til resonatorens kendte volumen.  	
	
	\subsubsection{Produktspecifikationer}
	\textit{Hardware:}\\
	\usbhøj\\
	\elektretto\\
	\daq\\
	\sprøjtebeholder\\	
	\snot\\	
	\højtalerholder til \usbhøj\\
	\modellervoks\\
	\plade\\
	\PC\\
	Termometer\\
	
	\textit{Software:}\\
	\labview\\
	\daqsoft\\
	\onlineg\\
	
		\subsubsection{Opstilling og opsætning}
	Flasken er skruet af tryksprøjten, og bunden er savet af i et lige snit. Denne flaske har funktionen som resonator i testen. Resonatoren vises i figur \ref{fig:resonator}. Resonatoren er placeret på pladen, og er tætnet med modellervoks i kanten mellem resonator og plade. Trykventilen på flasken er også tætnet med modellervoks.
Mikrofonen er monteret på resonatorens inderside med klæbemasse. Mikrofonen placeres mindre end resonatorhalsens længde fra  resonatorhalsens indre åbning. Mikrofonen er endvidere tilkoblet DAQ'en, som er tilsluttet PC'en. På PC'en er LabVIEW VI'et \texttt{optagefrekvenssignal05.vi} åbnet. Blokdiagrammet vises i figur \ref{fig:optagefrekvenssignal05}. En 3D-printet holder er skruet på flaskens gevind. Holderen sikrer, at højtaleren holdes i en afstand på $\geq$  diameter af portåbningen fra portindgangen. Højtaleren er tilsluttet PC'en, og \onlineg er åbnet i en browser, hvor pink noise genereres. 
	Et termometer sættes til at måle rumtemperaturen, imens testen udføres. 
	
	\begin{figure}[htb]
	\centering
		\includegraphics[width=4in]{bordtest61.jpg}	
		\caption{Opstillingen for testen fremgår af denne figur}
		\label{bordetest61op}
	\end{figure}
	
		\subsubsection{Udførsel}
			
			\begin{enumerate}
			\item Pink noise genereres fra \onlineg  ud gennem højtaleren. 
			\item Gentagelse af antal loops i blokdiagrammet sættes til 100, og der optages 4 sekunder af gangen. 
			\item Der trykkes \textit{Run} i VI'et. 
			\item Resonansfrekvens $f_{0}$ aflæses i \textit{Resonansfrekvens af kammer med objekt}.
			\end{enumerate}
			
	\subsubsection{Resultater}
	$f_{b}$ = 251 Hz
	
	\subsubsection{Diskussion}
	Resultatet er ikke tilnærmelsesvis det teoretiske på 171,094 Hz. 
	
	\subsubsection{Konklusion}
	Hypotesen forkastes. 
	
	\subsubsection{Aktion}
	Der udføres en ny test, hvor volumen af et objekt bestemmes. 

\subsection{I08-H3/M4/VI05\underline{O}}

\subsubsection{Testhypotese}
Volumen af den vandfyldte ballon i resonatoren kan bestemmes ud fra den målte resonansfrekvens $f_{b}$

\subsubsection{Teoretisk baggrundviden}
Resonansfrekvensen $f_{b}$ i et kammer indeholdende en vandfyldt ballon med et volumen på 0,676 L, kan udregnes ved brug af Helmholtz' ligning, givet ved dette udtryk \fb

hvor 
\begin{description}[align=left,labelindent=0.3cm]
\item $f_{b}$: resonansfrekvens i en resonator, indeholdende et objekt [$Hz$],\\
\item $W$: volumen af objekt [$m^3$],\\
\end{description}

(for yderligere specifikation se teoretisk baggrundsviden i I06, \ref{subsec:I06})

\subsubsection{Resonansfrevkensen $f_{b}$ af ballonen [V=0,676 L]} 
\hspace{1,5cm}
\begin{equation}
		f_{b} = \frac{c}{2\pi}\sqrt{\frac{S_{p}}{(V-W)(l_{p}+\Delta l)}}
	\end{equation}
$\Downarrow$
\begin{equation}
		f_{b} = \frac{{\SI{345,582}{\kelvin}}}{2\pi}\sqrt{\frac{{\SI{0,000962}{\meter}^2}}{({\SI{0,001671}{\meter}^3}-{\SI{0,000676}{\meter}^3})({\SI{0,034}{\meter}}+ {\SI{0.025354}{\meter}})}}	
\end{equation}
$\Downarrow$
\begin{equation}
		f_{b} = {\SI{221,7}{\hertz}}
\end{equation}

 \subsubsection{Produktspecifikationer}
	\textit{Hardware:}\\
	\usbhøj\\
	\elektretto\\
	\daq\\
	\sprøjtebeholder\\	
	\snot\\	
	\højtalerholder til \usbhøj\\
	\modellervoks\\
	\plade\\
	Ballon med vandvolumen på 0,676 L\\
	\PC\\
	Termometer\\
	
	\textit{Software:}\\
	\labview\\
	\daqsoft\\
	\onlineg\\
	
		\subsubsection{Opstilling og opsætning}
	Flasken er skruet af tryksprøjten, og bunden er savet af i et lige snit. Denne flaske har funktionen som resonator i testen. Ballonen placeres på pladen, og resonatoren sættes nedover. Kanten mellem resonator og plade tætnes med modellervoks. Trykventilen på flasken er også tætnet med modellervoks.
Mikrofonen er monteret på resonatorens inderside med klæbemasse. Mikrofonen placeres mindre end resonatorhalsens længde fra  resonatorhalsens indre åbning. Mikrofonen er endvidere tilkoblet DAQ'en, som er tilsluttet PC'en. På PC'en er LabVIEW VI'et \texttt{optagefrekvenssignal05.vi} åbnet. Blokdiagrammet vises i figur \ref{fig:optagefrekvenssignal05}. En 3D-printet holder er skruet på flaskens gevind. Holderen sikrer, at højtaleren holdes i en afstand på $\geq$ en diameter af halsåbningen fra halsens indgangen. Højtaleren er tilsluttet PC'en, og \onlineg er åbnet i en browser, hvor pink noise genereres. 
	Et termometer sættes til at måle rumtemperaturen, imens testen udføres. 
	
	\subsubsection{Udførsel}
			
			\begin{enumerate}
			\item Pink noise genereres fra \onlineg  ud gennem højtaleren. 
			\item Gentagelse af antal loops i blokdiagrammet sættes til 100, og der optages 4 sekunder af gangen. 
			\item Der trykkes \textit{Run} i VI'et. 
			\item Resonansfrekvens $f_{0}$ aflæses i \textit{Resonansfrekvens af kammer med objekt}.
			\end{enumerate}
			
	\subsubsection{Resultater}
	Den målte resonansfrekvens er $f_{b}$ = 321 Hz.  
	Volumen af ballonen $W$ udregnes ud fra den målte resonansfrekvens $f_{b}$. 
	Resonansfrekvens (udregnet i I06 \ref{subsec:I06}) $f_{0}$=171,094 Hz anvendes.  
	
\begin{equation}
		W = V\left(1-\left(\frac{f_{0}}{f_{b}}\right)^2\right)
\end{equation}
$\Downarrow$
\begin{equation}
		W = {\SI{0,001671}{\meter^{3}}}\left(1-\left(\frac{\SI{171,094}{\hertz}}{\SI{321}{\hertz}}\right)^2\right)
\end{equation}
$\Downarrow$
\begin{equation}
		W = {\SI{0,001196}{\meter^{3}}}={\SI{1,196}{\liter}}
\end{equation}

	\subsubsection{Diskussion}
	Den absolutte afvigelse er på knap det dobbelte. Skyldes dette unøjagtige mål, testopstillingen, softwaren eller andre ting? 
	
	\subsubsection{Konklusion}
	Hypotesen afkastes grundet den store afvigelse. 
	
	\subsubsection{Aktion}
	Der konfereres med fagperson, hvorefter der foretages en ny test med forskellige størrelser. 

\subsection{I09-H3/M4/VI05\underline{O}}

\subsubsection{Testhypotese}
Der opnåes mere præcise resultater ved at afmontere højtalerholderen.

\subsubsection{Produktspecifikationer}
	\textit{Hardware:}\\
	\usbhøj\\
	\elektretto\\
	\daq\\
	\sprøjtebeholder\\	
	\snot\\	
	\højtalerholder til \usbhøj\\
	Afstandsklods: hætte fra sprittusch ($\geq$ portens diameter)\\ 
	\modellervoks\\
	\plade\\
	\PC\\
	Termometer\\
	3 balloner:
	\begin{itemize}
	\item Gul ballon: 216 g. svarende til et volumen på: 0,000216 m3
\item Blå ballon: 383 g. svarende til et volumen på: 0,000383 m3
\item Grøn ballon: 516 g. svarende til et volumen på: 0,000516 m3	\end{itemize}
	
	\textit{Software:}\\
	\labview\\
	\daqsoft\\
	\onlineg\\
	
\subsubsection{Opstilling og opsætning}
	Flasken er skruet af tryksprøjten, og bunden er savet af i et lige snit. Denne flaske har funktionen som resonator i testen. Ballonen placeres på pladen, og resonatoren sættes nedover. Kanten mellem resonator og plade tætnes med modellervoks. Trykventilen på flasken er også tætnet med modellervoks.
Mikrofonen er monteret på resonatorens inderside med klæbemasse. Mikrofonen placeres mindre end resonatorhalsens længde fra  resonatorhalsens indre åbning. Mikrofonen er endvidere tilkoblet DAQ'en, som er tilsluttet PC'en. På PC'en er LabVIEW VI'et \texttt{optagefrekvenssignal05.vi} åbnet. Blokdiagrammet vises i figur \ref{fig:optagefrekvenssignal05}. I den del af testen, hvor der testes med højtalerholderen, er denne skruet på flaskens gevind. I testen uden højtalerholderen, indsættes en afstandsklods i stedet. Dette er for at sikre, at højtaleren holdes i en afstand på $\geq$ en diameter af halsåbningen fra halsens indgangen. Højtaleren er tilsluttet PC'en, og \onlineg er åbnet i en browser, hvor pink noise genereres. 
	Et termometer sættes til at måle rumtemperaturen, imens testen udføres. 
	
	\subsubsection{Udførsel}
			
			\begin{enumerate}
			\item Pink noise genereres fra \onlineg  ud gennem højtaleren. 
			\item Gentagelse af antal loops i blokdiagrammet sættes til 100, og der optages 4 sekunder af gangen. 
			\item Der trykkes \textit{Run} i VI'et. 
			\item Resonansfrekvens $f_{0}$ aflæses i \textit{Resonansfrekvens af kammer med objekt}.
			\end{enumerate}
	
	Der foretages først målinger med højtalerholderen og herefter uden. Der udføres et testsæt med resonansfrevkenser af det tomme kammer og derefter ballonerne. 
	
	\subsubsection{Resultater}
	
	
	Resonansfrekvensen ved målinger \textit{med} højtalerholderen:
	\begin{figure}[H]
	\centering
	\includegraphics[width=6in]{tomtkammer(m_holder)}
	\includegraphics[width=6in]{ballon216g(m_holder)}
	\includegraphics[width=6in]{ballon383g(m_holder)}
	\includegraphics[width=6in]{ballon516g(m_holder)}
	\end{figure}
	
	Resonansfrekvenser ved målinger \textit{uden} højtalerholderen:
	\begin{figure}[H]
	\centering
	\includegraphics[width=6in]{tomtkammer(u_holder)}
	\includegraphics[width=6in]{ballon216g(u_holder)}
	\includegraphics[width=6in]{ballon383g(u_holder)}
	\includegraphics[width=6in]{ballon516g(u_holder)}
	\end{figure}

	\subsubsection{Dikussion}
	Værdierne er mere præcise uden højtalerholderen. Dette skyldes muligvis, at der skal være frit luft omkring resonatorhalsen, så luften kan virke som en fjeder. 
	
	\subsubsection{Konklusion}
	Hypotesen godkendes. 

	\subsubsection{Aktion}
	Resultaterne skal behandles i Projektrapporten. Der skal endvidere testes uden højtalerholderen i et forsøg på at skabe endnu præcisere resulater. 



	

	