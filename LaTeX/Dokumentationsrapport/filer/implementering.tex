\chapter{Implementering}

\section{Indledning}
	Dette kapitel indeholder €€€Husk at skrive noget om at for at kunne udføre beskrevne tests, forventes det at man har et vist kendeskab til Labview
	  
		\subsection{Formål}
	
		\subsection{Læsevejledning}	
	
		
		\subsection{Versionshistorik}

 
\section{Enhedstest}
	Dette afsnit beskriver de indledende funktionstests, hvor hver enkelte selvstændige funktion i systemet afprøves. 
	
	\subsection{Højtaler ABS-224-RC}
	\label{etha}
		\subsubsection{Testformål}
		Det afprøves, om højtaleren kan generere en lyd.
		\subsubsection{Produktspecifikationer}
	
	
		\textit{Hardware:}\\
		\abshøj\\
		\højtalerkabel\\
		\pins\\
		\arduino\\
		\PC\\
		\usbkabel\\
	
		\textit{Software:}\\
		\labview\\
		\visa\\
		\vi\\
		\ardsw\
	
		\subsubsection{Opstilling og opsætning}
		Højtaleren er loddet til højtalerkablets ene ende, og pin headerne er loddet til kablets anden ende. 
		Pin headerne er isat Arduino'en i pin 46 (PL3(OC5A)), som et er digitalt PWM output, og til ground (GND). 
		Arduino'en er med et USB kabel koblet til PC'en. Testopstillingen er vist i figur \ref{fig:etha}.\\ 
	  
			\begin{figure}[htb]
			\centering
				\includegraphics[width=3in]{haArduino}
				\caption{Testopstilling for enhedstest af Højtaler ABS-224-RC.}	
				\label{fig:etha}
			\end{figure}
	
		På PC'en er VI'et \texttt{genererfrekvenssignal0.2.vi} åbnet i LabVIEW. Blokdiagrammet for VI'et er vist i figur \ref{fig:bdgenerer}.   \\   
	
			\begin{figure}[htb]
			\centering
				\includegraphics[width=4in]{genererfrekvenssignal02}
				\caption{I blokdiagrammet \texttt{genererfrekvenssignal0.2.vi} er der anvendt følgende LINX-VI'er; Initialize, Digital Write Square Wave samt Close.}	
				\label{fig:bdgenerer}
			\end{figure}	  
	
		\subsubsection{Udførsel}
			\begin{enumerate}
				\item I \textit{Enter Frequency} på frontpanelet i \texttt{genererfrekvenssignal0.2.vi}, indtastes 500. 
				\item Der trykkes på \textit{Run}. 
				\item Der lyttes efter lydsignal fra højtaleren.  
			\end{enumerate}
			
		\subsubsection{Resultater}
		Lydsignal blev generet og afspillet. 
		\subsubsection{Diskussion} 
		-
		\subsubsection{Konklusion}
		Højtaleren opfylder testen idet der generes en lyd. Samtidig konkluderes, at øvrigt anvendt HW og SW i denne test virker tilfredsstillende, og der vil derfor ikke blive lavet yderligere enhedstests på disse komponenter. 
	
		\subsubsection{Aktion}
		- 
	\subsection{Minijack PC Mikrofon}
		\subsubsection{Testformål}
		Det afprøves, om mikrofonen kan optage en tone.
		\subsubsection{Produktspecifikationer}
	
			\textit{Hardware:}\\
			\mikrofon\\
			\PC\\
	
			\textit{Software:}\\
			\labview\\
	
		\subsubsection{Opstilling og opsætning}
		Mikrofonen er sat i PC'ens minijack-stik. Testopstillingen er vist i figur \ref{fig:etmik}.\\ 
	  
			\begin{figure}[htb]
			\centering
				\includegraphics[width=3in]{mikopstilling}
				\caption{Testopstilling for enhedstest af Minijack PC Mikrofon.}	
				\label{fig:etmik}
			\end{figure}
	
			På PC'en er VI'et \texttt{optagefrekvenssignal0.1.vi} åbnet i LabVIEW. Blokdiagrammet for VI'et er vist i figur \ref{fig:bdoptage}.   \\   
	
			\begin{figure}[htb]
			\centering
				\includegraphics[width=4in]{optagefrekvenssignal01}
				\caption{I blokdiagrammet \texttt{optagefrekvenssignal0.1.vi} opsamles lydsignalet, som vises i en graf på frontpanelet.}	
				\label{fig:bdoptage}
			\end{figure}	  
	
		\subsubsection{Udførsel}
			\begin{enumerate}
				\item I VI'et \texttt{optagefrekvenssignal0.1.vi}, trykkes på \textit{Run}.  
				\item Der indtales en tone i mikrofonen. 
				\item På frontpanelet i VI'et ses frekvensudsving på en graf.  
			\end{enumerate}
		
		\subsubsection{Resultater}
		Mikronen har opfanget et frekvenssignal. 
		\subsubsection{Diskussion} 
		-
		\subsubsection{Konklusion}
		Mikrofonen opfylder testen idet der optages en tone. 
		\subsubsection{Aktion}
		- 
	
	\subsection{Tores højtaler}
		\subsubsection{Testformål}
		Det afprøves, om højtaleren kan generere en lyd.
		\subsubsection{Produktspecifikationer}
	
	
		\textit{Hardware:}\\
		\tores\\
		\højtalerkabel\\
		\kabelsko\\
		\pins\\
		\arduino\\
		\PC\\
		\usbkabel\\
	
		\textit{Software:}\\
		\labview\\
		\visa\\
		\vi\\
		\ardsw\
	
		\subsubsection{Opstilling og opsætning}
		Højtalerkablets ene ende er påsat kabelsko, som er påsat højtaleren. Til kablets anden ende er pin headere loddet fast. Pin headerne er isat Arduino'en i pin 46 (PL3(OC5A)), som et er digitalt PWM output, og til ground (GND). 
		Arduino'en er med et USB kabel koblet til PC'en. Testopstillingen er vist i figur \ref{fig:ethb}.\\ 
	  
			\begin{figure}[htb]
			\centering
				\includegraphics[width=3in]{hbArduino}
				\caption{Testopstilling for enhedstest af Tores højtaler.}	
				\label{fig:ethb}
			\end{figure}
	
		På PC'en er VI'et \texttt{genererfrekvenssignal0.2.vi} åbnet i LabVIEW. Blokdiagrammet for VI'et er vist i figur \ref{fig:bdgenerer}.   \\   
	
		
	
		\subsubsection{Udførsel}
			Udføres på samme vis, som ved enhedstest af \ref{etha} Højtaler ABS-224-RC. 
			
		\subsubsection{Resultater}
		Lydsignal blev generet og afspillet. 
		\subsubsection{Diskussion} 
		-
		\subsubsection{Konklusion}
		Højtaleren opfylder testen idet der generes en lyd. 
		\subsubsection{Aktion}
		- 

\section{Integrationstest}

	\subsection{Bordtest nr. 1} %% Skal måske omdøbes?!
	\label{bordtest1}
		\subsubsection{Testformål}
		Det afprøves, at generere et frekvenssignal som udsendes gennem højtaleren, og derefter opfanges af mikrofonen, hvor den højst målte frekvens til sidst angives.  
		\subsubsection{Produktspecifikationer}
		
		\textit{Hardware:}\\
		\abshøj\\
		\højtalerkabel\\
		\pins\\
		\arduino\\
		\PC\\
		\usbkabel\\
	
		\textit{Software:}\\
		\labview\\
		\visa\\
		\vi\\
		\ardsw\
		
		\subsubsection{Opstilling og opsætning}
		\textit{1. delopstilling}:\\
		Højtaleren er loddet til højtalerkablets ene ende, og pin headerne er loddet til kablets anden ende. 
		Pin headerne er isat Arduino'en i pin 46 (PL3(OC5A)), som et er digitalt PWM output, og til ground (GND). 
		Arduino'en er med et USB kabel koblet til PC'en. 		
		På PC'en er VI'et \texttt{genererfrekvenssignal0.2.vi} åbnet i LabVIEW. Testopstillingen for denne del er vist i figur \ref{fig:etha}, og blokdiagrammet er vist i figur \ref{fig:bdgenerer}.\\ 
 
		\textit{2. delopstilling}:\\
		Mikrofonen er sat i PC'ens minijack-stik. På PC'en er VI'et \texttt{optagefrekvenssignal0.1.vi} åbnet i LabVIEW. Testopstillingen for denne del er vist i figur \ref{fig:etmik}, og blokdiagrammet er vist i figur \ref{fig:bdoptage}.\\ 
		
		\subsubsection{Udførsel}
			\begin{enumerate}
				\item Højtaleren holdes manuelt således membranen står i lodret position. 
				\item Mikrofonen holdes manuelt, vendt mod højtaleren, i en afstand på 5 cm. 
				\item I VI'et \texttt{genererfrekvenssignal0.2.vi} indtastes den ønskede frekvens i den numeriske kontrol \textit{Enter Frequency}. 
					\begin{enumerate}
						\item Koden eksekveres ved at trykke på \textit{Run}. 
					\end{enumerate} 
				\item I VI'et \texttt{optagefrekvenssignal0.1.vi} trykkes på \textit{Run}. 
					\begin{enumerate}
						\item Den maksimale optagede frekvens aflæses i \textit{Max Frequency}. 
					\end{enumerate}	  
			\end{enumerate}
			
			Punkt 1-4 gentages med frekvenser på: 100 Hz, 150 Hz, 200 Hz, 400 Hz, 500 Hz, 600 Hz og 700 Hz. 
			
			\subsubsection{Resultater}
			Den maksimale optagede frekvens var ikke tilnærmelsesvis frekvensen på den udsendte tone.
			\subsubsection{Diskussion}
			Det ønskede resultat er frekvensen på den udsendte tone, hvilket ikke var tilfældet i denne test. Hvor er fejlen opstået; er der fejl i LabVIEW-kode eller hardware? 
			\subsubsection{Konklusion}
			Det er nødvendigt at undersøge om fejlen opstår i vores hardware eller software. 
			\subsubsection{Aktion}
			Det skal med en online tonegenerator undersøges, hvor fejlen er opstået. 

	\subsection{Bordtest nr. 2} %% Skal måske omdøbes?!
		\subsubsection{Testformål}
		Det afprøves, at generere et frekvenssignal fra en online tonegenerator, som udsendes gennem PC'ens højtaler, og derefter opfanges af mikrofonen, hvor den højst målte frekvens til sidst angives.  
		
		\subsubsection{Produktspecifikationer}
		
		\textit{Hardware:}\\
		\mikrofon\\
		\PC\\
	
		\textit{Software:}\\
		\labview\\
		\visa\\
		\vi\\
		\ardsw\\
		\texttt{onlinetonegenerator.com}
		
		\subsubsection{Opstilling og opsætning}
		Mikrofonen er sat i PC'ens minijack-stik. På PC'en er VI'et \texttt{optagefrekvenssignal0.1.vi} åbnet i LabVIEW. Testopstillingen for denne del er vist i figur \ref{fig:etmik}, og blokdiagrammet er vist i figur \ref{fig:bdoptage}.\\ 
		I en internetbrowser er hjemmesiden \texttt{www.onlinetonegenerator.com} åbnet, og PC'ens højtalere er slået til. 
		
		\subsubsection{Udførsel}
			\begin{enumerate}
				\item Mikrofonen holdes manuelt, vendt mod PC'ens højtaler, i en afstand på 5 cm. 
				\item I \texttt{onlinetonegenerator.com} genereres et signal med den ønskede frekvens. 
				\item I VI'et \texttt{optagefrekvenssignal0.1.vi} trykkes på \textit{Run}. 
					\begin{enumerate}
						\item Den maksimale optagede frekvens aflæses i \textit{Max Frequency}. 
					\end{enumerate}	  
			\end{enumerate}
			
			Punkt 1-4 gentages med frekvenser på: 100 Hz, 150 Hz, 200 Hz, 400 Hz, 500 Hz, 600 Hz og 700 Hz. 
			
			\subsubsection{Resultater}
			Den optagede frekvens var den generede udsendte frekvens (+/- 0.5 Hz). 
			\subsubsection{Diskussion}
			Der opnås nu pæne resultater, og der reflekteres over om resultaterne i \ref{bordtest1} skyldes fejl i højtaler i fejl i VI'et \texttt{genererfrekvenssignal0.2.vi}. 
			\subsubsection{Konklusion}
			Det konkluderes, at der ikke er fejl i VI'et \texttt{optagefrekvenssignal0.1.vi}. 
			\subsubsection{Aktion}
			Det skal undersøges, hvilken forskel der er på frekvenssignalet fra onlinetonegenerator.com og det generede frekvenssignal udsendt fra højtaleren ABS-224-RC.  
	
		\subsection{Bordtest nr. 3} %% Skal måske omdøbes?!
		\subsubsection{Testformål}
		Det undersøges, hvilken forskel der er på frekvenssignalet fra onlinegenerator.com og det genererede frekvenssignal udsendt fra højtaleren ABS-224-RC.  
		
		\subsubsection{Produktspecifikationer}
		
		\textit{Hardware:}\\
		\abshøj\\
		\højtalerkabel\\
		\pins\\
		\krympeflex
		\arduino\\
		\usbkabel\\
		\PC\\
		\mikrofon\\
	
		\textit{Software:}\\
		\labview\\
		\visa\\
		\vi\\
		\ardsw\\
		\texttt{onlinetonegenerator.com}
		
		\subsubsection{Opstilling og opsætning}
		\textit{1. delopstilling}:\\
		Højtaleren er loddet til højtalerkablets ene ende, og pin headerne er loddet til kablets anden ende. 
		Pin headerne er isat Arduino'en i pin 46 (PL3(OC5A)), som et er digitalt PWM output, og til ground (GND). 
		Arduino'en er med et USB kabel koblet til PC'en. 		
		På PC'en er VI'et \texttt{genererfrekvenssignal0.2.vi} åbnet i LabVIEW. Testopstillingen for denne del er vist i figur \ref{fig:etha}, og blokdiagrammet er vist i figur \ref{fig:bdgenerer}.\\ 
 
		\textit{2. delopstilling}:\\
		Mikrofonen er sat i PC'ens minijack-stik. På PC'en er VI'et \texttt{optagefrekvenssignal0.1.vi} åbnet i LabVIEW. Testopstillingen for denne del er vist i figur \ref{fig:etmik}, og blokdiagrammet er vist i figur \ref{fig:bdoptage}.\\  
		
		\textit{3. delopstilling}:\\
		I en internetbrowser er hjemmesiden \texttt{www.onlinetonegenerator.com} åbnet, og PC'ens højtalere er slået til. 
		
		\subsubsection{Udførsel}
			
			\textit{1. deltest}
			\begin{enumerate}
				\item Højtaleren holdes manuelt således membranen står i lodret position. 
				\item Mikrofonen holdes manuelt, vendt mod højtaleren, i en afstand på 5 cm. 
				\item I VI'et \texttt{genererfrekvenssignal0.2.vi} indtastes den ønskede frekvens i den numeriske kontrol \textit{Enter Frequency}. 
					\begin{enumerate}
						\item Koden eksekveres ved at trykke på \textit{Run}. 
					\end{enumerate} 
				\item I VI'et \texttt{optagefrekvenssignal0.1.vi} trykkes på \textit{Run}. 
					\begin{enumerate}
						\item Den maksimale optagede frekvens aflæses i \textit{Max Frequency}. 
					\end{enumerate}	 	
			\end{enumerate}
			
			
			\textit{2. deltest}			
			\begin{enumerate}
				\item Mikrofonen holdes manuelt, vendt mod PC'ens højtaler, i en afstand på 5 cm. 
				\item I \texttt{onlinetonegenerator.com} genereres et signal med den ønskede frekvens. 
				\item I VI'et \texttt{optagefrekvenssignal0.1.vi} trykkes på \textit{Run}. 
					\begin{enumerate}
						\item Den maksimale optagede frekvens aflæses i \textit{Max Frequency}. 
					\end{enumerate}	  
			\end{enumerate}
		
			Deltestene gentages med frekvenser på: 100 Hz, 150 Hz, 200 Hz, 400 Hz, 500 Hz, 600 Hz, 700, 1000 0g 1200 Hz, og resultaterne sammenholdes. 
			
			\subsubsection{Resultater}
			 Det blev observeret, at resultaterne fra det generede frekvenssignal i VI'et \texttt{genererfrekvenssignal0.2.vi} var grundtonens harmoniske overtoner, idet frekvensen udsendes som et firkantsignal. Kun ved højfrekvente signaler (<1 kHz), blev grundtonen opfanget. 
			 Ved at benytte \texttt{onlinetonegenerator.com}, kunne der udsendes et sinussignal med en given frekvens, som blev korrekt opfanget (+/- 0.5 Hz).     
			\subsubsection{Diskussion}
			Der opnås pæne resultater ved at bruge et sinussignal, men der er desværre meget kompliceret at generere sinussignaler til en Arduino.   
			\subsubsection{Konklusion}
			Det konkluderes, at der ikke er fejl i software og hardware, og de unøjagtige resultater skyldes firkantsignalets harmoniske overtoner.  
			\subsubsection{Aktion}
			Det skal undersøges, om det er muligt at filtrere firkantssignalets harmoniske overtoner fra, således firkantssignalet kan benyttes. 
		
		\subsection{Bordtest nr. 4} %% Skal måske omdøbes?!
		\subsubsection{Testformål}
		Det undersøges, om en resonatorligende beholder kan dæmpe de harmoniske overtoner fra firkantsignalet. 
		
		\subsubsection{Produktspecifikationer}
		
		\textit{Hardware:}\\
		\tores\\
		\højtalerkabel\\
		\kabelsko\\
		\pins\\
		\arduino\\
		\PC\\
		\usbkabel\\
	
		\textit{Software:}\\
		\labview\\
		\visa\\
		\vi\\
		\ardsw\\
		
		
		\subsubsection{Opstilling og opsætning}
		\textit{1. delopstilling}:\\
		Højtaleren er loddet til højtalerkablets ene ende, og pin headerne er loddet til kablets anden ende. 
		Pin headerne er isat Arduino'en i pin 46 (PL3(OC5A)), som et er digitalt PWM output, og til ground (GND). 
		Arduino'en er med et USB kabel koblet til PC'en. 		
		På PC'en er VI'et \texttt{genererfrekvenssignal0.2.vi} åbnet i LabVIEW. Testopstillingen for denne del er vist i figur \ref{fig:etha}, og blokdiagrammet er vist i figur \ref{fig:bdgenerer}.\\ 
 
		\textit{2. delopstilling}:\\
		Mikrofonen er sat i PC'ens minijack-stik. På PC'en er VI'et \texttt{optagefrekvenssignal0.1.vi} åbnet i LabVIEW. Testopstillingen for denne del er vist i figur \ref{fig:etmik}, og blokdiagrammet er vist i figur \ref{fig:bdoptage}.\\  
		
		\textit{3. delopstilling}:\\
		I en internetbrowser er hjemmesiden \texttt{www.onlinetonegenerator.com} åbnet, og PC'ens højtalere er slået til. 
		
		\subsubsection{Udførsel}
			
			\textit{1. deltest}
			\begin{enumerate}
				\item Højtaleren holdes manuelt således membranen står i lodret position. 
				\item Mikrofonen holdes manuelt, vendt mod højtaleren, i en afstand på 5 cm. 
				\item I VI'et \texttt{genererfrekvenssignal0.2.vi} indtastes den ønskede frekvens i den numeriske kontrol \textit{Enter Frequency}. 
					\begin{enumerate}
						\item Koden eksekveres ved at trykke på \textit{Run}. 
					\end{enumerate} 
				\item I VI'et \texttt{optagefrekvenssignal0.1.vi} trykkes på \textit{Run}. 
					\begin{enumerate}
						\item Den maksimale optagede frekvens aflæses i \textit{Max Frequency}. 
					\end{enumerate}	 	
			\end{enumerate}
			
			
			\textit{2. deltest}			
			\begin{enumerate}
				\item Mikrofonen holdes manuelt, vendt mod PC'ens højtaler, i en afstand på 5 cm. 
				\item I \texttt{onlinetonegenerator.com} genereres et signal med den ønskede frekvens. 
				\item I VI'et \texttt{optagefrekvenssignal0.1.vi} trykkes på \textit{Run}. 
					\begin{enumerate}
						\item Den maksimale optagede frekvens aflæses i \textit{Max Frequency}. 
					\end{enumerate}	  
			\end{enumerate}
		
			Deltestene gentages med frekvenser på: 100 Hz, 150 Hz, 200 Hz, 400 Hz, 500 Hz, 600 Hz, 700, 1000 0g 1200 Hz, og resultaterne sammenholdes. 
			
			\subsubsection{Resultater}
			 Det blev observeret, at resultaterne fra det generede frekvenssignal i VI'et \texttt{genererfrekvenssignal0.2.vi} var grundtonens harmoniske overtoner, idet frekvensen udsendes som et firkantsignal. Kun ved højfrekvente signaler (<1 kHz), blev grundtonen opfanget. 
			 Ved at benytte \texttt{onlinetonegenerator.com}, kunne der udsendes et sinussignal med en given frekvens, som blev korrekt opfanget (+/- 0.5 Hz).     
			\subsubsection{Diskussion}
			Der opnås pæne resultater ved at bruge et sinussignal, men der er desværre meget kompliceret at generere sinussignaler til en Arduino.   
			\subsubsection{Konklusion}
			Det konkluderes, at der ikke er fejl i software og hardware, og de unøjagtige resultater skyldes firkantsignalets harmoniske overtoner.  
			\subsubsection{Aktion}
			Det skal undersøges, om det er muligt at filtrere firkantssignalets harmoniske overtoner fra, således firkantssignalet kan benyttes. 


		
	
	
	

	