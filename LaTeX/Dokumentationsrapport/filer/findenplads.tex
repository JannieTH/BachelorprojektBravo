\chapter{Find en plads til det her}

\section{Bestemmelse af volumen af resonator}

Volumen af resonatoren skal kendes for at bestemme resonansfrekvensen i det tomme kammer. Volumen er udregnet ved at hælde vand i resonatoren, og derefter måle vandets vægt. Ud fra vandets vægt kunne volumen gives. 

\begin{equation}
  V=\frac{m}{\rho}
  \label{eq:vformel}
\end{equation}

\begin{equation}
V_{resonator}=\frac{\SI{1698}{\micro\gram \per \centi\meter^{3}}}{\SI{1000}{\kilo \gram \per \meter^{3}}}
\end{equation}

\begin{equation}
	V_{resonator}={\SI{1.698}{\meter^{3}}}={\SI{1.698}{\liter}}
\end{equation}

\subsection{Bestemmelse af volumen af resonator ekskl. halsvolumen}

Volumen af resonatoren ekskl. halsvolumen er udregnet på samme fremgangsmåde ved brug af ligning \ref{eq:vformel}. 

\begin{equation}
V_{resonator}=\frac{\SI{1671}{\micro\gram \per \centi\meter^{3}}}{\SI{1000}{\kilo \gram \per \meter^{3}}}={\SI{1.671}{\meter^{3}}}={\SI{1.671}{\liter}}
\end{equation}

\section{Udregning af volumen}

Helmholtz resonansen i en tom resonator er givet ved følgende formel: \fnul

hvor $f_{0}$: resonansfrekvens i en tom resonator [Hz],\\
$c$: lydens hastighed i luft,\\
$S_{p}$: tværsnitsareal af port,\\
$V$: statisk volumen af resonator,\\
$l_{p}$: længde af port,\\
$\Delta l$: endekorrektion\\

Lydens hastighed i luft er givet ved formlen: 
\c
hvor $T_{K}$ er givet ved \T. 
Derved kan lydens hastighed i luft ved en temperatur på 23\degree C bestemmes: 
\begin{equation}
		c = {\SI{331,5}{\meter / \second}}
		\sqrt{\frac{\SI{296,15}{\kelvin}}{\SI{273,15}{\kelvin}}} = {\SI{345,175}{\meter / \second}} 	\end{equation}
		
Tværsnitsarealet af porten bestemmes ved: 
\Sp 
hvor ${r}$ er radius er porten. 
Derved bliver $S_{p}$ ved en radius på ${\SI{1.75}{\centi \meter}}$:  
\begin{equation}
S_{p} = ({\SI{0,0175}{\meter}})^{2}\pi = {\SI{0,000962}{\meter^{2}}}	
\end{equation}  





