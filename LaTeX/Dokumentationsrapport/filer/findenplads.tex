\chapter{Find en plads til det her}

\section{Bestemmelse af volumen af resonator}

Volumen af resonatoren skal kendes for at bestemme resonansfrekvensen i det tomme kammer. Volumen er udregnet ved at hælde vand i resonatoren, og derefter måle vandets vægt. Ud fra vandets vægt kunne volumen gives. Der gøres opmærksom på at volumen af resonatorens hals \textit{ikke} medtages. 

Volumen af resonatoren $V$ er ekskl. port/ hals. Volumen er udregnet ved at hælde vand i resonatoren, og derefter måle vandets vægt. Ud fra vandets vægt kan volumen gives vha. følgende formel:  
\begin{equation}
  V=\frac{m}{\rho}
  \label{eq:vformel}
\end{equation}
$\Downarrow$
\begin{equation}
V_{resonator}=\frac{\SI{1,671}{\kilo\gram}}{\SI{1000}{\kilo \gram \per \meter^{3}}}={\SI{0,001671}{\meter^{3}}}
\end{equation}


\section{Udregning af volumen}

Helmholtz resonansen i en tom resonator er givet ved følgende formel: \f

hvor 
\begin{description}[align=left, labelwidth=1in,labelindent=0.5cm]
\item $f_{0}$: resonansfrekvens i en tom resonator [$Hz$],\\
\item $c$: lydens hastighed i luft [$m/s$],\\
\item $S_{p}$: tværsnitsareal af port [$m^2$],\\
\item $V$: statisk volumen af resonator [$m^3$],\\
\item $l_{p}$: længde af port [$m$],\\
\item $\Delta l$: endekorrektion [$m$]\\
\end{description}
Lydens hastighed i luft $c$ er givet ved formlen: 
\c
hvor $T_{K}$ er givet ved \T. 
Derved kan lydens hastighed i luft ved en temperatur på 23\degree C bestemmes: 
\begin{equation}
		c = {\SI{331,5}{\meter / \second}} \cdot
		\sqrt{\frac{\SI{296,15}{\kelvin}}{\SI{273,15}{\kelvin}}} = {\SI{345,175}{\meter / \second}} 	\end{equation}
		
Tværsnitsarealet $S_{p}$ af porten bestemmes ved: 
\Sp 
hvor ${r}$ er radius er porten. 
Derved bliver $S_{p}$ ved en radius på ${\SI{1.75}{\centi \meter}}$:  
\begin{equation}
S_{p} = ({\SI{0,0175}{\meter}})^{2}\pi = {\SI{0,000962}{\meter^{2}}}	
\end{equation}  

Endekorrektionen $\Delta l$ er en værdi som tillægges, som korrektion for den medsvingende luft, der i resonatoren fungerer som masse. 

$\Delta l$ gives ved: \deltal
$\Downarrow$
\begin{equation}
		\Delta l = 0,6 \cdot {\SI{0,0175}{\meter}} + \frac{8}{3\pi} \cdot {\SI{0,0175}{\meter}} = {\SI{0,025354}{\meter}} 
\end{equation}

Resonansfrekvens $f_{0}$ i en tom resonator ved en lufttemperatur på 23\degree C er altså:

\f
$\Downarrow$   
\begin{equation}
		f_{0} = \frac{\SI{345,175}{\meter / \second}}{2\pi}\sqrt{\frac{\SI{0,000962}{\meter^{2}}}{{\SI{0,001671}{\meter^{3}}}({\SI{0,034}{\meter}}+{\SI{0,025354}{\meter}})}} 
	\end{equation}	
$\Downarrow$
\begin{equation}
	f_{0} = {\SI{171,094}{\second}^{-1}} \approxeq {\SI{95,4}{\hertz}} 	
	\end{equation}
	




Volumen af resonatoren skal kendes for at verificere den målte resonansfrekvens i det tomme kammer. 
Volumen af resonatoren $V$ er ekskl. port/ hals. Volumen er udregnet ved at hælde vand i resonatoren, og derefter måle vandets vægt. Ud fra vandets vægt kan volumen gives vha. følgende formel:  
\begin{equation}
  V=\frac{m}{\rho}
  \label{eq:vformel}
\end{equation}
$\Downarrow$
\begin{equation}
V_{resonator}=\frac{\SI{1,671}{\kilo\gram}}{\SI{1000}{\kilo \gram \per \meter^{3}}}={\SI{0,001671}{\meter^{3}}}
\end{equation}


Helmholtz resonansen i en tom resonator er givet ved følgende formel: \f
hvor 
\begin{description}[align=left, labelwidth=1in,labelindent=0.5cm]
\item $f_{0}$: resonansfrekvens i en tom resonator [$Hz$],\\
\item $c$: lydens hastighed i luft [$m/s$],\\
\item $S_{p}$: tværsnitsareal af port [$m^2$],\\
\item $V$: statisk volumen af resonator [$m^3$],\\
\item $l_{p}$: længde af port [$m$],\\
\item $\Delta l$: endekorrektion [$m$]\\
\end{description}
Lydens hastighed i luft $c$ er givet ved formlen: 
\c
hvor $T_{K}$ er givet ved \T. 
Derved kan lydens hastighed i luft ved en temperatur på 23\degree C bestemmes: 
\begin{equation}
		c = {\SI{331,5}{\meter / \second}} \cdot
		\sqrt{\frac{\SI{296,15}{\kelvin}}{\SI{273,15}{\kelvin}}} = {\SI{345,175}{\meter / \second}} 	\end{equation}
		
Tværsnitsarealet $S_{p}$ af porten bestemmes ved: 
\Sp 
hvor ${r}$ er radius er porten. 
Derved bliver $S_{p}$ ved en radius på ${\SI{1.75}{\centi \meter}}$:  
\begin{equation}
S_{p} = ({\SI{0,0175}{\meter}})^{2}\pi = {\SI{0,000962}{\meter^{2}}}	
\end{equation}  

Volumen af resonatoren $V$ er ekskl. port/ hals. Volumen er udregnet ved at hælde vand i resonatoren, og derefter måle vandets vægt. Ud fra vandets vægt kan volumen gives vha. følgende formel:  
\begin{equation}
  V=\frac{m}{\rho}
  \label{eq:vformel}
\end{equation}
$\Downarrow$
\begin{equation}
V_{resonator}=\frac{\SI{1,671}{\kilo\gram}}{\SI{1000}{\kilo \gram \per \meter^{3}}}={\SI{0,001671}{\meter^{3}}}
\end{equation}

Endekorrektionen $\Delta l$ er en værdi som tillægges, som korrektion for den medsvingende luft, der i resonatoren fungerer som masse. 

$\Delta l$ gives ved: \deltal
$\Downarrow$
\begin{equation}
		\Delta l = 0,6 \cdot {\SI{0,0175}{\meter}} + \frac{8}{3\pi} \cdot {\SI{0,0175}{\meter}} = {\SI{0,157108}{\meter}} 
\end{equation}

Resonansfrekvens $f_{0}$ i en tom resonator ved en lufttemperatur på 23\degree C er altså:

\f
$\Downarrow$   
\begin{equation}
		f_{0} = \frac{\SI{345,175}{\meter / \second}}{2\pi}\sqrt{\frac{\SI{0,000962}{\meter^{2}}}{{\SI{0,001671}{\meter^{3}}}({\SI{0,034}{\meter}}+{\SI{0,157108}{\meter}})}} 
	\end{equation}	
$\Downarrow$
\begin{equation}
	f_{0} = {\SI{95,3497}{\second}^{-1}} \approxeq {\SI{95,4}{\hertz}} 	
	\end{equation}
	















