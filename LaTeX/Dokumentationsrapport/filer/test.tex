
\chapter{Dokumentation af accepttest}
	\section{Indledning}
	Dette kapitel indeholder dokumentation for udarbejdelsen af accepttesten, som er et værktøj til at validere de specificerede krav fra €.   
	 		  
		\subsection{Formål}
		
	
		\subsection{Læsevejledning}	
		
				
		\subsection{Versionshistorik}
		
		\vspace{5mm}
		
			\begin{figure*}[htb]
				
			\end{figure*}
			
			
	\section{Accepttest-protokol}
	Denne protokol beskriver alle de forhold og forudsætninger, som skal være opfyldt for at kunne udføre accepttest af den akustiske brystvolumenmåler. 
	  
		\subsection{Formål}
		Formålet med denne protokol er at specificere accepttest-aktivteterne gældnede for brystvolumenmåleren. 
		
		\subsection{Referencer}
		€€ Eksempelvis UC\#1
		
		\subsection{Forkortelser}
		€€
		  
		\subsection{Ansvar}
		\textit{		Ved underskrift på protokollen bekræfter: }\\
		
		Kunde - Projektansvarlig
		\begin{itemize}
			\item \textit{at indholdet er komplet og entydigt}
			\item \textit{at det tekniske indhold og metoder er i orden og dækkende for de planlagte tests}
			\item \textit{at den projektansvarlige er enig i omfang, metode og acceptkriterier}
		\end{itemize}
		
		Udviklere - Projektansvarlige
		\begin{itemize}
			\item \textit{at det tekniske indhold og metoder er i orden og dækkende for de planlagte tests}
		\end{itemize}
		
		\subsection{Udstyrsbeskrivelse}
		Den akustiske brystvolumenmåler består af en Helmholtz resonator, hvor der er påmonteret en højtaler og en mikrofon. Højtaleren og mikrofonen er koblet til en Arduino Mega 2560 R3. Arduino'en er koblet til en PC, hvor der skrives til Arduino i LabVIEW. 
		Brystvolumenmåleren benyttes af en plastikkirurg, som bruger BMV'en når han ønsker et objektivt volumenmål på et bryst. Når en måling intialiseres med BVM'en afsendes en lyd i resonatoren via højtaleren. Mikrofonen på resonatoren opsamler den reflekterede lyd, og udfra den reflekterede lyd, udregner en algoritme størrelsen på brystvolumen. 			
		\begin{figure}[htb]
			\centering
			\includegraphics[width=5in]{systembeskrivelse}
			\caption{Beskrivelse af systemets komponenter €€€€}
			\label{system}
		\end{figure}
		
		\subsection{Acceptkriterier}
		\textit{Acceptkriterierne er afledt af de forhold, der er specificeret i Kravspecifikationen, kapitel \ref{Kravspec}. Acceptkriterierne er specificeret i de enkelte testskemaer.} 
		
		\subsection{Metode}
		Dette afsnit specificerer de retningslinjer, hvorefter accepttesten skal afvikles. 
		
			\subsubsection{Dokumentsammenhæng}
			illustreres 
			€€ Indsæt oversigt, med inspiration fra s. 204 	
			
			\subsubsection{Gennerelle krav}
			Afviklingen af accepttesten skal overholde følgende nøglekrav: 
			
			\begin{itemize}
				\item \textit{Protokollen skal være godkendt, før aktiviteter udføres}
				\item \textit{Afvigelser skal registreres og rettelser udføres}
				\item \textit{Dataindsamling og registreringer skal udføres i overenstemmelse med >>god testpraksis<< }
				\item \textit{Data skal granskes for rigtighed og fuldstændighed }
				\item \textit{Accepttesten skal udarbejdes og godkendes, som bevis på den gennemførte accepttest}
			\end{itemize}
			
			\subsubsection{Acceptkriterier}
			 Dette afsnit beskriver, hvordan testskemaerne specifikt skal udfyldes. 
			 
			 \begin{itemize}
				\item \textit{Alle krav skal opfylde de specificerede acceptkriterier i hvert enkelte testskema.}
				\item \textit{De aktuelle observationer skal svare til de forventede observationer}
				\item \textit{I >>Godkendt<<-kolonnen skrives >>Ja<<, hvis acceptkriteriet er overholdt.} 
				\item \textit{I >>Godkendt<<-kolonnen skrives >>Nej<< samt afvigelsesnummer, hvis acceptkriteriet ikke er overholdt.}  \item \textit{Kolonnen >>Init./Dato<< skal udfyldes ved hver udførsel af test}
			\end{itemize}
			
€€€€ Husk an skrive at da vi ikke kører acceptesten, vil der ikke være udarbejede afvigelsesskemaer
			\subsubsection{Afvigelseshåndtering}	
			Afvigelser registreres under udførslen af accepttesten og dokumenteres ved udarbejdelse af afvigelsesbilag, hvor følgende punkter opfyldes:
			
			\begin{enumerate}[label=\alph*)]
				\item \textit{Afvigelse og årsag til afvigelsen beskrives}
				\item \textit{Aktion for opfyldelse af acceptkriterier angives}
				\item \textit{Resultat af gennemført aktion dokumenteres}	
			\end{enumerate}
			
			\subsubsection{Afslutning af accepttest-aktiviteter}
			Ved afslutningen af accepttest-aktivteterne, udarbejdes en accepttest-rapport. Denne rapport lukker accepttest-aktiviteterne, og fungerer som bevis for, at den gennemførte test af brystvolumenmåleren, er afsluttet med et tilfredsstillende resultat. 
			
			Accepttest-rapporten omhandler følgende punkter: 
			\begin{itemize}
				\item \textit{Konklusion på den gennemførte accepttest}
				\item \textit{Kopi af godkendt protokol}
				\item \textit{Afvigelsesblad med korrektive aktioner}
				\item \textit{Udfyldte testskemaer}
			\end{itemize}
			
		\subsection{Oversigt over testdokumenter}
		€€ Her angives de specifikke testdokumenter i en tabel. Obs, på eventuelle referencer til kravspec. 
		
		\subsection{Forudsætning for udførelse af accepttest}
		€€ Hvad forudsætter afgørelsen for igangsættelse af accepttest / når det og det er opfyldt, startes accepttest 
		
		\section{Accepttest}

		
\subsection{Funktionelle krav}

\subsubsection{Use Case 1}
\textbf{Test Case:} normalforløb \\
\textbf{Testforberedelse:} BVM skal være tændt. BVM skal være tændt. Dockingstation til kalibrering skal være tilgængelig. Ved brug af dockingsstation kendes volumen af tom resonator. \\

\begin{tabularx}{1\textwidth}{|l|X|}
\hline
\textbf{Krav nr.}              & UC1.1  \\ \hline
\textbf{Acceptkriterie}        & BVM er kalibreret \\ \hline
\textbf{Testmetode}            & Der trykkes på \textit{Kalibrér} og UI viser \textit{Kalibrering ok} \\ \hline
\textbf{Observation}           &  \\ \hline
\textbf{Godkendt {[}Ja/Nej{]}} &  \\ \hline
\textbf{Init./Dato}            &  \\ \hline
\end{tabularx}
				


\begin{tabularx}{1\textwidth}{|l|X|}
\hline
\textbf{Krav nr.}              & UC1.2-3  \\ \hline
\textbf{Acceptkriterie}        & Patientens bryst er omsluttet af BVM \\ \hline
\textbf{Testmetode}            & Der observeres efter luftlommer mellem BVM og hud \\ \hline
\textbf{Observation}           &  \\ \hline
\textbf{Godkendt {[}Ja/Nej{]}} &  \\ \hline
\textbf{Init./Dato}            &  \\ \hline
\end{tabularx}
		

\begin{tabularx}{1\textwidth}{|l|X|}
\hline
\textbf{Krav nr.}              & UC1.4  \\ \hline
\textbf{Acceptkriterie}        & BVM er placeret med et ensartet anlægstryk \\ \hline
\textbf{Testmetode}            & Der påføres en kontinuert tryk indtil alle trykdioder på BVM lyser grøn.  \\ \hline
\textbf{Observation}           &  \\ \hline
\textbf{Godkendt {[}Ja/Nej{]}} &  \\ \hline
\textbf{Init./Dato}            &  \\ \hline
\end{tabularx}
		
			

\begin{tabularx}{1\textwidth}{|l|X|}
\hline
\textbf{Krav nr.}              & UC1.5-6  \\ \hline
\textbf{Acceptkriterie}        & Der er foretaget en måling\\ \hline
\textbf{Testmetode}            & Der trykkes på \textit{M} og en progressbar angiver målingsstatus. Volumen angives på UI når måling er foretaget. \\ \hline
\textbf{Observation}           &  \\ \hline
\textbf{Godkendt {[}Ja/Nej{]}} &  \\ \hline
\textbf{Init./Dato}            &  \\ \hline
\end{tabularx}


\begin{tabularx}{1\textwidth}{|l|X|}
\hline
\textbf{Krav nr.}              & UC1.7  \\ \hline
\textbf{Acceptkriterie}        & BVM slukkes\\ \hline
\textbf{Testmetode}            & Der trykkes på \textit{O/I} \\ \hline
\textbf{Observation}           &  \\ \hline
\textbf{Godkendt {[}Ja/Nej{]}} &  \\ \hline
\textbf{Init./Dato}            &  \\ \hline
\end{tabularx}\\
	
		
\pagebreak
\textbf{Test Case:} alternativ flow\\
\textbf{Testforberedelse:} BVM skal være tændt. Dockingstation til kalibrering skal være tilgængelig. Ved brug af dockingsstation kendes volumen af tom resonator. \\


\begin{tabularx}{1\textwidth}{|l|X|}
\hline
\textbf{Krav nr.}              & UC1.A1  \\ 
\hline
\textbf{Scenarie}              & BVM melder fejl  \\ \hline
\textbf{Acceptkriterie}        & BVM melder klar til måling\\ \hline
\textbf{Testmetode}            & Kalibrering af BVM \\ \hline
\textbf{Observation}           &  \\ \hline
\textbf{Godkendt {[}Ja/Nej{]}} &  \\ \hline
\textbf{Init./Dato}            &  \\ \hline
\end{tabularx}




\subsection{Ikke-funktionelle krav}

\vspace{5mm}

\subsubsection{Test af usability krav}

\begin{tabularx}{1\textwidth}{|l|X|}
\hline
\textbf{Krav nr.}              & UK1  \\ \hline
\textbf{Acceptkriterie}        & ml. anføres efter talværdi \\ \hline
\textbf{Testmetode}            & UC1.5-6 følges. Det aflæses på UI om talværdi er angivet i ml.  \\ \hline
\textbf{Observation}           &  \\ \hline
\textbf{Godkendt {[}Ja/Nej{]}} &  \\ \hline
\textbf{Init./Dato}            &  \\ \hline
\end{tabularx}

\begin{tabularx}{1\textwidth}{|l|X|}
\hline
\textbf{Krav nr.}              & UK2  \\ \hline
\textbf{Acceptkriterie}        & Talværdi angives i digital talform med en højde på 1 cm. \\ \hline
\textbf{Testmetode}            & \begin{tabular}[l]{@{}l@{}}UC1.5-6 følges.\\ Det observeres på UI om talværdi er angivet\\ i digital form og skriftstørrelse måles\end{tabular}  \\ \hline
\textbf{Observation}           &  \\ \hline
\textbf{Godkendt {[}Ja/Nej{]}} &  \\ \hline
\textbf{Init./Dato}            &  \\ \hline
\end{tabularx}

\begin{tabularx}{1\textwidth}{|l|X|}
\hline
\textbf{Krav nr.}              & UK3  \\ \hline
\textbf{Acceptkriterie}        & Der anvendes ikke andet sprog end engelsk \\ \hline
\textbf{Testmetode}            & \begin{tabular}[l]{@{}l@{}}UC1 inkl. alternativ flow gennemføres og kontrollerer \\at sproget er engelsk\end{tabular} \\ \hline
\textbf{Observation}           &  \\ \hline
\textbf{Godkendt {[}Ja/Nej{]}} &  \\ \hline
\textbf{Init./Dato}            &  \\ \hline
\end{tabularx}


\vspace{5mm}

\subsubsection{Test af reliability krav}

\begin{tabularx}{1\textwidth}{|l|X|}
\hline
\textbf{Krav nr.}              & RK1  \\ \hline
\textbf{Acceptkriterie}        & Nøjagtighed på +/- 10 ml.  \\ \hline
\textbf{Testmetode}            & \begin{tabular}[l]{@{}l@{}} En testserie på 1000 volumenmålinger genereres hvorpå\\ nøjagtighed udregnes\end{tabular}  \\ \hline
\textbf{Observation}           &  \\ \hline
\textbf{Godkendt {[}Ja/Nej{]}} &  \\ \hline
\textbf{Init./Dato}            &  \\ \hline
\end{tabularx}

\begin{tabularx}{1\textwidth}{|l|X|}
\hline
\textbf{Krav nr.}              & RK2  \\ \hline
\textbf{Acceptkriterie}        & Afvigelse af præcision på højst 1\% af $f_{b}$  \\ \hline
\textbf{Testmetode}            & \begin{tabular}[l]{@{}l@{}} En testserie på 1000 målinger af $f_{b}$ genereres hvorpå\\ præcision udregnes\end{tabular}  \\ \hline
\textbf{Observation}           &  \\ \hline
\textbf{Godkendt {[}Ja/Nej{]}} &  \\ \hline
\textbf{Init./Dato}            &  \\ \hline
\end{tabularx}

\begin{tabularx}{1\textwidth}{|l|X|}
\hline
\textbf{Krav nr.}              & RK3  \\ \hline
\textbf{Acceptkriterie}        & \begin{tabular}[l]{@{}l@{}} Volumenangivelse skal\\ afrundes til nærmeste hele tal \end{tabular} \\ \hline
\textbf{Testmetode}            & \begin{tabular}[l]{@{}l@{}} UC1.5-6 følges og det kontrolleres\\ at volumen ikke er angivet som decimaltal\end{tabular}  \\ \hline
\textbf{Observation}           &  \\ \hline
\textbf{Godkendt {[}Ja/Nej{]}} &  \\ \hline
\textbf{Init./Dato}            &  \\ \hline
\end{tabularx}

\begin{tabularx}{1\textwidth}{|l|X|}
\hline
\textbf{Krav nr.}              & RK4  \\ \hline
\textbf{Acceptkriterie}        & Der må forekomme fejl ved hver 1000. måling  \\ \hline
\textbf{Testmetode}            & \begin{tabular}[l]{@{}l@{}} En testserie på 1000 volumenmålinger generes hvorpå\\ eventuelle fejl registreres \end{tabular}  \\ \hline
\textbf{Observation}           &  \\ \hline
\textbf{Godkendt {[}Ja/Nej{]}} &  \\ \hline
\textbf{Init./Dato}            &  \\ \hline
\end{tabularx}

\begin{tabularx}{1\textwidth}{|l|X|}
\hline
\textbf{Krav nr.}              & RK5  \\ \hline
\textbf{Acceptkriterie}        & Varighed af fejlløsning må vare højst 2 minutter  \\ \hline
\textbf{Testmetode}            & \begin{tabular}[l]{@{}l@{}} UC1.A1 følges og varighed registreres \end{tabular}  \\ \hline
\textbf{Observation}           &  \\ \hline
\textbf{Godkendt {[}Ja/Nej{]}} &  \\ \hline
\textbf{Init./Dato}            &  \\ \hline
\end{tabularx}

\begin{tabularx}{1\textwidth}{|l|X|}
\hline
\textbf{Krav nr.}              & RK6  \\ \hline
\textbf{Acceptkriterie}        & Kalibrering efter aktuel lufttemperatur  \\ \hline
\textbf{Testmetode}            & \begin{tabular}[l]{@{}l@{}} UC1.1 følges og temperatur aflæses på UI og\\ sammenholdes med målt temperatur i rummet \end{tabular}  \\ \hline
\textbf{Observation}           &  \\ \hline
\textbf{Godkendt {[}Ja/Nej{]}} &  \\ \hline
\textbf{Init./Dato}            &  \\ \hline
\end{tabularx}

\vspace{5mm}

\subsubsection{Test af performance krav}

\begin{tabularx}{1\textwidth}{|l|X|}
\hline
\textbf{Krav nr.}              & PK1  \\ \hline
\textbf{Acceptkriterie}        & Måling skal være foretaget inden for 10 sekunder  \\ \hline
\textbf{Testmetode}            & \begin{tabular}[l]{@{}l@{}} UC1 følges og varighed registreres \end{tabular}  \\ \hline
\textbf{Observation}           &  \\ \hline
\textbf{Godkendt {[}Ja/Nej{]}} &  \\ \hline
\textbf{Init./Dato}            &  \\ \hline
\end{tabularx}

\begin{tabularx}{1\textwidth}{|l|X|}
\hline
\textbf{Krav nr.}              & PK2  \\ \hline
\textbf{Acceptkriterie}        & Lydniveau ved måling skal være mindre end 75 dB \\ \hline
\textbf{Testmetode}            & \begin{tabular}[l]{@{}l@{}} UC1 følges og lydniveau registreres med lydmåler \end{tabular}  \\ \hline
\textbf{Observation}           &  \\ \hline
\textbf{Godkendt {[}Ja/Nej{]}} &  \\ \hline
\textbf{Init./Dato}            &  \\ \hline
\end{tabularx}

\vspace{5mm}

\subsubsection{Test af supportability krav}

\begin{tabularx}{1\textwidth}{|l|X|}
\hline
\textbf{Krav nr.}              & SK1  \\ \hline
\textbf{Acceptkriterie}        & Produktmateriale skal tåle rengøring med sprit \\ \hline
\textbf{Testmetode}            & \begin{tabular}[l]{@{}l@{}} Materiale rengøres med sprit og kontrolleres for skader \end{tabular}  \\ \hline
\textbf{Observation}           &  \\ \hline
\textbf{Godkendt {[}Ja/Nej{]}} &  \\ \hline
\textbf{Init./Dato}            &  \\ \hline
\end{tabularx}

\begin{tabularx}{1\textwidth}{|l|X|}
\hline
\textbf{Krav nr.}              & SK2  \\ \hline
\textbf{Acceptkriterie}        & UI skal tåle rengøring med sprit \\ \hline
\textbf{Testmetode}            & \begin{tabular}[l]{@{}l@{}} UI rengøres med sprit og kontrolleres for skader \end{tabular}  \\ \hline
\textbf{Observation}           &  \\ \hline
\textbf{Godkendt {[}Ja/Nej{]}} &  \\ \hline
\textbf{Init./Dato}            &  \\ \hline
\end{tabularx}

\vspace{5mm}

\subsubsection{Test af design krav}

\begin{tabularx}{1\textwidth}{|l|X|}
\hline
\textbf{Krav nr.}              & DK1  \\ \hline
\textbf{Acceptkriterie}        & Skal kunne anvendes uden brugermanual \\ \hline
\textbf{Testmetode}            & \begin{tabular}[l]{@{}l@{}} Minimum tyve plastikkirurger skal, uden kendskab til\\ brugermanual, intuitivt anvende BVM \end{tabular}  \\ \hline
\textbf{Observation}           &  \\ \hline
\textbf{Godkendt {[}Ja/Nej{]}} &  \\ \hline
\textbf{Init./Dato}            &  \\ \hline
\end{tabularx}

\vspace{2mm}
DK1 kan også anses for at være et usabilitykrav og kan testes iht. ANSI/AAMI HE75:20097/(R)2013(ref€€€€)

\begin{tabularx}{1\textwidth}{|l|X|}
\hline
\textbf{Krav nr.}              & DK2  \\ \hline
\textbf{Acceptkriterie}        & Anvendes på 90\% af tilfældigt udvalgte kvinder \\ \hline
\textbf{Testmetode}            & \begin{tabular}[l]{@{}l@{}} Ud af en testpopulation på 100 kvinder, må der findes\\ 10 kvinder som BVM ikke kan anvendes på\end{tabular}  \\ \hline
\textbf{Observation}           &  \\ \hline
\textbf{Godkendt {[}Ja/Nej{]}} &  \\ \hline
\textbf{Init./Dato}            &  \\ \hline
\end{tabularx}

\begin{tabularx}{1\textwidth}{|l|X|}
\hline
\textbf{Krav nr.}              & DK3  \\ \hline
\textbf{Acceptkriterie}        & Må ikke giver gener under måling \\ \hline
\textbf{Testmetode}            & \begin{tabular}[l]{@{}l@{}} Ergoterapeut godkender udformning \end{tabular}  \\ \hline
\textbf{Observation}           &  \\ \hline
\textbf{Godkendt {[}Ja/Nej{]}} &  \\ \hline
\textbf{Init./Dato}            &  \\ \hline
\end{tabularx}

\begin{tabularx}{1\textwidth}{|l|X|}
\hline
\textbf{Krav nr.}              & DK4  \\ \hline
\textbf{Acceptkriterie}        & Arbejdsstilling skal være ergonimisk korrekt \\ \hline
\textbf{Testmetode}            & \begin{tabular}[l]{@{}l@{}} Ergoterapeut godkender arbejdsstilling \end{tabular}  \\ \hline
\textbf{Observation}           &  \\ \hline
\textbf{Godkendt {[}Ja/Nej{]}} &  \\ \hline
\textbf{Init./Dato}            &  \\ \hline
\end{tabularx}

\begin{tabularx}{1\textwidth}{|l|X|}
\hline
\textbf{Krav nr.}              & DK5  \\ \hline
\textbf{Acceptkriterie}        & Ingen eksterne tilkoblinger med kabler \\ \hline
\textbf{Testmetode}            & \begin{tabular}[l]{@{}l@{}} Det kontrolleres, at der ikke er monteret kabler fra\\ eksterne enheder \end{tabular}  \\ \hline
\textbf{Observation}           &  \\ \hline
\textbf{Godkendt {[}Ja/Nej{]}} &  \\ \hline
\textbf{Init./Dato}            &  \\ \hline
\end{tabularx}

\begin{tabularx}{1\textwidth}{|l|X|}
\hline
\textbf{Krav nr.}              & DK6  \\ \hline
\textbf{Acceptkriterie}        &\begin{tabular}[l]{@{}l@{}} Dokumentation for overensstemmelse med klassificering\\ Im i hht. MDD 93/42/EEC \end{tabular} \\ \hline
\textbf{Testmetode}            & \begin{tabular}[l]{@{}l@{}} CE-certificering er underskrevet af bemyndiget organ  \end{tabular}  \\ \hline
\textbf{Observation}           &  \\ \hline
\textbf{Godkendt {[}Ja/Nej{]}} &  \\ \hline
\textbf{Init./Dato}            &  \\ \hline
\end{tabularx}

	
		