
\chapter{Design}
\section{Indledning}
Dette dokument beskriver systemets design og systemarkitektur. Dokumentet er en generel præsentation og beskrivelse af systemet, herunder hvorledes brystvolumenmåleren er opbygget, hardware-  og softwaremæssigt. \\
Grundet manglende kendskab til Helmholtz' resonansteori i forbindelse med brystvolumenmåling, er opbygning af systemets HW og SW udviklet på baggrund af testerfaringer og ved at konferere med fagfolk; herunder Tore A. Skogberg og Lars G. Johansen, lektorer i akustik ved Ingeniørhøjskolen, Aarhus Universitet. Derudover er der ud fra en systematisk litteratursøgning fundet en begrænset mængde videnskabelig litteratur omhandlende Helmholtz' resonansteori i forbindelse med voluminer af objekter.

Designafsnittet er altså udviklet på baggrund af det aktulle system. 

\section{Udviklingsværktøjer}
Selve softwaren til systemet udvikles i LabVIEW, version 14.0. Følgende tilføjelsespakker er anvendt:
\begin{itemize}
\item \visa
\item \vi
\item \ardsw	
\item \daqsoft 
\end{itemize}

Disse pakker anvendes til at interagere med \arduino{} samt \daq. 

	\subsection{Microsoft Visio}   
	Microsoft Visio er et tegneprogram, som anvendes til at illustrere udviklingsdiagrammer. 
	
\section{Versionsstyring}
\subsection{GitHub}
GitHub er et versionsstyringsprogram, som i projektet anvendes til versionsstyring af dokumenter og LabVIEW-kode. GitHub
bygger på open source versionsstyringssystemet Git, hvor der løbende opdateres ændringer, så det nyeste dokumentation og LabVIEW-kode altid er tilgængeligt. SourceTree er anvendt som user interface til GitHub-funktionerne. I SourceTree vises et overblik over ændringer, og under de enkelte filer, kan det observeres, hvad der er ændret i den pågældende version. Samtidig knyttes der en kommentar ved hvert commit/ ændring. Dette fremgår af figur \ref{fig:git}. 

\begin{figure}[htb]
\centering
\includegraphics[width=5in]{github.png}
\caption{SourceTree viser overblik over ændringer i enkelte filer}
\label{fig:git}	
\end{figure}


\section{Hardware}
Dette afsnit dokumenterer hardwaren i systemet \textit{Brystvolumenmåler}, og indeholder dermed beskrivelser af systemets fysiske dele samt deres funktionalitet. Beskrivelserne er udarbejdet ud fra et begrænset kendskab til Helmholtz' resonansteori i forbindelse med brystvolumenmåling. Disse beskrivelser der derfor anvendt som udgangspunkt for et indledende design, som kan anvendes til at teste frem mod erfaringer omkring hardwareudvælgelse. De fysiske deles specifikationer er derfor ikke uddybet i dette dokument, og begrundelser og argumenter for, hvorfor de brugte komponenter er valgt, er baseret på testerfaringer. 

\subsection{Block Definition Diagram}
Der er udarbejdet et Block Definition Diagram (BDD), som fremgår af figur \ref{fig:bdd}, til at danne et overordnet overblik over de indgående fysiske dele. Systemet \textit{Brystvolumenmåler} indeholder fem fysiske blokke, hhv. A/D-konvertering, en lydgivende kilde, en resonator, en lydopfanger samt et processeringselement. Flow portene beskriver, hvad der kan gå gennem blokken (ind og/eller ud).  
 
\begin{figure}[htb]
\centering
\includegraphics[width=5.5in]{bdd.png}
\caption{BDD over HW-komponenter i systemet \textit{Brystvolumenmåler}}
\label{fig:bdd}	
\end{figure}

\subsection{Internal Block Diagram}
Der er udarbejdet et internal block diagram (IBD), som fremgår af figur \ref{fig:ibd}. IBD'et anvendes til at give en mere konkret beskrivelse af, hvordan de forskellige komponenter interagerer med hinanden. Diagrammet anvendes til at definere, hvilke signaltyper systemet skal indeholde, for at sikre kommunikation mellem interne dele. 

\begin{figure}[htb]
\centering
\includegraphics[width=6in]{ibd.jpg}
\caption{IBD over HW-komponenter i systemet \textit{Brystvolumenmåler}}
\label{fig:ibd}
\end{figure}

I tabel \ref{table:overblok} fremgår en oversigt over hvilke funktioner samt signaler den enkelte blokke indeholder. 

\begin{table}[htb]
\centering
\caption{Oversigt over de enkelte blokkes funktioner og signaltyper}
\label{table:overblok}
\begin{tabular}{|l|l|l|}
\hline
\textbf{Bloknavn} & \textbf{Funktionsbeskrivelse} & \textbf{Signaltype}\\ \hline
Speaker & udsender lyd & frekvenssignal\\ \hline
Resonator & tilbagekaster lyd & resonansfrekvens\\ \hline
Microphone & opfanger reflekteret lyd & frekvenssignal \\ \hline
Arduino & A/D konvertering & frekvenssignal og resonansfrekvens \\ \hline
Dataprocessering & behandling af data & frekvenssignal og resonansfrekvens\\ \hline
\end{tabular}
\end{table}

\subsection{Bestemmelse af hardware} 
Udvælgelsen af hardware er, som nævnt, sket på baggrund af testerfaringer samt ved at konferere med fagfolk. Selve udvælgelsen er dermed en proces, som er kørt sideløbende med selve testprocessen i implementeringsfasen. 
I tabel \ref{table:anvendthw} fremgår en oversigt over det anvendte hardware. 

\begin{table}[htb]
\centering
\caption{Oversigt over det anvendte hardware}
\label{table:anvendthw}
\begin{tabular}{|l|l|l|}
\hline
\textbf{Funktion} & \textbf{ID} & \textbf{Navn} \\ \hline
\multirow{3}{*}{Højtaler} & H1 & ASB-224-RC \\ \cline{2-3} 
 & H2 & Wide Band 2.5'' SB65WBAC25-4  \\ \cline{2-3} 
 & H3 & Multimedia USB Speaker HP-1800 \\ \hline
 Resonator & res1 & Tryksprøjte NSG 150 - Neptun (modificeret) \\ \hline
\multirow{2}{*}{Mikrofon} & M1 & Minijack PC Mikrofon  \\ \cline{2-3} 
 & M2 & Logitech HD WEBCAM C270 \\ \cline{2-3}
 & M3 & Electret Microphone BOB12758 \\ \cline{2-3}
 & M4 & Electret Microphone Amplifier MAX4466\\ \hline
 \multirow{1}{*}{A/D konverter} & C1 & Arduino Mega 2560 \\ \cline{2-3}
 & C2 & \daq \\ \hline
 Dataprocesser & D1 & \labview \\ \hline
\end{tabular}
\end{table}

\subsubsection{Valg af højtaler}
Udvælgelsesprocessen af højtaleren fremgår af figur \ref{fig:flowlyd}. I diagrammet definerer \textit{E\#}, i hvilken enhedstest den pågældende komponent er testet. Endvidere fremgår en kort beskrivelse af årsagen til erstatningen for komponenten. Altså blev H1 erstattet med H2 da der opstod et behov for et bredere frekvensbånd i det udsendte lydsignal. Endvidere blev H2 erstattet med H3, som et resultatet af en problematik vedr. opfyldelse af Nyquists samplingsteori, hvorefter højtaleren måtte være en ekstern del af systemet. Det medførte et behov for en højtaler med et bredt frekvensbånd samt et minijack stik. 

\begin{figure}[htb]
\centering
\includegraphics[width=3in]{flowdiagramelyd.png}
\caption{Flowdiagram over de udførte enhedstests af lydgivende kilder. E\# definerer, i hvilken enhedstest den pågældende komponent er testet}
\label{fig:flowlyd}
\end{figure}

\subsubsection{Valg af resonator}
Da en resonator skulle vælges, blev det besluttet at finde en flaske af hårdt plast. I et byggemarked blev en tryksprøjte fundet. Tryksprøjten er af hårdt plast således den \textit{ikke svinger med} og transmittering af lyden undgåes. Selve trykmodulet kunne afmonteres, og på Prototypeværkstedet, Ingeniørhøjskolen Aarhus Universitet, blev bundet savet af i et lige snit. Tryksprøjten kunne derefter anvendes som en resonator (res1).   

\begin{figure}[htbp]
  \begin{minipage}[b]{0.5\linewidth}
    \centering
    \includegraphics[width=\linewidth]{tryk.jpg}
    \caption{Tryksprøjte NSG 150 - Neptun}
    \label{fig:tryks}
  \end{minipage}
  \hspace{0.5cm}
  \begin{minipage}[b]{0.5\linewidth}
    \centering
    \includegraphics[width=\linewidth]{resonator.jpg}
    \caption{Resonator (res1)}
    \label{fig:res}
  \end{minipage}
\end{figure}

\subsubsection{Valg af mikrofon}

Udvælgelsesprocessen af mikrofonen fremgår af figur \ref{fig:flowop}. I diagrammet definerer \textit{E\#}, i hvilken enhedstest den pågældende komponent er testet. Endvidere fremgår en kort beskrivelse af årsagen til erstatningen for komponenten. 

Under udførslen af integrationstest I05, blev det observeret, at resultatet i VI'et \texttt{optagefrekvenssignal02.vi} blev opfanget af PC'ens indbyggede mikrofon og ikke minijack PC mikrofonen (M1). Der opstod en mistanke om problemet da der med en kritisk tilgang blev reflekteret over de \textit{for pæne} resultater - altså var resultaterne ens uanset M1's placering indeni samt uden for resonatoren. Der blev derefter testet ved at udtage M1 fra PC'en, hvorefter resultaterne stadig var ens. Dette medvirkede til en ny enhedstest, E04-M1, hvor M1 blev placeret i et andet rum med en lukket branddør imellem. Da der ikke blev opfanget et signal i LabVIEW fra M1 blev det konstateret, at M1 ikke var aktiv. Årsagen til problemstillingen skyldes, at M1 har et 3-pols stik, og derfor mangler en pol til lyd input. PC'en indlæser derfor M1 som en højtaler, og forsøger dermed at udsende lyd gennem mikrofonen. 
Løsningen på denne problemstilling er at anvende en mikrofon med 4-pols stik, en adapter eller en mikrofon med USB-stik.
Det blev forsøgt at optage lyd med et headset med indbygget mikrofon, som har et 4-pols minijack stik. Headsettet blev indlæst på PC´en, som et headset, og derfor var det ikke muligt at vælge headsetmikrofonen som lydkilde under PC'ens konfigurationsindstillinger. Det var heller ikke muligt at få forbindelse til mikrofonen gennem LabView. Konklusionen på denne problemstilling er, at der må sidde et 3-pols minijack hun stik i PC´en. Der er foretaget en internetsøgning på indholdet af stik i en Macbook Pro 2009 model for at understøtte denne konklusion. Det lykkedes ikke at finde specifikationer som klart udspecificerer hvilket hun minijackstik, der er indbygget i omhandlende PC. Det vælges at gå videre til test med et USB-webkamera med indbygget mikrofon (M2) da dette kan være en hurtig løsning af problemstillingen. M2 blev koblet til computeren og under konfigurationsindstillingerne på PC'en var det nu muligt at vælge M2 som lydopfanger. Samtidig blev andre lydkilder i PC'en fravalgt, og der blev nu forsøgt at optage en lyd i LabVIEW i enhedstest E05-M2. Det blev konkluderet, at det er muligt at anvende en mikrofon med USB stik til indlæsning af lydsignaler i LabView. Med denne nye viden til rådighed, blev det undersøgt om det er muligt at anvende en adapter, så M1 kunne tilsluttes. Det lykkedes at finde en adapter med den ønskede funktion, dog med nogle dages leveringstid. Leveringstiden blev opvejet mod at tilslutte en electret breakout board microphone (M3) til en Arduino Mega 2560 og opsætte et program i LabVIEW til dette. Det blev vurderet, at arbejdsomfanget af M3 var minimalt og opvejede ventetiden. Sidenhen blev M3 dog erstattet af M4 da der opstod et behov for et brede frekvensbånd. 

\begin{figure}[htb]
\centering
\includegraphics[width=3in]{flowdiagrameop.png}
\caption{Flowdiagram over de udførte enhedstests af lydopfangende kilder. E\# definerer, i hvilken enhedstest den pågældende komponent er testet}
\label{fig:flowop}
\end{figure}
\subsubsection{Valg af A/D konverter}

I forbindelse med det indledende design, blev Arduino Mega 2560 (C1) valgt frem for DAQ (C2) grundet økonomiske omstændigheder. Dog blev det hurtigt konstateret, at C1 forårsagede uhensigtsmæssige udfordringer i forbindelse med frekvenssignalet, idet det kun er muligt at generere et firkantsignal igennem LINX MakerHub i LabVIEW. Det ønskede resultat var grundtonens frekvens, men i stedet blev firkantssignalets harmoniske overtoner opfanget som maksimum frekvens i FFT-funktionen i LabVIEW (integrationstest I04). Efter samtale med lektor Tore Arne Skogberg, blev det påpeget, at det burde være muligt at måle brystvolumen med firkantsignaler. De harmoniske overtoner kunne muligvis dæmpes ved hjælp af resonatoren da denne har funktion som et lavpasfilter, og dette skulle hermed testes som en løsning til denne problemstilling. Således blev det besluttet at arbejde videre med C1 da de økonomiske fordele stadig talte for. Der opstod endvidere en udfordring forårsaget af C1's begrænsede ydeevne. C1's højeste Loop Rate ligger på omkring 122Hz. I og med, at den daværende valgte højtaler, H1, er egnet til et frekvensområde fra 100 Hz op til 2 kHz vil der ikke kunne genereres et frekvenssignal, som er muligt at opfange. For at få en korrekt digital repræsentation af det analoge signal, skal der minimum samples med det dobblete af indgangssignalet. Således vil det indsendte signal være begrænset til maksimalt 50 Hz, for at undgå aliasering jv. Nyquists samplingsteori. Til sidst blev det konkluderet, at udfordringerne ikke blev opvejet af de økonomiske fordele, og det blev derfor besluttet at udskifte C1 til C2 da der kan samples med en højere frekvens, og samtidig anvendes sinussignal for at undgå udfordringer med harmoniske overtoner.      

Sidenhen blev det erfaret, at udviklingen af et VI til at generere en lyd gennem C2 til højtaleren H3 var mere udfordrende en først antaget. Der blev fundet flere guides på \texttt{www.NI.com} til at løse problemstillingen, og koden dertil fandtes meget simpel. Disse guides blev brugt til at bygge VI'et \texttt{generefrekvenssignal04.vi}, men der opstod fejl i forbindelse med opsætning af DAQ Assistant-modulet. Dette blev løst ved at vælge \textit{Generation mode} til \textit{1 sample(on Demand)} og \textit{Signal output range min} til 0. Løsningen blev fundet ved brug af \textit{trail and error} metoden sammen med vejleder. Der opstod herefter en ny fejl med kørslen af VI´et. Under fejlsøgning af denne problemstilling blev det opdaget at C2 har en maksimal samplingsrate på 150 Hz. Fejlsøgning blev afbrudt da denne nye viden satte udviklingen af VI´et i et nyt perspektiv. Med brug af C2 vil der kunne generes et maksimalt signal på 75Hz, når Nyquists samplingsteori opfyldes. I udviklingen af prototypen ønskes det, at kunne genere et bredt frekvensbånd fra 100 Hz op til 1000 Hz. Højtaleren H3's begrænsede frekvensspektre fra 100 Hz til 2 kHz  gør det ikke muligt at anvende C2 sammen med H3. 
På denne baggrund blev det besluttet ikke at anvende C2 til lydgenerering gennem H3 i det videre projektforløb. 

\subsubsection{Valg af dataprocessering}
LabVIEW blev valgt som databehandlingsprogram, da det på kort tid er muligt at producere og videreudvikle på et brugerdefineret program, som interagerer med real-world data og signaler.

\section{Software}
Systemets software er udviklet i LabVIEW, hvor koden udvikles i modulær kode, som afgrænser de enkelte funktionaliteter. Ved objektorienteret programmering beskrives koden typisk vha. klassediagrammer og 3-lags modellen, hvor de enkelte klassers ansvar og grænseflader defineres. I LabVIEW holdes koden simpel og i et logisk flow, som afspejler eksekveringsrækkefølgen. Ved at opdele koden i funktioner fremfor klasser er det ikke hensigtmæssigt at benytte 3-lags modellen. Ligeledes er softwareudviklingen afhængig af udviklingen af prototypen, og softwareudviklingen er derfor udarbejdet i takt med testprocessen og de opståede behov, og har ikke været forudbestemt. I tabel \ref{table:udvikletsw} fremgår en oversigt over det udviklede software. 

Den anvendte version af koden, som genererer et frekvenssignal (VI02G) vises i figur \ref{fig:swgf02}, og den version af koden, som optager og udregner volumen på et objekt (VI06\underline{O}) vises i figur \ref{fig:swop06}.  

\begin{figure}[htb]
\centering
\includegraphics[width=4in]{genererfrekvenssignal02.PNG}	
\caption{VI02G: den udviklede kode til generering af frekvenssignal}
\label{fig:swgf02}
\end{figure}

\begin{figure}[htb]
\centering
\includegraphics[width=6in]{optagefrekvenssignal06.jpeg}	
\caption{VI06\underline{O}: den udviklede kode til optagelse samt udregning af volumen på et objekt} 
\label{fig:swop06}
\end{figure}


\begin{table}[htb]
\centering
\caption{Oversigt over det udviklede software}
\label{table:udvikletsw}
\begin{tabular}{|l|l|l|l|}
\hline
\textbf{Funktion} & \textbf{ID} & \textbf{Navn} & \textbf{Ændring} \\ \hline
Generere frekvenssignal & VI02G & genererfrekvenssignal02.vi & Arduino \\ \hline
\multirow{4}{*}{\begin{tabular}[c]{@{}l@{}}Optage samt \\behandle frekvenssignal\end{tabular}} &
 VI01\underline{O} & optagefrekvenssignal01.vi & Graphics and sound VI \\ \cline{2-4} 
 & VI02\underline{O} & optagefrekvenssignal02.vi & +FFT \\ \cline{2-4} 
 & VI05\underline{O} & optagefrekvenssignal05.vi & +N opsamlinger \\ \cline{2-4} 
 & VI06\underline{O} & optagefrekvenssignal06.vi & DAQ + volumenudregning \\ \cline{2-4} 
 & VI08\underline{O} & optagefrekvenssignal08.vi & testversion \\ \hline
\end{tabular}
\end{table}



\subsection{User Interface}
I figur \ref{fig:ui} er et Mockup af test-brugergrænsefladen (GUI) vist. Denne GUI er langt fra færdigudviklet til brugeren, og skal kun anvendes i testfasen. Testeren har mulighed for at indtaste en række værdier; varighed for målingen, volumen af den tomme resonator samt resonansfrekvensen i den tomme resonator. Da softwaren anvendes i testsammenhæng kan den indstilles til at foretage store testsæt, og resultaterne gemmes derfor ned på en csv-fil. Der er derfor ikke nogen indikator af volumenmålingen på UI'en.   

\begin{figure}[htb]
\centering
\includegraphics[width=5in]{Frontpanel.PNG}
\caption{Brugergrænsefladen for det aktuelle system, som anvendes under testfasen}
\label{fig:ui}
\end{figure}

\subsection{Sekvensdiagram}
Der er i det indledende design ikke udviklet et sekvensdiagram ud fra UC1, da softwareudviklingen, som nævnt, er afhængig af udviklingen af prototypen. For at fremvise metoden er der udarbejdet et simpelt sekvensdiagram over den udviklede software. Sekvensdiagrammet kan i denne kontekst anvendes til at danne et overblik over den udviklede kode med henblik på en eventuel videreudvikling heraf. 

\begin{figure}[htb]
\centering
\includegraphics[width=6in]{sekvensdiagram.png}
\caption{Sekvensdiagram over brystvolumenmåleren}
\label{fig:sekvens}
\end{figure}



