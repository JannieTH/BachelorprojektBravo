
\chapter{Kravspecifikation}
\label{chap:kravspec}
	\section{Indledning}
	Dette kapitel indeholder kravspecifikationen for den akustiske brystvolumenmåler(BVM). Kravspecifikation er udarbejdet i samarbejde med projektets kunde, Pavia Lumholt, speciallæge i plastikkirurgi. 
	Kravspecifikationen er opbygget om FURPS+-modellen, som er en helhedsbetragtende teknik til definering af krav.
	FURPS+ er inddelt i funktionelle og ikke-funktionelle krav. F(functionality) definerer de funktionelle krav, og URPS+ (usability, reliability, perfomance, supportability og +) definerer de ikke-funktionelle krav. I denne kravspecifikation definerer +, designbegrænsninger.  
Kravspecifikationen er udarbejdet ud fra den koncepttuelle brystvolumenmåler. Grundet test og udviklingsmæssige udfordringer er denne model ikke blevet realiseret. Beskrivelse af den koncepttuelle brystvolumen er beskrevet i afsnit \ref{subsec:BVMopb} Brystvolumenmålerens opbygning 
		  
		\subsection{Formål}
		Kravspecifikation definerer de funktionelle og ikke-funktionelle krav, og fungerer som en bindende kontrakt mellem producent og kunde.
	
		\subsection{Læsevejledning}	
		Dokumentet indeholder en systembeskrivelse for den akustiske brystvolumenmåler (omtales herefter BVM). Systembeskrivelsen er en kort beskrivelse af BVM samt en illustration af måleren. De definerede krav er opdelt i funktionelle og ikke-funktionelle krav, og er beskrevet i de navnebeslægtede afsnit. Dokumentet indeholder ydermere en projektafgrænsning i form af MoSCoW-modellen samt et afsnit omhandlende projektets samarbejdspartnere.
		
		\subsection{Versionshistorik}
		
		\vspace{5mm}
		
			\begin{figure*}[htb]
				\centering
				\includegraphics[width=5in]{versionshisv01.png}
			\end{figure*}
		
	\section{Systembeskrivelse}
	
Systembeskrivelsen er opdelt i en konceptuelt og aktuel beskrivelse.
Denne systembeskrivelse tager udgangspunkt i den koncepttuelle BVM. Beskrivelse vil derfor være af en detaljegrad der afspejler det koncepttuelle niveau.

		\subsection{Den konceptuelle brystvolumenmåler} \label{subsec:BVMopb}
		
		Den konceptuelle BVM er bygget op af en resonator med en størrelse hvorpå den kan omslutte en patients bryst. Resonatoren har påmonteret en højtaler som sender lyd ind i resonatorporten. Indeni resonatoren er monteret en mikrofon til at opsamle resonansfrekvensen. I resonatorkanten, som tilslutter til brystet, er der påsat tryksensorer til detektering af anlægstrykket. Resonatoren er yderligere monteret med en passende mængde dioder til angivelse af et korrekt anlægstryk. Dioderne er placeret så det er synlige for plasktikkirurgen. Der er ydermere installeret en CPU til processering af data samt et display, med en størrelse, hvortil det er muligt at anvise progressbar for volumenmåling, det målte volumen samt relevante piktogrammer for procestilstanden. På resonatoren er der ligeledes påført tre knapper, en tænd-og sluk knap, målingsknap og en kalibreringsknap. Knappernes funktion er angivet med et piktogram til hver funktion. Et batteri er ligeledes tilkoblet så BVM bliver et trådløst device.   Et overbliksbillede af de forskellige komponenter som ingår i den konceptuelle BVM findes i figur \ref{fig:ksys} 
		
		
\vspace{5mm}  
		
		\begin{figure}[htb]
			\centering
				\includegraphics[width=5in]{Ksys}
				\caption{Diagrammet er en visuel beskrivelse af den konceptuelle brystvolumenmåler}	
				\label{fig:ksys}
			\end{figure}	     
		
	
		\subsection{Brystvolumenmåleren funktionalitet}
		Når en måling intialiseres med BVM'en afsendes en lyd fra højtalerne ind i resonatoren  Mikrofonen indeni resonatoren opsamler den opståede Helmholtzresonans. Igennem en A/D konvertering udregnes  udregner en algoritme størrelsen på brystvolumen. 
	
		\subsection{Aktørbeskrivelse}
		Systemets primære aktør er en plastikkirurg, som bruger BMV'en når han ønsker et objektivt mål på et bryst. Det er udelukkende plastikkirurgen, der  betjener BMV'en under en måling. Som sekundær aktør giver patient et input, sit bryst, til systemet. 
	
	\section{Funktionelle krav}	
	Dette afsnit beskriver de funktionelle krav, som er udarbejdet i samarbejde med Pavia Lumholt. Disse krav er præsenteret i et Use Case diagram samt i en Fully Dressed Use Case beskrivelse.  
	
	\pagebreak
	
		\subsection{Use Case diagram}
		I Use Case diagrammet, figur \ref{fig:UC1}, vises en Use Case for brugen af den akustiske brystvolumenmåler. På venstre side af Use Casen ses systemets primære aktør, og på højre side ses systemets sekundære aktør. Endvidere, vises systemets interessenter, nederst i diagrammet.  
	
			\begin{figure}[htb]
				\centering
					\includegraphics[width=4in]{UC1.png}
					\caption{Use Case diagrammet viser et overblik over Use Cases samt involverede aktører.}
					\label{fig:UC1}
			\end{figure}	 
	\pagebreak	
	\newpage	
  
  		\subsection{Use Case 1 - Udfør brystvolumenmåling}
  			
  			\begin{figure*}[htb]
  				\flushleft
  					\includegraphics[width=6in]{UCtabel1}
  			\end{figure*}
  			
  			
Det alternative flow i UC1 er medtaget, som et eksempel på anvendelsen heraf. Det er ikke muligt at fremlægge et specificeret alternativ flow da kendskabet til den endelige prototype ikke er indegående.  Det alternative flow må derfor betragtes, som et eksempel på håndteringen af Use Case-beskrivelsen. 
	
	\section{Ikke-funktionelle krav}
	
	\subsection{Usability krav}
	
\begin{tabularx}{1.1\textwidth}{|l|l|l|X|}
\hline
\textbf{\textbf{\begin{tabular}[c]{@{}l@{}}Krav \\ nr.\end{tabular}}} & \textbf{Krav} & \textbf{Kriterie} & \textbf{Baggrund for krav} \\ \hline
UK1 & \begin{tabular}[c]{@{}l@{}}Volumen angives i\\ afrundet milliliter {[}ml{]}\end{tabular} & \begin{tabular}[c]{@{}l@{}}ml anføres efter talværdi, \\ \textit{ex. 300 ml}\end{tabular} & \begin{tabular}[c]{@{}l@{}}Standard inden for \\ plastikkirurgi\end{tabular} \\ \hline
UK2 & \begin{tabular}[c]{@{}l@{}}Volumenangivelse \\ skal kunne aflæses på \\ en afstand af 50 cm.\end{tabular} & \begin{tabular}[c]{@{}l@{}}Talværdi angives i \\ digital talform \\ med en højde på 1 cm.\end{tabular} & Tænkt arbejdssituation \\ \hline
UK3 & \begin{tabular}[c]{@{}l@{}}Der anvendes pikto-\\grammer til visualisering\\ af processtatus \end{tabular} & \begin{tabular}[c]{@{}l@{}}Der anvendes udelukkende\\ visuelt symbolik\end{tabular} & \begin{tabular}[c]{@{}l@{}} Ingen sprogbarriere \end{tabular} \\ \hline
\end{tabularx}


\subsection{Reliability krav}

\begin{table*}[htb]
\begin{tabular}{|l|l|l|l|}
\hline
\textbf{\textbf{\begin{tabular}[c]{@{}l@{}}Krav \\ nr.\end{tabular}}} & \textbf{Krav}                                                                        & \textbf{Kriterie}                                                                           & \textbf{Baggrund for krav} \\ \hline
RK1           & \begin{tabular}[c]{@{}l@{}}Volumenangivelse skal\\ have høj nøjagtighed\end{tabular} & +/- 10 ml.                                                                                  & Kundekrav                  \\ \hline
RK2           & \begin{tabular}[c]{@{}l@{}}Volumenangivelse skal\\ have stor præcision\end{tabular}  & Afvigelse på 1\% af $f_{b}$                                                                      & Kundekrav                  \\ \hline
RK3           & \begin{tabular}[c]{@{}l@{}}Volumenangivelse skal\\ afrundes til nærmeste\\ hele tal\end{tabular}  &\begin{tabular}[c]{@{}l@{}} Volumenangivelse findes ikke\\ som decimaltal på UI\end{tabular} & Kundekrav                  \\ \hline
RK4           & \begin{tabular}[c]{@{}l@{}}Hyppighed af \\ fejltilstand\end{tabular}                 & \begin{tabular}[c]{@{}l@{}}Der må forekomme fejl\\ ved hver 1000. måling\end{tabular}       & Tænkt arbejdssituation     \\ \hline
RK5           & \begin{tabular}[c]{@{}l@{}}Straksafhandling af\\ fejltilstand\end{tabular}           & \begin{tabular}[c]{@{}l@{}}Varighed af fejlløsning\\ må vare højst 2 minutter\end{tabular}  & Tænkt arbejdssituation     \\ \hline

               
\end{tabular}
\end{table*}

	\subsection{Performance krav}
	\begin{table*}[htb]
\begin{tabular}{|l|l|l|l|}
\hline
\textbf{\textbf{\begin{tabular}[c]{@{}l@{}}Krav \\ nr.\end{tabular}}} & \textbf{Krav} & \textbf{Kriterie} & \textbf{Baggrund for krav} \\ \hline
PK1 & \begin{tabular}[c]{@{}l@{}}Måling skal foretages  \\ hurtigt\end{tabular} & \begin{tabular}[c]{@{}l@{}}Målingen skal være foretaget \\ inden for 10 sekunder\end{tabular} & Kundekrav \\ \hline
PK2 & \begin{tabular}[c]{@{}l@{}}Lydniveau må ikke \\ være skadeligt for \\ aktører\end{tabular} & \begin{tabular}[c]{@{}l@{}}Lydniveau skal være mindre \\ end 75 dB\end{tabular} & \begin{tabular}[c]{@{}l@{}}Anbefaling fra \\ Sundhedsstyrelsen\end{tabular} \\ \hline
\end{tabular}
\end{table*}

\pagebreak

\subsection{Supportability krav}

\begin{table*}[!h]
\begin{tabular}{|l|l|l|l|}
\hline
\textbf{\textbf{\begin{tabular}[c]{@{}l@{}}Krav \\ nr.\end{tabular}}} & \textbf{Krav} & \textbf{Kriterie} & \textbf{Baggrund for krav} \\ \hline
SK1 & \begin{tabular}[c]{@{}l@{}}Produktmateriale \\ skal tåle rengøring\end{tabular} & Skal tåle rengøring med sprit & Tænkt arbejdssituation \\ \hline
SK2 & UI skal tåle rengøring & Skal tåle rengøring med sprit & Tænkt arbejdssituation \\ \hline
\end{tabular}
\end{table*}

Ved en realisering af den konceptuelle BVM ville der endvidere i dette afsnit være listet krav til softwarearkitektur samt softlovgivning jvf. MDD/93/42/EEC. 

\subsection{Design krav}
\begin{table*}[!h]
\begin{tabular}{|l|l|l|l|}
\hline
\textbf{\textbf{\begin{tabular}[c]{@{}l@{}}Krav \\ nr.\end{tabular}}} & \textbf{Krav} & \textbf{Kriterie} & \textbf{Baggrund for krav} \\ \hline
DK1 & \begin{tabular}[c]{@{}l@{}}Skal være intuitiv at \\ anvende\end{tabular} & \begin{tabular}[c]{@{}l@{}}Funktioner skal være af \\ standardiseret design\end{tabular} & \begin{tabular}[c]{@{}l@{}}Organisatoriske og \\ økonomiske overvejelser\end{tabular} \\ \hline
DK2 & \begin{tabular}[c]{@{}l@{}}Skal kunne anvendes \\ på forskellige \\ bryststørrelser og \\ -former\end{tabular} & \begin{tabular}[c]{@{}l@{}}Skal kunne anvendes på \\ 90\% af tilfældigt udvalgte \\ kvinder\end{tabular} & Tænkt arbejdssituation \\ \hline
DK3 & \begin{tabular}[c]{@{}l@{}}Må ikke give gener \\ under måling\end{tabular} & \begin{tabular}[c]{@{}l@{}}Resonatoråbning må ikke\\ forvolde skader på huden\end{tabular} & Etiske overvejelser \\ \hline
DK4 & \begin{tabular}[c]{@{}l@{}}Må ikke forårsage \\ arbejdsskader\end{tabular} & \begin{tabular}[c]{@{}l@{}}Skal være ergonomisk at \\ anvende\end{tabular} & \begin{tabular}[c]{@{}l@{}}Arbejdsmiljømæssige \\ overvejelser\end{tabular} \\ \hline
DK5 & \begin{tabular}[c]{@{}l@{}}Design skal være \\ trådløst\end{tabular} & \begin{tabular}[c]{@{}l@{}}Ingen eksterne tilkoblinger \\ med kabler\end{tabular} & Kundekrav \\ \hline
DK6 & \begin{tabular}[c]{@{}l@{}}Design skal overholde \\ klassificering Im i hht. \\ MDD 93/42/EEC\end{tabular} & \begin{tabular}[c]{@{}l@{}}Dokumentation skal \\ foreligge\end{tabular} & \begin{tabular}[c]{@{}l@{}}Godkendelse af \\ medicinsk udstyr\end{tabular} \\ \hline
\end{tabular}
\end{table*}
	
	
		
	
	\section{Projektafgrænsning}
	MoSCoW-modellen er en prioriteringsmetode, som anvendes til afgræsning af projektet. Modellen beskriver, hvilke dele og krav i projektet, som skal opfyldes (\textbf{M}ust), bør opfyldes (\textbf{S}hould), kan opfyldes (\textbf{C}ould) og ikke vil opfyldes (\textbf{W}ould not have). Således gives en struktureret oversigt over, hvilke krav, der er vigtigst at få opfyldt inden for den givne tidsramme, og endvidere, hvilke krav, som efterfølgende med fordel kan implementeres, hvis tidensramme tillader det. Figur \ref{fig:MoSCoW} viser, hvordan de enkelte dele og krav i projektet prioriteres i henhold til MoSCoW-metoden.  
	
	\begin{figure}[htb]
				\centering
					\includegraphics[width=5in]{storrusser}
					\caption{MoSCoW-modellen definerer projektets afgrænsning}
					\label{fig:MoSCoW}
	\end{figure}	
		
	\section{Samarbejdspartnere}	 
	Kravspecifikationen er udarbejdet gennem et samarbejde med flere parter. 
	Først og fremmest er projektets kravspecifikation til den endelige prototype specificeret i et samarbejde med projektets kunde, speciallæge i plastikkirurgi, Pavia Lumholt.  
	Derudover er projektet tilknyttet en vejleder, lektor Samuel Alberg Thrysøe, med speciale i signalbehandling, som vejleder ved eventuelle problemstillinger.  
	Endvidere indgår eksterne konsulenter, som reviewer's på indholdet af kravspecifikationen.   
 	
  
 