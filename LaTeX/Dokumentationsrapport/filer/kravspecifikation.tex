
\chapter{Kravspecifikation}
\label{Kravspec}
	\section{Indledning}
	Dette kapitel indeholder kravspecifikationen for den akustiske volumenmåler til bryster. Kravspecifikation er udarbejdet i samarbejde med projektets kunde, Pavia Lumholt, speciallæge i plastikkirurgi. 
	  
		\subsection{Formål}
		Kravspecifikation definerer de funktionelle og ikke-funktionelle krav, og fungerer som en bindende kontrakt mellem producent og kunde.
	
		\subsection{Læsevejledning}	
		Dokumentet indeholder en systembeskrivelse for den akustiske brystvolumenmåler (omtales herefter BVM). Systembeskrivelsen er en kort beskrivelse af BVM samt en illustration af måleren. De definerede krav er opdelt i funktionelle og ikke-funktionelle krav, og er beskrevet i de navnebeslægtede afsnit. Dokumentet indeholder ydermere en projektafgrænsning i form af MoSCoW-modellen samt et afsnit omhandlende projektets samarbejdspartnere.
		
		\subsection{Versionshistorik}
		
		\vspace{5mm}
		
			\begin{figure*}[htb]
				\centering
				\includegraphics[width=5in]{versionshisv01.png}
			\end{figure*}
		
	\section{Systembeskrivelse}

		\subsection*{Brystvolumenmålerens opbygning}
		Den akustiske brystvolumenmåler består af en Helmholtz resonator, hvor der er påmonteret en højtaler og en mikrofon. Højtaleren og mikrofonen er koblet til en Arduino Mega 2560 R3. Arduino'en er koblet til en PC, hvor der skrives til Arduino i LabVIEW. \\ 
		
		\begin{figure}[htb]
			\centering
			\includegraphics[width=5in]{systembeskrivelse}
			\caption{Beskrivelse af systemets komponenter €€€€}
			\label{system}
		\end{figure}
	
		\subsection*{Brystvolumenmåleren funktionalitet}
		Når en måling intialiseres med BVM'en afsendes en lyd i resonatoren via højtaleren. Mikrofonen på resonatoren opsamler den reflekterede lyd, og udfra den reflekterede lyd, udregner en algoritme størrelsen på brystvolumen. 
	
		\subsection{Aktørbeskrivelse}
		Systemets primære aktør er en plastikkirurg, som bruger BMV'en når han ønsker et objektivt mål på et bryst. Det er udelukkende plastikkirurgen, der  betjener BMV'en under en måling. Som sekundær aktør giver patient et input, sit bryst, til systemet. 
	
	\section{Funktionelle krav}	
	Dette afsnit beskriver de funktionelle krav, som er udarbejdet i samarbejde med Pavia Lumholt. Disse krav er præsenteret i et Use Case diagram samt i en Fully Dressed Use Case beskrivelse.  
	
	\pagebreak
	
		\subsection{Use Case diagram}
		I Use Case diagrammet, figur \ref{fig:UC1}, vises en Use Case for brugen af den akustiske brystvolumenmåler. På venstre side af Use Casen ses systemets primære aktør, og på højre side ses systemets sekundære aktør. Endvidere, vises systemets interessenter, nederst i diagrammet.  
	
			\begin{figure}[htb]
				\centering
					\includegraphics[width=4in]{UC1.png}
					\caption{Use Case diagrammet viser et overblik over Use Cases samt involverede aktører.}
					\label{fig:UC1}
			\end{figure}	 
	\pagebreak	
	\newpage	
  
  		\subsection{Use Case \#1 - Udfør brystvolumenmåling}
  			
  			\begin{figure*}[htb]
  				\flushleft
  					\includegraphics[width=7in]{UC1tabel}
  			\end{figure*}
  			
	
	\section{Ikke-funktionelle krav}
	
	
	\subsection{Usability}
	
	
	
		\subsubsection{Tid}
 		Efter kalibrering må målingen maksimalt tage 10 sek. Dette er et kundekrav fra Pavia Lumholt.
 		BMV'en skal melde om kalibrereingsbehov hver 10. minut
		\subsubsection{Enheder}
 		Målingen skal angives til PK i milliliter(ml.)Plastikkirurger bruger milliliter enheder når de angiver bryststørrelse. 
 		\subsubsection{UI}
 		UI skal være en touch-screen, da denne form for skærm er rengøringsvenlig. 
 		Tekst skal være synligt på en halv meters afstand, da det skal være muligt at aflæse teksten når man står med BVM'en i nogelunde strakt arm.
 		Sproget skal være engelsk, da der ønskes et sprog som kan læses af formentlig alle plastikkirurger.
		 UI skal fejlmelde når der opstår uventet fejl 
 		\subsubsection{Lovgivning for medicinsk udstyr}
 		BMV'en skal overholde lovgivningen for et medicinsk device. BMV'en skal som et klasse I udstyr og et målingsudstyr opfylde bilag VII og være i overensstemmelse med processerne i bilag VI,V eller VI gældende for metrologisk udstyr.
 		\subsection{Kalibrering}
 		BMV'en skal kunne kalibreres efter temperatur og luftfugtighed, da disse kan have en indflydelse på målingens output. 
 		\subsection*{Nøjagtighed, præcision og linearitet}
 		BMV'en skal måle nøjagtige og præcice, for at målingerne er valide og kan bruge i praksis. 
 		Der skal kunne vises en linearitet ved målinger, så målingerne kan bruge i hele bryststørrelsesspektret. 
	
	\pagebreak
	
	\section{Projektafgrænsning}
	MoSCoW-modellen er en prioriteringsmetode, som anvendes til afgræsning af projektet. Modellen beskriver, hvilke dele og krav i projektet, som skal opfyldes (\textbf{M}ust), bør opfyldes (\textbf{S}hould), kan opfyldes (\textbf{C}ould) og ikke vil opfyldes (\textbf{W}ould not have). Således gives en struktureret oversigt over, hvilke krav, der er vigtigst at få opfyldt inden for den givne tidsramme, og endvidere, hvilke krav, som efterfølgende med fordel kan implementeres, hvis tidensramme tillader det. Figur \ref{fig:MoSCoW} viser, hvordan de enkelte dele og krav i projektet prioriteres i henhold til MoSCoW-metoden.  
	
	\begin{figure}[htb]
				\centering
					\includegraphics[width=4in]{MoSCoWv02}
					\caption{MoSCoW-model, hvor blablabla €€€}
					\label{fig:MoSCoW}
	\end{figure}	
		
	\section{Samarbejdspartnere}	 
	Kravspecifikationen er udarbejdet gennem et samarbejde med flere parter. 
	Først og fremmest er projektets kravspecifikation til den endelige prototype specificeret i et samarbejde med projektets kunde, speciallæge i plastikkirurgi, Pavia Lumholt.  
	Derudover er projektet tilknyttet en vejleder, lektor Samuel Alberg Thrysøe, med speciale i signalbehandling, som vejleder ved eventuelle problemstillinger.  
	Endvidere indgår eksterne konsulenter, som reviewer's på indholdet af kravspecifikationen.   
 	
  
 