\chapter{Resultater}
Der i projektet ikke opnået succes med at udvikle en prototype, som kan give et nøjagtigt og præcist objektivt mål for et brystvolumen. Projektarbejdet har derfor taget udgangspunkt i to systemer; et \textit{konceptuelt} og et \textit{aktuelt} system, som omtalt i kapitel 2.  

\section{Det udviklede system}
Dette afsnit beskriver det udviklede system, samt resultaterne af anvendelse af dets komponenter. 
Det udviklede system er det aktuelle system, som fremgår af figur \ref{fig:aktuellesys}. Systemet består af følgende dele: 

\begin{itemize}
\item resonator	
\item mikrofon 
\item højtaler
\item DAQ
\item softwareprogram
\item modellervoks
\item pink noise generator
\item termometer
\item afstandsklods
\end{itemize}

\begin{figure}[!h]
\centering
\includegraphics[width=4in]{aktuellesys.JPG}
\caption{Det aktuelle systems opstilling}
\label{fig:aktuellesys}	
\end{figure}

 Det aktuelle system består af en resonator, hvor en mikrofon, som er tilkoblet en DAQ, er påsat. En højtaler er tilsluttet en pink noise generator. Højtaleren holdes over resonatorens portåbning, og indsender pink noise gennem resonatorporten. Pink noise tilbagekastes i resonatoren, og resonansfrekvens opsamles i mikrofonen, hvor signalet opsamles via DAQ'en til softwareprogrammet i LabVIEW. I softwareprogrammet foretages en Fast Fourier Transform (FFT), hvorfra frekvensen med den største amplitude findes. Den resonansfrekvens anvendes i udregningen af det pågældende volumen.   

\subsection{Resonator}
Resonatoren er fremstillet af en tryksprøjte i hårdt plast. Tryksprøjtemodulet er afmonteret og bunden er savet af i et lige snit. Der er anvendt modellervoks til tætning mellem resonatorkant og underflade i et forsøg på at undgå luftutætheder. 

Volumen af resonatorkammeret er $={\SI{1.671}{\liter}}$
 
\subsection{Mikrofon}
Den anvendte mikrofon er af typen \textit{Electret Microphone Amplifier MAX4466}, og af det tilhørende datablad (bilag I) fremgår et frekvensspektre på 20 Hz - 20 KHz. Mikrofonen er påsat resonatoren med klæbemasse (også kendt som \textit{lærerens tyggegummi}).  
Mikrofonens optimale placering på resonatorens indvendige side er, i følge fundet litteratur \citep{RefWorks:11}, i en afstand af 280 mm fra portåbningen. Dette antages at være bestemt ud fra det anvendte systems skalering. Endvidere viser resultater fra en Ph.D.-afhandling omhandlende volumenmåling ved brug af Helmholtz' resonansteori, at ændringer i mikrofonens placering med hensyn til lydkildens placering ikke viser mærkbare ændringer i frekvens eller dens tilsvarende amplitude \citep{RefWorks:22}. På baggrund af anbefalinger fra TAS er mikrofonen i det aktuelle system placeret $\leq$ resonatorhalsens længde, hvilket er $\leq$ 37 mm, fra portåbningen. 

\subsection{Højtaler}
Den anvendte højtaler er af typen \textit{Multimedia USB Speaker HP-1800}. Det har ikke været muligt at finde et datablad, men i følge producentens specifikationer har højtalerne et frekvensspektre på 100 Hz - 20 kHz \citep{RefWorks:30}. Højtaleren er tilkoblet en PC, hvorigennem der genereres pink noise. Højtaleren skal holdes i en defineret afstand fra resonatorens portåbning.  På baggrund af samtale med TAS og hans testerfaringer, er denne afstand defineret til at være $\geq$ portens diameter \citep{RefWorks:31}. En holder, som opfyldte denne betingelse, blev 3D-printet og monteret på resonatorens hals. Højtalerenholderen fremgår af figur \ref{fig:resmedholder}. Testerfaringer (integrationstest I08) viste derefter, at systemet blev for lukket, og lufttilførelse er nødvendigt for at få luften til at virke som en fjeder. Holderen blev derefter afmonteret, og der blev i stedet anvendt en afstandsklods (en hætte fra en sprittusch) med den definerede længde. Det var ikke muligt at anvende en intern højtaler, da output kapaciteten på de anvendte analog-to-digital converters, ikke var tilstrækkelig til at opfylde Nyquists samplingsteori på output-pin'en. Der er derfor anvendt et eksternt lydoutput. 

\begin{figure}[!h]
\centering
\includegraphics[width=4in]{bordtest61.jpg}
\caption{Det udviklede system med 3D-printet højtalerholder monteret på resonatorhalsen}
\label{fig:resmedholder}	
\end{figure}

\subsection{Firkantsignal}
Der blev ikke opnået succes ved at anvende firkantssignal i de tidligere tests. Resultater fra den omtalte Ph.D.-afhandling viser endvidere, at anvendelse af firkantsignal til detektering af resonansfrekvenser, ikke har været succesfuldt da de ulige harmonier jævnligt faldt sammen med anden og tredje harmoni, forårsaget af resonatorens fysiske geometri og artefakter \citep{RefWorks:22}. Dette var også tilfældet i der udførte test med firkantsignal i projektet. 

\subsection{Pink noise}
Pink noise består af vilkårlige frekvenser med lige stor effekt. Frekvensdomænets output ligner tilnærmelsesvis en step funktion med ens effekt ved alle frekvenser. Ved at anvende pink noise som den indgående frekvens, kan resonansfrekvens altså observeres som en top i frekvensdomænet. 
Pink noise blev genereret via hjemmesiden \texttt{www.onlinetonegenerator.com}. 

\subsection{Termometer}
Et termometer anvendes i opstillingen da en stigning i lufttemperaturen medfører en forøgelse af lydens hastighed i luft, og derfor ændrer et volumen bestemt ud fra Helmholtz' ligning (beskrevet i afsnit \ref{subsec:efter}). 

\subsection{Software}
 I softwareprogrammet, som fremgår af figur \ref{fig:VI06O}, foretages en (FFT) for at få et frekvensspektre, hvorfra frekvensen med den største amplitude findes. 
 Varigheden af hvert enkelte mål skal være stor nok til at signalet kan nå at stabilisere sig \citep{RefWorks:22}. I testforløbet er det anvendt en måletid på hhv. 4 sekunder og 10 sekunder alt efter testsættets størrelse, og hvor hurtigt testerfaringerne ønskedes at opnåes. 
 
 \begin{figure}[!h]	
\centering
\includegraphics[width=6in]{optagefrekvenssignal06.jpeg}
\caption{Det udviklede softwareprogram i LabVIEW}
\label{fig:VI06O}
 \end{figure}

\section{Måleresultater}

I integrationstest I09 er der forsøgt at finde voluminer af tre forskellige balloner. Målingerne er foretaget hhv. \textit{med }og \textit{uden} højtalerholderen. 
Ballonernes specifikationer fremgår af oversigten i tabel \ref{table:balspec}. \\
\hspace{1.5cm}

\begin{table}[!h]
\centering
\caption{Oversigt over de anvendte balloners vægt og volumen}
\label{table:balspec}
\begin{tabular}{|c|l|l|}
\hline
\multicolumn{1}{|c|}{\textbf{\begin{tabular}[c]{@{}c@{}} Objekt \\ ID\end{tabular}}} & \multicolumn{1}{c}{\textbf{\begin{tabular}[c]{@{}c@{}} Vægt \\ {[}g{]}\end{tabular}}} & \multicolumn{1}{c|}{\textbf{\begin{tabular}[c]{@{}c@{}} Volumen\\ {[}m3{]}\end{tabular}}} \\ \hline
b1 & 216 & 0,000216  \\ \hline
b2 & 383 & 0,000383  \\ \hline
b3 & 516 & 0,000516  \\ \hline
\end{tabular}
\end{table}

\newpage

\subsection{Grafisk visning af måleresultaterne}

I figur \ref{fig:tomkam} vises måleresultater for det tomme kammer: 
\begin{figure}[H]
\centering
   \begin{subfigure}[b]{1\textwidth}
   \includegraphics[width=1\linewidth]{tomtkammer(m_holder)}
   \caption{}
   \label{fig:Ng1} 
\end{subfigure}

\begin{subfigure}[b]{1\textwidth}
   \includegraphics[width=1\linewidth]{tomtkammer(u_holder)}
   \caption{}
   \label{fig:Ng2}
\end{subfigure}

\caption[Two numerical solutions]{De målte resonansfrekvenser ($f_{0}$) for et tomt kammer. Graf (a) viser resultaterne \textit{med} højtalerholderen og graf (b) \textit{uden} højtalerholderen.}
\label{fig:tomkam}
\end{figure}

\newpage

I figur \ref{fig:b1} vises måleresultater for ballonen b1:
\begin{figure}[H]
\centering
   \begin{subfigure}[b]{1\textwidth}
   \includegraphics[width=1\linewidth]{ballon216g(m_holder)}
   \caption{}
   \label{fig:Ng1} 
\end{subfigure}

\begin{subfigure}[b]{1\textwidth}
   \includegraphics[width=1\linewidth]{ballon216g(u_holder)}
   \caption{}
   \label{fig:Ng2}
\end{subfigure}

\caption[Two numerical solutions]{De målte resonansfrekvenser ($f_{b1}$) for en vandfyldt ballon på 216 g. Graf (a) viser resultaterne \textit{med} højtalerholderen og graf (b) \textit{uden} højtalerholderen.}
\label{fig:b1}
\end{figure}

\newpage

I figur \ref{fig:b2} vises måleresultater for ballonen b2: 
\begin{figure}[H]
\centering
   \begin{subfigure}[b]{1\textwidth}
   \includegraphics[width=1\linewidth]{ballon383g(m_holder)}
   \caption{}
   \label{fig:Ng1} 
\end{subfigure}

\begin{subfigure}[b]{1\textwidth}
   \includegraphics[width=1\linewidth]{ballon383g(u_holder)}
   \caption{}
   \label{fig:Ng2}
\end{subfigure}

\caption[Two numerical solutions]{De målte resonansfrekvenser ($f_{b2}$) for en vandfyldt ballon på 383 g. Graf (a) viser resultaterne \textit{med} højtalerholderen og graf (b) \textit{uden} højtalerholderen.}
\label{fig:b2}
\end{figure}

\newpage

I figur \ref{fig:b3} vises måleresultater for ballonen b3: 
\begin{figure}[H]
\centering
   \begin{subfigure}[b]{1\textwidth}
   \includegraphics[width=1\linewidth]{ballon516g(m_holder)}
   \caption{}
   \label{fig:Ng1} 
\end{subfigure}

\begin{subfigure}[b]{1\textwidth}
   \includegraphics[width=1\linewidth]{ballon516g(u_holder)}
   \caption{}
   \label{fig:Ng2}
\end{subfigure}

\caption[Two numerical solutions]{De målte resonansfrekvenser ($f_{b3}$) for en vandfyldt ballon på 516 g. Graf (a) viser resultaterne \textit{med} højtalerholderen og graf (b) \textit{uden} højtalerholderen.}
\label{fig:b3}
\end{figure}

\subsection{Tabeloversigt af måleresultater}

\begin{table}[!htb]
\centering
\caption{Tabel over måleresultater opnået ved at måle volumen af vandballoner hhv. med og uden højtalerholder}
\label{table:resultater}
\begin{tabular}{|c|l|l|}
\hline
\textbf{\begin{tabular}[c]{@{}c@{}}Objekt \\ ID\end{tabular}} & \multicolumn{1}{c|}{\textbf{\begin{tabular}[c]{@{}c@{}}Med / uden \\ højtalerholder\\ {[}+/-{]}\end{tabular}}} & \multicolumn{1}{c|}{\textbf{\begin{tabular}[c]{@{}c@{}} Gennemsnitlig \\ resonansfrekvens \\  $f_{b}$ {[}Hz{]}\end{tabular}}} \\ \hline
\multirow{2}{*}{Tomt kammer} & + & 124,9875 \\ \cline{2-3} 
 & - & 217,12 \\ \hline
\multirow{2}{*}{b1} & + & 150,0025 \\ \cline{2-3} 
 & - & 182,105 \\ \hline
\multirow{2}{*}{b2} & + & 151,87 \\ \cline{2-3} 
 & - & 186,985 \\ \hline
\multirow{2}{*}{b3} & + & 159,1325 \\ \cline{2-3} 
 & - & 201,62 \\ \hline
\end{tabular}
\end{table}

Måleresultaterne verificeres ved at udregne det pågældende volumen ud fra Helmholtz' ligning, som eftervist i afsnit \ref{subsec:efter}.

\subsection{Teoretisk udregning af resonansfrekvensen i det tomme kammer ($f_{0}$)} 

Resonansfrekvensen $f_{0}$ i et tomt kammer udregnes ved brug af Helmholtz' ligning, givet ved dette udtryk \f

hvor 
\begin{description}[align=left,labelindent=0.3cm]
\item $f_{0}$: resonansfrekvens i en tom resonator [$Hz$],\\
\item $c$: lydens hastighed i luft [$m/s$],\\
\item $S_{p}$: tværsnitsareal af port [$m^2$],\\
\item $V$: statisk volumen af resonator [$m^3$],\\
\item $l_{p}$: længde af port [$m$],\\
\item $\Delta l$: endekorrektion [$m$]\\
\end{description}

Først udregnes lydens hastighed i en lufttemperatur på 23,7\degree C:
\begin{equation}
		c = {\SI{331,5}{\meter / \second}} \cdot \sqrt{\frac{T_{K}}{\SI{273,15}{\kelvin}}}	\end{equation}
 hvor 		
\begin{equation}
		T_{K} = t_{\degree C} + {\SI{273,15}{\kelvin}} = {\SI{23,7}{\degree C}} + {\SI{273,15}{\kelvin}} = {\SI{296,85}{\kelvin}}	
\end{equation}

og dermed er c: 
\begin{equation}
		c = {\SI{331,5}{\meter / \second}} \cdot \sqrt{\frac{\SI{296,85}{\kelvin}}{\SI{273,15}{\kelvin}}} = {\SI{345,582}{\kelvin}}	\end{equation}
		
Herefter findes tværsnitsarealet $S_{p}$ af porten:
\begin{equation}
		S_{p} = r^2\pi = (0,0175)^2 m \cdot \pi	= {\SI{0,000962}{\meter}^2}
\end{equation}

Endekorrektionsfaktoren bestemmes: 
\begin{equation}
		\Delta l = 0,6 \cdot r + \frac{8}{3\pi} \cdot r 
\end{equation} 
$\Downarrow$
\begin{equation}
\Delta l = 0,6 \cdot {\SI{0,0175}{\meter}} + \frac{8}{3\pi} \cdot {\SI{0,0175}{\meter}} = {\SI{0.025354}{\meter}} 
\end{equation}
		
Og dermed kan der ud fra ligning ligning 4.1 udledes: 

\begin{equation}
		f_{0} = \frac{{\SI{345,582}{\kelvin}}}{2\pi}\sqrt{\frac{{\SI{0,000962}{\meter}^2}}{{\SI{0,001671}{\meter}^3}({\SI{0,034}{\meter}}+ {\SI{0.025354}{\meter}})}}
\end{equation}	
$\Downarrow$
\begin{equation}
		f_{0} = {\SI{171,295}{\hertz}}
		\end{equation}

\subsection{Teoretisk udregning af resonansfrekvensen ($f_{b}$) i kammer indeholdende et objekt}

Resonansfrekvensen $f_{b}$ i et kammer indeholdende et objekt udregnes ved brug af Helmholtz' ligning, givet ved dette udtryk \fb

hvor 
\begin{description}[align=left,labelindent=0.3cm]
\item $f_{b}$: resonansfrekvens i en resonator, indeholdende et objekt [$Hz$],\\
\item $W$: volumen af objekt [$m^3$],\\
\end{description}


\subsubsection{Resonansfrevkensen af b1} 
\begin{equation}
		f_{b1} = \frac{{\SI{345,582}{\kelvin}}}{2\pi}\sqrt{\frac{{\SI{0,000962}{\meter}^2}}{({\SI{0,001671}{\meter}^3}-{\SI{0,000216}{\meter}^3})({\SI{0,034}{\meter}}+ {\SI{0.025354}{\meter}})}}	
\end{equation}
$\Downarrow$
\begin{equation}
		f_{b1} = {\SI{183,57}{\hertz}}
\end{equation}

\subsubsection{Resonansfrevkensen af b2} 
\begin{equation}
		f_{b2} = \frac{{\SI{345,582}{\kelvin}}}{2\pi}\sqrt{\frac{{\SI{0,000962}{\meter}^2}}{({\SI{0,001671}{\meter}^3}-{\SI{0,000383}{\meter}^3})({\SI{0,034}{\meter}}+ {\SI{0.025354}{\meter}})}}	
\end{equation}
$\Downarrow$
\begin{equation}
		f_{b2} = {\SI{195,108}{\hertz}}
\end{equation}

\subsubsection{Resonansfrevkensen af b3} 
\begin{equation}
		f_{b3} = \frac{{\SI{345,582}{\kelvin}}}{2\pi}\sqrt{\frac{{\SI{0,000962}{\meter}^2}}{({\SI{0,001671}{\meter}^3}-{\SI{0,000516}{\meter}^3})({\SI{0,034}{\meter}}+ {\SI{0.025354}{\meter}})}}	
\end{equation}
$\Downarrow$
\begin{equation}
		f_{b3} = {\SI{206,036}{\hertz}}
\end{equation}

\subsection{Udregning af volumen ud fra måleresultater samt teoretiske udregninger}

For at verificere målingerne samt udregningerne, foretages der \textit{backward engineering}, hvor volumen af objektet udregnes ud fra målte samt teoretiske resonansfrekvenser. Den teoretiske resonansfrekvens $f_{0}$ anvendes i udregningerne, da resultaterne af de målte $f_{0}$ er yderst afvigende, og en fejl i testudførslen ikke kan udelukkes.    

Ved at kombinere $f_{0}$ (ligning 4.1) og $f_{b}$ (ligning 4.10), kan volumen af objektet $W$ udledes, som eftervist i afsnit \ref{subsec:efter}. Følgende ligning anvendes til udregning af volumen ud fra $f_{b}$

\begin{equation}
		W_{b} = V\left(1-\left(\frac{f_{0}}{f_{b1}}\right)^2\right)
\end{equation}
 

\subsubsection{Volumen af b1}

\textit{Teoretisk $f_{b1}$}:
\begin{equation}
		W_{b1} = V\left(1-\left(\frac{{\SI{171,295}{\hertz}}
}{{\SI{183,57}{\hertz}}}\right)^2\right) = {\SI{0,000216}{\meter}^3} = {\SI{0,000216}{\liter}}
\end{equation}

\textit{$f_{b1}$ ud fra måling foretaget med højtalerholder}: 
\begin{equation}
		W_{b1} = V\left(1-\left(\frac{{\SI{171,295}{\hertz}}
}{{\SI{150,00}{\hertz}}}\right)^2\right) = {\SI{-0,000508}{\meter}^3} = {\SI{-0,000508}{\liter}}
\end{equation}

\textit{$f_{b1}$ ud fra måling foretaget uden højtalerholder}: 
\begin{equation}
		W_{b1} = V\left(1-\left(\frac{{\SI{171,295}{\hertz}}
}{{\SI{182,12}{\hertz}}}\right)^2\right) = {\SI{0,000193}{\meter}^3} = {\SI{0,000193}{\liter}}
\end{equation}

\hspace{1,5cm}
Resultaterne fremgår af oversigten i tabel \ref{table:b1af}.\\ 

\begin{table}[!h]
\centering
\caption{Tabel over den relative afvigelse af målt volumen af b1 i forhold til den sande volumen}
\label{table:b1af}
\begin{tabular}{|l|l|l|l|}
\hline
\textbf{$f_{b1}$ fundet ud fra:} & \multicolumn{1}{c|}{\textbf{\begin{tabular}[c]{@{}c@{}}Resonansfrevkens\\ {[}Hz{]}\end{tabular}}} & \multicolumn{1}{c|}{\textbf{\begin{tabular}[c]{@{}c@{}}Volumen\\ {[}l{]}\end{tabular}}} & \multicolumn{1}{c|}{\textbf{\begin{tabular}[c]{@{}c@{}}Relativ afvigelse\\ {[}\%{]}\end{tabular}}} \\ \hline
teoretisk udregning & 183,57  & 0,216 & 0 \\ \hline
\begin{tabular}[c]{@{}l@{}}måling foretaget med \\ højtalerholder\end{tabular} & 150,00 & -0,508  & -335 \\ \hline
\begin{tabular}[c]{@{}l@{}}måling foretaget uden \\ højtalerholder\end{tabular} & 182,12 & 0,193 & -11 \\ \hline
\end{tabular}
\end{table}

\subsubsection{Volumen af b2}

\textit{Teoretisk $f_{b2}$}:
\begin{equation}
		W_{b2} = V\left(1-\left(\frac{{\SI{171,295}{\hertz}}
}{{\SI{195,11}{\hertz}}}\right)^2\right) = {\SI{0,000383}{\meter}^3} = {\SI{0,000383}{\liter}}
\end{equation}

\textit{$f_{b2}$ ud fra måling foretaget med højtalerholder}:
\begin{equation}
		W_{b2} = V\left(1-\left(\frac{{\SI{171,295}{\hertz}}
}{{\SI{151,87}{\hertz}}}\right)^2\right) = {\SI{-0,000455}{\meter}^3} = {\SI{-0,000455}{\liter}}
\end{equation}

\textit{$f_{b2}$ ud fra måling foretaget uden højtalerholder}: 
\begin{equation}
		W_{b2} = V\left(1-\left(\frac{{\SI{171,295}{\hertz}}
}{{\SI{186,99}{\hertz}}}\right)^2\right) = {\SI{0,000269}{\meter}^3} = {\SI{0,000269}{\liter}}
\end{equation}

\hspace{1,5cm}
Resultaterne fremgår af oversigten i tabel \ref{table:b2af}.\\ 

\begin{table}[!h]
\centering
\caption{Tabel over den relative afvigelse af målt volumen af b2 i forhold til den sande volumen}
\label{table:b2af}
\begin{tabular}{|l|l|l|l|}
\hline
\textbf{$f_{b2}$ fundet ud fra:} & \multicolumn{1}{c|}{\textbf{\begin{tabular}[c]{@{}c@{}}Resonansfrevkens\\ {[}Hz{]}\end{tabular}}} & \multicolumn{1}{c|}{\textbf{\begin{tabular}[c]{@{}c@{}}Volumen\\ {[}l{]}\end{tabular}}} & \multicolumn{1}{c|}{\textbf{\begin{tabular}[c]{@{}c@{}}Relativ afvigelse\\ {[}\%{]}\end{tabular}}} \\ \hline
teoretisk udregning & 195,11  & 0,383 & 0 \\ \hline
\begin{tabular}[c]{@{}l@{}}måling foretaget med \\ højtalerholder\end{tabular} & 151,87 & -0,455  & -216 \\ \hline
\begin{tabular}[c]{@{}l@{}}måling foretaget uden \\ højtalerholder\end{tabular} & 186,99 & 0,269 & -30 \\ \hline
\end{tabular}
\end{table}

\subsubsection{Volumen af b3}

\textit{Teoretisk $f_{b3}$}:
\begin{equation}
		W_{b3} = V\left(1-\left(\frac{{\SI{171,295}{\hertz}}
}{{\SI{206,04}{\hertz}}}\right)^2\right) = {\SI{0,000516}{\meter}^3} = {\SI{0,000516}{\liter}}
\end{equation}

\textit{$f_{b3}$ ud fra måling foretaget med højtalerholder}:
\begin{equation}
		W_{b3} = V\left(1-\left(\frac{{\SI{171,295}{\hertz}}
}{{\SI{159,13}{\hertz}}}\right)^2\right) = {\SI{-0,000265}{\meter}^3} = {\SI{-0,000265}{\liter}}
\end{equation}

\textit{$f_{b3}$ ud fra måling foretaget uden højtalerholder}: 
\begin{equation}
		W_{b3} = V\left(1-\left(\frac{{\SI{171,295}{\hertz}}
}{{\SI{201,62}{\hertz}}}\right)^2\right) = {\SI{0,000465}{\meter}^3} = {\SI{0,000465}{\liter}}
\end{equation}

\hspace{1,5cm}
Resultaterne fremgår af oversigten i tabel \ref{table:b3af}.\\ 

\begin{table}[!h]
\centering
\caption{Tabel over den relative afvigelse af målt volumen af b3 i forhold til den sande volumen}
\label{table:b3af}
\begin{tabular}{|l|l|l|l|}
\hline
\textbf{$f_{b3}$ fundet ud fra:} & \multicolumn{1}{c|}{\textbf{\begin{tabular}[c]{@{}c@{}}Resonansfrevkens\\ {[}Hz{]}\end{tabular}}} & \multicolumn{1}{c|}{\textbf{\begin{tabular}[c]{@{}c@{}}Volumen\\ {[}l{]}\end{tabular}}} & \multicolumn{1}{c|}{\textbf{\begin{tabular}[c]{@{}c@{}}Relativ afvigelse\\ {[}\%{]}\end{tabular}}} \\ \hline
teoretisk udregning & 206,04  & 0,516 & 0 \\ \hline
\begin{tabular}[c]{@{}l@{}}måling foretaget med \\ højtalerholder\end{tabular} & 159,13 & -0,265  & -151 \\ \hline
\begin{tabular}[c]{@{}l@{}}måling foretaget uden \\ højtalerholder\end{tabular} & 201,62 & 0,465 & -10 \\ \hline
\end{tabular}
\end{table}


\section{Accepttest af systemet}
Den udarbejdede accepttest er ikke gennemført da kravspecifikationen som nævnt, er opbygget om det konceptuelle system. Accepttesten fremgår af Projektdokumentationen, kapitel 2. 

\section{Godkendelse af BVM som medicinsk udstyr}
Godkendelse af medicinsk udstyr kan opfattes som en forhindring i udviklings og produktionsprocessen af nye produkter, da dokumentationen bag en godkendelse kan være lang og meget omfattende. Heldigvis kan denne proces, med indsigt, systematik og struktureret planlægning, være en naturlig del af udviklings- og produktionsprocessen. Allerede i projektets opstartsfase bør produktets \textit{intended use} fastlægges og det bør undersøges, hvorledes dokumentationen skal udarbejdes for at opfylde gældende krav.
Der stilles regulatoriske krav, som skal dokumenteres allerede i den tidlig forskningsfase, hvor f.eks testdokumentation understøtter risikohåndteringen som er påkrævet. 

\begin{figure}[htb]
\centering	
\includegraphics[width=5in]{life}
\caption{Det medicinske udstyrs life cycle}
\label{fig:label}
\end{figure}

I henhold til at få et medicinsk udstyr markedsført i Europa, skal der foretages en godkendelse af produktet, hvilket opnås ved en CE-mærkning. Der stilles omfattende krav til produktet styret af \textit{the European Medical Devices Directive 93/42/EEC}{}(MDD 93/42/EEC). I udgangen af år 2016 eller begyndelsen af 2017 bliver MDD erstattet af \textit{the European Medical Device Regulation}{}(MDR), hvis regulativer skal være implementeret inden udgangen af 2019. 
  
Dette afsnit beskriver vejen til CE-mærket for brystvolumenmåleren. CE-mærkningen starter ved at definere produktet som medicinsk udstyr og derefter at klassificere produktet ud fra \textit{the European Commission's official guidance for Medical Devices - MEDDEV 2.4/1 Rev.9}. Ud fra klassifikationen tydeliggøres det i MEDDEV, hvilke bilag fra MDD, som skal opfyldes for at opnå CE-mærkning. 
Udover klassificeringen, findes gennerelle krav, som ethvert medicinsk udstyr skal opfylde. Disse væsentlige krav indeholder bl.a. en risikoanalyse, som vil blive udarbejdet for brystvolumenmåleren. 
Når dokumentationen for overensstemmelse med gældende krav er udarbejdet, kan CE-mærkning opnåes.
Redegørelsen for vejen til CE-mærkning vil ikke være fyldestgørende, idet ressourcerne hovedsageligt er brugt på det beskrevne proces- og testforløb. Udarbejdelsen af dette afsnit skal ses som en konsulterende redegørelse for håndteringen af de regulatoriske krav. 

\subsection{Definition af BVM som medicinsk udstyr}

BMV kategoriseres som et medicinsk udstyr ud fra definitionen af medicinsk udstyr i MDD 93/42/ EEC, artikel 1.2 (a) og B(e). BMV's anvendelsesformål er bestemmelse af volumen af et bryst med henblik på modificering af anatomien på en patient. Dette anvendelsesformål definerer derved brystvolumenmåleren som værende et medicinsk udstyr. 

\subsection{Klassificeringen af brystvolumenmåleren}

Ud fra MEDDEV guidelines klassificeres BMV'en som et klasse I produkt, ud fra regel 1: \\
\textit{"Devices that either do not touch the patient or contact intact skin only."} \\
Ydermere har produktet en målefunktion og derved skærpes kravene til CE-godkendelsen. Klassificeringen bliver derved en klasse Im.

\subsection{Vejen til CE-mærkning}

For klasse Im-udstyr, og dermed BVM, er der flere veje til at opnå CE-mærkning. I figur \ref{fig:Klas} fremgår det, hvilke bilag i MDD 93/42/EEC, som beskriver kravene til opfyldelse af overensstemmelseserklæring. Der gøres opmærksom på, at bilag VII \textit{skal} opfyldes sammen med \textit{enten} bilag II (pånær sektion 4), bilag IV, bilag V eller bilag VI. Bilag VII er en EF overensstemmelseserklæring, som blandt andet indeholder al den tekniske dokumentation. I bilag II, IV, V og VI stilles der krav til kvalitetssikringssystemer, hvor forskellen er omfanget af kravene til kvalitetssystemerne. Bilag II beskriver kravene til en fuld kvalitetssikring, hvor bilag IV, bilag V og bilag VI beskriver kvalitetssikringskrav til hhv. produktionverifikation, produktion og produkt, hvor kun de de metrologiske aspekter medtages. Producenten må derfor vurdere, hvordan graden af kvalitetssikring og økonomiske omkostninger skal afbalanceres.   

    Den harmoniserede standard DS/EN ISO 13485:2016 kan følges for at sikre overensstemmelse med kvalitetskravene i MDD. Følges hele standarden, er producenten fuldt ud i overensstemmelse med bilag II.
    Der kan opnåes overensstemmelse med de regulatoriske krav, omhandlende risikostyring som findes i bilag VII, ved at følge den harmoniserede standard ISO 14971:2007.
Ydermere skal der gøres opmærksom på at den konceptuelle BVM vil indeholde software. I de regulatoriske krav stilles der krav til software i medicinsk udstyr. Den harmoniserede standard IEC 62304:2006 kan følges for at opnå overensstemmelse med disse krav.   


\begin{figure}[htb]
\centering
\includegraphics[width=5in]{Klassificering}
\caption{Klassificeringen af det medicinske udstyr har indflydelse på hvilke bilag som skal opfyldes for at opnå CE-mærkningen. For klasse Im-udstyr \textit{skal} bilag VII opfyldes sammen med \textit{enten} bilag II (pånær sektion 4), bilag IV, bilag V eller bilag VI (MEDDEV 2. 4/1 Rev. 9 June 2010)}
\label{fig:Klas}
\end{figure}

\subsection{CE-mærkningen}
   
Når bilag VII samt en af de ovenstående kvalitetssikringsbilag er opfyldt, gennemgår og vurderer et selvvalgt bemyndiget organ om de metrologiske apsekter i produktet, lever op til kravene i MDD. Disse bemyndigede organer er private virksomheder, som er udvalgt af nationale sundhedsmyndigheder i EU. Når dokumentationen godkendes udstedes et certifikat, som giver producenten lov til at påsætte CE-mærket på sine produkter, og dermed markedsføre dem.    
 Producenten skal opbevare sin overensstemmelseserklæring i mindst fem år efter produktionen af sidste produkt. Producenten har samtidig det fulde ansvar for at holde sin dokumentation opdateret, således de opfyldte bilag til hver en tid kan godkendes af det bemyndigede organ. Det bemyndigede organ har ansvaret for løbende at vurdere dokumentationen.
Det gøres opmærksom på, at producenten er forpligtiget til at vedligeholde sit markedsovervågningssystem, som oprettes jvf. bilag VII. Efter markedsføringen skal producenten fortsat systematisk indsamle og vurdere erfaringer, som opnås ved brug af det medicinske udstyr på markedet. 

\subsection{Risikovurdering}
I dette afsnit eksemplificeres udførslen af en risikovurdering af anvendelsen af brystvolumenmåleren. Denne risikovurdering er en systematisk fremgangsmåde, hvor sporbarhed er essentielt. Her identificeres og vurderes risikofaktorer, og usikkerhed behandles. 
Det anbefales, at risikovurderingen udarbejdes af et tværfagligt team med eksperter på deres respektive områder, for at opnå en fyldestgørende og helhedsbetragtende risikovurdring.   
En risikovurdering består af en \textit{risikoanalyse} og en \textit{risikoevaluering}. I risikoanalysen identificeres en given fare, hvortil risikoen estimeres. I risikoevalueringen vurderes og vælges hvilke risikoniveauer, der er acceptable, og endvidere analyseres muligheder for en evt. risikoreduktion.

	\subsubsection{Risikoanalyse}
	Der findes forskellige analysemetoder til at identificere risici, og dette eksempel tager udgangspunkt i metoden kaldet \textit{Failure Mode and Effect Analysis}{} (FMEA). Ved brug af FMEA inddeles BMV'en i uafhængige undersystemer, som med fordel kan identificeres ud fra de udarbejdede BBD- og IBD-diagrammer. Ved denne inddeling opnås en kvalitativ og systematisk identificering af risikofaktorer.
	I udarbejdelsen af identificering af farer i forbindelse med BVM, er der blevet taget udgangspunkt i det fundne litteratur, sparring med fagpersoner samt egne udviklingserfaringer. Der er udarbejdet et par eksmepler på identificerede farer, som er blevet videreudviklet til en FMEA-tabel.  
	
	Risikofaktorerne vurderes derefter ud fra en kvantitativ scoring fra 1-10 i hht. \textit{Risk Priority Number}{} (RPN), som fremgår af figur \ref{fig:rpn}. \textit{Sandsynlighed}{}(S) er et begreb for, hvor ofte årsagen til fejltilstanden opstår. \textit{Konsekvens}{}(K) definerer, hvilken effekt fejltilstanden har. \textit{Detektion}{}(D) er et begreb for sandsynligheden for at detektere fejltilstanden. De angivne RPN-værdier er givet ud fra et estimat grundet manglende indsigt i samt data af den konceptuelle BVM. Tabel \ref{table:fmea} er udarbejdet, som et eksempel på, hvorledes FMEA kan anvendes. 	
	 
	\begin{figure}[htb]
	\centering
	\includegraphics[width=5in]{RPN}
	\caption	{RPN er produktet af sandsynlighed, konsekvens og detektion.}
	\label{fig:rpn}
	\end{figure}

\begin{landscape}
Tabel 4.5.5: Et udarbejdet eksempel på, hvorledes FMEA kan anvendes i en risikoanalyse
\centering
\begin{tabular}{|c|l|l|l|l|l|l|l|l|l|l|l|}
\hline
\multicolumn{1}{|l|}{\textbf{Enhed}} & \textbf{Enhed} & \textbf{Funktion} & \textbf{\begin{tabular}[c]{@{}l@{}}Fare eller\\ fejltilstand\end{tabular}} & \textbf{\begin{tabular}[c]{@{}l@{}}Effekt af \\ fare eller\\ fejltilstand\end{tabular}} & \textbf{\begin{tabular}[c]{@{}l@{}}Årsag til \\ fare eller\\ fejltilstand\end{tabular}} & \textbf{S} & \textbf{K} & \textbf{D} & \textbf{RNP} & \textbf{\begin{tabular}[c]{@{}l@{}}Ændrings-\\ dato\end{tabular}} & \textbf{\begin{tabular}[c]{@{}l@{}}Ansvar\\ {[}Init.{]}\end{tabular}} \\ \hline
R1 & Resonator & \begin{tabular}[c]{@{}l@{}}Reflekterer \\ indsendt lyd\end{tabular} & \begin{tabular}[c]{@{}l@{}}Afvigelse af \\ temperatur\end{tabular} & \begin{tabular}[c]{@{}l@{}}Fejl i volumen-\\ udregning\end{tabular} & \begin{tabular}[c]{@{}l@{}}Utæt kant-\\ afgrænsning\end{tabular} & 2 & 5 & 4 & 40 & \multicolumn{1}{c|}{-} & \multicolumn{1}{c|}{-} \\ \hline
R2 & Tryksensor & \begin{tabular}[c]{@{}l@{}}Detekterer \\ anlægstryk\end{tabular} & \begin{tabular}[c]{@{}l@{}}Ikke-ensartet\\ anlægstryk\end{tabular} & \begin{tabular}[c]{@{}l@{}}Fejl i volumen-\\ udregning\end{tabular} & Defekt sensor & 1 & 5 & 9 & 45 & \multicolumn{1}{c|}{-} & \multicolumn{1}{c|}{-} \\ \hline
R3 & ... & ... & ... & ... & ... & ... & ... & ... & ... & ... & ... \\ \hline
\end{tabular}
\label{table:fmea}
\end{landscape}

	\subsubsection{Risikoevaluering}  
	I risikoevalueringen vurderes og bestemmes, hvilke risikoniveauer der er acceptable samt hvilke niveauer der skal behandles mhp. risikoreduktion. Der sættes en tærskelværdi ud fra den definerede RPN-skala, som skelner mellem det acceptable og ikke-acceptable niveau. De risici, som overskider tærskelværdien skal reduceres. Denne risikoreduktion kan foretages ved anvendelse af princippet \textit{as low as reasonably practical}{} (ALARP). ALARP-niveauet er nået, når omkostningerne af yderligere reduktion bliver uhensigtsmæssigt disproportioneret i forhold til den ellers opnåede risikoreduktion.    

\subsection{Kvalitetssikringssystem}

Et kvalitetssystem designes til opfylde de regulatoriske krav som produktets klassificering pålægges i hht. ISO 13485. Det vil sige at det designes ud fra behov. 
Et kvalitetssikringssystem består af en organisationsstruktur, en ansvarsfordeling, procedure, specifikationer, processer og ressoucer. 




