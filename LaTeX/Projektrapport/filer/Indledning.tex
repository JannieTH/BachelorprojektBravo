\chapter{Indledning}

Indenfor det plastikkirurgiske fagområde, efterspørges en standardiseret metode til måling af et brystvolumen \citep{Refworks:15}. Der foretages i Danmark, et stigende antal operationer inden for  de to klassifikationer; KHAD (\textit{Korrigerende operationer på bryst}) og KHAE (\textit{Rekonstruktioner af bryst}) \citep{{RefWorks:13},{RefWorks:14}}. På nuværende tidspunkt findes der ingen klinisk accepteret teknik til brystvolumenmåling, da der mangler evidens for nøjagtigheden af målet \citep{RefWorks:4}. De mest pålidelige målemetoder er i dag 3D-modellering og MRI-scanning \citep{RefWorks:18}. Disse metoder er omkostningsfulde at anvende i praksis, og den mest benyttede metode er derfor anvendelse af en gennemsigtig, plastikskål, hvor plastikkirurgen subjektivt vurderer udfyldningen af skålen \citep{Refworks:15}. Dette er en hurtig og enkel metode, som læner sig op ad Grossman-Roudner-metoden. Ulempen ved denne metode er, at forskelle på volumenmålinger ikke kan undgås mellem afdelinger samt kirurger da der er tale om en subjektiv vurdering. En standardiseret målemetode vil etablere mere præcise nationale retningslinjer samt udjævne disse forskelle. Endvidere opnås tilfredsstillelse hos patienter, sundhedsmyndigheder samt forsikringsselskaber, da alle patienter for fair og lige behandling \citep{Refworks:15}.  

\begin{table}[htb]
\centering
\caption{Antallet af registrerede operationer på bryst defineret ud fra grupperinger}
\label{operationer}
\begin{tabular}{|l|l|r|r|r|}
\hline
\multicolumn{5}{|l|}{\textbf{REGISTREREDE OPERATINOER PÅ BRYST}} \\ \hline
GRUPPERING & REGION/SYGEHUS & 2012 & 2013 & 2014 \\ \hline
\multirow{2}{*}{\begin{tabular}[c]{@{}l@{}}KHAD \\ Korrigerende operationer på bryst\end{tabular}} & Hele landet & 5.206 & 5.504 & 5.507 \\ \cline{2-5} 
 & Privat & 1.803 & 2.403 & 2.414 \\ \hline
\multirow{2}{*}{\begin{tabular}[c]{@{}l@{}}KHAE\\ Rekonstruktioner af bryst\end{tabular}} & Hele landet & 1.568 & 1.864 & 2.066 \\ \cline{2-5} 
 & Privat & 42 & 39 & 56 \\ \hline
\end{tabular}
\end{table}

\section{Baggrund}

Pavia Lumholt, speciallæge i plastikkirurgi, er i gang med at udvikle en metode til at give et objektivt mål for brystvolumen.  
Lumholts metode fungerer efter Helmholtz' princip om resonans. Dette princip beskriver, at når luft presses ind i et hulrum, øges trykket, således luften presses ud og suges tilbage ind, hvilket sætter svininger igang. 
Lumholts metode består af en skal med en mindre hals, som omslutter brystet. Ved at indesende en lyd gennem halsen og opfange den reflekterede lyd, kan der bestemmes et volumen for brystet.
 Skallen fungerer som en resonator, hvori luften bevæger sig, når der indsendes en lyd gennem resonatorens hals (herefter omtalt som \textit{port}). Lyden bevæger luften, som opfører sig som en akustisk fjeder i resonatoren. Dette illustreres i  figur \ref{fig:Helmholtzteori}

\begin{figure}[htb]
\centering
\includegraphics[width=4in]{Helmholtzresonans}
\caption{€€ https://newt.phys.unsw.edu.au/jw/Helmholtz.html}
\label{fig:Helmholtzteori}
\end{figure}

Det har længe været kendt, at man ved brug af Helmholtz' resonansteori, kan bestemme et volumen ud fra resonansfrekvenser (€€ Artikel fra 1987). Fremgangsmåden er at måle resonansfrekvensen i den tomme resonator ($f_{0}$) og efterfølgende resonansfrekvensen i resonatoren, med et objekt placeret deri ($f_{b}$). Ved at kombinere disse to resonansfrevkenser, kan volumen af objektet ($W$) udledes, hvilket eftervises i afnit \ref{subsec:efter}.\\ 
\subsection{Eftervisning af volumenbestemmelse af objekt ud fra Helmholtz resonansteori}
\label{subsec:efter}

 Helmholtz resonansfrekvens i en resonator er givet ved dette udtryk \f

hvor 
\begin{description}[align=left,labelindent=0.3cm]
\item $f_{0}$: resonansfrekvens i en tom resonator [$Hz$],\\
\item $c$: lydens hastighed i luft [$m/s$],\\
\item $S_{p}$: tværsnitsareal af port [$m^2$],\\
\item $V$: statisk volumen af resonator [$m^3$],\\
\item $l_{p}$: længde af port [$m$],\\
\item $\Delta l$: endekorrektion [$m$]\\
\end{description}

Lydens hastighed i luft varierer afhængigt af den omgivende temperatur, og derfor gives $c$ ved udtrykket \c 

hvor $T_{K}$ er givet ved \T

Tværsnitsarealet $S_{p}$ af porten bestemmes ved \Sp 
hvor ${r}$ er radius af porten. 

Grundet luftens massefylde, transmitteres lyden gennem portens længde samt en yderligere merværdi. Denne merværdi udtrykkes ved en endekorrektion $\Delta l$, som gives ved \deltal. 

Når et objekt placeres i en resonator ændres resonansfrekvensen. Dette forhold udtrykkes ved \fb

hvor 
\begin{description}[align=left,labelindent=0.3cm]
\item $f_{b}$: resonansfrekvens i en resonator, indeholdende et objekt [$Hz$],\\
\item $W$: volumen af objekt [$m^3$],\\
\end{description}

Ved at kombinere $f_{0}$ og $f_{b}$, kan volumen af objektet $W$ udledes: 

\Wudl
$\Downarrow$
\Wudled
$\Downarrow$
\W

\section{Problemformulering}
Det er et problem for 

Lægen vurderer ved øjemål om hvorvidt brysterne er lige store. 
Der er to faktorer der kan ”snyde”: forskellige former for bryster kan gøre det svært at vurdere, hvilket der er størst  og når patienter ligger ned kan bryster opføre sig forskelligt.
 
BVM kan også anvendes ved korrektioner eller symmetriskabende operationer samt ved fedttransplantationer (hvor meget har vi opnået ved øget fylde) 

Fedttransplantation er en ny teknologi → hvad virker bedst?
 
Juridisk: økonomisk udgift for patient → læge skal dokumentere, at hun har fået det i, som lægen mener. Objektivt mål. 

Brystreduktion: til vurdering af, om det offentlige kan tilbyde reduktion (grænse omkring 900ml) → hvis under, kan det offentlige ikke tilbyde operation. 


- Et produkt, som er i sin spæde start, hvor der skal afgrænses
 Professionelt faglig målgruppe,  brug, udleje 
- Økonomi 
- Udvikling af ny løsning --> testes frem mod krav, erfaringer -> agil proces
- Produktets problemstilling
- Konkret problemformulering
- Er systemet bygget som en prototype eller et endeligt produkt? 
- Beskrivelse af det samlede system, der er tænkt realiseret i projektet (illustrationer)
- Systembeskrivelse: allerede her tages der udgangspunkt i lovgivningsmæssige krav mhp. overensstemmelse med Im 
- Produktets problemstilling
- Konkret problemformulering

\section{Systembeskrivelse}
- Er systemet bygget som en prototype eller et endeligt produkt? 
- Beskrivelse af det samlede system, der er tænkt realiseret i projektet (illustrationer)
- Systembeskrivelse: allerede her tages der udgangspunkt i lovgivningsmæssige krav mhp. overensstemmelse med Im

\section{Afgrænsning}
- Er systemet bygget som en prototype eller et endeligt produkt? 
- Beskrivelse af det samlede system, der er tænkt realiseret i projektet (illustrationer)
- Systembeskrivelse: allerede her tages der udgangspunkt i lovgivningsmæssige krav mhp. overensstemmelse med Im

