\chapter{Indledning}



- Produktets faglige baggrund 
	Målgruppe? 
- Produktets problemstilling
- Konkret problemformulering
- Beskrivelse af det samlede system, der er tænkt realiseret i projektet (illustrationer)
- Er systemet bygget som en prototype eller et endeligt produkt? 

- Et produkt, som er i sin spæde start, hvor der skal afgrænses
- Bestemmelse af brystvolumen kan anvendes i mange sammenhænge inden for plastikkirurgi, rekonstruktion, reduktion, garanti  
- Mangler en metode til et objektivt mål   
- Udvikling af ny løsning --> testes frem mod krav, erfaringer -> agil proces
- Professionelt brug, udleje 
- Økonomi 


- Systembeskrivelse: allerede her tages der udgangspunkt i lovgivningsmæssige krav mhp. overensstemmelse med Im 

Indenfor det plastikkirurgiske fagområde, efterspørges en metode til måling af et brystvolumen. Denne metode skal medvirke til at gøre en patientudvælgelse mere standardiseret og dermed tilfredsstille sundhedsmyndigheder og forsikringsselskaber for et objektiv målt kriterium, der udjævner forskellene mellem afdelinger og kirurger og etablerer mere præcise nationale retningslinjer. 


START
I Danamark er antallet af operationer på bryst inden for de seneste år stigende. I tabel \ref{operationer} fremgår antallet af operationer inden for de to klassifikationer; KHAD (\textit{Korrigerende operationer på bryst}) og KHAE (\textit{Rekonstruktioner af bryst})\citep{RefWorks:13, RefWorks:14}. Inden for det plastikkirurgiske fagområde, efterspørges en metode til måling af et brystvolumen.

\begin{table}[]
\centering
\caption{My caption}
\label{operationer}
\begin{tabular}{|l|l|r|r|r|}
\hline
\multicolumn{5}{|l|}{\textbf{REGISTREREDE OPERATINOER PÅ BRYST}} \\ \hline
GRUPPERING & REGION/SYGEHUS & 2012 & 2013 & 2014 \\ \hline
\multirow{2}{*}{\begin{tabular}[c]{@{}l@{}}KHAD \\ Korrigerende operationer på bryst\end{tabular}} & Hele landet & 5.206 & 5.504 & 5.507 \\ \cline{2-5} 
 & Privat & 1.803 & 2.403 & 2.414 \\ \hline
\multirow{2}{*}{\begin{tabular}[c]{@{}l@{}}KHAE\\ Rekonstruktioner af bryst\end{tabular}} & Hele landet & 1.568 & 1.864 & 2.066 \\ \cline{2-5} 
 & Privat & 42 & 39 & 56 \\ \hline
\end{tabular}
\end{table}




\section{Baggrund}



Helmholtz resonans teorien ligger til grund for udviklingen af brystvolumenmåleren. Helmholtzresonatoren er et lukket kammer med en lille halsåbning. Ved at påvirke luften i helmholtzresonatoren opstår Helmholtzresonansen hvor luften bevæger sig som en fjeder i resonatoren. Dette illustreres i  figur \ref{fig:Helmholtzteori}

\begin{figure}[htb]
\centering
\includegraphics[width=4in]{Helmholtzresonans}
\caption{€€ https://newt.phys.unsw.edu.au/jw/Helmholtz.html}
\label{fig:Helmholtzteori}
\end{figure}


 
Helmholtsresonansfrekvensen er givet i dette udtryk. 
\begin{eqnarray}
\label{eqn:fnul}
f_{0}&=&\frac{c}{2\pi}\sqrt{\frac{S_{p}}{V \left(l_{p}+\Delta l\right)}}
\end{eqnarray}
hvor c er lydens hastighed i luft, $S_{p}$ er tværsnitsarealet af resonanshalsen, $l_{p}$ er længden af resonanshalsen og $\Delta l$ er en forlængelsesværdi. 

Lydens transmitteres resonatorhalsens længde samt en merværdi grundet luftens massefylde. Denne merværdi udtrykkes ved $\Delta l$ og kan findes ud fra følgende formel. 
\begin{eqnarray}
\Delta l&=&0.6r+\frac{8}{3\pi}r
\end{eqnarray}
hvor r er radius af resonatorhalsen.

Ved at placerer et bryst i resonatoren vil resonansfrekvensen ændre sig og kan udtrykkes ved følgende ligning
\begin{eqnarray}
\label{eqn:fbryst}
f_{0}&=&\frac{c}{2\pi}\sqrt{\frac{S_{p}}{(V-W)\left(l_{p}+\Delta l\right)}}
\end{eqnarray}
hvor W er volumet af brystet. 

Ved at måle resonansfrekvensen af det tomme kammer og resonansfrekvensen af kammeret med brystet placeret deri, kan volumet af brystet bestemmes ved hjælp af ligning \ref{eqn:fnul} og ligning \ref{eqn:fbryst}. Husk reference €€









\section{Problemformulering}

\section{Afgrænsning}
