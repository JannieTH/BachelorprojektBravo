\chapter{Indledning}

Indenfor det plastikkirurgiske fagområde, efterspørges en standardiseret metode til måling af et brystvolumen \citep{RefWorks:15}. Der foretages i Danmark, et stigende antal operationer inden for  de to klassifikationer; KHAD (\textit{Korrigerende operationer på bryst}) og KHAE (\textit{Rekonstruktioner af bryst}) \citep{{RefWorks:13},{RefWorks:14}}. På nuværende tidspunkt findes der ingen klinisk accepteret teknik til brystvolumenmåling, da der mangler evidens for nøjagtigheden af målet \citep{RefWorks:4}. De mest pålidelige målemetoder er i dag 3D-modellering og MRI-scanning \citep{RefWorks:18}. Disse metoder er omkostningsfulde at anvende i praksis, og den mest benyttede metode er derfor anvendelse af en gennemsigtig plastikskål, hvor plastikkirurgen subjektivt vurderer udfyldningen af skålen \citep{RefWorks:15}. Dette er en hurtig og enkel metode, som læner sig op ad Grossman-Roudner-metoden \citep{RefWorks:26}. Ulempen ved denne metode er, at forskelle på volumenmålinger ikke kan undgås mellem afdelinger samt kirurger da der er tale om en subjektiv vurdering \citep{{RefWorks:7},{RefWorks:15},{RefWorks:27}}. En standardiseret målemetode vil etablere mere præcise nationale retningslinjer samt udjævne disse forskelle. Endvidere opnås tilfredsstillelse hos patienter, sundhedsmyndigheder samt forsikringsselskaber, da alle patienter for fair og lige behandling \citep{RefWorks:15}.  

\begin{table}[htb]
\centering
\caption{Antallet af registrerede operationer på bryst defineret ud fra grupperinger \citep{RefWorks:14}}
\label{operationer}
\begin{tabular}{|l|l|r|r|r|}
\hline
\multicolumn{5}{|l|}{\textbf{REGISTREREDE OPERATINOER PÅ BRYST}} \\ \hline
GRUPPERING & REGION/SYGEHUS & 2012 & 2013 & 2014 \\ \hline
\multirow{2}{*}{\begin{tabular}[c]{@{}l@{}}KHAD \\ Korrigerende operationer på bryst\end{tabular}} & Hele landet & 5.206 & 5.504 & 5.507 \\ \cline{2-5} 
 & Privat & 1.803 & 2.403 & 2.414 \\ \hline
\multirow{2}{*}{\begin{tabular}[c]{@{}l@{}}KHAE\\ Rekonstruktioner af bryst\end{tabular}} & Hele landet & 1.568 & 1.864 & 2.066 \\ \cline{2-5} 
 & Privat & 42 & 39 & 56 \\ \hline
\end{tabular}
\end{table}

\section{Baggrund}

Pavia Lumholt, speciallæge i plastikkirurgi, er i gang med at udvikle en metode til at give et objektivt mål for brystvolumen.  
Lumholts metode fungerer efter Helmholtz' princip om resonans. Dette princip beskriver, at når luft presses ind i et hulrum, øges trykket, således luften presses ud og suges tilbage ind, hvilket sætter svininger igang \citep{Refworks:22}. 
Lumholts metode består af en skal med en mindre hals, som omslutter brystet. Ved at indesende en lyd gennem halsen og opfange den reflekterede lyd, kan der bestemmes et volumen for brystet \citep{{Refworks:11}, {Refworks:20}, {Refworks:22}, {Refworks:21}}.
 Skallen fungerer som en resonator, hvori luften bevæger sig, når der indsendes en lyd gennem resonatorens hals (herefter omtalt som \textit{port}). Lyden bevæger luften, som opfører sig som en akustisk fjeder i resonatoren. Dette illustreres i  figur \ref{fig:Helmholtzteori}

\vspace{5mm}  
\begin{figure}[htb]
\centering
\includegraphics[width=4in]{Helmholtzresonans}
\caption{Helmholtz' princip, hvor ændring i tryk medvirker at luft opfører sig som en fjeder \citep{RefWorks:28}}
\label{fig:Helmholtzteori}
\end{figure}
\vspace{5mm}  


Det har længe været kendt, at man ved brug af Helmholtz' resonansteori, kan bestemme et volumen ud fra resonansfrekvenser \citep{Refworks:20}. Fremgangsmåden er at måle resonansfrekvensen i den tomme resonator ($f_{0}$) og efterfølgende resonansfrekvensen i resonatoren, med et objekt placeret deri ($f_{b}$). Ved at kombinere disse to resonansfrevkenser, kan volumen af objektet ($W$) udledes, hvilket eftervises i afnit \ref{subsec:efter}.\\ 

\subsection{Eftervisning af volumenbestemmelse af objekt ud fra Helmholtz resonansteori}
\label{subsec:efter}

I dette afsnit vises det, hvorledes der ud fra Helmholtz' ligning for resonansfrekvens, er opnået en ligning for volumenbestemmelse \citep{Refworks:22}. \\
 Helmholtz resonansfrekvens i en resonator er givet ved dette udtryk \f

hvor 
\begin{description}[align=left,labelindent=0.3cm]
\item $f_{0}$: resonansfrekvens i en tom resonator [$Hz$],\\
\item $c$: lydens hastighed i luft [$m/s$],\\
\item $S_{p}$: tværsnitsareal af port [$m^2$],\\
\item $V$: statisk volumen af resonator [$m^3$],\\
\item $l_{p}$: længde af port [$m$],\\
\item $\Delta l$: endekorrektion [$m$]\\
\end{description}

Lydens hastighed i luft varierer afhængigt af den omgivende temperatur, og derfor gives $c$ ved udtrykket \c 

hvor $T_{K}$ er givet ved \T

Tværsnitsarealet $S_{p}$ af porten bestemmes ved \Sp 
hvor ${r}$ er radius af porten. 

Grundet luftens massefylde, transmitteres lyden gennem portens længde samt en yderligere merværdi. Denne merværdi udtrykkes ved en endekorrektion $\Delta l$, som gives ved \deltal 

Når et objekt placeres i en resonator ændres resonansfrekvensen. Dette forhold udtrykkes ved \fb

hvor 
\begin{description}[align=left,labelindent=0.3cm]
\item $f_{b}$: resonansfrekvens i en resonator, indeholdende et objekt [$Hz$],\\
\item $W$: volumen af objekt [$m^3$],\\
\end{description}

Ved at kombinere $f_{0}$ (ligning 1.1) og $f_{b}$ (ligning 1.6), kan volumen af objektet $W$ udledes: 

\Wudl
$\Downarrow$
\Wudled
$\Downarrow$
\W
\vspace{5mm}  

\section{Problemformulering}
Målet med dette projekt er at udvikle et \textit{minimum viable product} (MVP) til volumenmåling af et bryst, i samarbejde med Pavia Lumholt, speciallæge i plastikkirurgi. Brystvolumenmåleren (herefter omtalt som BVM og systemet) bygger videre på de erfaringer, der er opnået ved tidligere prototyper, udviklet af Lumholt. Metoden til brystvolumenmåling baseres fortsat på Helmholtz' resonansteori, mens prototyperne erstattes med nyt hardware og software. Projektet er et udviklingsprojekt, hvor der testes frem mod erfaringer, som kan opstille krav til systemet. Det udviklede system skal kunne måle volumen af et bryst ved at benytte en resonator, hvori en indsendt lyd reflekteres og opfanges. 

Det færdigudviklede produkt henvender sig til klinisk brug af plastikkirurger samt til professionelt udleje. De økonomiske aspekter i forbindelse med udviklingen er dermed underordnet da der på nuværende tidspunkt ikke findes et lignende produkt på markedet.   

Udover udarbejdelse af udviklings- samt testproces, skal en redegørelse belyse hvilke regulatoriske krav, der skal opfyldes, for at opnå en medicinsk godkendelse samt CE-certificering. Endvidere skal en redegørelse belyse, hvilke metoder der kan benyttes i en risikostyring.

\section{Afgrænsning}
MoSCoW-modellen er en prioriteringsmetode, som anvendes til afgrænsning af projektet. Modellen beskriver, hvilke dele og krav i projektet, som skal opfyldes \textit{(Must have)}, bør opfyldes \textit{(Should have)}, kan opfyldes \textit{(Could have)} og ikke vil opfyldes \textit{(Would not have)}. Således gives en struktureret oversigt over, hvilke krav, der er vigtigst at få opfyldt inden for den givne tidsramme, og endvidere, hvilke krav, som efterfølgende med fordel kan implementeres, hvis tidsrammen tillader det. Figur \ref{fig:moscow} viser, hvordan de enkelte dele og krav i projektet prioriteres i henhold til MoSCoW-metoden.

De krav, som systemet skal opfylde i hht. til \textit{Must have}, dækker funktionerne til et MVP, hvor der er fokus på test af nøjagtighed og præcision. Endvidere skal der kunne fremvises dokumentation for test og for opnåelse af ny viden inden for de anvendte teorier. Desuden skal der foreligge en redegørelse for de regulatoriske krav samt en risikovurdering, gældende for prototypen.

De næst-prioriterede krav afspejles i \textit{Should have}. Disse krav beskriver overvejelserne for videreudviklingen af MVP'en. Der stræbes efter gøre prototypen håndholdt og trådløs, og med en integreret  brugergrænseflade. Endvidere er der gjort overvejelser omkring test af brystfantomer i forskellige størrelser, former og materialer samt linearitet heraf. Yderligere skal forhold vedr. temperatur- samt luftfugtighedsforholds påvirkning på målingerne undersøges, da de forventes at have en  betydning. 
Kravene i \textit{Should have} kræver en validering af den udviklede prototype igennem accepttest. Dette projekt vil derfor i højere grad fokusere på en verificering af den udviklede prototype i form af en accepttest, som tester funktionelle og ikke-funktionelle krav. 

\textit{Could have} og \textit{Would not have} beskriver kravene til den trinvise videreudvikling, hvis tidsrammen tilllader det. 

\begin{figure}[]
\includegraphics[width=5in]{moscow}
\caption{MoSCoW anvendt til prioritering af krav i udviklingsprocessen}
\label{fig:moscow}
\end{figure}


