\chapter{Indledning}

Indledning
Første egentlige afsnit i rapporten er indledningen, som skal få læseren til at forstå pro-duktet (i modsætning til forordet, der handler om rapporten). Produktets faglige baggrund og produktets problemstilling beskrives her.
I dette afsnit indsættes den konkrete projektformulering som I har udarbejdet på baggrund af et evt. projektoplæg.
For at oprette fokus på produktet beskrives det samlede system, der er tænkt realiseret i projektet. Anvend illustrationer for at uddybe systemets funktionalitet for læseren.
Beskrivelsen bør ligeledes fortælle, om systemet er bygget som en prototype eller et ende-ligt produkt.

 

Indenfor det plastikkirurgiske fagområde, efterspørges en metode til måling af brystvolumen, som kan gøre en patientudvælgelse mere standardiseret og dermed tilfredsstille sundhedsmyndigheder og forsikringsselskaber for et objektiv målt kriterium, der udjævner forskellene mellem afdelinger og kirurger og etablerer mere præcise nationale retningslinjer. 


\section{Baggrund}



Helmholtz resonans teorien ligger til grund for udviklingen af brystvolumenmåleren. Helmholtzresonatoren er et lukket kammer med en lille halsåbning. Ved at påvirke luften i helmholtzresonatoren opstår Helmholtzresonansen hvor luften bevæger sig som en fjeder i resonatoren. Dette illustreres i  figur \ref{fig:Helmholtzteori}

\begin{figure}[htb]
\centering
\includegraphics[width=4in]{Helmholtzresonans}
\caption{€€}
\label{fig:Helmholtzteori}
\end{figure}
 
Helmholtsresonansfrekvens er givet i dette udtryk. 

\begin{eqnarray}
f_{0}&=&\frac{c}{2\pi}\sqrt{\frac{S_{p}}{\left(l_{p}+\Delta l\right)}}
\end{eqnarray}

hvor c er lydens hastighed i luft, $S_{p}$ er tværsnitsarealet af resonanshalsen, $l_{p}$ er længden af resonanshalsen og $\Delta l$ er kompensationsværdien. 




\section{Problemformulering}

\section{Afgrænsning}
