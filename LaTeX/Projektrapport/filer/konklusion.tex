\chapter{Konklusion}

Der er i dette projekt udarbejdet et systematisk og struktureret testforløb, hvor krav til systemet er identificeret ud fra de opnåede testerfaringer. Det er ikke lykkes at nå frem til en prototype, som ved brug af Helmholtz' resonansteori kan måle et nøjagtigt og præcist volumen af et objekt eller bryst. Projektet har givet indblik i hvilke problemstillinger der skal løses inden en fungerende prototype kan implementeres i praksis, herunder endekorrektionsfaktorens påvirkning samt lydens transmittering gennem brystet. Resultatet af det samlede udviklingforløb giver incitament til at arbejde videre med udviklingen af prototypen. Der er derfor på baggrund af det konceptuelle system, udarbejdet en redegørelse, som belyser hvilke regulatoriske krav, der skal opfyldes, for at opnå en medicinsk godkendelse samt CE-certificering. Ydermere er der udarbejdet en redegørelse, som belyser, hvilke metoder, der kan benyttes til at identificere, håndtere samt reducere eventuelle risici og farer. 
   
