\section{Abstract}

\textbf{Background} There are currently no clinically accepted technique for breast volume measurement, given the lack of evidence for the accuracy of the measurement. The most reliable methods are cumbersome and costly to apply, and the commonly used method therefore consists of subjective assessments. Subjective assessments causes differences in volume measurements between surgeons and departments, which does not ensure equally patient care.  Therefore a standard measurement method is requested, which would establish more precise national guidelines and evens out volume measurement differences. The aim of the project is to develop a method to provide an objective measure of a breast volume using Helmholtz's resonance theory.\\
\textbf{Methods} The project is an agile development project, which systematically tests towards a product solution. The project is characterized by a comprehensive test process in which the emphasis is on reproducibility and traceability. The development process consists of four main phases, respectively: conceptual, a high-level product specification, development and test, and finally implementation. The project is managed with an Agile Stage-Gate model. The Scrum based tool, Pivotal Tracker, is used for organizing and managing the tasks of the project. \\
\textbf{Results and discussion} The prototype consists of a software program developed in LabVIEW, and a number of hardware components. A speaker emits pink noise, which is transmitted through the resonator port and into the resonator, where the resonance frequency then is sampled by a microphone. It did not succeed to use an internal speaker, as the output capacity of the used analog-to-digital converters, was not sufficient to satisfy the Nyquist sampling theorem on the output pin. Therefore an external audio output has been applied. Due to hardware challenges and lack of knowledge in the field, the development of the prototype has not reached further. In order to demonstrate techniques and skills, the project development is based on a conceptual and an actual system. In addition, an account of the road to CE- certification and risk management is devised to clarify how the conceptual product can be approved for marketing.\\
\textbf{Conclusion} It has not succeeded to measure a precise and accurate volume with the prototype. The project has therefore further given an insight into the problems which must be resolved before a working prototype can be implemented in practice, including the influence of the port length extension factor. The result of the overall development process does however incentive to proceed with the development.
    