\chapter{Metoder}

Dette kapitel omhandler de anvendte metoder i projektets organiserings-, planlægnings-  samt projektstyringsproces. Hensigten er at beskrive, hvorledes processer er opbygget og igangsat, samt med hvilket formål.  

\section{Projektorganisering}
\subsection{Samarbejdsaftale}
I projektets spæde opstart er der udformet og klarlagt, en samarbejdsaftale, fungerende som forventningsafstemning i gruppen. Denne aftale er anvendt som et værktøj til at få diskuteret vigtige parametre for samarbejdet, herunder mødetider, arbejdsform, målsætning, konflikthåndtering mm. \citep{RefWorks:12}. Samarbejdsaftalen fremgår af bilag A, \ref{App:samarbejdsaftale}. 

\subsection{Samarbejdspartnere}
	Projektoplægget er udarbejdet af Pavia Lumholt (PL), speciallæge i plastikkirurgi på OPA Privathospital Aarhus, i samarbejde med Samuel Alberg Thrysøe (SAT). PL agerer som kunden i projektet, og har henvendt sig med en idé, som han ønsker at få udviklet og medicinsk godkendt til klinisk anvendelse. I projektets opstart er der afholdt samarbejdspartnermøde, hvor der blev idé- og erfaringsudvekslet viden. Inden mødet har projektgruppen sørget for at fremsende en mødeindkaldelse samt at klarlægge roller som hhv. ordstyrer og referent. Der er lagt stor vægt på at fremstå professionelle idet gruppen repræsenterer uddannelsesinstitutionen. 
	
	\subsection{Kommunikation}

	\subsubsection{Mail}
	Med ønsket om en struktureret og organiseret arbejdstilgang, har projektgruppen oprettet en fælles mail, tilknyttet projektet. Her foregår al korrespondance med samarbejdspartner, vejleder samt implicerede fagfolk. På denne måde er det muligt at holde mailkorrespondancer adskilt fra private anliggender samt at logge disse mails ét samlet sted. 
	
	\subsubsection{Ekstern fildeling}
	For at gøre det lettilgængeligt at dele viden og udveksle filer, er der anvendt en fælles fildelingstjeneste på Google Drev, som kan tilgås af PL samt projektgruppen.  
	Projektgruppen har gjort PL bekendt med, at der forefindes risici ved at benytte en webbaseret tjeneste som Google Drev. PL er indforstået med dette, og accepterer brugen.
	 
	\subsubsection{Mødeindkaldelser og aktionsreferater}
	Som tidligere beskrevet, er det vigtigt for projektgruppen at fremstå professionelle, systematiske og organiserede i projektarbejdet. Således er der opbygget og oprettet en skabelon for mødeindkaldelser, som struktureret belyser informationer vedr. mødet. Her beskrives emne, formål samt hvad mødets resultat skal anvendes til. Yderemere beskrives mødedetaljer som tidspunkt, sted, mødedeltagere samt hvad der skal forberedes inden mødet, og hvad der evt. skal medbringes. Derudover stilles dagordenen, og en ansvarlig sættes for hvert punkt. Til sidst estimeres mødets varighed. Hensigten med at udsende disse informationer inden mødet, er at der foretages en forventningsafstemning inden mødet, og deltagere ved, hvad der skal være forberedt og medbringes. Mødeindkaldelsesskabelonen fremgår af bilag \ref{App:skabelon}.  
	
	Efter et endt møde, har projektgruppen udsendt et aktionsreferat fra det pågældende møde. Også her er der udarbejdet en struktureret skabelon, som beskriver emne samt formål med mødet, mødeleder, referent og tidspunkt samt varighed. Ud fra dagsordenen er der skrevet et resume til hvert punkt, og endvidere er beslutninger og aktioner sat op, hvor en ansvarlig samt en deadline er tilknyttet. På denne måde er det overskueliggjort, hvem der har hvilke ansvar inden et givent tidspunkt. Hensigten er at lette samarbejdet med implicerede mødedeltagere. Aktionsreferatskabelonen fremgår af bilag \ref{App:skabelonak}.        
	
\section{Projektplanlægning}  
\subsection{Den dynamiske tidsplan} 
I projektets begyndelse er der anvendt et online projektplanlægningsværktøj, Teamweek, som har fungeret som gruppens dynamiske tidsplan og interne kalender. Teamweek er tilpasset efter projektets behov, og større opgaver fra den statiske tidsplan er medtaget. Hensigten med at anvende den dynamiske tidsplan er, at have en let udgave af et Gantt-diagram, som giver overblik over tidsmæssige overlap mellem projektets faser. Figur \ref{fig:teamweek} viser et billede af den dynamiske tidsplans opbygning.  

\begin{figure}
\centering
\includegraphics[width=6in]{teamweek.png}	
\caption{Den dynamiske tidsplan med overlap, øjebliksbillede fra d. 29.09.16,}
\label{fig:teamweek}
\end{figure}

\subsection{Den statiske tidsplan}
\label{subsec:statisk}
\subsubsection{Stage-Gate model}
I projektets indledende faser, hvor der er arbejdet med konceptudvikling samt udkast til kravspecifikation og accepttest, der er anvendt en Stage-Gate model, som fremgår af figur \ref{fig:stagegate}. Stage-Gate modellen er opbygget af \textit{stages}, som repræsenterer projektets faser, og \textit{gates}, som repræsenterer de dertilknyttede kriterier samt deadlines. Inden deadline, skal det pågældende stage's kriterier være opfyldt og afkrydset i den tilhørende tjekliste, som fremgår af figur \ref{fig:tjekliste}. Gaten til næste stage åbnes først når kriterierne er opfyldt. Fordelene ved denne model er, at opdeling, specificering og eksekvering af de foreliggende opgaver, giver mulighed for at danne et helhedsbillede af projektets stages og gates sammenholdt med den tidsmæssige ramme. Dog har den også sine svagheder - den kan tildels sammenlignes med vandfaldsmodellen, og er derfor ikke hensigtsmæssig ved agile udviklingsprocesser. Der er derfor udarbejdet en ny tidsplan undervejs: den agile stage-gate model. 

\newpage
\begin{landscape}
\begin{figure}
\centering	
\includegraphics[width=9.5in]{stagegate.pdf}
\caption{Den anvendte Stage-Gate model}
\label{fig:stagegate}
\end{figure}
\end{landscape}

\begin{figure}[htb]
\centering
\includegraphics[width=4in]{tjekliste.png}
\caption{Tjekliste for opfyldelse af kriterier til fasen \textit{Konceptudvikling}}
\label{fig:tjekliste}
\end{figure}

\subsubsection{Agil Stage-Gate model}
I projekts udviklingsfase (herunder design, implementering samt integrationstest), blev Stage-Gate modellen udskiftet med en Agil Stage-Gate model. Den Agile Stage-Gate model dækker behovet for agilitet og dag-til-dag planlægning, som opstår under et test- og udviklingsforløb af et nyt produkt. Den Agile Stage-Gate er en model, der på nuværende tidspunkt under udvikling. Evidensen på denne nye udviklingsmetode er begrænset og består hovedsageligt af tidligere evidens, hvor der er eksperimenteret med Stage-Gate og Scrum. Derudover findes nyere empirisk evidens fra udviklingsprocesser i førende produktionsvirksomheder, så som LEGO, Coloplast, Grundfos og Danfoss \citep{{Refworks:8},{Refworks:24}}. Den Agile Stage-Gate er anvendt på samme vis som Stage-Gate modellen beskrevet ovenfor, men med agilitet i form af sprints (sprints uddybes i afsnit \ref{subsec:scrum}) og åbne gates. Fordelen ved at anvende den Agile Stage-Gate model er, at den dækker både mikro- og makroplanlægning, og det forventes derfor, at modellen vil opfylde behovet for klare milepæle og faste beslutningspunkter samt hastighed og flexibilitet. 
	   
	\newpage
\begin{landscape}
\begin{figure}
\centering	
\includegraphics[width=9.5in]{agilstagegate.pdf}
\caption{Den anvendte Stage-Gate model}
\label{fig:agilstagegate}
\end{figure}
\end{landscape}

\section{Projektstyring}
	
	\subsection{Scrum}
	\label{subsec:scrum}
	Der er i projektet anvendt elementer fra Scrum. Hver morgen afholdes \textit{Daily Scrum Meetings}, således gruppemedlemmer er opdateret på, hvad der er lavet siden sidst, hvad planen er for den pågældende dag samt eventuelle hindringer. Med henblik på at strukturere og overskueliggøre den dynamiske arbejdsproces, beskrevet i afsnit \ref{subsec:statisk}, er der i projektet anvendt den kendte iterative arbejdsmetode fra Scrum, hvor der løbende bliver prioriteret mellem opgaver. Herefter revuderes og planlægges delopgaver, og disse styres ud fra 7-dages-sprints. Dette gør, at produktet og resultater evalueres og testes løbende. I det efterfølgende afsnit, afsnit \ref{subsec:pivotal}, uddybes det, hvorledes denne styringsproces er anvendt.  
	
	\subsection{Pivotal Tracker}
	\label{subsec:pivotal}
	Pivotal Tracker er et webbaseret projektstyringsværktøj, som muliggør denne agile arbejdstilgang. I Pivotal Trackers icebox, er samtlige arbejdsopgaver defineret. Dette giver et overblik over foreliggende opgaver, og giver samtidig en ro over, at intet forglemmes. Arbejdsopgaverne defineres med en kort beskrivelse og tildeles points. Pointtildelingen sker ved brug af \textit{Planning poker}, som fremgår i figur \ref{fig:planningpoker}, hvorved der opnås enighed om opgavens arbejdsbyrde samt omfang. Denne arbejdsmetode skaber stor gennemsigtighed i arbejdsprocessen, og samtidig et fælles overblik over indholdet i opgaverne. 
	
	\begin{figure}[htb]
	\centering
	\includegraphics[width=2in]{Planningpoker}
	\caption{Anvendelse af Planning poker ved tildeling af points til arbejdsopgaver}
	\label{fig:planningpoker}	
	\end{figure}
	 
	Definererede arbejdsopgaverne ligger herefter med en kort beskrivelse samt pointestimat for omfanget i projektets icebox, klar til at blive flyttet over i backloggen. Backologgen indeholde de opgaver, som prioriteres, og Pivotal Tracker tilføjer automatisk opgaver til det igangværende sprint indtil \textit{Velocity}-grænsen opnås. Velocity er gennemsnittet af points, som gennemføres i løbet af et sprint. En opgaves status defineres ud fra en række forskellige states, herunder \textit{unstarted}, \textit{started}, \textit{finished}, \textit{delivered}, \textit{rejected} og \textit{accepted}. Denne arbejdsprocessen gør det dermed muligt, at en færdiggjort opgave kan afleveres til review hos det andet gruppemedlem, som derefter afviser eller godkender opgaven. Samtidig medvirker denne arbejdsproces til, at projektmedlemmer er inde over alt indhold indhold gennem projektprocessen.     
	
	Ved brug af \textit{Burnup chart'et} i Pivotal Tracker, kan der dannes et overblik over projektets fremgang, hvor der stræbes efter en lineær fremgang, således man undgår en tung arbejdsbyrde mod projektets slutning. Processen sammenholdes med tidsplanen, og ved en eksponentiel fremgang i Burnup chart'et, må en revidering af tidsplanen overvejes, for at opnå en realistisk arbejdsbyrde mod projektet udgang.  
	
	Pivotal Tracker har også den fordel, at den indeholder en komplet historik over de afsluttede sprints med dertilhørende opgaver. I denne log fremgår det, hvilke opgaver, der er udført i hvilken uge, og på den vis kan loggen benyttes som en opgave-logbog. Dog er der i projektet prioriteret at anvende en traditionel logbog, da overvejelser, refleksioner og erfaringer vægtes meget højt i arbejdsprocessen.      		
	
	\subsection{Logbog}
	Logbogen anvendes som et højt prioriteret værktøj i arbejdsprocessen, da projektets store omdrejningspunkt er udviklings- samt testproces. Logbogen benyttes til at dokumentere refleksioner, overvejelser og beslutninger, som er gjort under projektarbejdet. Hver morgen startes med Daily Scrum Meeting, hvorefter logbogen åbnes, og i forlængelse af Daily Scrum meeting, er dagordenen blevet  fastlagt. Logbogens opbygning, som fremgår af figur \ref{fig:logbog}, lægger op til en reflekterende og evaluerende granskning af procesforløbet. Således er procesforløbet løbende blevet evalueret og revideret i forhold til passende arbejdsmetoder. 
	
	\begin{figure}[htb]
	\centering
	\includegraphics[width=4in]{Logbogskabelon}
	\caption{Skabelon anvendt i projektets logbog}
	\label{fig:logbog}	
	\end{figure}	

\section{Udviklingsværktøjer}
	 
	\subsection{\LaTeX}
Det er i projektets indledende uger, prioriteret at bruge tid på at lære at anvende tekstformateringsprogrammet \LaTeX. Fordelene ved at anvende LaTeX, er at der kan fokuseres på at skabe det tekstuelle indhold, da der under skrivningen kun angives strukturelle og logiske kommandoer, som LaTeX derved bruger til at lave indholdfortegnelse, afsnitsinddeling, krydsreferencer, bibliografi mm. Den stilmæssige udformning af layoutet defineres i en særskilt fil, og på denne måde opnås en ensartet typografisk kvalitet, som er klar til udprintning. Hensigten er, at undgå Microsoft Word, hvor der ofte opstår formateringsudfordringer, ved projektarbejdets ende. 
    
	\subsection{RefWorks}
Det webbaserede referenceværktøj RefWorks, er benyttes til at holde styr på kilder fra anvendt litteratur. Projektgruppen har oprettet en fælles account til RefWorks, så alle referencer er samlet i én online database, og på denne måde kan tilgås fra enhver computer. Referencerne i RefWorks-databasen eksporteres til bibliografien i LaTeX, som danner en litteraturliste. Formålet med dette værktøj er, at gøre det problemfrit, at referere til anvendt litteratur. 

	\subsection{LabVIEW 14.0 Development System}
	LabVIEW er et udviklingsmiljø, der med grafisk og intuitiv programmering, gør det nemt at visualisere og kode teknisk software. Formålet med dette værktøj er, at der hurtigt (sammenlignet med traditionelle programmeringssporg) kan produceres et brugerdefineret program, som interagerer med real-world data og signaler. Hensigten er derfor at anvende det værktøj i databehandlingen af lydgenerering og lydopfangning. 
	
	\subsection{Microsoft Visio}   
	Microsoft Visio er et tegneprogram, som bruges til at illustrere forskellige diagrammer. Hensigten er at benytte dette værktøj til at tegne udviklingsdiagrammer.
	
	\subsection{Creately}
	Creately er et webbaseret tegneprogram, som også bruges til at illustrere forskellige diagrammer og modeller. Hensigten er at benytte dette program til at tegne farverige modeller og diagrammer. 
	
\section{Versionsstyring}
\subsection{GitHub}
GitHub er et versionsstyringsprogram, som i projektet anvendes til versionsstyring af dokumenter og LabVIEW-kode. 

I projektet er der gjort brug af versionsstyringssoftwaren GitHub. 
Til versionsstyring af projektdokumentationen og source kode anvendes GitHub, som
bygger på open source versionsstyringssystemet Git. Her opdateres der løbende ændringer,
så det nyeste dokumentation og source kode altid er tilgængeligt. Som user interface til
GitHub anvendes GitHub Desktop (figur: 3.1). I GitHub Desktop vises en tidslinje, for
hvornår, der er lavet ændringer. Under de enkelte filer kan det observeres, hvad der er
ændret i for den gældende version. Programmet giver yderligere indblik, i hvilke filer, der
lokalt er lavet ændringer i, som ikke er tilføjet repositoriet endnu.

dokumenterne fået et nyt versionsnummer og versionshistorikstabellen er opdateret i hvert
dokument, se tabellen nedenfor.
\section{Arbejdsfordeling}

\section{Opnåede erfaringer}
 