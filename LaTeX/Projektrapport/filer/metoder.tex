\chapter{Metoder}

\section{Litteratursøgning}

Der er praktiseret en omhyggelig, systematisk tilgang til litteratursøgningen, så resultatet af projektarbejdet bliver repræsentativt og uden bias. Søgeresultaterne danner grobund for den empiri, som projektarbejdet bygger videre på, og det er derfor vigtigt, at det er solidt nok til at bære analyser og konklusioner. Der er med andre ord, søgt, analyseret og vurderet ny viden som er relevant inden for projektets fagområde og rammer. For at overskueliggøre litteratursøgningen, er denne inddelt i en søgeproces bestående af fire faser, som vises i figur \ref{fig:littpro}, og som endvidere beskrives i nedenstående afsnit.  \\

	\begin{figure}[htb]
			\centering
				\includegraphics[width=5in]{Littproces}
				\caption{Litteratursøgningsprocessens fire faser}	
				\label{fig:littpro}
	\end{figure}
	
	\subsection{Søgestrategi}
	Søgestrategien, beskrevet i søgeprotokollen bilag €€, er udarbejdet med tanke på, at fremsøge det mest relevante information, ud fra gigantiske datamængder. 
	En søgestrategi der kombinerer ord i artiklernes titel og resumé med udvalgte emneord blev anvendt til at finde og screene artikler vedrørende den specifikke problemstilling. 
		
	€€€ Her nævnes søgeord, databser mm.
	
	\subsection{Dataindsamling}
	I screeningen beskrevet ovenfor, blev artikler udvalgt, hvis ordene i artiklens titel og resumé matchede de udvalgte emneord. Derefter blev de udvalgte artikler selvstændigt bedømt af JH og JR, og artikeludvælgelsen blev foretaget sammenholdt med følgende inklusionskriterier:\\
	
		\begin{enumerate}
			\item Valgte emneord med dertilhørende problemstillinger synes besvaret
			\item €€Noget GRADE-agtigt 
			\item €€Noget med studie-typen
			\item Etc. 
		\end{enumerate} 
		
	Alle artikler, som opfyldte ovenstående inklusionskriterier gik videre til dataudvælgelse. 	
		
	\subsection{Dataudvælgelse}	
	Information blev ekstraheret selvstændigt af JH og JR ved at bruge de foruddefinerede kriterier.  
	
	\subsection{Dataanalyse} 
	

 