\chapter{Metoder}

Dette kapitel omhandler de anvendte metoder i projektets organiserings-, planlægnings-  samt projektstyringsproces. Hensigten er at beskrive, hvorledes processer er opbygget og igangsat, samt med hvilket formål.  

\section{Projektorganisering}
\subsection{Samarbejdsaftale}
I projektets spæde opstart er der udformet og klarlagt, en samarbejdsaftale, fungerende som forventningsafstemning i gruppen. Denne aftale er anvendt som et værktøj til at få diskuteret vigtige parametre for samarbejdet, herunder mødetider, arbejdsform, målsætning, konflikthåndtering mm. \citep{RefWorks:12}. Samarbejdsaftalen fremgår af bilag A, \ref{App:samarbejdsaftale}. 

\subsection{Samarbejdspartnere}
	Projektoplægget er udarbejdet af Pavia Lumholt (PL), speciallæge i plastikkirurgi på OPA Privathospital Aarhus, i samarbejde med Samuel Alberg Thrysøe (SAT). PL agerer som kunden i projektet, og har henvendt sig med en idé, som han ønsker at få udviklet og medicinsk godkendt til klinisk anvendelse. I projektets opstart er der afholdt samarbejdspartnermøde, hvor der blev idé- og erfaringsudvekslet viden. Inden mødet har projektgruppen sørget for at fremsende en mødeindkaldelse samt at klarlægge roller som hhv. ordstyrer og referent. Der er lagt stor vægt på at fremstå professionelle idet gruppen repræsenterer uddannelsesinstitutionen. 
	
	\subsection{Kommunikation}

	\subsubsection{Mail}
	Med ønsket om en struktureret og organiseret arbejdstilgang, har projektgruppen oprettet en fælles mail, tilknyttet projektet. Her foregår al korrespondance med samarbejdspartner, vejleder samt implicerede fagfolk. På denne måde er det muligt at holde mailkorrespondancer adskilt fra private anliggender samt at logge disse mails ét samlet sted. 
	
	\subsubsection{Ekstern fildeling}
	For at gøre det lettilgængeligt at dele viden og udveksle filer, er der anvendt en fælles fildelingstjeneste på Google Drev, som kan tilgås af PL samt projektgruppen.  
	Projektgruppen har gjort PL bekendt med, at der forefindes risici ved at benytte en webbaseret tjeneste som Google Drev. PL er indforstået med dette, og accepterer brugen.
	 
	\subsubsection{Mødeindkaldelser og aktionsreferater}
	Som tidligere beskrevet, er det vigtigt for projektgruppen at fremstå professionelle, systematiske og organiserede i projektarbejdet. Således er der opbygget og oprettet en skabelon for mødeindkaldelser, som struktureret belyser informationer vedr. mødet. Her beskrives emne, formål samt hvad mødets resultat skal anvendes til. Ydermere beskrives mødedetaljer som tidspunkt, sted, mødedeltagere samt hvad der skal forberedes inden mødet, og hvad der evt. skal medbringes. Derudover stilles dagordenen, og en ansvarlig sættes for hvert punkt. Til sidst estimeres mødets varighed. Hensigten med at udsende disse informationer inden mødet, er at der foretages en forventningsafstemning inden mødet, og deltagere ved, hvad der skal være forberedt og medbringes. Mødeindkaldelsesskabelonen fremgår af bilag B, \ref{App:skabelon}.  
	
	Efter et endt møde, har projektgruppen udsendt et aktionsreferat fra det pågældende møde. Også her er der udarbejdet en struktureret skabelon, som beskriver emne samt formål med mødet, mødeleder, referent og tidspunkt samt varighed. Ud fra dagsordenen er der skrevet et resume til hvert punkt, og endvidere er beslutninger og aktioner sat op, hvor en ansvarlig samt en deadline er tilknyttet. På denne måde er det overskueliggjort, hvem der har hvilke ansvar inden et givent tidspunkt. Hensigten er at lette samarbejdet med implicerede mødedeltagere. Aktionsreferatskabelonen fremgår af bilag C, \ref{App:skabelonak}.        
	
\section{Projektplanlægning}  
\subsection{Den dynamiske tidsplan} 
I projektets begyndelse er der anvendt et online projektplanlægningsværktøj, Teamweek, som har fungeret som gruppens dynamiske tidsplan og interne kalender. Teamweek er tilpasset efter projektets behov, og større opgaver fra den statiske tidsplan er medtaget. Hensigten med at anvende den dynamiske tidsplan er, at have en let udgave af et Gantt-diagram, som giver overblik over tidsmæssige overlap mellem projektets faser. Figur \ref{fig:teamweek} viser et billede af den dynamiske tidsplans opbygning.  

\begin{figure}
\centering
\includegraphics[width=6in]{teamweek.png}	
\caption{Den dynamiske tidsplan med overlap, øjebliksbillede fra d. 29.09.16,}
\label{fig:teamweek}
\end{figure}

\subsection{Den statiske tidsplan}
\label{subsec:statisk}
\subsubsection{Stage-Gate model}
I projektets indledende faser, hvor der er arbejdet med konceptudvikling samt udkast til kravspecifikation og accepttest, der er anvendt en Stage-Gate model, som fremgår af figur \ref{fig:stagegate}. Stage-Gate modellen er opbygget af \textit{stages}, som repræsenterer projektets faser, og \textit{gates}, som repræsenterer de dertilknyttede kriterier samt deadlines. Inden deadline, skal det pågældende stage's kriterier være opfyldt og afkrydset i den tilhørende tjekliste, som fremgår af figur \ref{fig:tjekliste}. Gaten til næste stage åbnes først når kriterierne er opfyldt. Fordelene ved denne model er, at opdeling, specificering og eksekvering af de foreliggende opgaver, giver mulighed for at danne et helhedsbillede af projektets stages og gates sammenholdt med den tidsmæssige ramme. Dog har den også sine svagheder - den kan tildels sammenlignes med vandfaldsmodellen, og er derfor ikke hensigtsmæssig ved agile udviklingsprocesser. Der er derfor udarbejdet en ny tidsplan undervejs: den agile stage-gate model. 

\newpage
\begin{landscape}
\begin{figure}[htb]
\centering	
\includegraphics[width=9.5in]{stagegate.png}
\caption{Den anvendte Stage-Gate model}
\label{fig:stagegate}
\end{figure}
\end{landscape}

\begin{figure}[htb]
\centering
\includegraphics[width=4in]{tjekliste.png}
\caption{Tjekliste for opfyldelse af kriterier til fasen \textit{Konceptudvikling}}
\label{fig:tjekliste}
\end{figure}

\subsubsection{Agil Stage-Gate model}
\label{subsubsec:agil}
I projekts udviklingsfase (herunder design, implementering samt integrationstest), blev Stage-Gate modellen udskiftet med en Agil Stage-Gate model. Den Agile Stage-Gate model dækker behovet for agilitet og dag-til-dag planlægning, som opstår under et test- og udviklingsforløb af et nyt produkt. Den Agile Stage-Gate er en model, der på nuværende tidspunkt under udvikling. Evidensen på denne nye udviklingsmetode er begrænset og består hovedsageligt af tidligere evidens, hvor der er eksperimenteret med Stage-Gate og Scrum. Derudover findes nyere empirisk evidens fra udviklingsprocesser i førende produktionsvirksomheder, så som LEGO, Coloplast, Grundfos og Danfoss \citep{{Refworks:8},{Refworks:24}}. Den Agile Stage-Gate er anvendt på samme vis som Stage-Gate modellen beskrevet ovenfor, men med agilitet i form af sprints (sprints uddybes i afsnit \ref{subsec:scrum}) og åbne gates. Fordelen ved at anvende den Agile Stage-Gate model er, at den dækker både mikro- og makroplanlægning, og det forventes derfor, at modellen vil opfylde behovet for klare milepæle og faste beslutningspunkter samt hastighed og flexibilitet. 
	   
\newpage
\begin{landscape}
\begin{figure}[ht]
\centering
\includegraphics[width=10in]{agilstagegate}
\caption{Den anvendte Agile Stage-Gate model}
\label{fig:agilstagegate}
\end{figure}
\end{landscape}

\section{Projektstyring}
	
	\subsection{Scrum}
	\label{subsec:scrum}
	Der er i projektet anvendt elementer fra Scrum. Hver morgen afholdes \textit{Daily Scrum Meetings}, således gruppemedlemmer er opdateret på, hvad der er lavet siden sidst, hvad planen er for den pågældende dag samt eventuelle hindringer. Med henblik på at strukturere og overskueliggøre den dynamiske arbejdsproces, beskrevet i afsnit \ref{subsec:statisk}, er der i projektet anvendt den kendte iterative arbejdsmetode fra Scrum, hvor der løbende bliver prioriteret mellem opgaver. Herefter revuderes og planlægges delopgaver, og disse styres ud fra 7-dages-sprints. Dette gør, at produktet og resultater evalueres og testes løbende. I det efterfølgende afsnit, afsnit \ref{subsec:pivotal}, uddybes det, hvorledes denne styringsproces er anvendt.  
	
	\subsection{Pivotal Tracker}
	\label{subsec:pivotal}
	Pivotal Tracker er et webbaseret projektstyringsværktøj, som muliggør denne agile arbejdstilgang. I Pivotal Trackers icebox, er samtlige arbejdsopgaver defineret. Dette giver et overblik over foreliggende opgaver, og giver samtidig en ro over, at intet forglemmes. Arbejdsopgaverne defineres med en kort beskrivelse og tildeles points. Pointtildelingen sker ved brug af \textit{Planning poker}, som fremgår i figur \ref{fig:planningpoker}, hvorved der opnås enighed om opgavens arbejdsbyrde samt omfang. Denne arbejdsmetode skaber stor gennemsigtighed i arbejdsprocessen, og samtidig et fælles overblik over indholdet i opgaverne. 
	
	\begin{figure}[htb]
	\centering
	\includegraphics[width=2in]{Planningpoker}
	\caption{Anvendelse af Planning poker ved tildeling af points til arbejdsopgaver}
	\label{fig:planningpoker}	
	\end{figure}
	 
	Definererede arbejdsopgaverne ligger herefter med en kort beskrivelse samt pointestimat for omfanget i projektets icebox, klar til at blive flyttet over i backloggen. Backologgen indeholde de opgaver, som prioriteres, og Pivotal Tracker tilføjer automatisk opgaver til det igangværende sprint indtil \textit{Velocity}-grænsen opnås. Velocity er gennemsnittet af points, som gennemføres i løbet af et sprint. En opgaves status defineres ud fra en række forskellige states, herunder \textit{unstarted}, \textit{started}, \textit{finished}, \textit{delivered}, \textit{rejected} og \textit{accepted}. Denne arbejdsproces gør det dermed muligt, at en færdiggjort opgave kan afleveres til review hos det andet gruppemedlem, som derefter afviser eller godkender opgaven. Samtidig medvirker denne arbejdsproces til, at projektmedlemmer er inde over alt indhold gennem projektprocessen.     
	
	Ved brug af \textit{Burnup chart'et} i Pivotal Tracker, kan der dannes et overblik over projektets fremgang, hvor der stræbes efter en lineær fremgang, således man undgår en tung arbejdsbyrde mod projektets slutning. Processen sammenholdes med tidsplanen, og ved en eksponentiel fremgang i Burnup chart'et, må en revidering af tidsplanen overvejes, for at opnå en realistisk arbejdsbyrde mod projektet udgang.  
	
	Pivotal Tracker har også den fordel, at den indeholder en komplet historik over de afsluttede sprints med dertilhørende opgaver. I denne log fremgår det, hvilke opgaver, der er udført i hvilken uge, og på den vis kan loggen benyttes som en opgave-logbog. Dog er der i projektet prioriteret at anvende en traditionel logbog, da overvejelser, refleksioner og erfaringer vægtes meget højt i arbejdsprocessen.      		
	
	\subsection{Logbog}
	Logbogen anvendes som et højt prioriteret værktøj i arbejdsprocessen, da projektets store omdrejningspunkt er udviklings- samt testproces. Logbogen benyttes til at dokumentere refleksioner, overvejelser og beslutninger, som er gjort under projektarbejdet. Hver morgen startes med Daily Scrum Meeting, hvorefter logbogen åbnes, og i forlængelse af Daily Scrum meeting, er dagordenen blevet  fastlagt. Logbogens opbygning, som fremgår af figur \ref{fig:logbog}, lægger op til en reflekterende og evaluerende granskning af procesforløbet. Således er procesforløbet løbende blevet evalueret og revideret i forhold til passende arbejdsmetoder. 
	
	\begin{figure}[htb]
	\centering
	\includegraphics[width=4in]{Logbogskabelon}
	\caption{Skabelon anvendt i projektets logbog}
	\label{fig:logbog}	
	\end{figure}	

\section{Udviklingsværktøjer}
	 
	\subsection{\LaTeX}
Det er i projektets indledende uger, prioriteret at bruge tid på at lære at anvende tekstformateringsprogrammet \LaTeX. Fordelene ved at anvende LaTeX, er at der kan fokuseres på at skabe det tekstuelle indhold, da der under skrivningen kun angives strukturelle og logiske kommandoer, som LaTeX derved bruger til at lave indholdfortegnelse, afsnitsinddeling, krydsreferencer, bibliografi mm. Den stilmæssige udformning af layoutet defineres i en særskilt fil, og på denne måde opnås en ensartet typografisk kvalitet, som er klar til udprintning. Hensigten er, at undgå Microsoft Word, hvor der ofte opstår formateringsudfordringer, ved projektarbejdets ende. 
    
	\subsection{RefWorks}
Det webbaserede referenceværktøj RefWorks, er benyttes til at holde styr på kilder fra anvendt litteratur. Projektgruppen har oprettet en fælles account til RefWorks, så alle referencer er samlet i én online database, og på denne måde kan tilgås fra enhver computer. Referencerne i RefWorks-databasen eksporteres til bibliografien i LaTeX, som danner en litteraturliste. Formålet med dette værktøj er, at gøre det problemfrit, at referere til anvendt litteratur. 

	\subsection{LabVIEW 14.0 Development System}
	\label{subsec:labview}
	LabVIEW er et udviklingsmiljø, der med grafisk og intuitiv programmering, gør det simpelt at visualisere og kode teknisk software. Formålet med dette værktøj er, at der hurtigt (sammenlignet med traditionelle programmeringssporg) kan produceres et brugerdefineret program, som interagerer med real-world data og signaler. Hensigten er derfor at anvende det værktøj i databehandlingen af lydgenerering og lydopfangning. 
	
	\subsection{Microsoft Visio}   
	Microsoft Visio er et tegneprogram, som bruges til at illustrere forskellige diagrammer. Hensigten er at benytte dette værktøj til at tegne udviklingsdiagrammer.
	
	\subsection{Creately}
	Creately er et webbaseret tegneprogram, som også bruges til at illustrere forskellige diagrammer og modeller. Hensigten er at benytte dette program til at tegne farverige modeller og diagrammer. 
	
\section{Versionsstyring}
\subsection{GitHub}
GitHub er et versionsstyringsprogram, som i projektet anvendes til versionsstyring af dokumenter og LabVIEW-kode. GitHub
bygger på open source versionsstyringssystemet Git, hvor der løbende opdateres ændringer, så det nyeste dokumentation og LabVIEW-kode altid er tilgængeligt. SourceTree er anvendt som user interface til GitHub-funktionerne. I SourceTree vises et overblik over ændringer, og under de enkelte filer, kan det observeres, hvad der er ændret i den pågældende version. Samtidig knyttes der en kommentar ved hvert commit/ ændring. Dette fremgår af figur \ref{fig:git}. 

\begin{figure}[htb]
\centering
\includegraphics[width=5in]{github.png}
\caption{SourceTree viser overblik over ændringer i enkelte filer}
\label{fig:git}	
\end{figure}

\vspace{0.5cm}

\section{Udviklingsfaserne}
Dette afsnit omhandler de fire udviklingsfaser, der er gennemgået under projektprocessen. Hensigten med dette afsnit er at beskrive de anvendte metoder i hver enkelte fase, samt at eftervise anvendelsen af disse ved at inddrage eksempler.   
  
\subsection{Den første udviklingsfase: konceptudvikling}

Konceptudviklingsfasen består af projektadministrative opgaver, hvor processer og værktøjer er opsat og igangsat. Derudover er der afholdt møde med kunden \textit{(PL)}, hvor det overordnede koncept samt intended use/anvendelsesformål er fastlagt. Samtidig er de første problemstillinger samt afgrænsinger af projektet identificeret, hvorudfra der er udarbejdet den første version af MoSCoW-modellen (MoSCoW v01 fremgår af bilag D, \ref{App:moscowv01}). Endvidere er der foretaget en teoriundersøgelse for at forstå Hemlholtz' princip om resonans, og litteratursøgningsprocessen er planlagt. For at overskueliggøre litteratursøgningen, er denne inddelt i en søgeproces bestående af tre faser: søgestrategi, litteraturindsamling og litteraturudvælgelse. Denne proces beskrives i det kommende afsnit.  
	
\subsection{Den anden udviklingsfase: high-level produktspecifikation}
Der er praktiseret en omhyggelig, systematisk tilgang til litteratursøgningen, så resultatet af projektarbejdet bliver repræsentativt og uden bias. Søgeresultaterne danner grobund for den empiri, som projektarbejdet bygger videre på, og det er derfor vigtigt, at det er solidt nok til at bære analyser og konklusioner. Der er med andre ord, søgt, analyseret og vurderet ny viden som er relevant inden for projektets fagområde og rammer.  

	\subsubsection{Søgestrategi}
	Søgestrategien, beskrevet i søgeprotokollen bilag E \ref{App:sogeprotokol}, er udarbejdet med tanke på, at fremsøge det mest relevante information, ud fra gigantiske datamængder. Endvidere er søgestrategien udarbejdet med henblik på at gøre søgningen reproducerbar for at sikre troværdighed.  
	Søgestrategien, der kombinerer ord i artiklernes titel og resumé med udvalgte emneord, boolske operatorer samt sononymer, er anvendt til at finde og screene artikler vedrørende den specifikke problemstilling. Der er søgt i databaserne; PubMed, Web of Science, Cochrane, og der er anvendt citation tracking samt Google Scholar og derudover håndsøgninger i fagbøger.  
	
	\subsubsection{Litteraturindsamling}
	I screeningen beskrevet ovenfor, blev artikler udvalgt, hvis ordene i artiklens titel og resumé matchede de udvalgte emneord. Derefter blev de udvalgte artikler selvstændigt bedømt af JH og JR, og artikeludvælgelsen blev foretaget sammenholdt med følgende inklusionskriterier:
	
		\begin{enumerate}
			\item engelsk el. nordisk sprog 
			\item valgte emneord med dertilhørende problemstillinger synes besvaret
			\item kildekritisk opfyldelse
			\item studietyper
		\end{enumerate} 
		
	Alle artikler, som opfyldte ovenstående inklusionskriterier gik videre til litteraturudvælgelse. 	
		
	\subsubsection{Litteraturudvælgelse}	
	Artiklerne blev gennemgået med et kritisk øje, og relevant information blev efterfølgende ekstraheret selvstændigt af JH og JR. Det lykkedes ikke, at fremsøge litteratur vedr. kvinders følelse af og holdninger til at anvende et lignende system som BVM. Det blev derfor besluttet at der måtte - hvis tidsrammen tillod - udarbejdes en usabilityundersøgelse på en gruppe på minimum 15 personer i hht. retningslinjer fra \textit{Association of the Advancement of Medical Instrumentation} og \textit{American National Standards Institute} (AAMI/ANSI HE75:2009). \\
	
	\subsubsection{Kravspecifikation og accepttest} 
	Udover litteratursøgning, blev den første version af kravspecifikationen udarbejdet. Kravspecifikationen er udarbejdet på baggrund af det konceptuelle system, omtalt i afsnit \ref{sec:BVMopb}. I kravspecifikationen defineres de funktionelle krav ved anvendelse af et Use case diagram samt en fully dressed Use Case-beskrivelse. I Use Case diagrammet, som fremgår af figur \ref{fig:UC}, illustreres systemet \textit{Brystvolumenmåler}'s Use Case: \textit{udfør brystvolumenmåling}. Til venstre for systemet, vises systemets primære aktør; \textit{plastikkirurgen}, og til højre for systemet vises den sekundære aktør; \textit{patienten}. Nedenfor systemet, fremgår systemets interessenter. Da Use case diagrammet lå færdigt, blev der udarbejdet en fully dressed Use case beskrivelse, som fremgår af tabel \ref{UCtabel}.
	 Formålet med at lave en fully dressed beskrivelse er at klarlægge normalforløbet og alternative flows for brystvolumenmålingen. Endvidere er de ikke-funktionelle krav udarbejdet ud fra daværende kendte HW- og SW-krav.
	 
	 \begin{figure}[htb]
	\centering	
	\includegraphics[width=5in]{UC1.png}
	\caption{Use Case diagrammet
 giver overblik over Use case 1 samt involverede aktører}
	\label{fig:UC}
	\end{figure}
	
	\begin{figure}[htb]
	\includegraphics[width=6in]{UC1tabel}
	\caption{Fully dressed beskrivelse af UC1}
	\label{UCtabel}
	\end{figure}
	
	Efterfulgt af kravspecifikationen, er der udarbejdet en  accepttestprotokol (se Projektdokumentationsrapporten, afsnit 2.2). Denne protokol beskriver alle de forhold og forudsætninger, som skal være opfyldt for at kunne udføre accepttest af den akustiske brystvolumenmåler. Formålet med protokollen er at specificere accepttest-aktiviteterne, gældende for brystvolumenmåleren. 
	I selve accepttesten, forberedes en test til hvert punkt i normalforløbet, som indeholder et krav nr., acceptkriterie samt testmetode. Denne test er med til at verificere at alle krav, der er bestemt i samarbejde med kunden, er opfyldte. 	
	Der var et tydeligt behov, at der i udviklings- samt testfasen, måtte testes frem mod erfaringer, hvorudfra der kunne defineres mere specifikke krav. Udarbejdelse af kravspecifikation samt accepttest var derfor på \textit{high-level}, altså et højt, overordnet plan og under ingen omstændigheder endegyldigt. Der blev i denne fase udarbejdet en ny tidsplansmodel: den Agile Stage-Gate model, omtalt i afsnit \ref{subsubsec:agil}. Hensigten var at have en mere agil projektstyring. 
	
	\subsubsection{Planlægning af testforløb og -dokumentation}
	\label{subsubsec:test}
	Projektet er et udviklingsprojekt, hvor der systematisk testes frem til en produktløsning. Projektet er derfor præget af et omfattende testforløb, hvor der med en systematisk tilgang er lagt vægt på reproducerbarhed samt sporbarhed. 
Det indledende testforløb blev udarbejdet med inspiration fra “Projekteringshåndbogen”, skrevet af Søren Lyngsø-Petersen, som beskriver test af produktionsudstyr til Health Care branchen. Således blev Lyngsø-Petersens testforløbsmodel tilpasset projektets testforløb, som endte ud med at være et forløb inddelt i følgende fem faser:    

	\begin{enumerate}
		\item Enhedstest
		\item Integrationstest
		\item Accepttest
		\item Lab PoC
		\item Kvalificeringstest 
	\end{enumerate}

I første fase, enhedstest, testes de indgående komponenter, for at sikre disses funktion. I anden fase, integrationstest, er hovedformålet at foretage verificeringer af de forskellige funktionaliteter og processer i elementer, som anvendes i det videre testforløb. I tredje fase, accepttest, eftervises alle specificerede krav fra kravspecifikationen, som nævnt i forrige afsnit. I fjerde fase, Lab PoC, testes diverse forhold og hypoteser, og der udvikles mod en ny og bedre version af produktet. I femte fase, kvalificeringstest, foretages en systemvalidering til/med kunden. 
Udarbejdelsen af testforløbet afspejler “God testpraksis”, som er et begreb anvendt inden for Health Care Industrien \citep{RefWorks:29}. “God testpraksis” beskriver en testprocedure, hvor man udfører og dokumenterer sine tests på en måde, som gør dem valide, hvilket afspejles i den måde, hvorpå dokumentationen af de foretagede tests i projektet er opbygget. Projektets testudførelse består derfor af følgende tre forhold;

\begin{enumerate}
	\item beskrivelse af, hvordan testen skal udføres (testprocedure) 
	\item selve udførelsen af testen 
	\item dokumentation af testresultatet
	\end{enumerate}

Testproceduren beskriver den praktiske udførelse af testen, således den er reproducerbar og alle vil have mulighed for at udføre testen, uden at have nogen specifik baggrundsviden. Dokumentationen af hver enkelte test starter med en testhypotese for at afklare, hvad det forventede resultat er. Derefter specificeres det anvendte udstyr og komponenter, og testopstillingen samt - opsætningen beskrives meget udførligt for at sikre en korrekt udførelse af testen. Selve udførelsen af testen beskrives med en høj detaljegrad og kan muligvis forekomme nedladende, men da resultatet kan afhænge af, hvordan testen udføres er dette et nødvendigt forhold. Slutvis fremvises testresultaterne, og disse diskuteres efterfølgende for at sikre en refleksion over de opnåede resultater. Testen afrundes med en konklusion af resultatet i sammenhold med testhypotesen, og der planlægges en aktion for næste skridt. 
	
\subsection{Den tredje udviklingsfase: design, udvikling og test} 
I dette afsnit gives der eksempler på, hvordan projektet er gået fra at være kravorienteret til løsningsorienteret.
Den tredje udviklingsfase er en iterativ proces bestående af design, udvikling, test og opdagelser. Systemets design fastlægges i denne fase på baggrund af erfaringer fra den systematiske testproces, beskrevet ovenfor i afsnit  \ref{subsubsec:test}, samt overvejelser omkring opfyldelseskriterier i forbindelse med godkendelse af det medicinske udstyr. Det vil sige, at der ud fra testerfaringer designes, udvikles samt testes påny indtil det endelig mål opnåes, og den ønskede klassificering kan opfyldes. Systemets design er altså løbende udarbejdet og fastlagt efter testerfaringer samt ved at konferere med fagfolk; herunder Tore A. Skogberg og Lars G. Johansen, lektorer i akustik ved Ingeniørhøjskolen, Aarhus Universitet.

Under design af hardware er der brugt \textit{Block Definition Diagram} (BDD) og \textit{Internal Definition Diagram} (IBD). Diagrammerne benyttes til at beskrive systemet på en overskuelig måde ved opdeling i delsystemer, hvorfra funktioner og sammenhænge fremgår. Ud fra disse diagrammer er systemets udvikling påbegyndt. 

BDD og IBD er udarbejdet over hardware, og er anvendt til at sikre, at hvert enkelte hardwarekomponent kan kommunikere med hinanden. 

Det anvendte BDD, som fremgår af figur \ref{fig:bdd}, giver en black box repræsentation af systemet \textit{brystvolumenmåler}, sammen med sine fysisk sammensatte blokke, hhv. A/D-konvertering, en lydgivende kilde, en resonator, en lydopfanger samt et processeringselement. BDD anvendes til at give det overordnede overblik over, hvilke komponenter brystvolumenmåleren skal bestå af, og flow portene beskriver, hvad der kan gå gennem blokken (ind og/eller ud). 
    
\begin{figure}[htb]
\centering
\includegraphics[width=6in]{bdd.png}
\caption{BDD over brystvolumenmålerens hardwarekomponenter}
\label{fig:bdd}
\end{figure}

Det anvendte IBD, som fremgår af figur \ref{fig:ibd}, giver en white box repræsentation af systemet. IBD'et beskriver mere præcist, hvordan de forskellige komponenter interagerer med hinanden. 

\begin{figure}[htb]
\centering
\includegraphics[width=6in]{ibd.jpg}
\caption{IBD over brystvolumenmåleren}
\label{fig:ibd}
\end{figure}

Systemets software er udviklet i LabVIEW. Som nævnt i afsnit \ref{subsec:labview} udvikles programmet af modulær kode, som afgrænser de enkelte funktionaliteter. Ved objektorienteret programmering beskrives koden typisk vha. klassediagrammer og 3-lags modellen, hvor de enkelte klassers ansvar og grænseflader defineres. I LabVIEW holdes koden simpel og i et logisk flow, som afspejler eksekveringsrækkefølgen. Ved at opdele koden i funktioner fremfor klasser er det ikke hensigtmæssigt at benytte 3-lags modellen. 
Der er i denne fase ikke udviklet et sekvensdiagram ud fra UC1, da softwareudviklingen er afhængig af udviklingen af prototypen. Softwareudviklingen er derfor udarbejdet i takt med testprocessen og de opståede behov og har ikke været forudbestemt. Dog er der alligevel medtaget et simpelt sekvensdiagram over den udviklede software. Dette er gjort for at fremvise metoden samt for at præsentere den udviklede kode med henblik på en eventuel videreudvikling.

\begin{figure}[htb]
\centering
\includegraphics[width=6in]{sekvensdiagram.png}
\caption{Sekvensdiagram over brystvolumenmåleren}
\label{fig:sekvens}
\end{figure}

Overvejelser for valget af de enkelte hardwarekomponenter er resultater af testerfaringer samt anbefalinger af fagfolk, som nævnt ovenfor. 

Eksempelvis fremgår det af figur \ref{fig:flowop}, at der først er testet med én type mikrofon (\textit{M1}, hvorefter der er opdaget en eventuel fejl under udførslen af testen og resultatet af en dermed ny test af mikrofonen har belyst et krav om et bredere frekvensbånd. Derfra er der testet med en ny type højtaler, hvorefter der er opdaget et nyt krav osv.  

For at præsentere de udførte enhedstest på en systematisk og struktureret måde, er enhedstestene angivet med et tildelt ID. ID'et er opbygget af forkortelsen \textit{E} for enhedstest efterfulgt af et nummer, som definerer hvilket nummer enhedstest der er tale om. Endvidere er de anvendte hardwarekomponenter tildelt et ID defineret ud fra deres funktion og rækkefølge; eksempelvis \textit{H3} for \textit{højtaler tre}, altså den tredje anvendte højtaler. Disse to ID'er definerer tilsammen hvilken enhedstest der er tale om, samt hvilken hardwarefunktion der testes. 

\begin{figure}[htb]
\centering
\includegraphics[width=3in]{flowdiagramelyd.png}
\caption{Flowdiagram over de udførte enhedstests af lydgivende kilder}
\label{fig:flowlyd}
\end{figure}

\begin{figure}[htb]
\centering
\includegraphics[width=3in]{flowdiagrameop.png}
\caption{Flowdiagram over de udførte enhedstests af lydopfangere}	
\label{fig:flowop}
\end{figure}


\subsection{Den fjerde udviklingsfase: implementering}

Der er foretaget en kraftig revidering af kravspecifikationen, da der på baggrund af testerfaringerne kan udspecificeres mere konkrete krav. Disse krav er defineret ud fra metoden \textit{FURPS+} (functionality, usability, reliability, performance and supportability, hvor \textit{+} defineres som designkrav). FURPS+-metoden anvendes som en form for checkliste, således alle sider bliver medtaget. Kravspecifikationen afspejler det konceptuelle system, og accepttesten er derfor ikke udført.  

 