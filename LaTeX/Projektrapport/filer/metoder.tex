\chapter{Metoder}

Dette kapitel omhandler de anvendte metoder i projektets organiserings-, planlægnings-  samt projektstyringsproces. Hensigten er at beskrive, hvorledes processer er opbygget og igangsat, samt med hvilket formål.  

\section{Projektorganisering}
\subsection{Samarbejdsaftale}
I projektets spæde opstart er der udformet og klarlagt, en samarbejdsaftale, fungerende som forventningsafstemning i gruppen. Denne aftale er anvendt som et værktøj til at få diskuteret vigtige parametre for samarbejdet, herunder mødetider, arbejdsform, målsætning, konflikthåndtering mm. \citep{RefWorks:12}. Samarbejdsaftalen fremgår af bilag A, \ref{App:samarbejdsaftale}. 

\subsection{Samarbejdspartnere}
	Projektoplægget er udarbejdet af Pavia Lumholt (PL), speciallæge i plastikkirurgi på OPA Privathospital Aarhus, i samarbejde med Samuel Alberg Thrysøe (SAT). PL agerer som kunden i projektet, og har henvendt sig med en idé, som han ønsker at få udviklet og medicinsk godkendt til klinisk anvendelse. I projektets opstart er der afholdt samarbejdspartnermøde, hvor der blev idé- og erfaringsudvekslet viden. Inden mødet har projektgruppen sørget for at fremsende en mødeindkaldelse samt at klarlægge roller som hhv. ordstyrer og referent. Der er lagt stor vægt på at fremstå professionelle idet gruppen repræsenterer uddannelsesinstitutionen. 
	
	\subsection{Kommunikation}

	\subsubsection{Mail}
	Med ønsket om en struktureret og organiseret arbejdstilgang, har projektgruppen oprettet en fælles mail, tilknyttet projektet. Her foregår al korrespondance med samarbejdspartner, vejleder samt implicerede fagfolk. På denne måde er det muligt at holde mailkorrespondancer adskilt fra private anliggender samt at logge disse mails ét samlet sted. 
	
	\subsubsection{Ekstern fildeling}
	For at gøre det lettilgængeligt at dele viden og udveksle filer, er der anvendt en fælles fildelingstjeneste på Google Drev, som kan tilgås af PL samt projektgruppen.  
	Projektgruppen har gjort PL bekendt med, at der forefindes risici ved at benytte en webbaseret tjeneste som Google Drev. PL er indforstået med dette, og accepterer brugen.
	 
	\subsubsection{Mødeindkaldelser og aktionsreferater}
	Som tidligere beskrevet, er det vigtigt for projektgruppen at fremstå professionelle, systematiske og organiserede i projektarbejdet. Således er der opbygget og oprettet en skabelon for mødeindkaldelser, som struktureret belyser informationer vedr. mødet. Her beskrives emne, formål samt hvad mødets resultat skal anvendes til. Yderemere beskrives mødedetaljer som tidspunkt, sted, mødedeltagere samt hvad der skal forberedes inden mødet, og hvad der evt. skal medbringes. Derudover stilles dagordenen, og en ansvarlig sættes for hvert punkt. Til sidst estimeres mødets varighed. Hensigten med at udsende disse informationer inden mødet, er at der foretages en forventningsafstemning inden mødet, og deltagere ved, hvad der skal være forberedt og medbringes. Mødeindkaldelsesskabelonen fremgår af bilag \ref{App:skabelon}.  
	
	Efter et endt møde, har projektgruppen udsendt et aktionsreferat fra det pågældende møde. Også her er der udarbejdet en struktureret skabelon, som beskriver emne samt formål med mødet, mødeleder, referent og tidspunkt samt varighed. Ud fra dagsordenen er der skrevet et resume til hvert punkt, og endvidere er beslutninger og aktioner sat op, hvor en ansvarlig samt en deadline er tilknyttet. På denne måde er det overskueliggjort, hvem der har hvilke ansvar inden et givent tidspunkt. Hensigten er at lette samarbejdet med implicerede mødedeltagere. Aktionsreferatskabelonen fremgår af bilag \ref{App:skabelonak}.        
	
\section{Projektplanlægning}  
\subsection{Den dynamiske tidsplan} 
I projektets begyndelse er der anvendt et online projektplanlægningsværktøj, Teamweek, som har fungeret som gruppens dynamiske tidsplan og interne kalender. Teamweek er tilpasset, og større opgaver fra den statiske tidsplan er medtaget. Hensigten med at anvende den dynamiske tidsplan er, at have en let udgave af et Gantt-diagram, som giver overblik over tidsmæssige overlap mellem projektets faser. Figur \ref{fig:teamweek} viser et billede af tidsplanens opbygning.  

\begin{figure}
\centering
\includegraphics[width=6in]{teamweek.png}	
\caption{Den dynamiske tidsplan med overlap, d. 29.09.16,}
\label{fig:teamweek}
\end{figure}

\subsection{Den statiske tidsplan}
\subsubsection{Stage-Gate model}
I projektets indledende faser, hvor der er arbejdet med konceptudvikling samt udkast til kravspecifikation og accepttest, der er anvendt en Stage-Gate model, som fremgår af figur \ref{fig:stagegate}. Stage-Gate modellen er opbygget af \textit{stages}, som repræsenterer projektets faser, og \textit{gates}, som repræsenterer de dertilknyttede kriterier samt deadlines. Inden deadline, skal det pågældende stage's kriterier være opfyldt og afkrydset i den tilhørende tjekliste, som fremgår af figur \ref{fig:tjekliste}. Gaten til næste stage åbnes først når kriterierne er opfyldt. Fordelene ved denne model er, at opdeling, specificering og eksekvering af de foreliggende opgaver, giver mulighed for at danne et helhedsbillede af projektets stages og gates sammenholdt med den tidsmæssige ramme. Dog har den også sine svagheder - den kan tildels sammenlignes med vandfaldsmodellen, og er derfor ikke hensigtsmæssig ved agile udviklingsprocesser. Der er derfor udarbejdet en ny tidsplan undervejs: den agile stage-gate model. 

\newpage
\begin{landscape}
\begin{figure}
\centering	
\includegraphics[width=9.5in]{stagegate.pdf}
\caption{Den anvendte Stage-Gate model}
\label{fig:stagegate}
\end{figure}
\end{landscape}

\begin{figure}
\centering
\includegraphics[width=4in]{tjekliste.png}
\caption{Tjekliste for opfyldelse af kriterier til fasen \textit{Konceptudvikling}}
\label{fig:tjekliste}
\end{figure}

\subsubsection{Agil Stage-Gate model}
I projekts udviklingsfase (herunder design, implementering samt integrationstest), blev Stage-Gate modellen udskiftet med en Agil Stage-Gate model. Den Agile Stage-Gate model dækker behovet for agilitet og dag-til-dag planlægning, som opstår under et test- og udviklingsforløb af et nyt produkt. Den Agile Stage-Gate er en model, der på nuværende tidspunkt under udvikling. Evidensen på denne nye udviklingsmetode er begrænset og består hovedsageligt af tidligere evidens, hvor der er eksperimenteret med Stage-Gate og Scrum. Derudover findes nyere empirisk evidens fra udviklingsprocesser i førende produktionsvirksomheder, så som LEGO, Coloplast, Grundfos og Danfoss. Den Agile Stage-Gate er anvendt på samme vis som Stage-Gate modellen beskrevet ovenfor, men med agilitet i form af sprints (sprints uddybes i afsnit \ref{subsubsec:pivotal}) og åbne gates. Fordelen ved at anvende den Agile Stage-Gate model er, at den dækker både mikro- og makroplanlægning, og det forventes derfor, at modellen vil opfylde behovet for klare milepæle og faste beslutningspunkter samt hastighed og flexibilitet. 
	   
	\newpage
\begin{landscape}
\begin{figure}
\centering	
\includegraphics[width=9.5in]{agilstagegate.pdf}
\caption{Den anvendte Stage-Gate model}
\label{fig:agilstagegate}
\end{figure}
\end{landscape}

\subsection{Projektstyring}
	€€PivotalTracker, Planning poker, logbog OG agilt: opslagstavle, tavler, analoge oversigter 
	
	\subsubsection{Pivotal Tracker}
	\label{subsubsec:pivotal}
	Med henblik på at strukturere og overskueliggøre denne dynamiske arbejdsproces, er der i projektet anvendt elementer fra Scrum. Ved at bruge denne iterative arbejdsmetode bliver der løbende prioriteret mellem opgaver, hvorefter delopgaver revurderes og planlægges, og styres ud fra 7-dages-sprints. Dette gør, at produktet og resultater evalueres og testes løbende. Pivotal Tracker er et webbaseret projektstyringsværktøj, som muliggør denne agile arbejdstilgang. I Pivotal Trackers icebox, er samtlige arbejdsopgaver defineret. Dette giver et overblik over foreliggende opgaver, og giver samtidig en ro over, at intet forglemmes. Arbejdsopgaverne defineres med en kort beskrivelse og tildeles points. Pointtildelingen er sket ved brug af \textit{Planning poker}, som fremgår i figur \ref{fig:planningpoker}. 
	
	\begin{figure}[htb]
	\centering
	\includegraphics[width=3in]{Planningpoker}
	\caption{Anvendelse af Planning poker ved tildeling af points til arbejdsopgaver}
	\label{fig:planningpoker}	
	\end{figure}
	
	Herved opnås der enighed om opgavens arbejdsbyrde samt omfang. Dette har gjort, at der har været en stor gennemsigtighed i arbejdsprocessen, og samtidig et fælles overblik over indholdet i opgaverne. En opgaves status defineres ud fra en række forskellige states, herunder \textit{unstarted}, \textit{started}, \textit{finished}, \textit{delivered}, \textit{rejected} og \textit{accepted}. Arbejdsprocessen har på den måde muliggjort at en færdiggjort opgave afleveres til review hos det andet gruppemedlem, som derefter afviser eller godkender opgaven. Denne arbejdsproces har medvirket, at projektmedlemmerne har været inde over alt indhold gennem projektprocessen.  
    
    €HJÆLP€ 
	Definererede arbejdsopgaverne ligger herefter med en kort beskrivelse samt pointestimat for omfanget i projektets icebox, klar til at blive flyttet over i backloggen. Backologgen indeholde de opgaver, som prioriteres, og Pivotal Tracker tilføjer automatisk opgaver til det igangværende sprint indtil \textit{Velocity}-grænsen opnås. Velocity er gennemsnittet af points, som gennemføres i løbet af et sprint. Det har i projektarbejdet været en stor udfordring at definere, hvornår et sprint anses for at være \textit{Done}. €€Noget mere om dette!€€    
	
	Ved hjælp af \textit{Burn Up chart'et}, dannes der et overblik over projektets fremgang, og der stræbes efter en lineær fremgang, således man undgår en tung arbejdsbyrde mod projektets slutning. Figur \ref{fig:burnup} viser projektets Burn Up chart, som viser projektets arbejdesomfang sammenholdt med udførte opgaver. Processen har i følge Burn Up chart'et været tilnærmelsesvis lineær. Det fremgår tydeligt, at projektets første uge er præget af indkøring af dette nye værktøj. €Jannie, skriv noget klogt!€
	Processen sammenholdes med tidsplanen, og ved en eksponentiel fremgang i Burn Up chart'et, må en revidering af tidsplanen overvejes, for at opnå en realistisk arbejdsbyrde mod projektet udgang.  
		
	\begin{figure}[htb]
	\centering
	\includegraphics[width=4in]{burnup}
	\caption{Burn Up chart over projektets arbejdsomfang sammenholdt med udførte opgaver}
	\label{fig:burnup}	
	\end{figure}
	    	
	Pivotal Tracker har også den fordel, at den indeholder en komplet historik over de afsluttede sprints med dertilhørende opgaver. I denne log fremgår det, hvilke opgaver, der er udført i hvilken uge, og på den vis kan loggen benyttes som en opgave-logbog. Dog er der i projektet prioriteret at anvende en traditionel logbog, da overvejelser og refleksioner vægtes meget højt i arbejdsprocessen.	
	
	\subsubsection{Logbog}
	Logbogen er anvendt som et højt prioriteret værktøj i arbejdsprocessen, da projektets store omdrejningspunkt er udviklings- samt testproces. Logbogen er benyttet til at dokumentere refleksioner, overvejelser og beslutninger, som er gjort under projektarbejdet. Hver morgen er startet med, at logbogen er blevet åbnet, og i forlængelse af Daily Scrum meeting, er dagordenen blevet  fastlagt. Logbogens opbygning, som fremgår af figur \ref{fig:logbog}, lægger op til en reflekterende og evaluerende granskning af procesforløbet. Således er procesforløbet løbende blevet evalueret og revideret i forhold til passende arbejdsmetoder. Projektgruppen har fundet denne arbejdsmetode tung, men yderst fordelagtig, da ofte vigtige refleksioner og overvejelser hurtigt kan blive forglemt. 
	
	\begin{figure}[htb]
	\centering
	\includegraphics[width=4in]{Logbogskabelon}
	\caption{Skabelon anvendt i projektets logbog}
	\label{fig:logbog}	
	\end{figure}
	
	\subsection{Agile, analoge værktøjer}
	

\section{Udviklingsværktøjer}
	€€LaTeX+ RefWorks, LabVIEW, Visio, Creately, 

	\subsubsection{\LaTeX}
Det blev i projektets indledende uger, prioriteret at bruge tid på at lære at anvende tekstformateringsprogrammet \LaTeX. Fordelene ved at anvende LaTeX, er at der kan fokuseres på at skabe det tekstuelle indhold, da der under skrivningen kun angives strukturelle og logiske kommandoer, som LaTeX derved bruger til at lave indholdfortegnelse, afsnitsinddeling, krydsreferencer, bibliografi mm. Den stilmæssige udformning af layoutet defineres i en særskilt fil, og på denne måde opnås en ensartet typografisk kvalitet, som er klar til udprintning.   
	\subsubsection{RefWorks}
Det online referenceværktøj RefWorks, er benyttet til at holde styr på kilder fra anvendt litteratur. Projektgruppen har oprettet en fælles account til RefWorks, så alle referencer er samlet i én online database, og på denne måde kan tilgås fra enhver computer. Referencerne i RefWorks-databasen eksporteres til bibliografien i LaTeX, som danner en litteraturliste. På denne måde har det i rapportskrivningen været problemfrit at referere til anvendt litteratur.  
	
	



\section{Versionsstyring}
 	€€Dropbox og GitHub

\section{Arbejdsfordeling}

\section{Opnåede erfaringer}
 