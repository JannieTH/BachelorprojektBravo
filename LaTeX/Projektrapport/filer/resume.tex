\section*{Resumé}

\textbf{Baggrund} Der findes på nuværende tidspunkt ingen klinisk accepteret teknik til brystvolumenmåling, da der mangler evidens for nøjagtigheden af målet. De mest pålidelige metoder er omstændelige og omkostningsfulde at anvende, hvorfor den hyppigt anvendte metode består af en subjektiv vurdering. En subjektiv vurdering forårsager forskelle på volumenmålinger mellem kirurger og afdelinger, og sikrer dermed ikke en lige patientbehandling. Der efterspørges derfor en standardiseret målemetode, som kan etablere mere præcise nationale retningslinjer samt udjævne volumenmålingsforskelle. Projektet omhandler udviklingen af en metode til at give et objektivt mål for et brystvolumen, ved brug af Helmholtz' resonansteori. \\
\textbf{Metoder} Projektet er et agilt udviklingsprojekt, hvor der systematisk testes frem mod en produktløsning. Projektet er derfor præget af et omfattende testforløb, hvor der er lagt vægt på reproducerbarhed samt sporbarhed. 
Udviklingsfasen består af fire overordnede faser, hhv. \textit{konceptudvikling}, \textit{high-level produktspecifikation}, \textit{design, udvikling og test} samt \textit{implementering}.
Projektets tidsplan er opbygget som en Agil Stage-Gate model. Projektstyringen af arbejdsopgaverne er organiseret gennem Scrumværktøjet Pivotal Tracker.\\
\textbf{Resultater og diskussion} Den udviklede prototype består af et software program udviklet i LabVIEW, samt en række hardwarekomponenter. En højtaler udsender pink noise, som sendes gennem resonatorens port og ind i resonatoren, hvorefter resonansfrekvensen opfanges af en mikrofon. Det var ikke muligt at anvende en intern højtaler, da output kapaciteten på de anvendte analog-to-digital converters, ikke var tilstrækkelig til at opfylde Nyquist samplingsteori på output-pin'en. Der er derfor anvendt et eksternt lydoutput. Grundet HW-udfordringer og manglende kendskab på fagområdet, er udviklingen af prototypen ikke nået videre, og der er derfor skiftevis arbejdet med det \textit{konceptuelle} og det \textit{aktuelle} system. Dette er for at kunne eftervise metoder samt færdigheder. 
Endvidere er en redegørelse for vejen til CE-certificering samt risikostyring udarbejdet til at belyse, hvorledes det konceptuelle produkt kan godkendes til markedsføring.\\
\textbf{Konklusion} Det har med prototypen ikke været muligt at måle et præcist og nøjagtigt volumen. Projektet har yderligere givet et indblik i hvilke problemstillinger der skal løses inden en fungerende prototype kan implementeres i praksis, herunder endekorrektionfaktorens påvirkning. Resultatet af det samlede udviklingforløb giver dog incitament til at arbejde videre med udviklingen af prototypen.   