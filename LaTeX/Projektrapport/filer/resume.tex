\section{Resumé}

\textbf{Baggrund} \\
Der findes på nuværende tidspunkt ingen klinisk accepteret teknik til brystvolumenmåling, da der mangler evidens for nøjagtigheden af målet. De mest pålidelige metoder er omstændige og omkostningsfulde at anvende i praksis, og den hyppigt anvendte metode består af en subjektiv vurdering ved brug af en gennemsigtig plastikskål. En subjektiv vurdering forårsager forskelle på volumenmålinger mellem afdelinger og kirurger, og sikrer dermed ikke en lige patientbehandling. Der efterspørges derfor en standardiseret målemetode, som kan etablere mere præcise nationale retningslinjer samt udjævne volumenmålingsforskelle. Pavia Lumholt, speciallæge i plastikkirurgi, er i gang med at udvikle en metode til at give et objektivt mål for brystvolumen. På baggrund af Helmholtz' resonansteori er det muligt at foretage en volumenmåling af et objekt, ved at indsende og opfange lyd i en resonator.   

\textbf{Metoder} \\
Projektet er et udviklingsprojekt, hvor der gennem en agil arbejdstilgang, systematisk er testet frem mod en produktløsning. Projektet er derfor præget af et omfattende testforløb, hvor der med en systematisk tilgang er lagt vægt på reproducerbarhed samt sporbarhed. 
Udviklingsfasen bestod overordnet af fire faser, hhv. \textit{konceptudvikling}, \textit{high-level produktspecifikation}, \textit{udvikling og test} samt \textit{implementering}.
Projektets overordnede tidsplan er opbygget som en agil Stage- Gate model, hvor de enkelte udviklingsfaser er fastlagt med deadlines. Projektstyringen af arbejdsopgaverne er organiseret gennem Scrumværktøjet Pivotal Tracker, som har givet et overblik over de enkelte ugers sprints.

\textbf{Resultater og diskussion}\\
Den udviklede prototype består af et software program
udviklet i LabVIEW, samt en række hardwarekomponenter. Prototypen er opbygget om Helmholtz' resonansteori, hvor en højtaler genererer pink noise som sendes gennem resonatorens port og ind i resonatoren, hvorefter resonansfrekvensen opfanges af en mikrofon. Det var ikke muligt at anvende en intern højtaler, da output kapaciteten på de anvendte analog-to-digital converters, ikke var tilstrækkeligt til at sample på output-pin'en med over 150 Hz. Der er derfor anvendt en ekstern lydkilde. Grundet HW-udfordringer er udviklingen af prototypen ikke nået videre, og der er derfor skiftevis arbejdet med det \textit{konceptuelle} og det \textit{aktuelle} system. Dette er for at kunne eftervise metoder samt færdigheder. 
Herudover er en redegørelse for vejen til CE-certificering samt risikostyring udarbejdet til at belyse, hvorledes det konceptuelle produkt kan godkendes til markedsføring.
 
\textbf{Konklusion} \\
 Projektet har ikke vist, at det er muligt at måle et præcist og nøjagtigt volumen. Projektet har yderligere givet
et indblik i hvilke problemstillinger der skal løses inden en fungerende prototype kan implementeres i praksis, herunder hvorvidt endekorrektionfaktorens påvirkning.
Resultatet af det samlede udviklingforløb giver dog incitament til at arbejde videre med udviklingen af prototypen.  