\newpage
\appendix	
\chapter{Bilag}

\section{Bilag A}

\subsection{Samarbejdsaftale}
\label{App:samarbejdsaftale}

\subsubsection{Mødeaftaler}
Det aftales, at det primære arbejde udføres i vores tildelte grupperum på Ingeniørhøjskolen. Om nødvendigt kan arbejde udføres i mere idégenererende omgivelser, for at komme ud af et eventuelt Writer’s block. Arbejdstiden vil ligge primært i hverdagene, og i tidsrummet fra 8-9 tiden til 15-16 tiden, afhængigt af, hvordan det passer med aflevering og afhentning af børn i institution. Det er indforstået, at weekender og aftener kan blive inddraget til projektarbejdet for at overholde tidsplanen.

\subsubsection{Arbejdsform}
Arbejdet vil hovedsageligt være individuelt da det ellers ikke er muligt at overholde vores tidsplan. I nogle områder vil være fordelagtigt at udarbejde i fællesskab, eksempelvis kravspecifikationen. Vi vil sikre at vi begge bliver inddraget i hinandens arbejde ved daglige Scrum-møder samt interne reviews af færdigskrevne afsnit. 

\subsubsection{Målsætning}  
Med dette projekt ønskes der at udarbejde et fungerende produkt samt at vise en systematisk, velstruktureret tilgang til arbejdsprocessen og produktudviklingen. Det ønskes, at projektet udarbejdes så det til eksamen er muligt at indløse topkarakteren. Selvom ambitionsniveauet er højt, skal det ikke gå ud over den gode stemning i gruppearbejdet, og der skal være plads til hyggesnak og kaffepauser. Det skal ligeledes være i orden at have en off-dag, og der er selvfølgelig intet problem ved at man må tilgodese sine børn ved sygdom eller andre forældre-situationer. Der skal gøres plads til individuelle behov i projektarbejdet.

\subsubsection{Relationer til uddannelsesinstitution}
Det ønskes at anvende teori og erfaringer fra de beståede fag. Ligeledes ønskes det at anvende de ressourcer, f.eks. undervisere, som kan være os behjælpelige med svære problemstillinger. Ydermere ønskes det at gøre brug af materiale stillet til rådighed fra Ingeniørhøjskolen, Aarhus Universitets bibliotek. 

\subsubsection{Konfliktløsning}
Skulle der, mod forventning, opstå konflikter i projektarbejdet vil der først og fremmest blive indledt en samtale omkring konflikten. Hver holdning skal respekteres, og findes der ikke en løsning må en tredjepart involveres og fungere som konfliktløser. Denne tredjepart vil formentlig være den tildelte vejleder til projektet.   

\subsubsection{Evaluering og vurdering}
På et ugentligt fredagsmøde vil gruppen, over en kold øl, overordnet drøfte og vurdere, hvordan samarbejdet fungerer. Dette vil være en mundtlig begivenhed, og der vil ved disse møder ikke blive noteret et referat, med mindre der har været en konflikt. Dette vil noteres i den daglige logbog.

\subsubsection{Gruppelogbog} 
Det ønskes at føre en logbog på daglig basis. Logbogen skal være velstruktureret og  indeholde vigtige faglige refleksioner og overvejelser om elementer fra dagens arbejde, som kan være nyttig viden til senere arbejde - her tænkes specielt på projektrapporten. Derudover medtages ekstraordinære begivenheder såsom møde med projektejer, Pavia Lumholt, eller vejleder.

     

\newpage

\section{Bilag B}

\subsection{Skabelon til mødeindkaldelse}
\label{App:skabelon}
\begin{figure}[htb]
\centering
\includegraphics[width=6in]{mode}	
\end{figure}

\newpage

\section{Bilag C}
\subsection{Skabelon til aktionsreferat}
\label{App:skabelonak}
\begin{figure}[htb]
\centering
\includegraphics[width=6in]{Aktionsreferat.png}	
\end{figure}

\newpage

\section{Bilag D}
\subsection{Den første version af MoSCoW-modellen}
\label{App:moscowv01}
\begin{figure}[htb]
\centering
\includegraphics[width=6in]{moscowv01}	
\end{figure}

\newpage

\section{Bilag E}
\subsection{Søgeprotokoller til litteratursøgningsprocessen}
\label{App:sogeprotokol}

\newpage

\section{Bilag F}
\subsection{Udfyldt logbog}
Af de næste side fremgår den udarbejdede logbog
\label{App:logbog}
\includepdf[pages={1-},scale=0.75]{Logbog}


\newpage
\section{Bilag G}
\subsection{Mødeindkaldelser samt aktionsreferater fra vejledermøder}
Af de næste sider fremgår mødeindkaldelser samt aktionsreferater fra møder med samarbejdspartner. 
\includepdf[pages={1-},scale=0.75]{vejl}

\newpage
\section{Bilag H}
\subsection{Mødeindkaldelser samt aktionsreferater fra samarbejdspartnermøder}
Af de næste sider fremgår mødeindkaldelser samt aktionsreferater fra møder med samarbejdspartner. 
\includepdf[pages={1-},scale=0.75]{Samarb}

\newpage
\section{Bilag I}
De næste 15 sider viser databladet for mikrofonen \textit{Electret Microphone MAX4466}. \includepdf[pages={1-},scale=0.75]{MAX4465-MAX4469}



 