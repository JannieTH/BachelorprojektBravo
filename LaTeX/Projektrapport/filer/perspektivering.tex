\chapter{Perspektivering}

\section{Design}
\subsection{Resonator}
Det burde overvejes at kantafgrænse resonatoren med et silikonebaseret tætningsbånd for at undgå ændringer i resonansfrekvensen grundet utætheder. Samtidig er dette et forhold, som burde undersøges nærmere i en testcase. Resonatorens størrelse i forhold til objektet burde envidere også undersøges, da der forefindes modstridende anbefalinger heraf. 
Derudover burde det overvejes at producere resonatoren i rustfrit stål for at reducere temperaturvariation af kammeret, og endvidere for at reducere støj i form af medsvingninger, transmitteringer mm.  
Forhold som længde samt diameter af resonatorhalsen er også parametre der kan medføre en ikke-ideel opførelse i resonatoren, og burde ligeledes undersøges nærmere. 

\subsection{Endekorrektionsfaktoren}
Endekorrektionsfaktoren skal undersøges dybdegående for at opnå valide måleresultater. 

\section{Det videre testforløb}
Det videre testforløb er, efter erfaringer og refleksioner over resonatorens opførelse, skitseret i følgende tabel. Ud fra disse specificerede undersøgelser, testes der mod at nå det konceptuelle produkt.  

\begin{tabularx}{1.1\textwidth}{|l|l|l|X|}
\hline
\textbf{\textbf{\begin{tabular}[c]{@{}l@{}}Videre\\test nr.\end{tabular}}} & \textbf{Undersøgelse} & \textbf{Testmetode} & \textbf{Materialer} \\ \hline
VT1 & \begin{tabular}[c]{@{}l@{}}Opførelse af $f_{b}$ på\\ kropslignende materiale\end{tabular} & \begin{tabular}[c]{@{}l@{}}Teste på forskellige\\ typer brystfantomer \end{tabular} & \begin{tabular}[c]{@{}l@{}} Gelatine, kyllingebryster,\\svinekød og -hud\end{tabular}\\ \hline

VT2 & 
\begin{tabular}[c]{@{}l@{}}Linearitet mellem\\ bryststørrelser og\\ volumenbestemmelser på\\ kropslignende materiale\end{tabular} & 
\begin{tabular}[c]{@{}l@{}}Teste på forskellige\\ typer brystfantomer \end{tabular} & 
\begin{tabular}[c]{@{}l@{}} Gelatine, kyllingebryster,\\svinekød og -hud\end{tabular}\\ \hline
VT3 & 
\begin{tabular}[c]{@{}l@{}}Betydning af resonators\\ kantafgrænsning\end{tabular} & 
\begin{tabular}[c]{@{}l@{}}Teste ud fra et komplet\\ lukket system samt\\ et system med kendte\\ åbninger\end{tabular} & \begin{tabular}[c]{@{}l@{}} Resonator, hvor bund\\ kan skrues fast og\\ forsegles\end{tabular}\\ \hline
VT4 &
\begin{tabular}[c]{@{}l@{}}Betydning af resonators\\ udformning og\\ størrelse\end{tabular} & 
\begin{tabular}[c]{@{}l@{}}Teste med runde og\\ 
firkantede resonatorer\\ i forskellige størrelser\end{tabular} &
\begin{tabular}[c]{@{}l@{}}Firkantede og runde\\ resonatorer bygget \\af træ, stål eller 3D-print\end{tabular}\\ \hline
VT5 &
\begin{tabular}[c]{@{}l@{}}Betydning af placering\\ for hhv.  lydkilde og\\ lydopfanger \end{tabular} & 
\begin{tabular}[c]{@{}l@{}}Teste med forskellige\\ placeringer af lydgiver\\ og lydopfanger\end{tabular} &
        - \\ \hline
VT6 &
\begin{tabular}[c]{@{}l@{}}Betydning af luft-\\temperatur samt \\ luftfugtighed\end{tabular} & 
\begin{tabular}[c]{@{}l@{}}Teste med forskellige\\ temperaturer og \\ luftfugtigheder\end{tabular} &
\begin{tabular}[c]{@{}l@{}}Varmekilde og vand\end{tabular} \\ \hline
\end{tabularx}


\section{Andre anvendelsesmuligheder}
Udviklingen af det konceptuelle system til volumenmåling af et bryst, antages at kunne anvendes til ammemonitorering. I følge Neonatalafdelingen AUH, \textit{Landsforeningen Præmatures Vilkår} og \textit{Dansk Præmatur Forening} er der et stort fokus på amning af præmature spædbørn, og i ammeetableringsfasen er mængden af indtaget modermælk en vigtig parameter. Samtidig vurderes det, at der er en stor målgruppe for hjemmemonitorering af spædbørns modermælksindtag. Det var et stort ønske at udføre en usabilitytest og undersøge dette anvendelsesområde nærmere, men dette blev ikke opnået grundet prioriteringer samt den tidsmæssige ramme.  

Der findes et par konkurrende teknologier på markedet, www.milksense.com og www.mymomsense.com, men ingen af producenterne reklamerer med validiteten af disse produkter, hvilket kan tolkes som man vil. 


