\chapter{Systembeskrivelse}
Dette kapitel omhandler systembeskrivelse af brystvolumenmåleren (BVM). Der tages udgangspunkt i to systembeskrivelser: en \textit{konceptuel} og en \textit{aktuel}. Den konceptuelle beskrivelse introducerer den \textit{tænkte} BVM, hvor den aktuelle beskrivelse introducerer BVM'ens status. Denne inddeling skyldes, at der grundet udviklingsmæssige udfordringer (€REf til road blocks procesrapport), ikke er opnået en prototype, hvor færdigheder inden for dokumentation i form af kravspecifikation samt accepttest ellers kunne fremstilles. Endvidere tages der udgangspunkt i den konceptuelle beskrivelse i redegørelsen for medicinsk godkendelse samt risikostyring for at afspejle disse færdigheder.   

\section{Den konceptuelle brystvolumenmåler} \label{sec:BVMopb}
	
	Den konceptuelle BVM er bygget op af en resonator med en størrelse, hvorpå den kan omslutte en patients bryst. Resonatoren har påmonteret en højtaler som sender lyd ind i resonatorporten. Inde i resonatoren er der monteret en mikrofon til at opsamle resonansfrekvensen. I resonatorkanten, som tilslutter til brystet, er der påsat tryksensorer til detektering af anlægstrykket. Resonatoren er yderligere monteret med en passende mængde dioder til angivelse af et korrekt anlægstryk. Dioderne er placeret så de er synlige for plastikkirurgen. Der er ydermere installeret en CPU til processering af data samt et display, med en størrelse, hvorpå det er muligt at anvise en progressbar for volumenmåling, det målte volumen samt relevante piktogrammer for procestilstanden. På resonatoren er der ligeledes påført tre knapper, en tænd- og sluk-knap, en målingsknap og en kalibreringsknap. Knappernes funktion er angivet med et piktogram, beskrivende for hver funktion. Et batteri er ligeledes tilkoblet så BVM er en trådløs enhed. Et overbliksbillede af de forskellige komponenter som indgår i den konceptuelle BVM, findes i figur \ref{fig:ksys} 
		
\vspace{5mm}  
		
		\begin{figure}[htb]
			\centering
				\includegraphics[width=5in]{Ksys.png}
				\caption{Diagrammet er en visuel beskrivelse af den konceptuelle brystvolumenmåler}	
				\label{fig:ksys}
			\end{figure}	     
		
	
		\subsection{Brystvolumenmålerens funktionalitet}
		Når en måling initialiseres med BVM'en, afsendes en lyd fra højtaleren ind i resonatoren. Mikrofonen, der er monteret inde i resonatoren, opsamler den opståede Helmholtz resonans. Igennem en A/D konvertering udregner en algoritme størrelsen på brystvolumen. 
	
		\subsection{Aktørbeskrivelse}
		Systemets primær aktør er en plastikkirurg, som bruger BMV'en når han ønsker et objektivt mål for volumen på et bryst. Det er udelukkende plastikkirurgen, der  betjener BMV'en under en måling. Som sekundær aktør giver patient et input, brystet, til systemet.
		 
